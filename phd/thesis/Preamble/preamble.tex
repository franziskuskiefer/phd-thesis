% ******************************************************************************
% ****************************** Custom Margin *********************************

% Add `custommargin' in the document class options to use this section
% Set {innerside margin / outerside margin / topmargin / bottom margin}  and
% other page dimensions
\ifsetCustomMargin
  \RequirePackage[left=37mm,right=30mm,top=35mm,bottom=30mm]{geometry}
  \setFancyHdr % To apply fancy header after geometry package is loaded
\fi

% *****************************************************************************
% ******************* Fonts (like different typewriter fonts etc.)*************

% Add `customfont' in the document class option to use this section

\ifsetCustomFont
  % Set your custom font here and use `customfont' in options. Leave empty to
  % load computer modern font (default LaTeX font).
  \RequirePackage{helvet}
\fi

% *****************************************************************************
% **************************** Custom Packages ********************************

% ************************* Algorithms and Pseudocode **************************

%\usepackage{algpseudocode}


% ********************Captions and Hyperreferencing / URL **********************

% Captions: This makes captions of figures use a boldfaced small font.
%\RequirePackage[small,bf]{caption}

\RequirePackage[labelsep=space,tableposition=top]{caption}
\renewcommand{\figurename}{Fig.} %to support older versions of captions.sty


% *************************** Graphics and figures *****************************

%\usepackage{rotating}
%\usepackage{wrapfig}

% Uncomment the following two lines to force Latex to place the figure.
% Use [H] when including graphics. Note 'H' instead of 'h'
%\usepackage{float}
%\restylefloat{figure}

% Subcaption package is also available in the sty folder you can use that by
% uncommenting the following line
% This is for people stuck with older versions of texlive
%\usepackage{sty/caption/subcaption}
\usepackage{subcaption}

% ********************************** Tables ************************************
\usepackage{booktabs} % For professional looking tables
\usepackage{multirow}

%\usepackage{multicol}
%\usepackage{longtable}
%\usepackage{tabularx}


% ***************************** Math and SI Units ******************************

\usepackage{amsfonts}
\usepackage{amsmath}
\usepackage{amssymb}
\usepackage{siunitx} % use this package module for SI units

% ***************************** My Custom Packages ******************************

\usepackage[list-header=chapter]{acro}
\usepackage{xspace}

% ******************************* Line Spacing *********************************

% Choose linespacing as appropriate. Default is one-half line spacing as per the
% University guidelines

% \doublespacing
% \onehalfspacing
% \singlespacing


% ************************ Formatting / Footnote *******************************

% Don't break enumeration (etc.) across pages in an ugly manner (default 10000)
%\clubpenalty=500
%\widowpenalty=500

%\usepackage[perpage]{footmisc} %Range of footnote options


% *****************************************************************************
% *************************** Bibliography  and References ********************

%\usepackage{cleveref} %Referencing without need to explicitly state fig /table

% Add `custombib' in the document class option to use this section
\ifuseCustomBib
   \RequirePackage[square, sort, numbers, authoryear]{natbib} % CustomBib

% If you would like to use biblatex for your reference management, as opposed to the default `natbibpackage` pass the option `custombib` in the document class. Comment out the previous line to make sure you don't load the natbib package. Uncomment the following lines and specify the location of references.bib file

%\RequirePackage[backend=biber, style=numeric-comp, citestyle=numeric, sorting=nty, natbib=true]{biblatex}
%\bibliography{References/references} %Location of references.bib only for biblatex

\fi

% changes the default name `Bibliography` -> `References'
\renewcommand{\bibname}{References}


% *****************************************************************************
% *************** Changing the Visual Style of Chapter Headings ***************
% This section on visual style is from https://github.com/cambridge/thesis

% Uncomment the section below. Requires titlesec package.

%\RequirePackage{titlesec}
%\newcommand{\PreContentTitleFormat}{\titleformat{\chapter}[display]{\scshape\Large}
%{\Large\filleft{\chaptertitlename} \Huge\thechapter}
%{1ex}{}
%[\vspace{1ex}\titlerule]}
%\newcommand{\ContentTitleFormat}{\titleformat{\chapter}[display]{\scshape\huge}
%{\Large\filleft{\chaptertitlename} \Huge\thechapter}{1ex}
%{\titlerule\vspace{1ex}\filright}
%[\vspace{1ex}\titlerule]}
%\newcommand{\PostContentTitleFormat}{\PreContentTitleFormat}
%\PreContentTitleFormat


% ******************************************************************************
% ************************* User Defined Commands ******************************
% ******************************************************************************

% *********** To change the name of Table of Contents / LOF and LOT ************

%\renewcommand{\contentsname}{My Table of Contents}
%\renewcommand{\listfigurename}{My List of Figures}
%\renewcommand{\listtablename}{My List of Tables}


% ********************** TOC depth and numbering depth *************************

\setcounter{secnumdepth}{2}
\setcounter{tocdepth}{2}


% ******************************* Nomenclature *********************************

% To change the name of the Nomenclature section, uncomment the following line

%\renewcommand{\nomname}{Symbols}


% ********************************* Appendix ***********************************

% The default value of both \appendixtocname and \appendixpagename is `Appendices'. These names can all be changed via:

%\renewcommand{\appendixtocname}{List of appendices}
%\renewcommand{\appendixname}{Appndx}

% ******************************** Draft Mode **********************************

% Uncomment to disable figures in `draftmode'
%\setkeys{Gin}{draft=true}  % set draft to false to enable figures in `draft'

% These options are active only during the draft mode
% Default text is "Draft"
%\SetDraftText{DRAFT}

% Default Watermark location is top. Location (top/bottom)
%\SetDraftWMPosition{bottom}

% Draft Version - default is v1.0
\SetDraftVersion{v0.1}

% Draft Text grayscale value (should be between 0-black and 1-white)
% Default value is 0.75
%\SetDraftGrayScale{0.8}


%% Todo notes functionality
%% Uncomment the following lines to have todonotes.

\ifsetDraft
	\usepackage[colorinlistoftodos]{todonotes}
	\newcommand{\mynote}[1]{\todo[size=\small,inline,color=blue!30]{#1}} %author=fk,
\else
	\newcommand{\mynote}[1]{}
	\newcommand{\listoftodos}{}
\fi

% Example todo: \mynote{Hey! I have a note}


%****************************************************************************************************
% 9. Handy macros
% ****************************************************************************************************
\newcommand{\Send}{\ensuremath{\mathsf{Send}}\xspace}
\newcommand{\Execute}{\ensuremath{\mathsf{Execute}}\xspace}
\newcommand{\Reveal}{\ensuremath{\mathsf{Reveal}}\xspace}
\newcommand{\Test}{\ensuremath{\mathsf{Test}}\xspace}
\newcommand{\Corrupt}{\ensuremath{\mathsf{Corrupt}}\xspace}

% we have to get back the old marginpar for todonotes
\let\marginpar\oldmarginpar

% make href and footnote
\newcommand{\myhref}[2]{\href{#1}{#2}\footnote{\url{#1}}}

% Generel Model Variables
\newcommand{\A}{\ensuremath{\mathcal{A}}\xspace}

% UC variables
\newcommand{\UCZ}{\ensuremath{\mathcal{Z}}\xspace}
\newcommand{\UCS}{\ensuremath{\mathcal{S}}\xspace}
\newcommand{\UCF}{\ensuremath{\mathcal{F}}\xspace}

% General notation
\newcommand{\algout}{\ensuremath{\gets}\xspace}
\newcommand{\ralgout}{\ensuremath{\stackrel{\$}{\gets}}\xspace}
\newcommand{\rin}{\ensuremath{\in_R}\xspace}

\newcommand{\bits}{\ensuremath{\{0,1\}}\xspace}
\newcommand{\bigo}{\ensuremath{\mathcal{O}}\xspace}
\newcommand{\secpar}{\ensuremath{\lambda}\xspace}

\newcommand{\pk}{\ensuremath{\mathtt{pk}}\xspace}
\newcommand{\sk}{\ensuremath{\mathtt{sk}}\xspace}

% Numbers
\newcommand{\NN}{\ensuremath{\mathbb{N}}\xspace}
\newcommand{\PP}{\ensuremath{\mathbb{P}}\xspace}
\newcommand{\ZZ}{\ensuremath{\mathbb{Z}}\xspace}
\newcommand{\RR}{\ensuremath{\mathbb{R}}\xspace}
\newcommand{\FF}{\ensuremath{\mathbb{F}}\xspace}
\newcommand{\GG}{\ensuremath{\mathbb{G}}\xspace}

% Caligraphic Letters
\newcommand{\cA}{\ensuremath{\mathcal{A}}\xspace}
\newcommand{\cB}{\ensuremath{\mathcal{B}}\xspace}
\newcommand{\cC}{\ensuremath{\mathcal{C}}\xspace}
\newcommand{\cD}{\ensuremath{\mathcal{D}}\xspace}
\newcommand{\cF}{\ensuremath{\mathcal{F}}\xspace}
\newcommand{\cI}{\ensuremath{\mathcal{I}}\xspace}
\newcommand{\cK}{\ensuremath{\mathcal{K}}\xspace}
\newcommand{\cL}{\ensuremath{\mathcal{L}}\xspace}
\newcommand{\cM}{\ensuremath{\mathcal{M}}\xspace}
\newcommand{\cO}{\ensuremath{\mathcal{O}}\xspace}
\newcommand{\cS}{\ensuremath{\mathcal{S}}\xspace}
\newcommand{\cX}{\ensuremath{\mathcal{X}}\xspace}
\newcommand{\cZ}{\ensuremath{\mathcal{Z}}\xspace}

% Passwords
\newcommand{\pwd}{\ensuremath{\mathtt{pwd}}\xspace}
\newcommand{\pwdv}{\ensuremath{\mathtt{{\bf pwd}}}\xspace}

% crs
\newcommand{\crs}{\ensuremath{\mathtt{crs}}\xspace}

% functions
\newcommand{\KGen}{\ensuremath{\mathsf{KGen}}\xspace}
\newcommand{\Enc}{\ensuremath{\mathsf{Enc}}\xspace}
\newcommand{\Dec}{\ensuremath{\mathsf{Dec}}\xspace}
\newcommand{\Exp}{\ensuremath{\mathsf{Exp}}\xspace}
\newcommand{\PRF}{\ensuremath{\mathsf{PRF}}\xspace}

\newcommand{\Adv}{\ensuremath{\mathsf{Adv}}\xspace}
\newcommand{\ccatwo}{\ensuremath{\mathsf{IND-CCA2}}\xspace}
\newcommand{\AKESEC}{\ensuremath{\mathsf{AKE-SEC}}\xspace}

% PAKE model
\newcommand{\pid}{\ensuremath{\mathtt{pid}}\xspace}
\newcommand{\sid}{\ensuremath{\mathtt{sid}}\xspace}
\newcommand{\ssid}{\ensuremath{\mathtt{ssid}}\xspace}
\newcommand{\trans}{\ensuremath{\mathtt{trans}}\xspace}
\newcommand{\key}{\ensuremath{\mathtt{k}}\xspace}
\newcommand{\mIn}{\ensuremath{\mathtt{m}_{\mathrm{in}}}\xspace}
\newcommand{\mOut}{\ensuremath{\mathtt{m}_{\mathrm{out}}}\xspace}
\newcommand{\state}{\ensuremath{\mathtt{state}}\xspace}
\newcommand{\used}{\ensuremath{\mathtt{used}}\xspace}
\newcommand{\role}{\ensuremath{\mathtt{role}}\xspace}
\newcommand{\client}{\ensuremath{\mathtt{client}}\xspace}
\newcommand{\server}{\ensuremath{\mathtt{server}}\xspace}
\newcommand{\NULL}{\ensuremath{\mathtt{NULL}}\xspace}

\newcommand{\cFPAKE}{\ensuremath{\cF_{\mathrm{PAKE}}}\xspace}
\newcommand{\NS}{\ensuremath{\mathtt{NS}}\xspace}
\newcommand{\TP}{\ensuremath{\mathtt{TP}}\xspace}
\newcommand{\NK}{\ensuremath{\mathtt{NK}}\xspace}

% SPHF
\newcommand{\SPHFF}{\ensuremath{\text{SPHF}^x}\xspace}
\newcommand{\aux}{\ensuremath{\mathtt{aux}}\xspace}

\newcommand{\hk}{\ensuremath{\mathtt{k_h}}\xspace}
\newcommand{\hp}{\ensuremath{\mathtt{k_p}}\xspace}

\newcommand{\HKGen}{\ensuremath{\mathtt{KGen_H}}\xspace}
\newcommand{\PKGen}{\ensuremath{\mathtt{KGen_P}}\xspace}
\newcommand{\Hash}{\ensuremath{\mathtt{Hash}}\xspace}
\newcommand{\PHash}{\ensuremath{\mathtt{PHash}}\xspace}

% general abbrevs.
\newcommand{\ie}{i.\,e.\xspace}
\newcommand{\Ie}{I.\,e.\xspace}
\newcommand{\eg}{e.\,g.,\xspace}
\newcommand{\Eg}{E.\,g.,\xspace} 
\newcommand{\etal}{et al.\xspace}
