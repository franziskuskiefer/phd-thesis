\chapter{Password-based Authentication in the Single-Server Setting} \label{ch:vpake}
Using passwords for authentication is it's most common use case.
As discussed in the introduction, this is still done with the inherently flawed approach of \ac{HTML}-over-\ac{TLS} despite the existence of secure \ac{PAKE} protocols.
In this section we propose new mechanisms to register user accounts with secure password policy checks and give a new verifier-based \ac{PAKE} protocol to authenticate with these accounts.

This chapter proposes a new framework for verifier-based password authenticated key exchange in the two-party setting spanning including password registration and the actual authentication.
This chapter is based on work in \cite{Kiefer13a,Kiefer2012,KieferM14b,KieferM15a,DongK15a,ManulisSKD15a}.

%********************************** %Registering Passwords  **************************************
\section{Registering Passwords} \label{sec:vpake-registration}

\subsection{Zero-knowledge Password Policy Checks}
\mynote{ZKPPC from ESORICS'14 \cite{KieferM14b}}

\subsection{Blind Password Registration}
\mynote{BPR from ???(PETS'15/ePrint'15)  \cite{KieferM15a}}

\subsection{More efficient Blind Password Registration}
\mynote{BPR using SPC ???(ePrint/ESORICS'15) \cite{DongK15a}}
\mynote{performance trade-off discussion (registration VS authentication)}

%********************************** %PAKE  **************************************
\section{Verifier-based Password Authenticated Key Exchange} \label{sec:vpake-pake}
\mynote{how to use the passwords registered before}

\subsection{Building VPAKE from PAKE}

\begin{figure}[htbp]
\centering
\scalebox{1.0}{\begin{tikzpicture}
\draw[] (-3.5,.5) rectangle (12.2,-2);

\node[party,align=center] (client) at (.5,0) {{$C~ (\pwd)$}};
\node[party,align=center,text width=10em] (server) at (9.5,0) {$S~ (C, \ver, r)$};

\node[state, align=left] at (0.5,-.5) [stateS, align=left]{$\ver\gets f(\pwd, r)$};

\node[dummyState] (clientR) at (2.75,-0.6){};
\node[dummyState] (serverR) at (7.5,-0.6){};
\draw[pil] (serverR) -- node[above, align=center] {$r$} (clientR);

\node[dummyState] (clientPAKE) at (2.75,-1.6){};
\node[dummyState] (serverPAKE) at (7.5,-1.6){};
\draw[pil,<->] (clientPAKE) -- node[above, align=center] {PAKE on \ver} (serverPAKE);

\end{tikzpicture}}
\caption{Generic VPAKE construction from PAKE}
\label{fig:genericVPAKE}
\end{figure}

\begin{figure}[htbp]
\centering
\scalebox{1.0}{\begin{tikzpicture}
\draw[] (-3.5,.5) rectangle (12.2,-2);

\node[party,align=center] (client) at (.5,0) {{$C~ (\pwdv)$}};
\node[party,align=center,text width=10em] (server) at (9.5,0) {$S~ (C, \ver, r)$};

\node[state, align=left] at (0.5,-0.5) [stateS, align=left]{$\forall \pwd_i\in\pwdv:~ \ver_i\gets f(\pwd_i, r)$};
\node[stateS, align=left] at (11.25,-0.5) {draw fresh randomness $r'$};

\node[state, align=left] at (0.5,-1.5) [stateS, align=left]{$\forall \pwd_i\in\pwdv:~ H_i\gets g(\ver_i, r')$};
\node[stateS, align=left] at (11.25,-1.5) {$H\gets g(\ver, r')$};

\node[dummyState] (clientR) at (2.75,-0.6){};
\node[dummyState] (serverR) at (7.5,-0.6){};
\draw[pil] (serverR) -- node[above, align=center] {$r, r'$} (clientR);

\node[dummyState] (clientPAKE) at (2.75,-1.6){};
\node[dummyState] (serverPAKE) at (7.5,-1.6){};
\draw[pil,<->] (clientPAKE) -- node[above, align=center] {OPAKE on $(\{\bm H, \bar{\ver}\}, (H, \ver))$} (serverPAKE);

\end{tikzpicture}}
\caption{Generic Oblivious VPAKE construction from suitable PAKE}
\label{fig:genericOVPAKE}
\end{figure}

\mynote{generic description of how build VPAKE from PAKE}
\mynote{VPAKE security model}

\subsection{VPAKE Protocols}
\mynote{Verifier-based PAKE from ESORICS'14  \cite{KieferM14b}}
\mynote{tSOKE \cite{ManulisSKD15a}}

\subsection{Oblivious VPAKE}
\mynote{OVPAKE for password trials \cite{Kiefer2012,Kiefer13a}}

%********************************** %Implementation/Demo  **************************************
\section{Implementation and Demo} \label{sec:vpake-demo}
\mynote{Online demo and description of implementation containing\\
* optional registration (BPR style)\\
* server config\\
* authentication (tSOKE style)\\
* optional OPAKE}
