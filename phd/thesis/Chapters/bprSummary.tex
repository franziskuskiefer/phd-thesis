\section{Blind Password Registration --- Discussion and Analysis} \label{sec:discussion}

Sections \ref{sec:zkppc}-\ref{sec:spc-bpr} propose three password registration protocols with different security guarantees and performance.
In this section we compare all three protocols considering security and performance.

\subsection{Security} \label{sec:bpr-security-analysis}
The \ac{ZKPPC}-based approach has no formal security model for the password registration protocol.
However, it is clear (as discussed before) that it does not fulfil the security model proposed for \ac{BPR} protocols in Section \ref{sec:bpr} but requires a weaker definition.
While the \ac{SPC}-based approach has a security model, it differs from the \ac{BPR} security model proposed in Section \ref{sec:bpr}.
It thus remains to compare security guarantees given by the \ac{SPC}-based approach with guarantees from the \ac{BPR} protocol in Section \ref{sec:bpr}.
We call these protocols BPR1 and BPR2 respectively in the following.

First note that BPR2 seems not provable in the game-based security model BPR1 is proven in.
This is due to the fact that the reduction to the one-more RSA problem does not work.
The idea would be to say that the client is only able to register a non-compliant password if he is able to break one-more RSA, \ie by creating an
element in $\hat{\ClientSet}$ that was not sent to the server in any $u_i$. 
The problem is that the challenger has to build $\hat{\ServerSet}$ and therefore breaks the ability to actually generate that forgery. 
Alternatively, using only the requested elements in $\hat{\ServerSet}$ allows the client to create a forgery, but lets the protocol fail because the client uses elements in $\hat{\ClientSet}$ that are not in $\hat{\ServerSet}$.

Further, BPR2 has with the one-more RSA assumption and the random oracle model much stronger assumptions on the model than BPR1.
This leads to the conclusion that BPR1 is the preferable protocol when strong security guarantees are required.

\subsection{Performance \& Implementation} \label{sec:performance}

While the \ac{BPR} protocol from Section \ref{sec:bpr} has stronger security guarantees than the \ac{SPC}-based \ac{BPR} protocol, performance of the \ac{SPC}-based \ac{BPR} protocol is significantly better as discussed in the following.
However, password registration is a task that is performed rarely, such that it is acceptable for it to take a noticeable time. 
This is in contrast to authentication, which is performed regularly and thus has to have execution time that is not noticeable by the user.

\subsubsection{Performance}
We first compare the two zero-knowledge-based \ac{BPR} protocols from Section \ref{sec:zkppc} and \ref{sec:bpr}.
In the implementation of both protocols one can adopt several tricks aiming to improve performance. 
First, we can pre-compute and reuse values $g^{\pi_i}$ on client and server side.
Further, computation of $b^i$ can be performed in a way that allows to re-use previously calculated values and the implementation of the proof can be optimised allowing the client to use $\pi$.
Considering this, we can estimate performance of the \ac{BPR} protocol by counting the number of exponentiations as follows.
Note that we do not count exponentiations with small exponents.
% Client:
% \begin{itemize}
%   \item $94$ exponentiations to compute all $g^{\pi_i}$
%   \item $3n$ exponentiations to compute commitments $C_i, C'_i$ and $C^\ast_i$
%   \item \PoM needs $2\sum_i |\omega_i|-n$
%   \item \PoE needs $7$
%   \item \PoS needs $2n+12$
% \end{itemize}
%
% Server:
% \begin{itemize}
%   \item $94$ exponentiations to compute all $g^{\pi_i}$
%   \item \PoM needs $2\sum_i |\omega_i|$
%   \item \PoE needs $n+8$
%   \item \PoS needs $4n+8$
% \end{itemize}
The client in the \ac{BPR} protocol has to perform $4n+2\sum_i |\omega_i| + 113$ exponentiations.
The server must perform $5n + 2\sum_i |\omega_i| + 16$ exponentiations if $g^{\pi_i}$ is pre-computed and re-used.
In contrast, the generic approach for \ac{ZKPPC} requires $3n+3\sum_i (|\omega^\ast_i|-1)+7$ exponentiations on the client side and $3\sum_i |\omega^\ast_i|+8$ on the server side.
Note that $\omega^\ast_i$ in the \ac{ZKPPC} context depends on the maximum password length and thus contains all characters from $\omega_i$ \emph{plus} all characters from $\omega_i$ shifted by $j = 1,\ldots, \pmax$ positions.
Therefore, the cost of a \ac{ZKPPC}-based password registration protocol is given by $3n+2\pmax\sum_i (|\omega_i|-1)+7$ exponentiations on the client side and $2\pmax\sum_i |\omega_i|+8$ on the server side. 
A \ac{ZKPPC}-based password registration protocol is thus much less efficient than the \ac{BPR} protocol from Section \ref{sec:bpr}: in the optimal case where $n=\pmax$ the difference can be estimated by  $2(n-1)\sum_i|\omega_i| - 2n^2 - n - 106$ additional exponentiations for the client and $2(n-1)\sum_i|\omega_i| - 5n - 8$ additional exponentiations for the server.

The \ac{SPC}-based \ac{BPR} protocol from Section \ref{sec:spc-bpr} is based on mainly symmetric primitives, which yields a significant lower complexity.
In particular, the server in the \ac{SPC}-based \ac{BPR} protocol has to perform $\pmin+|\ServerSet|$ exponentiations where the size of the server set $|\ServerSet|$ is $\sum_i x_i|\omega_i|$ for $x_i$ being the number of characters required from subset $\omega_i$.
The client performs $n$ exponentiations.
We refrain from counting symmetric operations as they do not have significant impact on the performance.

% \paragraph{Implementation}
% We implement an unoptimised prototype of the \ac{BPR} protocol over the \ac{NIST} P-192 elliptic curve \cite{nistEC} in Python using the Charm framework \cite{charm13} and measure its performance.
% To this end we set $b=10^5$ in order to achieve security guarantees for all reasonable password lengths and policies.
% We also implement the \ac{ZKPPC} approach in order to compare its performance with our \ac{BPR} implementation.
% The performance tests (completed on a laptop with an Intel Core Duo P8600 at 2.40GHz) underline the theoretical findings from the previous paragraph. 
% In particular, execution of the direct \ac{BPR} protocol with a password of length $10$ and policy $(dl, 5)$ needs $0.72$ seconds on the client and $0.67$ seconds on the server side while the \ac{ZKPPC} execution requires $9.1$ seconds on the client and $8.9$ seconds on the server side with a maximum password length of $10$.
% Increasing the maximum password length to $20$ slows down the client to $22.7$ and the server to $22.2$ seconds.
% Our measurements show that our \ac{BPR} protocol is \emph{at least} $10$ times faster than the \ac{ZKPPC}-based registration approach.
% With the overall running time of $1.5$ seconds for 10-character passwords, $2.5$ seconds for 15-character passwords, and $3.3$ seconds for 20-character passwords the proposed \ac{BPR} protocol can be deemed practical.

% \subsubsection{Implementation and Evaluation (BPR)}\label{sec:evaluation}
\subsubsection{Implementation}
We implemented a prototypes of all three password registration protocols and measured their real world performance. 
The implementations are in C and use OpenSSL 1.0.2.c-1\footnote{\url{https://www.openssl.org}} for underlying cryptographic operations. 
In the experiments, we set the security parameter to 80-bit and use 1024-bit RSA keys and \mbox{{SHA-}1}. 
The two zero-knowledge-based protocols use the \ac{NIST} P-192 elliptic curve \cite{nistEC}.
All experiments were run on a laptop with an Intel Core Duo P8600 at 2.40GHz with 8GB RAM.

\begin{table}[!t]
%\setlength\tabcolsep{5pt}
\begin{center}
\scalebox{0.77}{
\begin{tabular}{lrrrrrrrrrr}
\toprule
\multirow{2}{*}{}&\multicolumn{2}{ c }{$(P1,10)$}&\multicolumn{2}{ c }{$(P2,10)$}&\multicolumn{2}{ c }{$(P4,20)$}&\multicolumn{2}{ c }{$(P2,20)$}&\multicolumn{2}{ c }{$(P2,40)$}\\
\cmidrule{2-11}
&Total&Pol-ck&Total&Pol-ck&Total&Pol-ck&Total&Pol-ck&Total&Pol-ck\\
\midrule
ZKPPC (Section \ref{sec:zkppc}) & 14,461 & 14,453 & 8,264 & 8,257 & 41,594 & 41,578 & 72,993 & 72,976 & 499,070 & 499,028 \\
\midrule
BPR (Section \ref{sec:bpr}) & 1,216 & 1,204 & 682 & 672 & 1,365 & 1,346 & 2,314 & 2,296 & 5,569 & 5,533 \\
\midrule
SPC (Section \ref{sec:spc-bpr}) & 132 & 5 & 247 & 11 & 489 & 23 & 266 & 11 & 300 & 11\\
% \hline
% Improvement (ZKPPC - SPC)&580$\times$&&275$\times$&&80$\times$&&35$\times$&&1649$\times$&\\
\bottomrule
\end{tabular}}
\caption{Protocol Performance (Running Time in Milliseconds)} \label{tab::perf}
\end{center}
\end{table} 

The running time of the protocols are shown in Table \ref{tab::perf}. We measured the running time with different policies and password lengths. The passwords are printable \ac{ASCII} strings. 
The alphabet is partitioned into the four usual classes: digits, lower case, upper case and symbols. 
We used three policies $P1, P2$ and $P4$ in the experiments, which require at least one, two and four characters in all character classes respectively. 
In the first row of the table, the pairs indicate the policy and the password length that were used in the experiment, \eg $(P1, 10)$ means policy $P1$ is used and the password was $10$ characters long. 

The table shows the total running time as well as the time spent on checking the policies (Pol-ck) in each protocol. 
As expected, the \ac{ZKPPC} approach is significantly slower than the two \ac{BPR} protocols.
The main differences in all protocols comes from the policy checking time.
Differences between the general \ac{ZKPPC} approach and \ac{BPR} come from the set membership proofs and therefore differ depending on the password length and its relation to the password policy. 
The measurements further show that the actual time needed for policy checking in the \ac{SPC}-based \ac{BPR} protocol is indeed independent of the password length (it depends on the password policy as everything else is negligible).

% The cost of the zero-knowledge proof protocol is $6\cdot n\cdot\sum_{i=1}^n\omega_i$ exponentiations, where $n$ is the password length in the experiments, and $\omega_i$ is the size of character class to which the i$th$ character in the password belongs. 
% In our protocol,  policy checking is done by using the \ac{SPC} protocol and the cost is mainly the \ac{OBI} protocol which is based on symmetric cryptography. The cost of the \ac{OBI} protocol is $4.32\cdot |\hat{S}|\cdot \lambda$ hash operations, where $\lambda$ is the security parameter.
% More concretely, in setting $(P1,20)$,  the zero-knowledge proof based policy checking requires around 200,000 exponentiations while our \ac{OBI} based \ac{SPC} requires only less than 33,000 hash operations.

% \begin{figure}[!bp]
% \centering
% \includegraphics[width=0.6\textwidth]{Figs/break}
% \caption{Time Breakdown}\label{fig::break}%\vspace{-5mm}
% \end{figure}

\paragraph{Discussion SPC-based BPR}
% We also show the running time for each step in our protocol (see Fig. \ref{fig::break}). 
The time for computing $u_i$ and $u_i'$ is linear in the password length, and the time for computing $\hat{\cS}$ and executing \ac{SPC} is linear in the size of $\hat{\cS}$. 
The most costly step is in the setup phase when the server computes the encrypted version of the alphabet $\hat{S}$.
A possible optimisation is to take this step offline. 
Since the computation of $\hat{S}$ does not depend on the client's password, the server can generate a random RSA key pair and pre-compute $\hat{S}$ before engaging with the client. 
The keys and pre-computed values can be stored together. 
Later when a client sends a registration request, the server can retrieve them and run the protocol. 
If this step is taken offline, then the online computation cost is very small.
