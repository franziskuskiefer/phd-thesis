\section[SPC-based Password Registration]{More efficient Blind Password Registration from Set-based Policy Checking}\label{sec:spc-bpr}

The \ac{BPR} protocol proposed in the previous section is secure and has reasonable performance.
However, it uses an approach closely related to the \ac{ZKPPC}-based approach from Section \ref{sec:zkppc} and thus shares many of its shortcomings.
In this section we propose an entirely different approach to \acl{BPR} based on set theory rather than zero-knowledge proofs, which allows us to build a significantly faster protocol but also requires a different security model.
Note that the \ac{BPR} protocol in this section is based on a more general approach that can be used in other scenarios than password policy checks.
To this end we first introduce the notion of \acl{SPC}.
In order to make the following technical sections easier to understand we give an examples in Section \ref{spc:example} on how the \ac{SPC} protocol works.
The idea behind this approach is to use \ac{PSI} with data transfer, which allows the client to retrieve data (shares of a secret) if and only if the intersection between the client's set and the server's set is of a certain form that is described in the server's policy.
If the client is then able to prove to the server that he was able to reconstruct the secret, the server believes that the client's set fulfils the server's policy.

\subsection{Secure Set-based Policy Checking (SPC)} \label{sec:spc}
In \ac{SPC} a server holds a public policy $f$ as defined in Section \ref{sec:spc-passwords} and a client holds a private set \ClientSet. 
The goal is to allow the server to check whether \ClientSet satisfies $f$ without learning anything else about $\ClientSet$. 

\begin{definition}[Secure Set-based Policy Checking, SPC]\label{def:SPC}
Set-based policy checking is executed between client \Client with a private set \ClientSet and server \Server with a public policy $f=(\ServerSet,\Gamma_\ServerSet)$.
Server and client retrieve $f(\ClientSet)$ as result. 
We call a set-based policy checking protocol secure iff it fulfils the following three notions.
\begin{enumerate}
  \item Correctness: Honest execution of the protocol with $f(\ClientSet)=\true$ is accepted by the server with overwhelming probability.
  \item Client Privacy: Server \Server learns nothing about the client set \ClientSet other than $f(\ClientSet)$.
  \item Soundness: A client \Client holding \ClientSet with $f(\ClientSet)\not=\true$ has negligible probability in getting \Server to accept the \ac{SPC} execution. \eod
\end{enumerate}
\end{definition}

\noindent
Definition \ref{def:SPC} says in particular that an \ac{SPC} protocol provides both participants with the result of $f(\ClientSet)$ while the server learns nothing about \ClientSet more than it can can infer from the result and public information.
%Depending on the application, the server's set \ServerSet can be private or public. 
%(If \ServerSet is secret, the client learns nothing. If \ServerSet is public, there are no new information to learn from the execution.)
%Definition \ref{def:SPC} says in particular that an \ac{SPC} protocol provides both participants with the result of $f(\ClientSet)$ while the client learns nothing new about \ServerSet and the server learns nothing about \ClientSet.
%Depending on the application, the server's set \ServerSet can be private or public. 
%(If \ServerSet is secret, the client learns nothing. If \ServerSet is public, there are no new information to learn from the execution.)
Formally, privacy is defined using the simulation-based approach.
Privacy of an \ac{SPC} protocol $\Pi$ in particular holds if its execution in the real model emulates execution of the ideal model with a trusted party for \ac{SPC} functionality $f_\Pi$, \ie
$ \{\IDEAL_{f_\Pi, \cB(z)}(x, y)\}_{x, y, z} \stackrel{c}{\equiv} \{\REAL_{\Pi, \cA(z)}(x, y)\}_{x, y, z}$.
Functionality $f_\Pi$ outputs $f(\ClientSet)$ to the server and client, \ie $f_{\Pi}(\ClientSet, f)=(f(\ClientSet), f(\ClientSet))$.

% \setlength{\textfloatsep}{21pt}
% \setlength{\belowcaptionskip}{2pt}

\subsubsection{Instantiation}
% \begin{figure}[!b]
% \centering
% \scalebox{0.75}{
% \begin{tikzpicture}[framed] %[scale=0.7, every node/.style={scale=0.7}, ]
% \matrix (m)[matrix of nodes, column  sep=.1cm,row  sep=1mm, ampersand replacement=\&,
% 		nodes={draw=none, anchor=center,text depth=1pt},
% 		column 1/.style={nodes={minimum width=18em}}, %,text width=15em
% 		column 2/.style={nodes={minimum width=4em}},
% 		column 3/.style={nodes={minimum width=20em, align=right, text width=20em}}]{
% 	\node[align=center](client){$\Client~ (\ClientSet)$}; \draw[]($(client.south west)+(.5,0)$)--($(client.south east)-(.5,0)$); \& \& \node[align=center](server){$\Server$ $(f$ $=$ $(\ServerSet,\Gamma_\ServerSet))$}; \draw[]($(server.south west)+(.5,0)$)--($(server.south east)-(.5,0)$);\\ [1mm]
% 	
% 	\& \& choose a secret \secret $\in_R$ $\mathbb{Z}_p$ \\
% 	\& \& generate matrix $M$ and function $\rho$ from $f$\\
% 	\parbox{18em}{}\&  $f, M, \rho$\& generate shares $\secret_{i}$ s.t. $\rho(i)$ $\in$ $\ServerSet$ \\
% 	\parbox{18em}{get shares from $G_{\ClientSet\cap\ServerSet}$: $\{\secret_i|\rho(i)\in \ServerSet\cap \ClientSet\}$} \&
% 	  \ac{OBI} \& \\
% 	\parbox{18em}{if possible: recover $\secret$ from the shares} \& \& \\
% 	\parbox{18em}{else: abort and output $\bot$} \& \& \\
% 	\parbox{18em}{$m_\secret\gets H(\secret,\Client,\Server,\trans)$} \& 
% 	$m_\secret$ \&
% 	accept iff $m_\secret$ \verify $H(\secret,\Client,\Server,\trans)$ \\
% };
%
% \draw[<-] (m-4-2.south west) -- (m-4-2.south east);
% \draw[<->] (m-5-2.south west) -- (m-5-2.south east);
% \draw[->] (m-8-2.south west) -- (m-8-2.south east);
% \end{tikzpicture}}
% \caption{Secure Set-based Policy Checking}%\vspace{-5mm}
% \label{fig:spc}
% \end{figure}

\begin{figure*}[t]
\begin{center}
\scalebox{0.95}{\small
\begin{tabular}{ l c l }
\toprule
{\bf Client \Client} & & {\bf Server \Server} \\
Input: \ClientSet & & Input: $f=(\ServerSet,\Gamma_\ServerSet)$ \\
\midrule
 & & choose a secret \secret $\in_R$ $\mathbb{Z}_p$ \\
 & & generate matrix $M$ \\
 & & and function $\rho$ from $f$ \\
 & $\xleftarrow{\makebox[2cm]{$f, M, \rho$}}$ & generate shares $\secret_{i}$ s.t. $\rho(i)$ $\in$ $\ServerSet$ \\
get shares from $G_{\ClientSet\cap\ServerSet}$: $\{\secret_i|\rho(i)\in \ServerSet\cap \ClientSet\}$ & $\xleftrightarrow{\makebox[2cm]{\ac{OBI}}}$ & \\
if possible: recover $\secret$ from the shares & & \\
else: abort and output $\bot$ & & \\
$m_\secret\gets H(\secret,\Client,\Server,\trans)$ & $\xrightarrow{\makebox[2cm]{$m_\secret$}}$ & accept if $m_\secret$ \verify $H(\secret,\Client,\Server,\trans)$\\
\bottomrule
\end{tabular}
}
\end{center}
\caption{Secure Set-based Policy Checking}
\label{fig:spc}
\end{figure*}

\noindent
The high-level idea of this protocol is to use a \ac{PSI} protocol with data transfer and \ac{LSSS} so that the server can check the client's set of password characters using efficient symmetric operations.
In particular, the server sends a set of secret shares, built according to the server's policy, to the client such that the latter can rebuild the secret only if the client's set is qualified. 
Although in general we can use any \ac{PSI} with data transfer protocol (\eg \citet{fre04}), \ac{OBI} is chosen here because of its efficiency. 
\ac{OBI} is very efficient due to the fact that it relies mostly on symmetric operations. 
While the performance of \ac{OBI} can be further improved by the modifications proposed by \citet{Pinkas0Z14}, based on hash table plus oblivious transfer that is more efficient than \ac{OBI}, the new \ac{PSI} protocol cannot be used in our case because it does not support data transfer.

An overview of the \ac{SPC} protocol is depicted in Fig.~\ref{fig:spc}. 
Let $f=(\ServerSet,\Gamma_\ServerSet)$ be the server's policy defined over its set $\ServerSet$ and $\ClientSet$ be the client's set. 
The two parties want to check $f(\ClientSet)$, \ie whether $\ClientSet$ satisfies $f$. 
Both parties get a collision resistant hash function $H$, security parameter \secpar, and prime $p$ of length \secpar as common input. 
The server can map policy $f$ into a set of secret shares $\mathfrak{S}$ by first choosing a random secret \secret from $\ZZ_p$.
Then \Server generates \ac{LSSS} matrix $M$ and the mapping function $\rho$ from policy $f$, and computes shares for \secret. 
Each share $\secret_i \in \mathfrak{S}$ is associated with an element $\rho(i)\in \ServerSet$. 
Depending on the policy, it is possible that an element in $s_j\in\ServerSet$ is associated with multiple shares, \ie there exist multiple rows $i$ in $M$ such that $\rho(i)=s_j$. 
In this case, the server concatenates all $\secret_i$ associated with $s_j$ so that $\mathfrak{S}$ still has cardinality $|\ServerSet|$. 
The two parties then run the \ac{OBI} protocol with the server using $(\ServerSet, \mathfrak{S})$ and the client using $\ClientSet$ as their inputs. 
In the \ac{OBI} protocol, the server's garbled Bloom filter is built using elements in $\ServerSet$ as keys and shares in $\mathfrak{S}$ as data items.
The data items to be transferred in \ac{OBI} must be at least $\lambda$-bits and of equal length. 
We assume a proper scheme (\eg zero padding) to pad elements in $\mathfrak{S}$ to the correct length.
At the end of the \ac{OBI} protocol the client gets a set of shares $\{\secret_i|\rho(i)\in \ServerSet\cap \ClientSet\}$. 
If $\ClientSet$ satisfies policy $f$, then the shares obtained from the \ac{OBI} protocol will allow the client to reconstruct secret \secret, otherwise the client learns nothing about \secret and aborts. 
In the last step, by proving knowledge of secret \secret, the client can convince the server that its set is compliant with policy~$f$.

\subsubsection{Example}\label{spc:example}
To facilitate understanding of \ac{SPC} we give a short example of the protocol in Figure \ref{fig:spc}.
Let $f=(\ServerSet, \Gamma_\ServerSet)=(\ServerSet_1\cup\ServerSet_2, (t_1,t_2))=(\{a,b\}\cup\{A,B\}, (1,1))$ denote the server's policy.
This translate to; an authorised set must contain at least one element from $\{a,b\}$ and one element from $\{A,B\}$, \ie $|\ClientSet\cap\ServerSet_1|\geq t_1$ and $|\ClientSet\cap\ServerSet_2|\geq t_1$.
The server chooses a random element $\secret\rin\ZZ_p$.
Now the server generates an appropriate matrix $M$ and function $\rho$ for $f$ as follows.
First we can rewrite $f$ as boolean formula $(a\vee b)\wedge(A\vee B)$ or $((a,b,1),(A,B,1),2)$, which is the format used as input to the algorithm by \citet{LiuC10}.
The algorithm is used to transform the policy into matrix $M$:
\[M=
\begin{pmatrix}
  1 & 1 \\
  1 & 2 \\
\end{pmatrix}
~~ L=
\begin{pmatrix}
  (a,b,1) \\
  (A,B,1) \\
\end{pmatrix}
% \]
% \[
\qquad\qquad M=
\begin{pmatrix}
  1 & 1 \\
  1 & 1 \\
  1 & 2 \\
\end{pmatrix}
~~ L=
\begin{pmatrix}
  a \\
  b \\
  (A,B,1) \\
\end{pmatrix}
\]
\[M=
\begin{pmatrix}
  1 & 1 \\
  1 & 1 \\
  1 & 2 \\
  1 & 2 \\
\end{pmatrix}
~~ L=
\begin{pmatrix}
  a \\
  b \\
  A \\
  B \\
\end{pmatrix}
\]
Function $\rho(i)$ is given by $\rho(1)=a,\rho(2)=b,\rho(3)=A,\rho(4)=B$.
Matrix $M$ and function $\rho$ is sent to the client.
For the \ac{OBI} protocol the server further prepares $\ServerSet_d=(\share_1,\share_2,\share_3,\share_4)$ with secret shares $\share_1=\share_2\rin\ZZ_p$ and $\share_3=\share_4=2\share_1-\share$ (according to $M$).
Execution of the \ac{OBI} protocol provides the client with shares $\share_i$ for all $\{\secret_i|\rho(i)\in \ServerSet\cap \ClientSet\}$, \ie all elements that are in the client's set \ClientSet as well as in the server's set \ServerSet.
If the client retrieves a qualified set of shares, \ie $\share_1$ and ($\share_3$ or $\share_4$), or $\share_2$ and ($\share_3$ or $\share_4$), he can reconstruct \secret, \eg $\secret=2\share_1-\share_3$ and thus prove to the server that his set is policy compliant.

\subsubsection{Security Analysis}
\begin{lemma}[Correctness]\label{lem:correct}
  Let $\ClientSet$ and $\ServerSet$ denote sets from some universe and $f=(\ServerSet,\Gamma_\ServerSet)$.
  Assuming the used \ac{OBI} and \ac{LSSS} algorithms are correct, then the \ac{SPC} protocol from Fig. \ref{fig:spc} is correct, \ie honest execution of the protocol with $f(\ClientSet) = \true$ is accepted by the server with overwhelming probability.\end{lemma}
\begin{proof}
%First we argue that the proposed protocol is correct, \ie the server and the client are always able to determine $f(\ClientSet)$ if the protocol is executed honestly.
Assuming correct \ac{LSSS} and \ac{OBI}, correctness can be proved by the following argument.
LSSS guarantees correct sharing of secret \secret, \ie for any authorised set $\ServerSet_i\in \Gamma_\ServerSet$ there is a set of shares associated with the elements in $\ServerSet_i$ that can reconstruct \secret and for any other set the secret cannot be reconstructed. 
OBI guarantees that for each element in the intersection of $\ServerSet\cap \ClientSet$, the client is able to obtain the share associated with it. 
All elements in $\ClientSet$ that are not in the intersection are irrelevant as by definition they are not in $\ServerSet$ thus not in any authorised set. 
Then the client can reconstruct \secret iff there exists a subset of $\ServerSet\cap \ClientSet$ that is an authorised set. 
If there is such an authorised set then $f(\ClientSet)=\true$. Eventually, the hash value $m_\secret$ is the same on the client and server if the same key \secret, transcript \trans, and participants identifiers \Client and \Server are used.
\end{proof}

\begin{lemma}[Privacy]\label{lem:SPCprivacy}
 Let $\ClientSet$ and $\ServerSet$ denote sets from some universe, $f=(\ServerSet,\Gamma_\ServerSet)$ a policy and $f_\SPC(\ClientSet, f)=(f(\ClientSet), f(\ClientSet))$. 
 If the \ac{OBI} protocol is secure and the \ac{LSSS} is correct, the \ac{SPC} protocol from Figure \ref{fig:spc} securely realises $f_\SPC$ in the presence of a malicious server or client.
\end{lemma}

\begin{proof}
For the security proof we consider an ideal \ac{OBI} function $f_\OBI$, \ie show computational indistinguishability between the world $\HYBRID^{f_\OBI}_{\Pi, \cA(z)} (x, y)$ and ideal world $\IDEAL_{f_\SPC, \cB(z)}(x, y)$ to prove security. 
First, since we are in the $\HYBRID^{f_\OBI}$ world, every \ac{OBI} operation and message is forwarded to the \ac{OBI} functionality $f_\OBI$.
Since we only require \ac{LSSS} correctness here, we do not use the ideal functionality.
In the following we show that the hybrid world $\HYBRID^{f_\OBI}_{\Pi, \cA(z)}$ is computationally indistinguishable from $\IDEAL_{f_\SPC, \cB(z)}$ and hence the real world $\REAL_{\Pi, \cA(z)}$. 
We give a simulator $\cB_\Server(z)$ that simulates a malicious server in the ideal world. 
Note that we omit auxiliary input $z$ if not needed. 
We build $\cB_\Server$ that on input of the server's policy $f$, and access to the real world adversary $\cA_\Server$ that plays server \Server, generates $\view_\Server$ that is indistinguishable from $\view_{\cA\Server}$ of $\cA_\Server$.

$\cB_\Server$ starts by invoking server $\cA_\Server$ with $f$ and $z$ to receive share-generating matrix $M$ and $\rho$ and policy $f'$ from adversary $\cA_\Server$ in the first protocol message. 
Then as input to the \ac{OBI} functionality $\cB_\Server$ receives the server's \ac{OBI} input  from $\cA_\Server$, \ie a set $\ServerSet'$ and shares $\ServerSet_d=\{\secret_i\}$. 
$\cB_\Server$ then sends $\ServerSet'$ to the trusted party. 
If the trusted party replies with $\bot$, the simulation terminates and $\cB_\Server$ outputs whatever $\cA_\Server$ outputs. 
If the trusted party replies with $\true$, 
simulator $\cB_\Server$ generates a set \ClientSet with $f'(\ClientSet)=\true$ based on $f'$ and $\ServerSet'$. 
If the trusted party replies with $\texttt{false}$, simulator $\cB_\Server$ generates a set \ClientSet with $f(\ClientSet)=\texttt{false}$ based on $f'$ and $\ServerSet'$. 
Building \ClientSet from $\ServerSet'$ and $f'$ is straightforward. 
$\cB_\Server$ sends client and server input to the \ac{OBI} functionality to retrieve the server's and client's view on the \ac{OBI} execution. 
Further, $\cB_\Server$ tries to recover \secret from the output of $f_\OBI$ using the combination algorithm of \ac{LSSS}. 
If the client can reconstruct \secret, $\cB_\Server$ generates $m_\secret$ using $M, \rho, f'$ and $\ServerSet'$, the (simulated) \ac{OBI} transcript, \secret, \Client, and \Server and sends it to $\cA_\Server$. 
Otherwise, if $\cB_\Server$ cannot recover \secret from the $f_\OBI$ output, it terminates the session with $\bot$.
Eventually $\cB_\Server$ outputs the transcript and whatever $\cA_\Server$ returns on terminating as $\view_\Server$.

We claim that the output of an honest client in the ideal execution is indistinguishable from the client's output in the real world.
This is easy to see as the client always receives $f$ and the possibility to compute $f(\ClientSet)$ using either the public server set \ServerSet, or the output of the \ac{OBI}/$f_\OBI$ execution.
Note that the evaluation of $f$ depends on the set used by $\cA_\Server$ in the \ac{OBI} execution ($\ServerSet'$).
Since this strategy is the same in the real and the ideal world, the claim follows.
Indistinguishability of $\view_\Server$ and $\view_{\cA\Server}$ follows from the following observations.
If the server's input to \ac{OBI} is not correct, the protocol terminates.
It is easy to see that the transcript containing $f, M, \rho$, and \ServerSet, and the \ac{OBI} execution in $\view_\Server$ and $\view_{\cA\Server}$ is identical.
Since the input to the hash function $H$ is equivalent in both worlds, $\view_\Server$ and $\view_{\cA\Server}$ are indistinguishable.

\end{proof}

\noindent
Lemma \ref{lem:SPCprivacy} proves that the \ac{SPC} protocol ensures client privacy, \ie \ac{SPC} does not leak any information about the client's set.
We now give a lemma to show soundness of our \ac{SPC} protocol that concludes the security analysis of the proposed \ac{SPC} protocol.

\begin{lemma}[Soundness]\label{lem:SPCsecurity}
  Let $\ClientSet$ and $\ServerSet$ denote sets from some universe and $f=(\ServerSet,\Gamma_\ServerSet)$ a policy.
  Assuming the used \ac{OBI} and \ac{LSSS} algorithms are secure and $H$ is collision resistant, then the \ac{SPC} protocol from Fig. \ref{fig:spc} is sound in the presence of a malicious client, \ie the server accepts the protocol with negligible probability if $f(\ClientSet)\not=\true$.
\end{lemma}

\begin{proof}
Lemma \ref{lem:SPCsecurity} can be proven by showing that convincing \Server to accept the protocol despite the fact that $f(\ClientSet)\not=\true$ implies a collision in $H$.
First, it is clear that the attacker is not able to recover the correct secret \secret from the \ac{OBI} interaction with \Server since this would break either \ac{OBI} or \ac{LSSS} security.
Now it follows directly that any attacker that is able to generate a message $m_\secret$ such that $m_\secret=H(\secret,\Client,\Server,\trans)$ found a collision in $H$ as it does not know the correct secret \secret.
\end{proof}


\subsubsection{Applications}\label{sec:generalisation}
While \ac{SPC} can be used in many different scenarios, it will be detailed for use in password-policy checks in order to build \ac{BPR} protocols in Section \ref{sec:spc-bpr-protocol}.
Before that we want to give some ideas of other \ac{SPC} use-cases.
Note that these applications might have additional requirements on the used sets (such as authorised sets or secret server sets) we do not detail here.
These are possible applications of \ac{SPC} and would require further rigorous analysis before usage.

\paragraph{Policy checks for Access Control}
In a role-based access control scenario \cite{FerraioloK09} a user has to have a certain role (or combination of roles) in order to access a resource.
\ac{SPC} can be used in this case to verify whether a client has necessary roles that allow it to access the resource.
The server in this case defines \ServerSet to contain identifiers for all allowed roles $\ServerSet_i$ and $\Gamma_\ServerSet$ according to the access policy.
The user's set \ClientSet contains the client's role identifiers $c_i$.
Access should be granted if and only if the \ac{SPC} protocol is successful, \ie the user can convince the server that he has all necessary roles.

\paragraph{Policies for Friendship Analysis}
One popular application of set based protocols is friendship analysis.
This test should determine whether two parties become friends or not depending on the number of mutual friends.
\ac{SPC} can be used in this scenario as a very efficient alternative while increasing privacy.
Using \ac{SPC} further allows to build subsets in friend sets, such as colleagues, family etc., which in turn makes the friendship-test more ``accurate'' while leaking as little information about the friendship relations as possible.

\paragraph{Genome Testing}
\citet{BaldiBCGT11} propose protocols to perform privacy preserving genome testing, such as paternity tests. 
The tests can often be reduced to check a set of Single Nucleotide Polymorphism that are present in a patient's genome against some predefined sets. 
Although it is not exactly policy checking, \ac{SPC} can be used in this setting too. 


\subsection{SPC-based Blind Password Registration}\label{sec:spc-bpr-protocol}

The connection between \acl{SPC} and password policies is easy to see, since password policies are defined in terms of thresholds and subsets over an alphabet, which is a set of characters.
We can use the algorithm proposed by \citet{LiuC10} to translate such threshold access structures efficiently into \acp{MSP}. 
It is also not difficult to see how \ac{SPC} can be applied in the password policy checking setting, since a password can be seen as a set of characters as described in Section \ref{sec:spc-passwords}. 

\subsubsection{The Password Registration Protocol}
An overview of the proposed password registration protocol is given in Fig. \ref{fig:bpr-spc}. 
To simplify the presentation, we assume the protocol is run over a secure channel, \eg implemented as a server authenticated \ac{TLS} channel. 
The secure channel will address common network-based attacks such as replay, eavesdropping and man-in-the-middle. 
The server publishes its password policy $f=(\ServerSet,\Gamma_\ServerSet)$ where \ServerSet is a set transformed from alphabet $\Sigma$ according to Section \ref{sec:spc-passwords} and $\Gamma_\ServerSet$ is a threshold access structure defined over \ServerSet. 
Other public parameters include security parameter \secpar, a pseudorandom function family $f_k$, and three hash functions $H_1, H_2$, and $H_3$. 
The client has a private password \pwd which will be used in the protocol as its input, \ie we assume the client received policy $f$ already and entered an appropriate password $\pwd\in\cD_f$. 
The protocol has two phases, a setup phase and a policy checking phase. 
In the setup phase the client commits to its password, and each party blinds its set. 
The blinded sets are then used in the policy checking phase. In the policy checking phase, the server checks the password policy with a secure \ac{SPC} protocol using the blinded sets.

The protocol proceeds as follows. 
In the setup phase, the server runs the RSA key generation algorithm on input of security parameter $\lambda$ to generate an RSA key pair $\pk=(e,N), \sk=(d,N)$ using $\KGen(\secpar)$. 
The public key is sent to the client while the secret key is kept private. 
Upon receiving the server's public key $\pk=(e,N)$, the client's first step is to compute a key $k=H_1(\pwd)$ for the pseudorandom function by hashing the password using $H_1$. The client uses the password pre-processing function \PwdMap to generate $\ClientSet\gets\PwdMap(\pwd)$ and to compute $r_i=f_k(i)$ and $u_i=H_2(c_i)\cdot r_i^e$ for each $c_i\in\ClientSet$.
The result $(u_1,\ldots,u_v)$ is sent to the server.
For each $i\in[1,v]$ the server computes $u_i'=u_i^d$ and returns $(u_1',\ldots,u_v')$ back to the client. 
Eventually, upon receiving $(u_1',\ldots,u_v')$, the client creates an empty set $\hat{\ClientSet}$ and for $i\in[1,v]$ puts $u_i'\cdot r_i^{-1}=(H_2(c_i))^d$ into $\hat{\ClientSet}$. 
The server also creates an empty set $\hat{\ServerSet}$ and for $i\in[1,w]$ puts $(H_2(s_i))^d$ into $\hat{\ServerSet}$. The set $\hat{\ServerSet}$ is partitioned into $m$ subsets according to the character classes. 
The server also generates $\hat{f}$ from $f$ by replacing $\ServerSet$ with $\hat{\ServerSet}$.
The policy checking phase is essentially an execution of  the \ac{SPC} protocol using $\hat{\ClientSet}$ and $\hat{f}$ as inputs. 
At the end of the \ac{SPC} protocol the server learns whether the client's password satisfies the policy or not. 
If the \ac{SPC} execution is successful, the server computes the hash of the product of the client's $u_i$ values $h\gets H_3(\prod_{i=1}^v u_i)$, and stores the password verifier $\ver = (h,e,N,d,\bm{u})$, where $(e,N,d)$ is the RSA key pair from the setup phase and $\bm{u}=\{u_1,\dots,u_v\}$ is the vector of client ``commitments''.
Note that storing $(d, \bm{u})$ as part of \ver is only a technical necessity in order to prove security.
Further note that it is important for $e$ in \KGen is chosen uniformly at random here.
(Usually $e$ is set $2^{16}+1$ in real world implementation due to efficiency reasons.
This however would break the protocol as $u_i$ could be pre-computed in this case and thus enable more efficient attacks than brute-force dictionary attacks.)

% \begin{figure}[!t]
% \centering
% \scalebox{0.7}{
% \begin{tikzpicture}[framed] %[scale=0.7, every node/.style={scale=0.7}, ]
% \matrix (m)[matrix of nodes, column  sep=.1cm,row  sep=1mm, ampersand replacement=\&,
% 		nodes={draw=none, anchor=center,text depth=1pt},
% 		column 1/.style={nodes={minimum width=18em}}, %,text width=15em
% 		column 2/.style={nodes={minimum width=10em}},
% 		column 3/.style={nodes={minimum width=17em, align=right, text width=17em}}]{
% 	\node[align=center](client){$C\ (f, \pwd)$}; \draw[]($(client.south west)+(.5,0)$)--($(client.south east)-(.5,0)$); \& {\it Setup Phase} \& \node[align=center](server){$S$ $(f,\ServerSet$ $=$ $\{s_1,\dots,s_w\})$}; \draw[]($(server.south west)+(.5,0)$)--($(server.south east)-(.5,0)$);\\ [1mm]
% 	
% 	\parbox{18em}{$k\gets H_1(\pwd)$} \& 
% 	  $(e,N)$ \& 
% 	  $(e,d,N)$ $\ralgout$ $\KGen(\lambda)$ \\  %Define policy $f$
% 	\parbox{18em}{$\ClientSet=\{c_1,c_2,\ldots,c_v\}\gets\PwdMap(\pwd)$} \&  \&  \\
% 	\parbox{18em}{$\forall i\in[1,v]:~ r_i\gets f_k(i),~ u_i\gets H_2(c_i)\cdot r_i^e$} \& 
% 	  $(u_1,\cdots,u_v)$ \& 
% 	  abort if $v$ $<$ \pmin \\
% 	\parbox{18em}{$\hat{\ClientSet} = \emptyset$} \& 
% 	  $(u'_1,\cdots,u'_v)$ \& 
% 	  $\forall i$ $\in$ $[1,v]$: $u'_i$ $=$ $u_i^d$;~ $\hat{\ServerSet}$ $=$ $\emptyset$ \\
% 	\parbox{18em}{$\forall i\in[1,v]:~ \hat{\ClientSet}\cup u'_i\cdot r_i^{-1}$} \& \&
% 	  $\forall i$ $\in$ $[1,w]$: $\hat{\ServerSet}$ $\cup$ $(H_2(s_i))^d$ \\
% \parbox{18em}{} \& \&
% 	  $\hat{f}$ $\gets$ $f,\hat{\ServerSet}$ \\
% 	\hfill \& \hfill \& \hfill \\
% 	\parbox{18em}{} \& {\it Policy Checking Phase} \& \\
% 	\parbox{18em}{} \& {$\hat{\ClientSet}\rightarrow$ Run \ac{SPC}$\leftarrow \hat{f}$} \& 
% 	  If \ac{SPC} execution is successful \\
% 	\parbox{18em}{} \& \hfill \& %$\forall c_i\in\pwd$ but not in \ClientSet:
% 	  store $\ver$ $\gets$ $(H_3(\prod_{i=1}^v u_i), e, N, d, \bm{u})$ \\
% };
%
% \draw[<-] (m-2-2.south west) -- (m-2-2.south east);
% \draw[->] (m-4-2.south west) -- (m-4-2.south east);
% \draw[<-] (m-5-2.south west) -- (m-5-2.south east);
% \draw[dashed] (m-8-1.south west) -- (m-8-3.south east);
% \draw[<->] (m-10-2.south west) -- (m-10-2.south east);
% \end{tikzpicture}}
% \caption{Password Registration using secure SPC}%\vspace{-5mm}
% \label{fig:bpr-spc}
% \end{figure}

\begin{figure*}[t]
\begin{center}
\begin{tabular}{ l c l }
\toprule
{\bf Client \Client} & & {\bf Server \Server} \\
Input: $f, \pwd$ & & Input: $f,\ServerSet=\{s_1,\dots,s_w\}$ \\
\midrule
 & {\it Setup Phase} & \\
% \cmidrule{2-2}
$k\gets H_1(\pwd)$ & $\xleftarrow{\makebox[2cm]{$(e,N)$}}$ & $(e,d,N)$ $\ralgout$ $\KGen(\lambda)$ \\
$\ClientSet\gets\PwdMap(\pwd)$ & & \\
$\forall i\in[1,v]:~ r_i\gets f_k(i)$ & & \\
$u_i\gets H_2(c_i)\cdot r_i^e$ & $\xrightarrow{\makebox[2cm]{$(u_1,\cdots,u_v)$}}$ & abort if $v$ $<$ \pmin \\
$\hat{\ClientSet} = \emptyset$ & $\xleftarrow{\makebox[2cm]{$(u'_1,\cdots,u'_v)$}}$ & $\forall i \in [1,v]: u'_i=u_i^d;~ \hat{\ServerSet}=\emptyset$ \\
$\forall i\in[1,v]:~ \hat{\ClientSet}\cup u'_i\cdot r_i^{-1}$ & & $\forall i \in [1,w]: \hat{\ServerSet} \cup (H_2(s_i))^d$ \\
 & & $\hat{f} \gets f,\hat{\ServerSet}$ \\
 & {\it Policy Checking Phase} & \\
% \cmidrule{2-2}
 & $\hat{\ClientSet}\rightarrow$ Run \ac{SPC}$\leftarrow \hat{f}$ & \\
 & & If \ac{SPC} is successful \\
 & & store verifier \\
 & & $(H_3(\prod_{i=1}^v u_i), e, N, d, \bm{u})$ \\
\bottomrule
\end{tabular}
\end{center}
\caption{Password Registration using secure SPC}
\label{fig:bpr-spc}
\end{figure*}

\paragraph{Password Length Hiding (Enhanced Protocol)} 
The protocol in Figure~\ref{fig:bpr-spc} leaks the password length to the server. By counting the number of blinded characters $u_i$, the server learns the password length $v$. 
This is intentional because this peripheral information leakage allows the server to efficiently enforce the minimal password length in the policy. 
However, in cases where the password length is considered sensitive, it can be hidden from the  server at small additional cost.
(This is is contrast to the \ac{BPR} protocol proposed in the previous section, where it is expensive to achieve this.)

The client generates a set $\ClientSet'\subseteq\ClientSet$ and uses it in the setup phase to generate $\hat{\ClientSet}$. 
$\ClientSet'$ contains only significant characters to fulfil $f$. 
That is, the client first takes characters from $\ClientSet$ according to character class $A_i$ and threshold $t_i$, and puts them in $\ClientSet'$.
If the size of $\ClientSet'$ is smaller than the minimal password length \pmin, the client pads it with other characters in $\ClientSet$ that are not in $\ClientSet'$ yet. 
In the setup phase, the client only uses characters in $\ClientSet'$ and obtains the corresponding $\hat{\ClientSet}$. In this process, the server learns the size of $\ClientSet'$ and can check whether this is equal to the minimal password length \pmin required by the policy. 
The client then uses this $\hat{\ClientSet}$ in the policy checking phase to convince the server about the password complexity. 
If the server accepts, all characters in $\ClientSet\setminus\ClientSet'$ that have not been sent to the server are put into an additional $u^\ast=r_{v+1}^e\cdot\prod u_i$ with $r_i\gets f_k(i),~ u_i\gets H_2(c_i)\cdot r_i^e$ for $c_i\in\ClientSet\setminus\ClientSet'$. 
This value $u^\ast$ is then sent to the server and is multiplied with the other $u_i$ values the server received in the setup phase.
This product is then used to generate verifier \ver, \ie $\ver\gets(H_3(r_{v+1}^e\prod_{i=1}^v u_i), e, N, d, \bm{u})$. 
Note that we require $r_{v+1}^e$ as a multiplicand when computing $u^\ast$. 
Without this, the server could learn the client's password length when $\ClientSet\setminus\ClientSet'=\emptyset$ because the client would have nothing to send in this case. 

\subsubsection{Security Analysis} \label{sec:bpr:security}
We now analyse the security of the \ac{BPR} protocol. 
Recall that in the password registration protocol, the two parties have different security requirements. For the server, privacy is not a concern since its input, the policy, is public. 
On the other hand, the server cares about the soundness of the protocol because an unsound protocol would allow a user to register a password that is not policy compliant. 
For the client, privacy is the main concern while soundness is not. Since the policy is public, the client can check the policy by itself and can detect if the server cheats. 
We therefore refrain from using an over-complicated security definition and use the following comprehensible security model.
Note that the \ac{BPR} model from Section \ref{sec:bpr} is not usable for this protocol (cf. Section \ref{sec:bpr-security-analysis} for more details).
Let $\ver\gets\phi(\pwd,r)$ denote a password verifier, computed from a password \pwd and some randomness $r$, and $\PwdMap(\pwd)$ a function to generate set \ClientSet from password \pwd.
\begin{enumerate}
  \item Privacy: A malicious server must not be able to retrieve more information from the protocol than the password verifier and the result of the policy verification, \ie the protocol must implement the functionality $f_{\Pi}=(f(\ClientSet), (f(\ClientSet), \ver))$. Furthermore, the verifier must not give a malicious server advantage in terms of dictionary attacks. 
  
  \item Soundness: The server accepts a password verifier $\ver\gets\phi(\pwd,r)$ if and only if (i) the password is compliant with the server's policy, \ie $f(\ClientSet)=\true$ for $\ClientSet\gets\PwdMap(\pwd)$, and (ii) the verifier is uniquely defined by the password and some server known randomness, \ie there exists no password $\pwd'\not=\pwd$ such that $\phi(\pwd',r)=\ver$ and it is not possible to find randomness $r'\not=r$ in polynomial time such that $\phi(\pwd,r')=\ver$.
\end{enumerate}

\noindent
In the following we show that the enhanced version of the previously defined protocol satisfies those properties.
Note that the simple version satisfies the same properties but in a weaker version, \ie we would have to replace dictionary $\cD_f$ in Lemma \ref{lem:privacy} with $\cD_{f,|\pwd|}$, where $\cD_{f,|\pwd|}$ denotes the dictionary that contains all passwords of size $|\pwd|$ that are policy compliant with respect to $f$.
Note that $H_2$ has to be modelled as random oracle here in order to use the one-more RSA assumption.
For the other two hash functions $H_1$ and $H_3$ it is sufficient to assume collision resistance.

\begin{lemma}[Privacy]\label{lem:privacy}
If $f_k$ is a secure pseudorandom function family, $H_1$ is collision resistant, $H_2$ a random oracle, and the one-more RSA assumption holds in the used group, the enhanced password registration protocol offers privacy with respect to a malicious server and dictionary $\cD_f$.
\end{lemma}

\begin{proof}
  We first show that the server is simulatable, \ie the protocol realises the functionality $f_\Pi$, and then show that the verifier \ver is of no help when performing a dictionary attack.
  
  We start with the simulation by building a simulator $\cB_\Server$, simulating a malicious server.
  $\cB_\Server$ starts by invoking adversary $\cA_\Server$ with $(P, \ServerSet)$ that is playing the role of server \Server in the protocol, and is provided with $(e,N)$ as a result.
  Using $(e,N)$, $\cB_\Server$ generates $(u_1, \dots, u_l)$ as $u_i=\alpha_i r_i^e$ for $\alpha_i\rin \ZZ^\ast_N$, $r_i\gets f_{H_1(\pwd)}(i)$, and some $l=\pmin$ where $\pwd\in\cD_f$, and returns it to $\cA_\Server$.
  The random oracle $H_2$ is honestly simulated by $\cB_\Server$.
  % with the same values used for characters $c_i$, and random values for all other queries.
  After receiving $(u'_1, \dots, u'_l)$ the simulator builds $\hat{\ClientSet}$ according to protocol specification and uses it together with $\hat{f}$, which is provided by $\cA_\Server$, to simulate the \ac{SPC} execution.
  Eventually, $\cB_\Server$ gives the result of the \ac{SPC} execution as well as $u^\ast\rin\ZZ^\ast_N$ to $\cA_\Server$ and outputs whatever $\cA_\Server$ returns on termination.
  It is easy to see that the client's view after the protocol is identical in the real and ideal world as the protocol execution is correct and all server values despite $d$ are public.
  Further, the adversary's view is computationally indistinguishable from the simulator's output since all client messages have the same distribution in both worlds.
  
  To see that a malicious server is not able to use the values retrieved in the protocol to perform a dictionary attack over $\cD_f$ more efficient than without executing the protocol, \ie we show that the adversary is not able to perform an attack on the retrieved elements $u_i, i\in[1,l]$ and $u^\ast$ that contain information about the password, which is faster than a dictionary attack over $\cD_f$.
  Since the client is essentially creating blind RSA signatures on $H_2(c_i)$ in $u_i$, those values are indistinguishable from random elements.
  While this would be true in the statistical sense if $r_i$ would be chosen uniformly at random, this is not the case here.
  However, it is easy to see that in order to verify an element $u_i$, the server has to compute $r_i$, which either requires an offline dictionary attack on $\cD_f$ to compute $k\gets H_1(\pwd)$, or yields either a collision in $H_1$ or breaks pseudorandomness in $f_k$.
  Therefore, the fastest way for a server to retrieve the password is to perform an offline dictionary attack on $\cD_f$.
  
\end{proof}

\begin{lemma}[Soundness]\label{lem:soundness}
 The enhanced password registration protocol is sound with respect to a malicious client under the one-more RSA assumption if $H_1$ and $H_3$ are collision resistant hash functions, and $H_2$ a random oracle.
\end{lemma}

\begin{proof}
  First note that the used \ac{SPC} protocol is secure and therefore guarantees that the server accepts iff the elements in $\hat{\ClientSet}$ are compliant with policy $f$ with respect to $\hat{\ServerSet}$.
  We therefore only have to show that (i) the client is not able to use different elements in \ClientSet than in $\hat{\ClientSet}$, \ie the password \pwd actually satisfies $f$, and (ii) the password verifier $\ver$ is uniquely defined by $(\pwd, e, N, d, \bm{u})$, \ie there exists no password $\pwd'\not=\pwd$ that generates the same verifier as \pwd and it is not possible to find randomness $(e',N', d')\not=(e, N, d)$ in polynomial time that generates the same verifier as $(e,N, d)$.
  
  (i) We claim that the mapping from \ClientSet to $\hat{\ClientSet}$ is an injective function such that the client is not able to build $\hat{\ClientSet}$ from a password $\pwd'\not=\pwd$.
  The elements in $\hat{\ClientSet}$ have the form $\hat{c_i}=(H_2(c_i))^d$ where $d$ is the server's secret RSA key.
  If the attacker is able to generate $\hat{c_i}=(H_2(c_i))^d$ from $c_i\not\in\ClientSet$, we can use it to build a successful attacker on the one-more RSA assumption.
  
  (ii) We claim that the password verifier $\ver=(H_3(r_{v+1}^e\prod_{i=1}^{v} u_i), e, N, d, \bm{u})$ is uniquely identified by $(\pwd, e, N, d, \bm{u})$.
  In this case we assume the client chose a policy compliant password \pwd and performed the protocol honestly.
  The claim is easy to see since 
  $
    H_3(r_{v+1}^e\prod_{i=1}^{v} u_i) = 
    H_3\left(f_{H_1(\pwd)}(v+1)^e \cdot \prod_{i=1}^{v} \left(H_2(c_i)\cdot f_{H_1(\pwd)}(i)\right)\right).
  $
  We can in particular either find collisions in $H_1$ or $H_3$, or distinguish between $f_k$ and a random function.
\end{proof}
