%************************************************
\chapter{Conclusion \& Future Work}\label{ch:futurework}
%************************************************
The work at hand gives an introduction into password-based cryptography with focus on \acl{PAKE}.
It summarises work done on \ac{PAKE} models as well as two-party and two-server \ac{PAKE} protocols.
Future work includes, but is not limited to, aforementioned research on two-server protocols, password policy checking and real world \ac{PAKE} implementations.
While already in progress, expected results on ongoing work is described in the following.

\paragraph{PAKE as Drop-in Replacement for Current Techniques}\hspace*{1em}\\
We explore challenges and possibilities for deploying \ac{PAKE} as a drop-in replacement for current \ac{HTML} form based authentication mechanisms.
This includes a prototype that allows servers to easily integrate \ac{PAKE} as login mechanism and thus replace currently used insecure techniques.
On the client side it allows users to log on to a website the way they are familiar with on desktop and mobile devices but using trusted input and secure \ac{PAKE} protocols in the background.

\paragraph{Password Policy Checker}
To tackle the inherent threat of two-party \ac{PAKE} protocol that server compromise and malicious servers pose we investigate possible solutions for secure password registration algorithms.
This involces in particular the aforementioned blind password policy checker.

\paragraph{Two-Server Protocols}
Two-server password-based protocols solve many of the inherent problems in the two-party case that stem from the use of low-entropy passwords.
However, lack of research in this area offers many possibilities for future work.
We will look here at two-server password authenticated secret sharing and two-server \ac{PAKE} protocols.


%\fk[inline]{ToDo: Future Work}
%\begin{itemize}
%	\item password policy checker
%	\item 2PASS (original intention of before 2PAKE)
%	\item System paper on PAKE as drop-in replacement for TLS-based approach
%\end{itemize}

%\section{Overall Topics of my PhD}
%\begin{itemize}
%	\item Secure online Login with passwords (practical aspects)
%	\item Security against malicious servers (opake, 2pake, 2pass, policy checker)
%	\item Better understanding of security models
%\end{itemize}

%\noindent
%The overall topic of my PhD is password-based authentication.
%Thereby I tackle the following challenges
%\begin{itemize}
%	\item deploying PAKE for website login
%	\item security against malicious servers
%	\item user challenges
%	\item understanding of password-based security models
%\end{itemize}
