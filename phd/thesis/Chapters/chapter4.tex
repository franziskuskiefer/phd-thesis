\chapter{Password Authentication Framework in the Two-Server Setting} \label{ch:2pake}

While \ac{PAKE} solves one of the most pressing problems in user authentication, it is still vulnerable to offline dictionary attacks once the server is compromised.
To alleviate the impact of password leaks on the server threshold \ac{PAKE} has been proposed.
However, research in this area is rather limited compared to the single-server case.
This chapter proposes a complete framework to use two-server \ac{PAKE} for user authentication, consisting of a blind password registration protocol and a \ac{UC}-secure two-server \ac{PAKE} protocol.

This chapter is based on work in \cite{KieferM14b,KieferM15b,KieferM15c}.

\paragraph{Introduction}

Considering that ``password-cracking tools'' such as Hashcat \cite{hashcat} and John the Ripper \cite{JohnTheRipper} are very efficient, it is safe to assume that leaked password hashes are not safer than un-hashed ones when compromised by an attacker \cite{NarayananS05a,WeirAMG09,DellAmicoMR10,Bonneau12}.
The notion of threshold and two-server password authenticated key-exchange \cite{FordK00,MacKenzieSJ02} has been proposed where the password is not stored on a single server but split between a number of servers such that leakage of a password database on a non-qualified subset does not reveal the password.
The two-server setting is regarded as more practical (in comparison to a more general threshold setting) given that if one server is compromised a notification to change the password can be sent out to the clients.
Two-server password authenticated key-exchange protocols (2PAKE) \cite{Brainard2003,SzydloK05,Katz2005} split the client's password $\pwd$ into two shares $\share_1$ and $\share_2$ such that each share is stored on a distinct server.
During the authentication phase both servers collaborate in order to authenticate the client.
Yet, no server alone is supposed to learn the plain password.
A second, more recent development in two-server (and threshold) password protocols is password authenticated secret sharing (PASS) \cite{Bagherzandi2011,CamenischLN2012,JareckiKK14} where a client stores shares of a (high-entropy) secret key on a number of servers and uses a (low-entropy) password to authenticate the retrieval process.

Registering password shares for 2PAKE/2PASS protocols however makes it impossible for the servers to verify their password policies upon registration unless the password is transferred to each of them in plain. 
This however, would imply that the client trusts both servers to securely handle its password, which contradicts the purpose and trust relationships of multi-server protocols.
The use of two-server password protocols in a remote authentication setting, therefore, requires a suitable password registration procedure in which none of the servers would receive information enabling it (or an attacker in control of the server) to deliberately or inadvertently recover the client's password.
This registration procedure must further allow for policy compliance checks to be performed by the servers since secret sharing per se does not protect against ``weak'' passwords.
A trivial approach of sending $\share_1$ and $\share_2$ to the corresponding servers over secure channels is not helpful here since it is not clear how the two servers can perform the required compliance check.
To alleviate a similar problem in the verifier-based PAKE setting, Kiefer and Manulis~\cite{KieferM14c} introduced the concept of zero-knowledge password policy checks, where upon registration the client can prove to the server the compliance of its chosen password with respect to the server's policy without disclosing the actual password.
In this work, we propose the concept of blind password registration for two-server password protocols and thus show how to realise secure registration of password shares in a way that protects against at most one malicious server (if both servers are malicious, the attacker obviously gets the password), yet allows both servers to check password compliance against their mutual password policy.
Two-server Blind Password Registration (2BPR) is not vulnerable to offline dictionary attacks as long as one server remains honest.
This is in contrast to the single-server setting where an attacker is always able to perform offline dictionary attacks on password verifiers after compromising a server. %password BPR protocol
Our main contribution is the 2BPR security model and the corresponding protocol for secure registration of 2PAKE/2PASS passwords. We show how secure distribution of password shares can be combined with an appropriate policy-compliance proof for the chosen password in a way that does not reveal the password and can still be verified by both servers.
Our 2BPR protocol can be used to enforce policies over the alphabet of all 94 printable ASCII characters, including typical requirements on password length and character types.

\paragraph{Two-Server Password Authenticated Key Exchange}
Before diving into technical preliminaries in the next section we want to discuss two-server PAKE and security guarantees it should provide.
We discuss informally several security requirements for 2PAKE protocols, which is the basis for the ideal functionality \FTWOPAKE defined later.
% Depending on the 2PAKE protocol not all of them can be achieved.
First, note that the adversary has full control over the communication channel between the client and the servers as usual.
% \emph{Security against offline dictionary attacks}
The first and foremost security requirement of password protocols is security against \emph{offline dictionary attacks} and therefore has to be fulfilled by 2PAKE protocols as well.
In particular, no eavesdropping adversary must be able to perform an offline dictionary attack on the exchanged messages.
% \emph{Security against impersonation attacks}
Further, a malicious server must not be able to \emph{impersonate a registered client} in a 2PAKE execution.
An attacker, even with knowledge and capabilities of one of the two servers, must have success probability that is not significantly better than the success probability of a brute-force attacker when running a 2PAKE protocol on behalf of a registered client.
% \emph{Security against malicious server}
The BPR game-based security notion for PAKE and two-server PAKE, which is derived from the AKE security notion, captures security by testing whether an attacker is able to distinguish between a real session key generated by a (two-server) PAKE protocol, and a randomly chosen session key.
This implies in particular that the client agrees on two independent session keys with the two servers in the two-server PAKE setting.
UC-security in contrast requires simulatability of a (two-server) PAKE protocol and is therefore harder to instantiate because the protocol has to be simulated in the case the adversary is able to guess the correct password (in which case the game-based model simply aborts the protocol execution and declares the adversary won the game).

% The strongest requirement on a 2PAKE protocol is the security against a malicious server.
% Here, a malicious server, or attacker controlling the server, must not be able to distinguish the session key between the client and the other server.
% The malicious server may be active or passive in its activities.
% Note again that not all server necessarily compute keys.

We want to further elaborate the difference between 2PAKE protocols where both servers compute the same key, compared to a 2PAKE protocol where the servers compute different keys.
Note that asymmetric key generation may also imply that only one server calculates a key while the second server only assists in the computation.
In the symmetric setting both servers calculate the same session key as result of the 2PAKE protocol.
Even though it is the first natural extension to the single server PAKE scenario, to the best of our knowledge no such protocol has been proposed yet.
This may be due to the fact that corruption of a single server compromises the session key of any execution of a 2PAKE protocol that involves this server.
In the asymmetric setting both servers generate different session keys as result of the 2PAKE protocol, possibly none.
Katz et al. proposed the first password-only 2PAKE provably secure in the standard model, based on the Katz-Ostrovsky-Yung (KOY) protocol \cite{Katz_Ostrovsky_Yung_2001} in 2005 \cite{KatzMTB05}.
While their protocol is symmetric in its execution the client computes two independent session keys, one with each server.
Other asymmetric 2PAKE protocols have been proposed \cite{Yang_Deng_Bao_2006,Jin_Wong_Xu_2007} where the server interacts with only one server.
% In this setting the second server may not even interact with the client or compute a session key.
% Before discussing these possible 2PAKE classes and their security properties we give an informal overview on security requirements on 2PAKE protocols.
% \emph{Symmetric Key Generation}
% Key generation in 2PAKE protocols however can have different forms.
% We distinguish here between \emph{symmetric} and \emph{asymmetric} key generation.
% In this case 2PAKE offers resistance against offline dictionary attacks by the server and server compromise attacks of a single server that leak the server's password database.
% However, an attacker controlling one of the two servers calculates the same session key as the honest server and the client.
% A 2PAKE protocol secure in this setting can only protect against impersonation attacks, and password database leaks while they can not prevent malicious servers from attacking it.
% \emph{Asymmetric Key Generation}
In this work we stick to the asymmetric setting where only one server computes a session key.
This however can be easily extended to a 2PAKE protocol that computes an independent session keys for each server.
% However, instead of using a game-based model such as the BPR derivatives, we propose an ideal functionality for the UC framework to show security of our 2PAKE protocol.
To this end we give the first definition of a 2PAKE UC functionality.
It brings all benefits UC-security carries in the PAKE setting such as universal composability, security holds with arbitrary (related) password choices, security on execution with non-matching password (shares), etc.
For a comprehensible overview of advantages using UC in the PAKE setting we refer to \cite{Canetti2005}.


\paragraph{Application to existing 2PAKE/2PASS protocols}
Our 2BPR protocol can be used to register passwords for two-server protocols such as two-server password authenticated key exchange (2PAKE) and two-server password authenticated secret sharing (2PASS).
2BPR can be used with two-server protocols that adopt additive password sharing in $\ZZ_q$ or multiplicative sharing in $G$.
This includes 2PAKE protocols from \cite{Katz2005,Kiefer14}, which do not consider password registration such that our protocol can simply be used as part of the registration process.
Integration of 2BPR into 2PASS on the other hand is more involved as password registration is part of the 2PASS protocol, i.e. the secret sharing phase.
2PASS protocols in general can be divided in two stages: password and secret registration/sharing and secret reconstruction.
While the approach from Bagherzandi et al. \cite{Bagherzandi2011}, as well as subsequent work using similar approaches \cite{PryvalovK14,CamenischLLN14,JareckiKK14}, does not actually share the password and could therefore use other means to verify policy compliance of a prospective password, the UC-secure 2PASS protocol from Camenisch, Lysyanskaya and Neven \cite{CamenischLN2012} uses multiplicative password sharing in $G$.
To use our 2BPR protocol in conjunction with the setup protocol from \cite{CamenischLN2012} we redefine the encoded password to $g^{\pi}$ with $\pi\gets\pwdint(\pwd)$ such that shares are computed as $g^{\pi}=g^{\share_0}g^{\share_1}$.
The first message (step 1) from the setup protocol in \cite{CamenischLN2012} can piggyback the first 2BPR protocol message.
The subsequent three messages between the client and each server are performed between step 1 and step 2, while the inter-server communication can be piggybacked on step 2 and step 3.
In addition to checking correctness of shares done in the setup of \cite{CamenischLN2012} the servers can now verify the 2BPR proofs and thus the password's policy compliance.
This adds three flows to the setup protocol of \cite{CamenischLN2012} in order verify policy compliance of password shares.

\paragraph{Outline}
\hfill\\
\mynote{do outline}

\section{Modeling Passwords and Policies}

\subsection{Passwords}
We adopt the reversible, structure-preserving encoding scheme from \cite{KieferM14c} that (uniquely) maps strings of printable ASCII characters to integers.
We use \pwd for the ASCII password string, $c_i=\pwd[i]$ for the $i$-th ASCII character in \pwd, and integer $\pi$ for the encoded password string.
The encoding proceeds as follows: $\pi\gets\pwdint(\pwd)=\sum_{i=0}^{n-1}b^{i} (\ASCII(c_i)-32)$ for the password string \pwd and $\pi_i\gets\chrint(c_i)$ $=\ASCII(c_i)-32$ for the $i$-th \emph{unshifted} ASCII character in \pwd.
Note that $n$ denotes the length of \pwd and $b\in\NN$ is used as shift base.
(We refer to \cite{KieferM14c} for a discussion on the shift base $b$. Note, however, that shift base related attacks on the password verifier from \cite{KieferM14c} are not possible in our two-server setting.) The \ASCII function returns the decimal ASCII code of a character.
In our protocol we will consider the case where password strings \pwd are chosen uniformly at random or according to some distribution with min-entropy $\beta$ from dictionary \cD.

\begin{remark}\label{rem:passwords}
While password distribution is important for the security of a password registration protocol in the verifier-based PAKE setting \cite{KieferM14c}, the password distribution plays a different role in the two-server setting.
Since the server stores only a password share instead of a password verifier, offline dictionary attacks from an attacker controlling only one of the two servers are impossible.
% (Note that this does not mean online dictionary attacks on a registered password are not possible. However, this is out of scope of the security model for a password registration protocol.)
The notion for 2BPR proposed in this work is therefore independent of the password, chosen by the client.
Note however that the password strength still continues to play a role in the usage of 2PAKE/2PASS protocols, where it influences the probability of successful online dictionary attacks.
% That said, one has to be careful not to assume that the client's password choice is not important for the overall security of the system.
% However, this only comes into play in subsequent use of the password but \emph{not} in the setup.
\end{remark}

\subsection{Password Sharing}
We focus on the additive password sharing of client passwords, i.e. $\pi=\share_0+\share_1 \mod q$ over a prime-order group $G_q$.
Such sharing has been used in various two-server PAKE protocols, including  \cite{Katz2005,Yang_Deng_Bao_2006,Jin_Wong_Xu_2007,Kiefer14}.
To be used in combination with two-server password authenticated secret sharing protocols such as \cite{CamenischLN2012} one can define the password as $g^\pi$ and thus create a multiplicative sharing $g^\pi=g^{\share_0}g^{\share_1}$.
Password shares are created as $\share_0\rin\ZZ_q$ and $\share_1=\pi-\share_0 \mod q$.
We remark that other sharing options such as xor \cite{Brainard2003,SzydloK05} have been used in literature but these are not supported by our protocol.


\subsection{Password Policies}
We represent password policies using an approach based on \cite{KieferM14c}, i.e. a password policy $f=(R,\pmin)$ consists of a simplified regular expression $R$ that defines ASCII subsets that must be present in the chosen password string and the minimum length \pmin of the password string.
Note that we do not require a maximum password length \pmax.
$R$ is defined over the \emph{four} ASCII subsets $\Sigma=\{d,u,l,s\}$ with digits $d$, upper case letters $u$, lower case letters $l$ and symbols $s$, and gives the minimum frequency of a character from the subset that is necessary to fulfil the policy; for instance, $R=ulld$ means that the password string must contain at least one upper case letter, two lower case letters and one digit.
In the two-server password setting each of the two servers may have its own policy, i.e. $f_0$ and $f_1$.
The registered passwords must comply with their \emph{mutual policy} defined as $f=f_0\cap f_1=(\max(R_0,R_1),\max(\pmin_0,\pmin_1))$, where $\max(R_0,R_1)$ is the regular expression with the maximum number of characters from each of the subsets $u,l,d,s$ from the two expressions $R_0$ and $R_1$.
%we define the intersection of two policies $f_0$ and $f_1$ that passwords registered through our 2BPR protocol will need to fulfil, namely
A mutual policy is fulfilled, i.e. $f(\pwd)=\true$, iff $f_0(\pwd)=\true$ and $f_1(\pwd)=\true$, and not fulfilled, i.e. $f(\pwd)=\false$, iff $f_0(\pwd)=\false$ or $f_1(\pwd)=\false$.
We mainly operate on the integer representation $\pi$ of a password \pwd throughout this paper and sometimes write $f(\pi)$, which means $f(\pwd)$ for $\pi\gets\pwdint(\pwd)$.
Further note that a character $c_i\in\pwd$ is called \emph{significant} if it is necessary to fulfil a policy expression $R$ and we say the corresponding set $R_j\in R$ is the according \emph{significant set}.


%********************************** %2PAKE Registration  **************************************
% \mynote{2BPR ???(ePrint/ESORICS'15) \cite{KieferM15c}}
\section{Two-Server Blind Password Registration} \label{sec:2pake-registration}
\ac{2BPR} allows a client to register password shares with two servers for later use in \ac{2PAKE}/\ac{PPSS} protocols and prove that the shares can be combined to a password that complies with the mutual password policy of both servers, without disclosing the password.
% The password shares can later be used in two-server PAKE protocols or two-server password protected secret sharing (cf. Section \ref{sec:application}).
A \ac{2BPR} protocol is executed between client $\Client$ and two servers $\Server_0$ and $\Server_1$ with password policies $f_0$ and $f_1$ respectively.
\Client interacts with $\Server_0$ and $\Server_1$ in order to distribute shares of a freshly chosen password string \pwd and prove its compliance with the mutual policy, \ie $f_0(\pwd)=\true$ and $f_1(\pwd)=\true$.
A \ac{2BPR} protocol between an honest client $\Client$ and two honest servers $\Server_0$ and $\Server_1$ is correct if $\Server_0$ and $\Server_1$ accept their password shares if the client is able to prove the following statement for $f=f_0\cap f_1$:
\begin{align}
  (\pwd,\share_0,\share_1):~ \pwdint(\pwd)=\share_0+\share_1 ~ \wedge ~ f(\pwd)=\true.
  \label{eq:zkppc-pok}
\end{align}
Note that the \ac{2BPR} protocol can be used to register new clients or to register new passwords for existing clients. The following definition formally captures the functionality of \ac{2BPR} protocols.

\begin{definition}[Two-Server Blind Password Registration]\label{def:2bpr}
A \ac{2BPR} protocol is executed between a client $\Client$ and two servers $\Server_0$ and $\Server_{1}$, holding a password policy $f_b$ each, such that the servers, when honest, eventually accept password shares $\share_b$ of a policy compliant, client chosen password \pwd iff $f(\pwd)=\true$ for $f=f_0\cap f_1$, $\pwdint(\pwd)=\share_b+\share_{1-b}$ and $b\in\{0,1\}$.
\eod
\end{definition}

\noindent
Definition~\ref{def:2bpr} requires that password shares $\share_0$ and $\share_1$ can be combined to the policy-compliant integer password $\pi$. The corresponding verification must therefore be part of the \ac{2BPR} protocol. 
Otherwise, the client could register password shares $\share_0$ and $\share_1$ that can both be combined to a policy compliant password in the respective proofs with the servers, but combining $\share_0$ and $\share_1$ might result in a password that is \emph{not} policy compliant, \ie $f(\share_0+\share')=\true$ and $f(\share_1+\share'')=\true$ but $f(\pi)\not=\true$.
This further ensures that servers hold valid password shares, which is crucial for the security of \ac{2PAKE}/\ac{PPSS} protocols that should be executed later with these password shares.
We assume that the servers distributed their policy to the client before the protocol starts.
We further assume that the servers communicate with each other in order to confirm correctness of the password shares.
This requirement can be relaxed by requiring the client to forward messages (authenticated and confidential) between the two servers.
For simplicity however we assume direct communication between the two servers in our model.


\subsection{Model}\label{sec:securitymodel}
Security of \ac{2BPR} protocols, given according to Eq. (\ref{eq:zkppc-pok}), has to guarantee that the client knows the sum $\pwdint(\pwd)$ of the password shares $\share_0$ and $\share_1$, and that \pwd fulfils both password policies $f_0$ and $f_1$ in case both servers accept the registration.
This model is related to the \ac{BPR} model from Chapter \ref{ch:vpake} Section \ref{sec:bpr} but requires handling of two servers and server corruption.
We translate Eq. (\ref{eq:zkppc-pok}) into a game-based security model that captures the relation using two different security definitions.
The first notion, like in \ac{BPR}, can be directly converted from Eq. (\ref{eq:zkppc-pok}) and captures \acl{PC} of the password.
In particular, if both servers are honest while accepting their password shares in the \ac{2BPR} protocol, the combination $\pi$ of the shares represents a password compliant with the mutual password policy $f=f_0\cap f_1$, \ie $f(\share_b+\share_{1-b})=\true$.
The second notion relates to the fact that servers should not learn anything about the password and therefore called \ac{PB}, in contrast to \ac{BPR}, where the server is always able to perform offline dictionary attacks on the password verifier.
\ac{PB} requires that a malicious server $\Server_b$ must not be able to learn anything else from the \ac{2BPR} execution about the client's password than the fact that it is policy compliant.

We observe that the blindness property has to hold for all possible password policies and all compliant passwords. It should not be possible to mount an offline dictionary attack after observing \ac{2BPR} protocol executions or while gaining access to and controlling at most one of the two servers.

\paragraph{Setup and Participants}
Protocol participants $\Client,\Server_0,\Server_1$ with $\Client$ from the universe of clients and $\Server_1,\Server_2$ from the universe of servers have common inputs, necessary for the execution of the protocol. %, such as group parameters, the common reference string, the server's public keys and according policies $f_1,f_2$.
Instances of protocol participants \Client or \Server are denoted $\Client_i$, $\Server_{0,i}$ or $\Server_{1,i}$.
Protocol participants without specified role are denoted by $P$, and $\Server_b$ and $\Server_{1-b}$ for unspecified servers.
A client can only register one password with one server pair, but can register passwords at an arbitrary number of server pairs.
Client $\Client$ and servers $\Server_b$ are unique and used as identifier on the server, \ie as \emph{username} to store alongside the password share on $\Server_{1-b}$.
We say a client registers a client, server pair at a server, \ie a client $\Client$ registers a password share for $(\Client, \Server_{1-b})$ at a server $\Server_b$ and a password share for $(\Client,\Server_{b})$ at server $\Server_{1-b}$.
Further, a server only allows a single registration from a client, server pair such that any attempt to register a password with a server that already stores a password share for this client, server pair overwrites existing entries, \ie resets the password.
An entry $(\Client,\Server_{1-b},\share_b)$ is only stored on server $\Server_b$ if the \ac{2BPR} protocol is successful.

\paragraph{Oracles}
To interact with protocol participants, \ac{PPT} adversary \cA has access to \Setup, \Send, \Execute and \Corrupt oracles.
% We say a message $m$ is \emph{rogue} if it is either generated by the adversary or by an oracle and then modified by the adversary or by an oracle but for a different session.

\begin{itemize}
  \item $\Setup(\Client,\Server_0,\Server_1, \pwd')$ creates new instances of all participants and stores identifiers of the other parties to each participant.
        To this end the client receives the server policies $f_0\cap f_1=f$ and either chooses a new policy compliant password $\pwd\in\cD_f$ if $\pwd'=\bot$ or uses $\pwd=\pwd'$.
%         This oracle models the initial registration request by the client and returns the first server messages $m_0$ and $m_1$ with $m_b$ honestly generated if $m'_b=\bot$ or $m_b\gets m'_b$ otherwise for $b\in\bits$.

  \item $\Execute(\Client,\Server_0,\Server_1)$ models a passive attack and executes a \ac{2BPR} protocol between new instances of $\Client$, $\Server_0$ and $\Server_1$. %, initiated with \Setup.
%         If there exists an entry $(C,\Server_1,\share_0)$ at $\Server_0$ or $(C,\Server_2,\share_1)$ at $\Server_2$, the oracle aborts.
        It returns the protocol transcript and the internal state of all corrupted parties.
%         Note that communication between the protocol is confidential such that the adversary only learns ciphertexts unless he corrupted one of the participants.

  \item $\Send_\Client(\Client,\Server_{b,j},m)$ sends message $m$, allegedly from client $\Client$, to server instance $\Server_{b,j}$ for $b\in\{0,1\}$.
        If $\Server_{b,j}$ does not exist, the oracle aborts. %or there exists an entry $(C,\Server_2,\share_1)$ at $\Server_1$ or $(C,\Server_1,\share_2)$ at $\Server_2$,
        Note that any instance $\Server_{b,j}$ was thus set up with \Setup and therefore has an according partner instance $\Server_{1-b,j}$.
        If all participants exist, the oracle returns the server's answer $m'$ if there exists any.
        Necessary inter server communication is performed in $\Send_\Client$ queries.
        If $m=\bot$, server $\Server_{b,j}$ returns its first protocol message if it starts the protocol.
        % if both servers are honest.
%         If $\Server_{1-b}$ is corrupt, the oracle additionally returns message $m'_\Server$ from $\Server_{b,i}$ to $\Server_{1-b,i}$ if such a message is sent in the protocol.
%         This oracle can model a passive attack if the message $m$ is not rogue, or an active attack if $m$ is rogue.

  \item $\Send_\Server(\Server_{b},\Client_{j},m)$ sends message $m$, allegedly from server $\Server_{b}$ for $b\in\{0,1\}$, to client instance $\Client_{j}$.
        If $\Client_j$ does not exist, the oracle aborts.
        Note that any instance $\Client_j$ was set up with \Setup and therefore has an according partner instance $\Server_{1-b,i}$.
        If all participants exist, the oracle returns the client's answer $m'$ if there exists any.
        If $m=\bot$, server $\Server_{b}$ returns its first message if he starts the protocol.
%         This oracle can model a passive attack if the message $m$ is not rogue, or an active attack if $m$ is rogue.

  \item $\Send_{\Server\Server}(\Server_{b},\Server_{1-b,j},m)$ sends message $m$, from server $\Server_{b}$ for $b\in\{0,1\}$, to server instance $\Server_{1-b,j}$.
        If $\Server_{1-b,j}$ does not exist, the oracle aborts.
        Note that any $\Server_{1-b,j}$ was set up with \Setup.
        If all participants exist, the oracle returns the server's answer $m'$ if there exists any.

  \item $\Corrupt(\Server_{b})$ allows the adversary to corrupt a server $\Server_b$ and retrieve its internal state, \ie stored messages and randomness, and the list of stored password shares $(\Client, \Server_{1-b}, \share_b)$. % if no instance of $\Server_b$ is active.
      $\Server_b$ is marked \emph{corrupted}.
%       Otherwise the oracle aborts.
\end{itemize}

\noindent
Note that we allow the adversary to register passwords with servers such that we do not require the existence of a client instance $\Client_i$ after a successful registration (client identities $\Client$ are unique but not secret and can therefore be used by the adversary).

\paragraph{Policy Compliance}
\ac{PC} is the first natural security property of \ac{2BPR} protocols, requiring that a password set up with a \ac{2BPR} protocol by a client is compliant with the policy of the two servers $f(\pwd)=\true$.
The attacker here plays the role of the client that tries to register a password \pwd that is \emph{not} policy compliant on two honest servers.

% XXX: \fk{We should think about if it is sensible to allow password reset. In real the world the server would require the client to login before changing the password (authentication is done via e-mail usually if password is not known anymore). Therefore the adversary should be allowed to change passwords for triple $(C,\Server_0,\Server_1)$ that executed the protocol successfully already.}

\begin{definition}[Policy Compliance]\label{def:pc}
\acl{PC} of a \ac{2BPR} protocol holds if for every \ac{PPT} adversary $\cA$ with access to $\Setup$ and $\Send_\Client$ oracles the probability that two server instances $\Server_{b,i}$ and $\Server_{1-b,j}$ exist after $\cA$ stopped that accepted $(\Client,\Server_{1-b},\share_b)$, $(\Client,\Server_{b},\share_{1-b})$ respectively, with $f(\share_b+\share_{1-b})=\false$ is negligible.
\eod
\end{definition}

\paragraph{Password Blindness}
The second security property requires that every password, chosen and set up by an honest client must remain hidden from an adversary even if he corrupts one of the two used servers, gets its internal state and controls its actions.
We model this with a distinguishing experiment where the attacker, after interacting with the oracles, outputs a challenge comprising two passwords $\pwd_0$ and $\pwd_1$, two clients $\Client_0$ and $\Client_1$, and a pair of servers $\Server_0$ and $\Server_1$.
After randomly assigning the two passwords to the two clients, the adversary interacts with the oracles again and has to decide which client uses which password, \ie guess the random bit $b$.
The notion is formalised in the following definition.

\begin{definition}[Password Blindness]\label{def:zk}
The \acl{PB} property of a \ac{2BPR} protocol $\Pi$ holds if for every \ac{PPT} adversary $\cA$ there exists a negligible function $\varepsilon(\cdot)$ such that
% for $t$ passwords set-up for $(\Client,\Server_0,\Server_1)$
\[ \Adv_{\Pi, \cA}^{\mathrm{PB}}=\left|\Pr[\Exp_{\Pi, \cA}^{\mathrm{PB}}=1]-\frac{1}{2}\right|\leq\varepsilon(\secpar). \]

\noindent
$\Exp_{\Pi, \cA}^{\mathrm{PB}}:$ \\
\hspace*{2em} $(\Client_0, \Client_1, \Server_0,\Server_1,\pwd_0,\pwd_1)\gets\cA_1^{\Setup(\cdot),\Send_\Server(\cdot),\Send_{\Server\Server}(\cdot), \Execute(\cdot),\Corrupt(\cdot)}$ $(\secpar,\cD,\{\Client\},\{\Server\})$ \\
\hspace*{2em} check $\pwd_0,\pwd_1\in\cD_{f_0\cap f_1}$, $|\pwd_0|=|\pwd_1|$, $\Client_0,\Client_1\in\{\Client\}$ and $\Server_0,\Server_1\in\{\Server\}$\\
\hspace*{2em} $ b'\gets\cA^{\Setup'(\cdot),\Send_\Server(\cdot),\Send_{\Server\Server}(\cdot), \Execute(\cdot),\Corrupt(\cdot)}(\secpar,\cD,\{\Client\},\{\Server\})$\\
\hspace*{2em} if $\Server_0$ or $\Server_1$ is not corrupted, return $b=b'$; otherwise return $0$ \\

\noindent
$\Setup'$ (in contrast to \Setup) uses $\pwd_b$ for client $\Client_0$ and $\pwd_{1-b}$ for $\Client_1$ with $b\rin\bits$ instead of choosing a random password or using the provided one.
\eod
\end{definition}


% ==================================================================================
% Two-Server BPR Protocol
% ==================================================================================

\subsection{An Efficient Two-Server BPR Protocol}\label{sec:framework}
Before describing technical details, we give a high-level description of our \ac{2BPR} protocol.
We assume that client $\Client$ selected two servers $\Server_0$ and $\Server_1$ to register with, and server authenticated channels between each of the two servers and the client as well as between the two servers have been established.
Note that the connection between the two servers is necessary in order to verify the correctness of the password shares.
% If a direct connection between the two servers is not possibleA weaker security notion where correctness is not necessary is possible.
Server authenticated and confidential channels between each server and the client is a rather weak assumption that can be implemented with a \ac{TLS} channel \cite{rfc5246,JagerKSS12,KrawczykPW13}.
This essentially prevents the attacker from querying the $\Send_\Server$ or $\Send_{\Server\Server}$ oracle without prior corruption of the according server and further provides an attacker only with encrypted transcripts unless he performs active attacks or corrupts a participating server.
% We describe the protocols internal functionality in the following paragraph without explicit description of the encrypted channel.
The protocol further relies on a common setup, given a \ac{CRS}.
The \ac{CRS} contains all necessary parameters, \ie $\crs=(g, h, p)$ with generators $g$ and $h$ for group of prime-order $p$ where $\log_g(h)$ is not known.
At the beginning of the registration phase the client commits to the integer representation $\pi$ of the chosen password string \pwd and sends this commitment together with a password share $\share_b$ to the corresponding server $\Server_b$, $b\in\bits$, along with auxiliary information that is needed to perform the \ac{PC} proof.
For the latter, the client needs to prove the knowledge of $\pi$ in the commitment such that $\pi = \share_0 + \share_1$ and that it fulfils both policies $f_1$ and $f_2$.
Thus, servers $\Server_0$ and $\Server_1$ eventually register the new client, accept and store the client's password share, iff each $\Server_b$ holds $\share_b$ such that $\share_0+\share_1=\pi$ for $\pi\gets\pwdint(\pwd)$ and $f(\pwd)=\true$ for $f=f_0\cap f_1$.

The general protocol is similar to the \ac{BPR} protocol proposed in Chapter \ref{ch:vpake} Section \ref{sec:bpr} for the single server setting.
While \acl{PoM} and \ac{PoS} are essentially the same, a \ac{PoC} replaces the \ac{PoE} from \ac{BPR} to prove correctness of shares.
We further require zero-knowledge proofs that are secure against malicious verifiers (servers) since the attacker is allowed to corrupt one server in the protocol, \ie we do not assume semi-honest servers.

\subsubsection{Protocol Overview}
% \begin{figure}[!b]
% \centering
% \footnotesize
% \scalebox{0.8}{\begin{tikzpicture}%[framed]
% \draw[] (-3.5,1) rectangle (12.3,-13);
% \draw ($(-3,.7)!0.55!(12,.7)$) node[draw] {\textit{I -- Client Preparation}};
%
% \node[party,align=center] (client) at (.5,0) {\uline{$\Client~ (\Server_0,\Server_{1}, f_0, f_1, \pwd, \crs)$}};
%
% \node[state, align=left] at (0,-.5) [stateS, align=left]{Encode $\pi\gets\pwdint(\pwd)$;};
% \node[state, align=left] at (0,-1.2) [stateS, align=left]{Compute password shares: $\share_{0}\rin\ZZ_p$,\\ \hspace*{1em} $\share_{1}=\pi-\share_{0}$};
%
% \node[state, align=left] at (0,-2.3) [stateS, align=left]{Commit to shares: \\\hspace*{1em} $\fC_0=g^{\share_{0}}h^{r_{0}}$, $\fC_{1}=g^{\share_{1}}h^{r_{1}}$\\ \hspace*{1em} $\fD_0=\fC_0g^{\share_{1}}$, $\fD_{1}=\fC_{1}g^{\share_{0}}$};
%
% % PHASE II
%
% % \draw[dashed] (-2.5, -3.2) -- (11.5, -3.2);
% \draw ($(-3,-3.5)!0.55!(12,-3.5)$) node[draw] {\textit{II -- Password Registration}};
%
% \node[party,align=center] (client2) at (.5,-4.2) {\uline{$\Client~ (\Server_0,\Server_{1}, f=f_0\cap f_1, \pwd, \crs)$}};
% \node[align=center,text width=13em] (server) at (9.6,-4.2) {\uline{$\Server_b~ (\Client,\Server_{1-b}, f=f_0\cap f_1, \crs)$}};
%
% \node[state, align=left] at (0.0,-4.9) [stateS, align=left]{Commit to all characters in \pwd:\\\hspace*{1em} $C_i=g^{\pi_i}h^{r_i};~ C'_i=C_ih^{r'_i}$};
%
% \node[state, align=left] at (0.0,-5.8) [stateS, align=left]{Shuffle $\bm C'$ s.t. each $C'_i=C_{\phi(i)}h^{r'_{\phi(i)}}$ with permutation $\phi$ over $[1,|\pwd|]$;};
%
% \node[state, align=left] at (0,-6.8) [stateS, align=left]{For each $c_i\in\pwd$ identify appropriate set $\omega_{\phi(i)}$ and build $\vect{\omega}$ from it;};
% \node[state, align=left] at (0,-7.6) [stateS, align=left]{Execute ZK proofs with the server};
%
% \node[state, align=left] at (9.75,-8.0) [stateSmS, align=left]{choose challenges};
% \node[state, align=left] at (9.75,-9) [stateSmS, align=left]{Proceed if \\\hfill $|\bm C|=|\bm C'| \geq \pmin$, $\PoM$,\\\hfill $\PoC$ and $\PoS$ all holds};
%
% \node[dummyState] (clientCom) at (2.75,-7.6){};
% \node[dummyState] (clientCh) at (2.75,-8.2){};
% \node[dummyState] (clientRes) at (2.75,-8.8){};
% \node[dummyState] (clientResF) at (2.75,-9.4){};
%
% \node[dummyState] (serverCom) at (7.5,-7.6){};
% \node[dummyState] (serverCh) at (7.5,-8.2){};
% \node[dummyState] (serverRes) at (7.5,-8.8){};
% \node[dummyState] (serverResF) at (7.5,-9.4){};
%
% \draw[pil] (clientCom) -- node[above, align=center] {$\Comm_{\PoM}, \Comm_{\PoC}, \Comm_{\PoS}, n$} (serverCom);
% \draw[pil] (serverCh) -- node[above, align=center] {$\Ch_\PoC, \Ch_\PoM, \Ch_\PoS$} (clientCh);
% \draw[pil] (clientRes) -- node[above, align=center] {$\Res_{\PoM}, \Res_{\PoC}, \Res_{\PoS}$} (serverRes);
%
% % PHASE III
%
% % \draw[dashed] (-2.5, -10.2) -- (11.5, -10.2);
% \draw ($(-3,-10.5)!0.55!(12,-9.7)$) node[draw] {\textit{III -- Share Verification}};
%
% \node[party] (server2) at (.5,-10.7) {\uline{$\Server_0~ (\Client, \Server_{1}, f=f_0\cap f_1, \crs)$}};
% \node[state, align=left] at (0,-11.4) [stateS, align=left]{$\fD'_{1}=\fC_{1}g^{\share_{0}}$};
% \node[state, align=left] at (0,-12.2) [stateS, align=left]{If $\fD'_0=\fD_0$\\\hspace*{1em} store $(\Client,\Server_{1},\share_{0})$};
% \node[dummyState] (server11Ch) at (2.75,-11.4){};
% \node[dummyState] (server12Ch) at (2.75,-12.2){};
%
% \node[party,align=center] (server3) at (9.5,-10.7) {\uline{$\Server_{1}~ (\Client, \Server_0, f=f_0\cap f_1, \crs)$}};
% \node[state, align=left] at (9.75,-11.4) [stateSmS, align=left]{$\fD'_{0}=\fC_{0}g^{\share_{1}}$};
% \node[state, align=left] at (9.75,-12.2) [stateSmS, align=left]{If $\fD'_{1}=\fD_{1}$\\\hspace*{1em} store $(\Client,\Server_{0},\share_{1})$};
%
% \node[dummyState] (server21Ch) at (7.5,-11.4){};
% \node[dummyState] (server22Ch) at (7.5,-12.2){};
%
% \draw[pil] (server11Ch) -- node[above, align=center] {$\fD'_{1}$} (server21Ch);
% \draw[pil] (server22Ch) -- node[above, align=center] {$\fD'_{0}$} (server12Ch);
%
% \end{tikzpicture}}
% \caption[Two-Server BPR Protocol]{Two-Server BPR Protocol --- A High-Level Overview\\
% {\small $\vect{\omega}$ contains character sets of $c_{\phi(i)}$ ordered according to permutation $\phi$, used in \ac{PoM}}}
% %\linebreak \PoS proves correctness of shuffle according to $\phi$
% \label{fig:protocol-overview}
% \end{figure}

\begin{figure*}[tbhp]
\begin{center}
\scalebox{0.95}{
\begin{tabular}{ l c l }
\toprule
{\bf Client \Client} & & \\
Input: $\Server_0,\Server_{1}, f=f_0\cap f_1$ & &\\
\hspace*{2.8em} $\pwd, \crs$ & & \\
\midrule
& \textit{I -- Client Preparation} & \\
% \cmidrule{2-2}
encode $\pi\gets\pwdint(\pwd)$ & & \\
compute password shares: & & \\
\hspace*{1em} $\share_{0}\rin\ZZ_p$, $\share_{1}=\pi-\share_{0}$ & & \\
commit to shares: & & \\
\hspace*{1em} $\fC_0=g^{\share_{0}}h^{r_{0}}$, $\fC_{1}=g^{\share_{1}}h^{r_{1}}$ & & \\
\hspace*{1em} $\fD_0=\fC_0g^{\share_{1}}$, $\fD_{1}=\fC_{1}g^{\share_{0}}$ & & \\
\midrule
{\bf Client \Client} & & {\bf Server $\Server_b$} \\
Input: $\Server_0,\Server_{1}, f=f_0\cap f_1$ & & Input: $\Client,\Server_{1-b}$ \\
\hspace*{2.8em} $\pwd, \crs$ & & \hspace*{2.8em} $f=f_0\cap f_1, \crs$ \\
\midrule
& \textit{II -- Password Registration} & \\
% \cmidrule{2-2}
commit to all characters: & & \\
\hspace*{1em} $C_i=g^{\pi_i}h^{r_i};~ C'_i=C_ih^{r'_i}$ & & \\
shuffle $\bm C'$ s.t. $C'_i=C_{\phi(i)}h^{r'_{\phi(i)}}$ & & \\
with permutation $\phi$ & & \\
for $c_i\in\pwd$ identify set $\omega_{\phi(i)}$ & & \\
execute \PoC, \PoM, \PoS & & \\
 & $\xrightarrow{\makebox[4cm]{$\Comm_{\PoM}, \Comm_{\PoC}, \Comm_{\PoS}$}}$ & \\
 & $\xleftarrow{\makebox[4cm]{$\Ch_{\PoM}, \Ch_{\PoC}, \Ch_{\PoS}$}}$ & choose challenges\\
 & $\xrightarrow{\makebox[4cm]{$\Res_{\PoM}, \Res_{\PoC}, \Res_{\PoS}$}}$ & Proceed if\\
 & & $|\bm C|=|\bm C'| \geq \pmin, \PoM$\\
 & & $\PoC$ and $\PoS$ all holds\\
\midrule
{\bf Server $\Server_0$} & & {\bf Server $\Server_1$} \\
Input: $\Client, \Server_{1}$ & & Input: $\Client,\Server_{0}$ \\
\hspace*{2.8em} $f=f_0\cap f_1, \crs$ & & \hspace*{2.8em} $f=f_0\cap f_1, \crs$ \\
\midrule
& \textit{III -- Share Verification} & \\
% \cmidrule{2-2}
$\fD'_{1}=\fC_{1}g^{\share_{0}}$ & $\xrightarrow{\makebox[4cm]{$\fD'_{1}$}}$ & $\fD'_{0}=\fC_{0}g^{\share_{1}}$ \\
If $\fD'_0=\fD_0$ & $\xleftarrow{\makebox[4cm]{$\fD'_{0}$}}$ & If $\fD'_{1}=\fD_{1}$ \\
\hspace*{1em} store $(\Client,\Server_{1},\share_{0})$ & & \hspace*{1em} store $(\Client,\Server_{0},\share_{1})$ \\
\bottomrule
\end{tabular}}
\end{center}
\caption[Two-Server BPR Protocol]{Two-Server BPR Protocol --- A High-Level Overview\\
 {\small $\vect{\omega}$ contains character sets of $c_{\phi(i)}$ ordered according to permutation $\phi$, used in \ac{PoM}}}
\label{fig:protocol-overview}
\end{figure*}

\noindent
In Figure \ref{fig:protocol-overview} we give an overview of the \ac{2BPR} protocol involving a client $\Client$ and two servers $\Server_b$, $b\in\bits$.
The protocol proceeds in three phases.
In the first phase \emph{(Client Preparation)} the client, encodes the chosen password \pwd to $\pi$, computes shares $\share_{0}$ and $\share_{1}$, and computes commitments $\fC_0, \fC_1, \fD_0, \fD_1$ to the shares and the password.

In the second phase \emph{(Password Registration)} $\Client$ interacts with each server $\Server_b$, $b\in\bits$ over a server-authenticated and confidential channel. $\Client$ computes a commitment $C_i$ for each encoded character $\pi_i\gets\chrint(c_i)$, $c_i\in\pwd$, and a second commitment $C'_i$ as a re-randomised version of $C_i$.
The set $\bm C'$ containing the re-randomised commitments $C'_i$, is then shuffled and used to prove in the \acl{PoM} protocol that each character committed to in $C'_i\in\bm C'$ is a member of some character set $\omega_{\phi(i)}$, chosen according to policy $f$.
Note that \ac{PoM} must be performed over the \emph{shuffled} set $\bm C'$ of commitments as the server would otherwise learn the type (lower/upper case, digit, or symbol) of each password character.
To further prove that transmitted commitments ${\bm C}, \fC_b$, and $\fD_b$ are correct, namely that the product of commitments in $\bm C$ commits to password \pwd, $\fC_b$ contains the correct share $\share_b$, and $\fD_b$ contains \pwd, client and server execute the \acl{PoC} protocol. 
Finally, the client proves to each server that set $\bm C'$ is a shuffle of set $\bm C$ by executing the \acl{PoS} protocol. 
This proof is necessary to finally convince both servers that 
(1) the characters committed to in $\bm C'$ are the same as the characters in the commitments in $\bm C$, which can be combined to password \pwd (as follows from the \ac{PoC} protocol) and 
(2) each commitment $C_i$ is for a character $c_i\in\pwd$ from some set $\omega_{i}$, chosen according to policy $f$ (as follows from the \ac{PoM} protocol).  
%, yet without revealing the set $\omega_{i}$
For all three committed $\Sigma$ protocols (\ac{PoM}, \ac{PoC}, \ac{PoS}) we use variables as defined in Chapter \ref{ch:prelims} Section \ref{sec:commited-sigma}.
If each server $\Server_b$, $b\in\bits$ successfully verifies all three committed $\Sigma$ protocols and the length of the committed password $\pwd$ is policy-conform, then both servers proceed with the last phase. 

In the third phase \emph{(Share Verification)} the two servers $\Server_0$ and $\Server_1$ interact with each other over a mutually-authenticated and confidential channel. Each $\Server_b$ computes its verification value $\fD'_{1-b}$ and sends it to $\Server_{1-b}$.
Upon receiving $\fD'_{b}$, $\Server_b$ checks it against $\fD_b$ to verify that the client used the same password with both servers in the second phase, \ie that $\share_b+\share_{1-b}=\pi$.
If this verification is successful, $\Server_b$ stores the client's password share $(\Client,\Server_{1-b},\share_b)$ and considers $\Client$ as being registered.

\subsubsection{Specification}\label{sec:protocol}
In the following we give a detailed description of the \ac{2BPR} protocol.
To this end we describe the three proofs \ac{PoC}, \ac{PoM} and \ac{PoS} detailing on their computations.
We describe the interaction between client $\Client$ and server $\Server_b$ and therefore only consider one policy $f_b$.
Note that \Client and each server $\Server_b$ perform the same protocol.
If both servers accept, the password fulfils the policy $f=f_b\cap f_{1-b}$.
The following largely recalls the zero-knowledge proofs from Section \ref{sec:bpr-proofs} in Chapter \ref{ch:vpake} but with committed $\Sigma$ proofs.

We first describe the client's pre-computations such as password encoding and sharing before giving a detailed description of the proofs.
The protocol operates on a group \GG of prime-order $p$ with generator $g$.
Further, let $h,f_i\rin \GG$ for $i\in[-4,m]$ denote random group elements such that their discrete logarithm with respect to $g$ is unknown.
Public parameters of the protocol are defined as $(p,g,h,\bm f, H)$ with $\bm f = \{f_i\}$ where $m$ is at least $n=|\pwd|$, and hash function $H$.
In practice $m$ can be chosen big enough, \eg $100$, in order to process all reasonable passwords.
Note that we use the range $i\in[0,n-1]$ for characters $\pwd[i]$ and their commitments $C_i$, but $[1,x]$ for most other ranges.

\paragraph{Client Preparation}
We assume that password policies $f_0$ and $f_1$ are known by the client.
This can be achieved by distributing them beforehand with other set-up parameters.
The client encodes password $\pwd\in\cD_f$ to $\pi\gets\pwdint(\pwd)$.
The password is shared by choosing a random $\share_{b}\rin\ZZ_p$ and computing $\share_{1-b}=\pi-\share_{b}$.
The client then commits to both password shares $\fC_b=g^{\share_{b}}h^{r_b}$ and $\fC_{1-b}=g^{\share_{1-b}}h^{r_{1-b}}$ with $r_b,r_{1-b}\rin\ZZ_p$ and computes commitments to the entire password $\pi$ with the same randomness, \ie $\fD_b=\fC_b g^{\share_{1-b}}$ and $\fD_{1-b}=\fC_{1-b} g^{\share_{b}}$.
For the following proofs the client further encodes every character $c_i\in\pwd$ as $\pi_i\gets\chrint(c_i)$.
% The remainder of the protocol is performed separately between the client and each server $\Server_b$.

\paragraph{Password Registration}
The client iterates over all encoded characters $\pi_i$ to perform the following operations:
\begin{itemize}
  \item commit to $\pi_i$ by computing $C_i=g^{\pi_i}h^{r_i}, C'_i=C_i h^{r_i'}$ for $r_i,r'_i\rin\Zrp$;
  \item choose a random permutation $\phi(i)$ over $[0,n-1]$ to shuffle $C'_i$;
  \item if $\pi_i$ is \emph{significant} for any $R_j\in R$, set $\omega_{\phi(i)}\gets R_j$, otherwise $\omega_{\phi(i)}\gets\Sigma$ (all \ac{ASCII} characters).
\end{itemize}
Let further $l_i\in\NN$ denote the index in $\omega_{\phi(i)}$ such that $c_i=\omega_{\phi(i)}[l_i]$.
Values $(C_i$, $C'_i$, $\omega_{\phi(i)}$, $\phi(i)$, $l_i$, $\pi_i$, $r_i$, $r'_i)$ are used in the following zero-knowledge proofs.
The client combines previously computed values $\bm C = \{C_{i}\}$.
Shuffled commitments $C'_{\phi(i)}$ and sets $\omega_{\phi(i)}$ are combined according to the shuffled index $\phi(i)$, \ie $\bm C' = \{C'_{\phi(i)}\}$ and $\vect{\omega} = \{\omega_{\phi(i)}\}$.
Once these computations are finished $\Client$ and $\Server_b$ proceed with the protocol.
In the following we describe the three proofs \ac{PoM}, \ac{PoC} and \ac{PoS} and define their messages.
Similar to the single-server case the following proofs are based on techniques first introduced in by \citet{CramerDS94,Schnorr91,Chaum93}.

\paragraph{Proof of Correctness (\PoC)}
This proof links the password shares, sent to each server, to the proof of \ac{PoM} and shows knowledge of the other password share.
We define the proof of correctness for an encoded password $\pi$, which proves that share $\share_{b}$ can be combined with a second share $\share_{1-b}$ such that $\pi=\share_{b}+\share_{1-b}$ and that the received commitments to password characters $c_i$ can be combined to a commitment to that same password $\pi$.
\PoC is defined as a committed zero-knowledge proof between $\Client$ and $\Server_b$ for the statement
\begin{align*}
&\ZKP\{(\pi,r_{1-b},r_b,r_{Cb}): \fC_{1-b}g^{\share_{b}}=g^{\pi}h^{r_{1-b}}  \wedge  \prod_{i=0}^{n-1}C_i^{b^i}=g^{\pi}h^{r_{Cb}}  \wedge  \fD_{b}=g^{\pi}h^{r_{b}}\}. 
\end{align*}
% for newly generated $C_{1-b}=g^{\pi_{1-b}}h^{r_{1-b}}$ and $D_{b}=C_{b}g^{\pi_{1-b}}$ with fresh $C_{1-b}=g^{\pi_{1-b}}h^{r_{1-b}}$ from the server sub-protocol.
$C_i=g^{\pi_i}h^{r_i}$ are character commitments from the set-up stage and $r_{Cb}=\sum_{i=0}^{n-1}b^i r_i$ is the combined randomness from character commitments $C_i$.
$\fC_{1-b}=g^{\share_{1-b}}$ $h^{r_{1-b}}$, $\fD_{b}=\fC_{b}g^{\share_{1-b}}$, and $\fC_{b}=g^{\share_{b}}h^{r_{b}}$ are share and password commitments from the client's preparation phase.
This connects the link of the password commitment to the product of the character commitments with the proof of knowledge of the combined password $\pi=\share_b+\share_{1-b}$.
Messages for \ac{PoC} are computed as follows:

\begin{enumerate}
  \item %[$\Com_{\PoC}$:]
    The client chooses random $k_{\pi}$, $k_{\rho b}$, $k_{\rho (1-b)}, k_{\rho C}\rin\ZZ_p$, computes $t_{C(1-b)}=g^{k_{\pi}}h^{k_{\rho (1-b)}}$, $t_{C}=g^{k_\pi}h^{k_{\rho C}}$ and $t_{Db}=g^{k_{\pi}}h^{k_{\rho b}}$.
    The first message with $r_{\Comm_{\PoC}}\rin\ZZ_p$ is then given by
    \begin{align*}      
    & \Comm_{\PoC}=g^{H(\fC_{1-b}g^{\share_b}, \{C_i\}, \fD_b, t_{C(1-b)}, t_C, t_{Db})} h^{r_{\Com_{\PoC}}}.
    \end{align*}

  \item %[$\Ch_{\PoC,b}$:]
    After receiving $\Com_{\PoC}$ from the client the server chooses a random challenge $\Ch_{\PoC,b}\rin\ZZ_p$ and sends it back to the client.

  \item %[$\Res_{\PoC1}$:]
    After receiving challenge $\Ch_{\PoC, b}$, the client computes $s_{\pi}=k_{\pi}+\Ch_{\PoC,b}\pi$, $s_{\rho (1-b)}=k_{\rho{(1-b)}}+\Ch_{\PoC,b}r_{1-b}$, $s_{\rho C}=k_{\rho C} + \Ch_{\PoC,b} \sum_{i=0}^{n-1}b^i r_i$ and $s_{\rho b}=k_{\rho{b}}+\Ch_{\PoC,b}r_{b}$ before computing the next message with $r_{\Res_{\PoC}}\rin\ZZ_p$
    \[\Res_{\PoC1}=g^{H(s_{\pi}, s_{\rho (1-b)}, s_{\rho C}, s_{\rho b})} h^{r_{\Res_{\PoC}}}.\]

  \item %[$\Res_{\PoC2}:$]
    Eventually the client sets the decommitment message to
    \begin{align*}
    & \Res_{\PoC2}=(\share_b, \fC_{1-b}, \{C_i\}, \fD_b, t_{C(1-b)}, t_C, t_{Db}, s_{\pi}, s_{\rho (1-b)}, s_{\rho C}, s_{\rho b}, r_{\Com_{\PoC}}, r_{\Res_{\PoC}}).
    \end{align*}
\end{enumerate}
% and sets $\Com_{\PoC}=(t_{C(1-b)}, t_C, t_{Db})$.
% commits to the password share $\pi$ in $C_{1-b}=g^{\pi_{1-b}}h^{r_{1-b}}$ and $D_b=g^{\pi}h^{r_b}$,
% and sets $\Res_{\PoC,b}=(s_{\pi}, s_{\rho (1-b)}, s_{\rho C}, s_{\rho b})$.

\noindent
$\Res_{\PoC1}$ and $\Res_{\PoC2}$ form together $\Res_{\PoC}$.
The server verifies the proof by checking the following:
\[
  \Com_{\PoC} \verify g^{H(\fC_{1-b}g^{\share_b}, \{C_i\}, \fD_b, t_{C(1-b)}, t_C, t_{Db})} h^{r_{\Com_{\PoC}}} ~~~~
  \Res_{\PoC1} \verify g^{H(s_{\pi}, s_{\rho (1-b)}, s_{\rho C}, s_{\rho b})} h^{r_{\Res_{\PoC}}}
\]
\[
  g^{s_{\pi}}h^{s_{\rho (1-b)}} \verify t_{C(1-b)}(\fC_{1-b}g^{\share_b})^{\Ch_{\PoC,b}} ~~~~
  g^{s_{\pi}}h^{s_{\rho C}} \verify t_{C}(\prod_{i=0}^{n-1} C_i^{b^i})^{\Ch_{\PoC,b}} ~~~~
  g^{s_{\pi}}h^{s_{\rho b}} \verify t_{Db}\fD_{b}^{\Ch_{\PoC,b}}
\]
% \begin{itemize}[leftmargin=*]
%   \item $\Com_{\PoC} \verify g^{H(\fC_{1-b}g^{\share_b}, \{C_i\}, \fD_b, t_{C(1-b)}, t_C, t_{Db})}$ $h^{r_{\Com_{\PoC}}}$
%   \item $\Res_{\PoC1} \verify g^{H(s_{\pi}, s_{\rho (1-b)}, s_{\rho C}, s_{\rho b})} h^{r_{\Res_{\PoC}}}$
%   \item $g^{s_{\pi}}h^{s_{\rho (1-b)}} \verify t_{C(1-b)}(\fC_{1-b}g^{\share_b})^{\Ch_{\PoC,b}}$
%   \item $g^{s_{\pi}}h^{s_{\rho C}} \verify t_{C}(\prod_{i=0}^{n-1} C_i^{b^i})^{\Ch_{\PoC,b}}$
%   \item $g^{s_{\pi}}h^{s_{\rho b}} \verify t_{Db}\fD_{b}^{\Ch_{\PoC,b}}$
% \end{itemize}
% and $g^{s'_{\pi}}h^{s_{\rho 2}} \verify t_{C1}(C_1g^{\share_2})^c$.

\paragraph{Proof of Membership (\PoM)}
\ac{PoM} proves $c_{\phi(i)}\in\omega_{\phi(i)}$ for every character $c_{\phi(i)}\in\pwd$ running over the shuffled set of commitments $\bm C'$, \ie
\begin{equation*}
\ZKP\{\{\pi_i,r_i\}_{i\in[0,n-1]} : ~ C'_{\phi(i)}=g^{\pi_{i}}h^{r_{i}} ~ \wedge ~ \pi_{\phi(i)}\in \omega_{\phi(i)}\}.
\end{equation*}
Note that the proof uses the shuffled commitments $C'_{\phi(i)}$ and not $C_i$ and recall that $c_{\phi(i)}$ belongs to the character set $\omega_{\phi(i)}$ (we sometimes write $\pi_i\in\omega_i$ for $\pi_i\gets\chrint(c_i)$).
This \ac{PoM} is essentially the proof used in \ac{BPR} (cf. Chapter \ref{ch:vpake} Section \ref{sec:bpr}) in committed form.

\begin{enumerate}
  \item %[$\Comm_\PoM$:]
    To prove that every $C'_{\phi(i)}$ commits to a value in the according set $\omega_{\phi(i)}$ the client computes the following values for the first move of the proof:
    \begin{align*}
    \text{--}~ & \forall \pi_j\in\omega_{\phi(i)} \wedge \pi_j\not=\pi_{\phi(i)} :~ s_j\rin\ZZ_p, \Ch_j\rin\ZZ_p \text{ and } t_j=g^{\pi_j}h^{s_j}(C'_{\phi(i)}/g^{\pi_j})^{\Ch_j} \\      
    \text{--}~ & k_{\rho_i}\rin\ZZ_p;\quad t_{l_{\phi(i)}}=g^{\pi_i}h^{k_{\rho_i}}
    \end{align*}
%     \]
    % j\in[1,|\omega_i|] \wedge i\not={l_i}
%     \[
%     \]
    Values $(\bm t_{\phi(i)}, \bm s_{\phi(i)}, \bm c_{\phi(i)}, k_{\rho_i})$, with $\bm t_{\phi(i)}=t_j \cup \{t_{l_{\phi(i)}}\}$, $\bm s_{\phi(i)}=\{s_j\}$, and $\bm c_{\phi(i)}=\{\Ch_j\}$ are stored for future use.
    Note that $t_{l_{\phi(i)}}$ has to be added at the correct position $l_{\phi(i)}$ in $\bm t_{\phi(i)}$.
    A commitment $\Comm_{\PoM}=g^{H(\vect{\omega}, \bm C', \bm t_{\phi(i)})} h^{r_{\Comm_\PoM}}$ with $r_{\Comm_\PoM}\rin\ZZ_p$ is computed as output with $\vect{\omega}=\{\omega_{\phi(i)}\}$.
%     After computing the proof and commitment for every $C'_{\phi(i)}$ the client sets the message $\Comm_\PoM=\{\Comm_{\PoM, i}\}$.

  \item %[$\Ch_\PoM$:]
    The server stores received values, checks them for group membership, and chooses a random challenge $\Ch_\PoM=\Ch\rin\ZZ_p$.

  \item %[$\Res_{\PoM1}$:]
    After receiving the challenge $c$ from the server, the client computes the following verification values for all commitments $C'_{\phi(i)}$ (note that $s_j$ and $\Ch_j$ for all $j\not= l_{\phi(i)}$ are chosen already):
    \begin{align*}      
     & \Ch_{l_{\phi(i)}}=c\oplus \bigoplus_{j=1,j\not=l_{\phi(i)}}^{|\omega_{\phi(i)}|} \Ch_j
     && s_{l_{\phi(i)}}=k_{\rho_{\phi(i)}} - \Ch_{l_{\phi(i)}}(r_{i}+r'_{\phi(i)}),
    \end{align*}
    where $i$ is the index of $C'_{\phi(i)}$ before shuffling.
    The client then combines $\bm s = \bm s_{\phi(i)} \cup \{s_{l_{\phi(i)}}\}$ and $\bm c = \bm c_{\phi(i)} \cup \{\Ch_{l_{\phi(i)}}\}$.
    Note again that the set union has to consider the position of $l_{\phi(i)}$ to add the values at the correct position.
    A commitment $\Res_{\PoM1}=g^{H(\bm s, \bm c)} h^{r_{\Res_\PoM}}$ with $r_{\Res_\PoM}\rin\ZZ_p$ computed is as output.
    % and $\bm s_{\pi}=\{s_{\pi_i}\}$
%     The response message $\Res_\PoM$ is then set to $\{\Res{_\PoM, i}\}$.

  \item %[$\Res_{\PoM2}$:]
    Eventually the client sets the decommitment message with $\bm t = \{\bm t_{\phi(i)} \}$, $\vect{\omega}=\{\omega_{\phi(i)}\}$, $\bm r_{\Comm_\PoM}=\{r_{\Comm_\PoM i}\}$, , $\bm r_{\Res_\PoM}=\{r_{\Res_\PoM i}\}$, and $\bm C'=\{C'_{\phi(i)}\}$ to
    \[\Res_{\PoM2}=(\vect{\omega}, \bm C', \bm t, \bm s, \bm c, \bm r_{\Comm_\PoM}, \bm r_{\Res_\PoM}).\]
\end{enumerate}

\noindent
$\Res_{\PoM1}$ and $\Res_{\PoM2}$ form together $\Res_{\PoM}$.
To verify the proof, \ie to verify that every commitment $C'_{\phi(i)}$ in $\bm C'$ commits to a character $c_i$ from either a subset of $\Sigma$ if significant or $\Sigma$ if not, the server verifies the following for every set $\omega_{\phi(i)} \in \vect{\omega}$ with $i\in[0,n-1]$:
\begin{itemize}
%   \item $\Comm_\PoM \verify g^{H(\vect{\omega}, \bm C', \bm t)} h^{r_{\Comm_\PoM}}$
% 
%   \item $\Res_\PoM \verify g^{H(\bm s, \bm c)} h^{r_{\Res_\PoM}}$

  \item Let $\Ch_j\in\bm c_i$ for $\bm c_i\in\bm c$ and verify
        $\Ch \verify \bigoplus_{j=1}^{|\omega_i|}\Ch_j$

  \item Let $\pi_j\in\omega_{\phi(i)}$, $\bm s_i \in \bm s$, $\bm t_i\in \bm t$, and $\bm c_i \in \bm c$, and verify
        $\displaystyle \bm t_{i}[j] \verify g^{\pi_j}h^{\bm s_i[j]}(C'_i/g^{\pi_j})^{\bm c_i[j]}$
        for all $j\in[1,|\omega_{\phi(i)}|]$
\end{itemize}
The server further verifies commitments 
\[
  \Comm_\PoM \verify g^{H(\vect{\omega}, \bm C', \bm t)} h^{r_{\Comm_\PoM}} \text{ and } \Res_\PoM \verify g^{H(\bm s, \bm c)} h^{r_{\Res_\PoM}}.
\]
The verification of the proof is successful iff all equations above are true \emph{and} \vect{\omega} contains all significant characters for $f_b$.

\paragraph{Proof of Shuffle (\PoS)}
The proof of correct shuffling \ac{PoS} is committed version of the \ac{PoS} used for \ac{BPR} in Chapter \ref{ch:vpake} Section \ref{sec:bpr}.
Note that indices for commitments $C$ and $C'$ run from $1$ to $n$ and index ranges in the following change frequently.

\begin{enumerate}
  \item %[$\Comm_\PoS$:]
    In the first move, the client (prover) builds a permutation matrix and commits to it.
    First he chooses random $A'_j\rin\ZZ_p$ for $j\in[-4,n]$.
    Let $A_{ij}$ denote a matrix with $i\in[-4,n]$ and $j\in[0,n]$, \ie of size $(n+5)\times (n+1)$, such that a $n\times n$ sub-matrix of $A_{ij}$ is the permutation matrix (built from permutation $\phi$).
    Further, let $\phi^{-1}$ be the inverse shuffling function.
    This allows us to write the shuffle as $C'_{i}=\prod_{j=0}^{n}C_{j}^{A_{ji}}=C_{\kappa_i}h^{r'_{\kappa_i}}$ with $C_0=h$ and $\kappa_i=\phi^{-1}(i)$ for $i\in[1,n]$.
    The matrix $A_{ij}$ is defined with $A_{w0}\rin\ZZ_p, A_{-1v}\rin\ZZ_p$ and $A_{0v}=r'_{\phi(v)}$ for $w\in[-4,n]$ and $v\in[1,n]$.
    The remaining values in $A_{ij}$ are computed as follows for $v\in[1,n]$:
    \begin{align*}      
    & A_{-2v}=\sum_{j=1}^{n} 3A_{j0}^2 A_{jv}; && A_{-3v}=\sum_{j=1}^{n} 3A_{j0} A_{jv}; && A_{-4v}=\sum_{j=1}^{n} 2A_{j0} A_{jv}
    \end{align*}

% \begin{figure}[!h]
% % \begin{wrapfigure}[15]{r}{0.5\textwidth}
% % \vspace*{-2em}
% \centering
% \begin{tikzpicture}
%   \matrix [matrix of math nodes, text height=1em, minimum width=11em] (m) { %
%     \hfill  & A_{-4,v}=\sum_{j=1}^{n} 2A_{j0} A_{jv} \\
%     \hfill  & A_{-3,v}=\sum_{j=1}^{n} 3A_{j0} A_{jv} \\
%     \hfill  & A_{-2,v}=\sum_{j=1}^{n} 3A_{j0}^2 A_{jv} \\
%     A_{w0}\rin\ZZ_p & A_{-1,v}\rin\ZZ_p \\
%     \hfill  & A_{0,v}=r'_{\phi(v)} \\
%     \hfill  & \hfill \\
%     \hfill  & A_{ij} := \phi \\
%     \hfill  & \hfill \\
%   };
%   \draw[] (m-1-1.north west) -- (m-1-2.north east) -- (m-8-2.south east) -- (m-8-1.south west) -- (m-1-1.north west);
%   \draw[] (m-1-2.north west) -- (m-1-2.south west) -- (m-1-2.south east);
%   \draw[] (m-2-2.north west) -- (m-2-2.south west) -- (m-2-2.south east);
%   \draw[] (m-3-2.north west) -- (m-3-2.south west) -- (m-3-2.south east);
%   \draw[] (m-4-2.north west) -- (m-4-2.south west) -- (m-4-2.south east);
%   \draw[] (m-5-2.north west) -- (m-5-2.south west) -- (m-5-2.south east);
%   \draw[] (m-6-2.north west) -- (m-8-2.south west);
% \end{tikzpicture}
% \caption{$A_{ij}$}\label{fig:matrix}
% % \end{wrapfigure}
% \end{figure}

    \noindent
    After generating $A_{ij}$ the client commits to it in $(C'_0, \tilde{f}, \bm f', w, \tilde{w})$ for $\bm f'=\{f'_v\}$ with $v\in[0,n]$:
    \begin{align}
    & f'_v=\prod_{j=-4}^{n} f_j^{A_{jv}};~ \tilde{f}=\prod_{j=-4}^{n} f_j^{A'_{j}}; && \tilde{w}=\sum_{j=1}^n A_{j0}^2 - A_{-40} \nonumber\\
    & C'_0=g^{\sum_{j=1}^{n} \pi_j A_{j0}} h^{A_{00}+\sum_{j=1}^{n} r_jA_{j0}}; && w=\sum_{j=1}^n A_{j0}^3-A_{-20}-A'_{-3} \label{eq:pos2}
    \end{align}
    %  \text{ for } v\in[0,n]

    \noindent
    Note that $C'_0=\prod_{j=0}^n C_j^{A_{j0}}=h^{A_{00}}\prod_{j=1}^n C_j^{A_{j0}}$, but Eq. (\ref{eq:pos2}) saves $n-1$ exponentiations.
    The output is then created as
    $\Comm_\PoS=g^{H(\{C_i\}, \{C'_{\phi(i)}\}, C'_0, \tilde{f}, \bm f', w, \tilde{w})} h^{r_{\Comm_\PoS}}$ with $r_{\Comm_\PoS}\rin\ZZ_p$.

\item %[$\Ch_\PoS$:]
    When receiving $\Comm_\PoS$ the server chooses $\bm c = \{\Ch_v\}$ with $\Ch_v\rin\ZZ_p$ for $v\in[1,n]$ and sets $\Ch_\PoS=\bm c$.

\item %[$\Res_{\PoS1}$:]
    After receiving challenges $\bm c$ from the server, the client computes the following verification values $(\bm s, \bm s')$ for $\bm s = \{s_v\}$ and $\bm s' = \{s'_v\}$ with $v\in[-4,n]$ and $\Ch_0=1$:
    \begin{align*}
     & s_v = \sum_{j=0}^{n} A_{vj}\Ch_j; && s'_v = A'_v + \sum_{j=1}^{n} A_{vj}\Ch_j^2
    \end{align*}

    \noindent
    The client sets $\Res_{\PoS1}=g^{H(\bm s, \bm s')} h^{r_{\Res_\PoS}}$ with $r_{\Res_\PoS}\rin\ZZ_p$.

\item %[$\Res_{\PoS2}$:]
    Eventually, the client sends the decommitment message to the server
    \begin{align*}
      & \Res_{\PoS2}=(C'_0, \tilde{f}, \bm f', w, \tilde{w}, \bm s, \bm s', r_{\Comm_\PoS}, r_{\Res_\PoS}).
    \end{align*}
    Note that $\{C_i\}$ and $\{C'_{\phi(i)}\}$ are omitted here as they are part of $\Res_{\PoC2}$, $\Res_{\PoM2}$ respectively, already.
    If this proof is used stand-alone, those values have to be added to $\Res_{\PoS2}$.

\end{enumerate}

\noindent
$\Res_{\PoS1}$ and $\Res_{\PoS2}$ form together $\Res_{\PoS}$.
The server verifies now that the correctness of the commitments
$\Comm_\PoS \verify g^{H(\{C_i\}, \{C'_{\phi(i)}\}, C'_0, \tilde{f}, \bm f', w, \tilde{w})} h^{r_{\Comm_\PoS}}$ and
$\Res_{\PoS1} \verify g^{H(\bm s, \bm s')} h^{r_{\Res_\PoS}}$,
and that the following equations hold for a randomly chosen $\alpha\rin\ZZ_p$ and $C_0=h$:
\begin{align*}
  & \prod_{v=-4}^{n} f_v^{s_v+\alpha s'_v}  \verify  f'_0\tilde{f}^\alpha \prod_{j=1}^n {f'_j}^{\Ch_j+\alpha \Ch_j^2};
  &&  \prod_{v=0}^n C_v^{s_v}  \verify  \prod_{j=0}^n {C'_j}^{\Ch_j} \\
  & \sum_{j=1}^n (s_j^3 - \Ch_j^3)  \verify  s_{-2} + s'_{-3} + w;
  &&  \sum_{j=1}^n (s_j^2 - \Ch_j^2)  \verify  s_{-4} + \tilde{w} 
\end{align*}

\noindent
The server accepts the proof iff all those verifications succeed.
This concludes the proof of correct shuffling.

\paragraph{Share verification}
To verify that the client used the same password \pwd and shares $\share_0,\share_1$ with both servers $\Server_0$ and $\Server_1$, the servers compute the commitment $\fD'_b$ from the share commitment $\fC_b$ and their share $\share_{1-b}$, and exchange it.
Comparing $\fD'_b$ with the value $\fD_b$ received from the client, the server verifies share correctness.
This concludes the \ac{2BPR} protocol and each server $\Server_b$ stores $(\Client,\Server_{1-b},\share_b)$ if all checks were successful.

\subsubsection{Security Analysis}
We show that the proposed \ac{2BPR} protocol is secure using the model from Section \ref{sec:securitymodel} and therefore offers \acl{PC} and \acl{PB}.
Note that \ac{PoM}, \ac{PoC}, and \ac{PoS} are minor modification from the proofs used for the \ac{BPR} protocol in Chapter \ref{ch:vpake} Section \ref{sec:bpr}.
We therefore omit proofs here and only give according lemmata.

\begin{lemma}\label{lem:poc2}
  The \ac{PoC} protocol from Section \ref{sec:protocol} is a concurrent zero-knowledge proof if the discrete logarithm problem in the used group \GG is hard and $H:\bits^\ast\mapsto\ZZ_p$ is a collision resistant hash function.
\end{lemma}

\begin{lemma}\label{lem:pom2}
  The \ac{PoM} protocol from Section \ref{sec:protocol} is a concurrent zero-knowledge proof if the discrete logarithm problem in the used group \GG is hard and $H:\bits^\ast\mapsto\ZZ_p$ is a collision resistant hash function.
\end{lemma}

\begin{lemma}\label{lem:pos2}
  The \ac{PoS} protocol from Section \ref{sec:protocol} is a concurrent zero-knowledge proof of knowledge of shuffling $\phi$ if the discrete logarithm problem in the used group \GG is hard and $H:\bits^\ast\mapsto\ZZ_p$ is a collision resistant hash function.
\end{lemma}

% \noindent
% Due to space limitations, the proof of Lemma \ref{lem:pos} is given in Appendix \ref{app:pos}.

\begin{theorem}\label{theo:pc}
  If \GG is a DL-hard group of prime-order $p$ with generators $g$ and $h$, and $H$ a collision resistant hash function, the construction in Figure \ref{fig:protocol-overview} provides \acl{PC} according to Definition \ref{def:pc}.
\end{theorem}

% \noindent
% Due to space limitations, the proof of Theorem \ref{theo:pc} is given in Appendix \ref{app:pc}.
%
\begin{theorem}\label{theo:zk}
  If \GG is a DL-hard group of prime-order $p$ with generators $g$ and $h$, and $H$ a collision resistant hash function, the construction in Figure \ref{fig:protocol-overview} provides \acl{PB} according to Definition \ref{def:zk}.
\end{theorem}

% \noindent
% Due to space limitations, the proof of Theorem \ref{theo:zk} is given in Appendix \ref{app:zk}.

\noindent
Regarding combination of \ac{PoC} and \ac{PoM} it is worth noting that they do not use common values and can be therefore regarded as independent.
Further, both proofs do not need rewinding as we do not build extractors.

% =================================================================
% PoC Correctness \PoC
% =================================================================

% \begin{proof}[Proof of Lemma \ref{lem:poc}]
% First note that \PoC is actually a proof of knowledge.
% However, it is sufficient that it is a proof of correctness such that we only show that a malicious prover without knowledge of a valid witness has negligible probability of convincing a verifier of the correctness of his statement.
% \emph{Correctness} follows by inspection.
% \emph{Zero-knowledge} follows from the underlying $\Sigma$-protocol and the fact that we can build a simulator with knowledge of the commitment trapdoor without rewinding for any (malicious) verifier.
% In particular, simulator \SIM generates the common reference string \crs such that it knows $\tau$ for $h=g^\tau$, sends commitments $\Comm_\PoC=g^{\alpha}h^{\rho1}$ and $\Res_{\PoC1}=g^{\beta}h^{\rho2}$ with $\alpha,\beta,\rho1,\rho2\rin\ZZ_p$ to the verifier (server), retrieves the challenge $\Ch_\PoC=c$ from the verifier, and generates $\Res_{\PoC1}$ as follows: choose random values
% \begin{align*}  
%   & \share_b, s_\pi, s_{\rho(1-b)}, s_{\rho C}, s_{\rho b}\rin\ZZ_p;~ \fC_{1-b}, \fC_i, \fD_b\rin G~ \forall i\in[0,n-1]
% \end{align*}
% and compute \vspace*{-1em}
% \begin{align*}  
%  t_{C(1-b)}=g^{s_\pi}h^{s_{\rho(1-b)}}(\fC_{1-b}g^{\pi_b})^c;~ t_C=g^{s_\pi}h^{s_{\rho C}} (\prod_{i=0}^{n-1}b^i C_i)^c;~ t_{Db}=g^{s_\pi}h^{s_{\rho b}}\fD_b^c 
% \end{align*}\vspace*{-2em}
% \begin{align*}
%   r_1,r_2 \text{ s.t. } & \alpha+\tau\rho_1\equiv H(C_{1-b}g^{\pi_b}, \{C_i\}, D_b, t_{C(1-b)}, t_C, t_{Db}) + \tau r_1; \\
%   & \beta + \tau\rho_2 \equiv H(s_{\pi}, s_{\rho (1-b)}) + \tau r_2
% \end{align*}
%
% \noindent
% \emph{Soundness} follows from the collision resistance of $H$ and the soundness of the underlying $\Sigma$-proof.
% In particular, a prover without knowledge of the trapdoor $\tau$ has to compute values $(\share_b$, $\fC_{1-b}$, $\{C_i\}$, $\fD_b$, $t_{C(1-b)}$, $t_C$, $t_{Db}$, $s_{\pi}$, $s_{\rho (1-b)}$, $s_{\rho C}$, $s_{\rho b}$, $r_{\Com_{\PoC}}$, $r_{\Res_{\PoC}})$ that verify, which implies that he can either compute the discrete logarithm of $h$, \ie find the trapdoor $\tau$, produces a collision in $H$, or breaks soundness of the $\Sigma$-proof, \ie breaks the binding property of Pedersen commitments.
% In particular, to break the soundness of the underlying $\Sigma$ proof the attacker without knowledge of $(\pi,r_{1-b},r_b,r_{Cb})$ has to be able to generate values such that the \PoC zero-knowledge proof holds for given $\fC_{1-b}g^{\pi_b}$, $\bm C$ and $\fD_b$, which is equivalent to breaking the binding property of Pedersen commitments.
% \end{proof}
%
% % =================================================================
% % PoS Shuffle \PoS
% % =================================================================
%
% \begin{proof}[Proof of Shuffle]
% % This proof follows from the discussions in \cite{JareckiL00,Damgard00}.
% %proof in \cite{KieferM15a} and
% % For completeness we recall the proof here for our construction.
% We have to prove soundness and zero-knowledge of the \PoS protocol, completeness follows from inspection.
% We start with proving the \emph{zero-knowledge} property.
% Note again that the simulator works without rewinding here.
% The simulator first chooses $h=g^\tau$ in the \crs such that he knows the trapdoor $\tau$.
% To this end, we compute $\Comm_\PoS, \Res_{\PoS1}, \Res_{\PoS2}$ such that they are indistinguishable from a view of any verifier, given $(q,g,h,\bm f, \bm C, \bm C')$:
% % C'_0, \tilde{f}, \{f'_j\}, w, \tilde{w}, \{c_j\}, \{s_j\}, and $\{s'_j\}$,
% \begin{align*}
% %   \[   
%     & f'_j, c_j, s_v, s'_v, \alpha,\beta,\rho1,\rho2 \rin \ZZ_p \text{ for } v\in[-4,n], j\in[1,n] \\
% %   \]
% %   \[
%    & \Comm_\PoS=g^{\alpha}h^{\rho1};~~ \Res_{\PoS1}=g^{\beta}h^{\rho2};~~
%     C'_0 = \frac{\prod_{j=0}^n C_j^{s_j}}{\prod_{j=1}^n {C'_j}^{c_j}};~~
%     f'_0 = \frac{\prod_{v=-4}^n f_v^{s_v}}{\prod_{j=1}^n {f'_j}^{c_j}} \\
% %   \]
% %   \[
%    & \tilde{f} = \frac{\prod_{v=-4}^n f_v^{s'_v}}{\prod_{j=1}^n {f'_j}^{c^2_j}};~~
%     w = \sum_{j=1}^n (s_j^3 - c_j^3) - s_{-2} - s'_{-3};~~
%     \tilde{w} = \sum_{j=1}^n (s_j^2 - c_j^2) - s_{-4} \\
% %   \]
% %   \[
%    & r_1,r_2 \text{ s.t. } \alpha+\tau\rho_1\equiv H(\bm C, \bm C', C'_0, \tilde{f}, \bm f', w, \tilde{w}) + \tau r_1;~ \beta + \tau\rho_2 \equiv H(\bm s, \bm s') + \tau r_2 \\
% %   \]
% \end{align*}
%
% \vspace*{-1em}
% \noindent
% Note that \PoS does not send messages $\bm C$ and $\bm C'$ in the last step as defined for committed zero-knowledge proofs.
% This is because they have been chosen in \PoM and \PoS already.
% Considering \PoS alone those two messages have to be chosen by the simulator and made part of $\Res_{\PoS2}$ as well.
%
%
% To prove \emph{soundness} of the \PoS scheme we show how to construct an extractor $E$ that extracts the matrix $A_{vj}$ and $A'_v$ for $v\in[-4,n], j\in[0,n]$.
% Note that this includes extraction of the permutation matrix, \ie $\phi$, and the re-randomisation values $r'_j$, but \emph{not} the content, \ie the characters.
% Further note that the committed execution of the proof does not change how we build the extractor except for the following observations.
% If the prover returns commitments $(\bm C, \bm C')\not=(\hat{\bm C}, \hat{\bm C'})$ after rewinding, we either have a collision in $H$ or can compute $\log_g(h)$.\footnote{While this case is not possible here because the commitments are sent in \PoS and \PoC respectively, this case has to be considered when executing the three proofs in parallel.}
% The same argument applies for the rest of the input to the hash function $(C'_0 \tilde{f}, \bm f', w \tilde{w})$, \ie it either stays the same over all rewindings, or we can compute a collision in $H$ or the discrete logarithm of $h$.
% With this in mind we can specify the extractor as follows.
% First, we see that there exists an extractor $E$ using $n+1$ linearly independent challenge sets $\bm c$ that is able to extract $A_{ij}$ from $s_v=\sum_{j=0}^n A_{vj}c_j$ that fulfils the equation $\prod_{v=-4}^n f_v^{s_v}=\prod_{j=0}^n {f'_j}^{c_j}$, and $A'_v$ from $s'_v=A'_v + \sum_{j=1}^n A_{vj}c_j^2$ that fulfils the equation $\prod_{v=-4}^n f_v^{s'_v}=\prod_{j=1}^n {f'_j}^{c^2_j}$ for $v\in[-4,n]$ and $j\in[0,n]$.
% In this case we also know that the prover either built $s_v$ and $s'_v$ correct, or is able to create (non-trivial) integers $\{\beta_v\}$ for $v\in[-4,n]$ as $\sum_{j=0}^n A_{vj}c_j - s_v$ or $\sum_{j=1}^n A_{vj}c^2_j - s'_v$ such that $\prod_{v=-4}^n f_v^{\beta_v} = 1$, which is impossible under the discrete logarithm assumption.
% Second, equations $\sum_{j=1}^n (s_j^3 - c_j^3)=s_{-2} + s'_{-3} + w$ and $\sum_{j=1}^n (s_j^2 - c_j^2)= s_{-4} + \tilde{w}$ ensure that the $n\times n$ sub-matrix used in $A_{ij}$ is indeed a permutation matrix, using the fact that every $n\times n$ permutation matrix $A_{ab}$ satisfies the following two equations:
% \begin{align*}
%   & \sum_{j=1}^n A_{jc} A_{jd} A_{je} = 1 \text{ if } (c=d=e) \text{ otherwise } 0; \\  
%   & \sum_{j=1}^n A_{jc} A_{jd} = 1 \text{ if } (c=d) \text{ otherwise } 0.
% \end{align*}
% % \[
% % \]
% % \[
% % \]
% Eventually, $\prod_{v=0}^n C_v^{s_v}=\prod_{j=0}^n {C'_j}^{c_j}$ guarantees that, for well-formed $s_v$, the prover knows the used randomness to create the shuffled commitments $C'_j$ for $j\in[1,n]$.
% \end{proof}

% =================================================================
% Policy Compliance
% =================================================================

\begin{proof}[Proof of Theorem \ref{theo:pc}]
To prove \ac{PC} we show how to build an adversary on the soundness properties of the three proofs \ac{PoC}, \ac{PoM}, and \ac{PoS} such that \ac{PC} follows directly from Lemmata \ref{lem:poc2}, \ref{lem:pom2}, \ref{lem:pos2}.
We first show how to build a successful attacker on the soundness of \ac{PoC} and \ac{PoM} using a successful attacker on \ac{PC}.
The \ac{PC}-adversary has access to \Setup and $\Send_\Client$ oracles.

\game{0} This game corresponds to the correct execution of the protocol.

\game{1} In this game we change how $\Send_\Client(\Client, \Server_{b,j}, m)$ queries are answered.
If message $m$ from the adversary is parsed as $(\Comm_\PoM,\Comm_\PoC,\Comm_\PoS)$, $\Comm_\PoM$ is used by the challenger as output to the \ac{PoM} verifier.
This provides the challenger with challenge $\Ch_\PoM$ that is used as reply to the $\Send_\Client$ query (other challenges are generated at random).
If message $m$ from the adversary is parsed as $(\Res_{\PoM1},\Res_{\PoC1},\Res_{\PoS1})$ or $(\Res_{\PoM2},\Res_{\PoC2},\Res_{\PoS2})$ and the first $\Send_\Client$ query from that session was forwarded to the verifier, $\Res_{\PoM1}$, $\Res_{\PoM2}$ respectively, is used as output to the \ac{PoM} verifier.

It is easy to see that the challenger breaks soundness of \ac{PoM} if the adversary uses a password $\pwd\not\in\cD_f$ and \ac{PoM} verifies successfully.
If this is the case we have the desired contradiction.
It is therefore safe to assume for the following experiments that $\pwd\in\cD_f$.

\game{2}
In this game we change again how $\Send_\Client(\Client, \Server_{b,j}, m)$ queries are answered.
If message $m$ from the adversary is parsed as $(\Comm_\PoM,\Comm_\PoC,\Comm_\PoS)$, $\Comm_\PoC$ is used by the challenger as output to the \ac{PoC} verifier.
This provides the challenger with challenge $\Ch_\PoC$ that is used as reply to the $\Send_\Client$ query (other challenges are generated at random).
If message $m$ from the adversary is parsed as $(\Res_{\PoM1},\Res_{\PoC1},\Res_{\PoS1})$ or $(\Res_{\PoM2},\Res_{\PoC2},\Res_{\PoS2})$ and the first $\Send_\Client$ query from that session was forwarded to the verifier, $\Res_{\PoC1}$, $\Res_{\PoC2}$ respectively, is used as output to the \ac{PoC} verifier.

It is easy to see that the challenger breaks soundness of \ac{PoC} if $\share_0+\share_1\not=\pi$, \ie the password share $\share_b$ can not be used with a second share $\share_{1-b}$ to rebuild the password $\pi$ committed to in $\bm C$, \ie $\sum_i b^i\pi_i\not=\pi$.
We further see that the second share $\share_{1-b}$ has to be stored on server $\Server_{1-b}$, \ie the attacker has not performed the set-up with $\Server_b$ and $\Server_{1-b}$ with shares that do not combine to the same encoded password $\pi$.
Otherwise we can break the binding property of Pedersen commitments.
In particular, the attacker has to generate commitments $\fC_0, \fC_1, \fD_0$ and $\fD_{1}$ such that $\fC_0g^{\share_1}=\fD_0$ or $\fC_1g^{\share_0}=\fD_1$.
We can therefore safely assume that the password share $\share_b$ received by server $\Server_b$ can be combined with the second share $\share_{1-b}$ of server $\Server_{1-b}$ to an encoded password $\pi$ with according character commitments $C_i$.


\game{3}
In this game we change how $\Send_\Client(\Client_i, \Server_{b,j}, m)$ queries are answered.
If message $m$ from the adversary is parsed as $(\Comm_\PoM,\Comm_\PoC,\Comm_\PoS)$, $\Comm_\PoS$ is used by the challenger as output to the \ac{PoS} verifier.
This provides the challenger with challenge $\Ch_\PoS$ that is used as reply to the $\Send_\Client$ query (other challenges are generated at random).
If message $m$ from the adversary is parsed as $(\Res_{\PoM1},\Res_{\PoC1},\Res_{\PoS1})$ or $(\Res_{\PoM2},\Res_{\PoC2},\Res_{\PoS2})$ and the first $\Send_\Client$ query from that session was forwarded to the verifier, $\Res_{\PoS1}$, $\Res_{\PoS2}$ respectively, is used as output to the \ac{PoS} verifier.

It is easy to see that if the attacker is rewindable, the challenger can act as a knowledge extractor for \ac{PoS}.
In particular, we can extract shuffling function $\phi$ and re-randomiser $\{r'_i\}$ to break soundness of \ac{PoS}.
It is therefore safe to assume that $\bm C'$ is a shuffle of $\bm C$.
This concludes the proof of Theorem \ref{theo:pc} by observing that the password shares stored on both servers can be combined to a policy compliant password.
\end{proof}


% =================================================================
% Blindness
% =================================================================

\begin{proof}[Proof of Theorem \ref{theo:zk}]
We give a sequence of games that each changes the way an oracle is computed.
In the last game all interaction of a server with a client is simulated and therefore password independent such that an attacker can only guess which client used which password.

\game{0}
This is the correct execution of the protocol.

\game{1}
The challenger computes \crs such that it knows trapdoor $\tau=\log_g(h)$.
This does not change anything else.

\game{2}
The challenger changes the way \Execute oracles are answered if at least one of the participating servers is not corrupted.
In case both servers are corrupted the attacker retrieves the password anyway and can not win the game.
Instead of correctly executing the protocol all messages from zero-knowledge proofs are simulated and messages between the servers chosen accordingly.
However, two correct password shares are stored on the two servers.
This allows future corruption of the servers without noticing this change.
Note that as long as the attacker did not corrupt any of the participating servers, \Execute does not provide any information to the attacker since the communication is encrypted between the client and each server as well as between the two servers.
In this case the change from Game$_1$ to Game$_2$ is not noticeable.
In case the attacker corrupted one of the participating servers, he is able to decrypt the messages and therefore gets the communication between the client and the corrupted server.
While the attacker receives the protocol transcript now, he is not able to distinguish it from a correct execution unless we can build a distinguisher for one of the three zero-knowledge proofs, which is covered by Lemmata \ref{lem:poc2}, \ref{lem:pom2}, and \ref{lem:pos2}.

\game{3}
In this game the challenger changes the way $\Send_{S}$ and $\Send_{\Server\Server}$ are answered if the second participating server is not corrupted.
Recall that $\Send_\Server$ queries are only accepted by the client if \Server has been corrupted before and provided the attacker with necessary keys.
Instead of computing answers to $\Send_\Server$ correctly the challenger simulates the zero-knowledge proofs (Lemmata \ref{lem:poc2}, \ref{lem:pom2}, \ref{lem:pos2}).
We further change how $\Send_{\Server\Server}$ is answered by sending $\fD'_b=\fD_b$ instead of computing it correctly.
To allow future corruption of the honest server without noticing this change we save an appropriate share.
This change to the simulation is not noticeable by an attacker unless we can build a distinguisher against one of the three zero-knowledge proofs.
In this last game all proofs are simulated such that an attacker can only guess which password is used by which client.
\end{proof}


% =================================================================
% Implementation & Performance
% =================================================================
\subsection{Performance Discussion}
Considering similarities between \ac{2BPR} and \ac{BPR}, performance of the proposed \ac{2BPR} protocol is very similar to the performance of the \ac{BPR} protocol from Section \ref{sec:bpr} Chapter \ref{ch:vpake}.
We therefore refrain from implementing this protocol and refer to the performance discussion on \ac{BPR} (Section \ref{sec:performance} Chapter \ref{ch:vpake}) for details of the \ac{BPR} protocol, which reflects the time between the client and a single server in the \ac{2BPR} case.
The additional time needed for committed zero-knowledge proofs is negligible compared to the time needed for \ac{PoM}.

% \mynote{redo when implementation is done}
%
% % \subsubsection{Implementation}
% We implement an unoptimised prototype of the \ac{2BPR} protocol from Section \ref{sec:framework} over the NIST P-192 elliptic curve \cite{nist} in Python using the Charm framework \cite{charm13} to estimate performance of the proposed protocol.
% % In order to be able to compare this implementation with the approach from \cite{KieferM14c}, we set $b=10^5$.
% % Note however that the performance influence of $b$ is negligible.
% The performance tests (completed on a laptop with an Intel Core Duo P8600 at 2.40GHz) underline the claim that the proposed is practical.
% In particular, execution of the \ac{2BPR} protocol with a password of length $10$ and policies $(dl, 5)$ and $(ds, 7)$ needs $1.4$ seconds on the client and $0.68$ seconds on each server.
% With $2.76$ seconds overall runtime for a password of length $10$, $4.59$ for a password of length $15$ and $6.34$ for a password of length $20$ the proposed \ac{2BPR} protocol is deemed practical.
% Also note that the client execution can be parallelised performing the proofs with $\Server_0$ and $\Server_1$ at the same time, which allows to significantly decrease the execution time.
% The source code is available from \url{https://franziskuskiefer.de/data/2bpr.zip}.
%

\subsection{Application to 2PAKE/PPSS protocols}
The \ac{2BPR} protocol can be used to register passwords for two-server protocols such as \ac{2PAKE} and two-server \ac{PPSS}.
\ac{2BPR} can be used with two-server protocols that adopt additive password sharing in $\ZZ_p$ or multiplicative sharing in \GG.
This includes the \ac{2PAKE} protocols by \citet{Katz2012a}, which does not consider password registration such that our protocol can simply be used as part of the registration process.
This equally applies to the \ac{2PAKE} protocols proposed in Sections \ref{sec:twoserverpake} and \ref{sec:uc2pake} of this chapter.
Integration of \ac{2BPR} into \ac{PPSS} on the other hand is more involved as password registration is part of the \ac{PPSS} protocol, \ie the secret sharing phase.
Two-server \ac{PPSS} protocols in general can be divided in two stages: password and secret registration/sharing and secret reconstruction.
While the approach from \citet{Bagherzandi2011}, as well as subsequent work using similar approaches \cite{PryvalovK14,CamenischLLN14,JareckiKK14}, does not actually share the password and could therefore use other means to verify policy compliance of a prospective password, the \ac{UC}-secure two-server \ac{PPSS} protocol from \citet{Camenisch2012} uses multiplicative password sharing in \GG.
To use our \ac{2BPR} protocol in conjunction with the setup protocol from \citet{Camenisch2012} we redefine the encoded password to $g^{\pi}$ with $\pi\gets\pwdint(\pwd)$ such that shares are computed as $g^{\pi}=g^{\share_0}g^{\share_1}$.
The first message (step 1) from the setup protocol in \cite[Figure 4]{Camenisch2012} can piggyback the first \ac{2BPR} protocol message.
The subsequent three messages between the client and each server are performed between step 1 and step 2, while the inter-server communication can be piggybacked on step 2 and step 3.
In addition to checking correctness of shares done in the setup of \cite{Camenisch2012} the servers can now verify the \ac{2BPR} proofs and thus the password's policy compliance.
This adds three flows to the setup protocol of \cite{Camenisch2012} in order verify policy compliance of password shares.



%********************************** %2PAKE  **************************************
% \section{Two-Server PAKE} \label{sec:2pake-uc}

\section{Distributed Smooth Projective Hashing} \label{sec:dsphf-2pake}

Smooth projective hashing allows to compute the hash value of an element from a set in two different ways:
either by using a secret hashing key on the element, or utilising the public projection key and some secret information proving that the particular element is part of a specific subset under consideration.
In addition, smooth projective hash values guarantee to be uniformly distributed in their domain as long as the input element is not from a specific subset of the input set.
This allowed \citet{Canetti2005} to build the first \ac{UC} secure \ac{PAKE} protocol.
All subsequently proposed \ac{UC} secure \ac{PAKE} protocols follow a similar approach and use \acp{SPHF} to ensure \ac{UC} security.
In this section we extend \acp{SPHF} to work in the three-party setting and thus enable us to build \ac{UC} secure \ac{PAKE} protocols.
% These features make them a popular building block in many protocols such as \ac{CCA}-secure public key encryption, blind signatures, password authenticated key exchange, oblivious transfer, zero-knowledge proofs, commitments and verifiable encryption.

% Smooth projective hash functions (SPHF) are due to Cramer and Shoup \cite{Cramer2002} who used them to construct \ac{CCA}-secure public key encryption schemes and analyse mechanisms from \cite{Cramer_Shoup_1998}.
% The first use of SPHFs in the construction of a password authenticated key exchange (PAKE) protocol is due to Gennaro and Lindell \cite{Gennaro2003}, who introduced additional requirements to the SPHF such as pseudorandomness that was later extended in \cite{Katz2011}.
% The SPHF-based approach taken in \cite{Gennaro2003} was further helpful in the ``explanation'' of the KOY protocol from \cite{Katz_Ostrovsky_Yung_2001}, where those functions were implicitly applied.
%
% Abdalla et al. \cite{Abdalla2009} introduced conjunction and disjunction of languages for smooth projective hashing that were later used in the construction of blind signatures \cite{Blazy2012,cryptoeprint:2013:034}, oblivious signature-based envelopes \cite{Blazy2012}, and authenticated key exchange protocols for algebraic languages \cite{Hamouda2013}.
% Blazy et al. \cite{Blazy2012} demonstrate more general use of smooth projective hashing in designing round-optimal privacy-preserving interactive protocols.

In particular, this section extends research on \aclp{SPHF} by considering divergent parameterised languages in a single smooth projective hash function that allows multiple parties to jointly evaluate the result of the function. 
To this end we propose the notion of \ac{D-SPHF} that enables joint hash computation for special languages, which is then used to construct a new \ac{2PAKE} framework and show how it helps to explain the protocol from \citet{Katz2012a} in the following section.
% Actually, the authors of \cite{AbdallaP06} already built a group PAKE protocol using smooth projective hashing in a multi-party party protocol.
% However, they assume a ring structure such that the smooth projective hashing is only used between two parties.


%===============================================================================
\subsection{Extending Smooth Projective Hashing}\label{sec:sphff}
%===============================================================================
We introduce an extended notion of smooth projective hashing that allows us to distribute the computation of the hash value.
The new notion of extended \ac{SPHF} is defined in the following setting:
We consider a language $\Laux$ with words (ciphertexts) $C$ that are ordered sets of $n=x+1$ ciphertexts $(C_0,\dots,C_x)$.
Parameter \aux, a language is indexed with, allows us to easily describe languages that differ only in the secret part $\aux'$.
The secret variable information $\aux'$ is chosen from the additive group $(\PP,+)=(\ZZ_p^{+},+)$ with a (sharing) function $h:\PP\mapsto\PP^x$.
Let $L^\cL_\aux$ denote the language of ciphertexts encrypting the secret part $\aux'$ from $\aux$ with public key $\pk$ from $\aux$ using encryption scheme $\cL$.
For all $C_i, i\in\{1,\dots,x\}$ it must hold that $C_i\in L^\cL_{\aux_i}$ where $\aux_i=(\pk,\aux'_i)$ with $\aux'_i=h(\aux')[i]$.
For $C_0$ it must hold that $C_0\in L^\cL_{\aux}$.
Furthermore, the ciphertexts must offer certain homomorphic properties such that there exists a modified decryption algorithm $\Dec'$ and a combining function $g$ such that $\Dec'_{\pi}(C_0)=\Dec'_{\pi}(g(C_1,\dots,C_x))$, where $\pi$ denotes the secret key for the corresponding public key $\pk$ from \crs.

The idea of extended \ac{SPHF} is to be able to use the \ac{SPHF} functionality not only on a single ciphertext, but on a set of ciphertexts with specific properties, \ie over encrypted shares of $\aux'$.
Due to the nature of the words considered in extended \ac{SPHF} they produce two different hash values.
One can think of the two hash values as $h_0$ for $C_0$ and  $h_x$ for $C_1,\dots,C_x$.
The hash value $h_0$ can be either computed with knowledge of the hash key $\hk_0$ or with the witnesses $w_1,\dots,w_x$ that $C_1,\dots,C_x$ are in $L^\cL_{\aux_i}$ each.
The hash value $h_x$ can be computed with knowledge of the hash keys $\hk_1,\dots,\hk_x$ or with the witness $w_0$ that $C_0$ is in $L^\cL_{\aux}$.

\begin{definition}[Extended SPHF]\label{def:symgensphf}
Let $L_\aux$ denote a language such that $C=(C_0,C_1,\dots,\allowbreak C_x)\in L_\aux$ if there exists a witness $w=(w_0,w_1,\dots,w_x)$ proving so and there exist functions $h(\aux')=(\aux'_1,\dots,\aux'_x)$ and $g:\GG^l\mapsto\GG^{l'}$ as described above.
An extended smooth projective hash function for language $L_\aux$ with $\Gamma\in\GG^{k\times n}$ consists of the following six algorithms:

\begin{itemize}
	\item $\HKGen(\Laux)$ generates a hashing key $\hk_i\in\ZZ_p^{1\times n}$ for $i\in\{0,\dots,x\}$ and language $\Laux$.
	
	\item $\PKGen(\hk_i,\Laux)$ derives the projection key $\hp_i=\Gamma \odot \hk_i\in\GG^{1\times k}$ for $i\in\{0,\dots,x\}$.
	
	\item $\Hash_x(\hk_0,\Laux,C_1,\dots,C_x)$ outputs hash value
	\[h_x=\Theta^x_{\aux}(C_1,\dots,C_x)\odot\hk_0.\]
	
	\item $\ProjHash_x(\hp_0,\Laux,C_1,\dots,C_x,w_1,\dots,w_x)$ returns hash value
	\[h_x=\prod^{x}_{i=1}(\lambda^i\odot \hp_0), \text{ where } \lambda^i=\Omega(w_i,C_i).\]
	
	\item $\Hash_0(\hk_1,\dots,\hk_x,\Laux,C_0)$ outputs hash value
	\[h_0=\prod^{x}_{i=1}(\Theta_{\aux}^0(C_0)\odot\hk_i)=\Theta_{\aux}^0(C_0)\odot \sum^x_{i=1}\hk_i.\]
	
	\item $\ProjHash_0(\hp_1,\dots,\hp_x,\Laux,C_0,w_0)$ returns hash value
	\[h_0=\prod^{x}_{i=1}(\lambda^0\odot \hp_i), \text{ with } \lambda^0=\Omega(w_0,C_0).\] \eod
\end{itemize}
\end{definition}

\noindent
The correctness of the scheme can be easily verified by checking that $\Hash_x=\ProjHash_x$ and $\Hash_0=\ProjHash_0$.

\subsubsection{Security Analysis}
We refine definitions of smoothness and pseudorandomness to account for the two different hash functions.
Therefore, we add both hash values to the indistinguishable sets, as well as the vector of projection keys.
We start with smoothness of extended \ac{SPHF}.
The smoothness proven in Theorem \ref{theo:smoothnessc} follows directly from the proof given in \cite[Appendix D.3]{cryptoeprint:2013:034} and follows the same approach for smoothness proofs as in previous works on \ac{SPHF} \cite{cryptoeprint:2013:034,Gennaro2003,Katz2011}.
Recall that we are only concerned with \emph{adaptive smoothness}.
Let $\overline{\hp}$ denote the vector of projection keys $\hp_i$ for $i=0,\dots,x$.
For any functions $f,f'$ to $\Omega\setminus\Laux$ (for output space $\Omega$ of $f,f'$) the following distributions are statistically $\varepsilon$-close:
\begin{align*}
& \{(\overline{\hp},h_0,h_x) ~|~ h_0\algout\Hash_0(\hk_1,\dots,\hk_x,\Laux,f(\hp_0));~ h_x\algout\Hash_x(\hk_0,\Laux, \\
& f'(\hp_1,\dots,\hp_x)); \forall i\in\{0,\dots,x\}:~ \hk_i\ralgout\HKGen(\Laux); \hp_i\algout\PKGen(\hk_i,\Laux)\} \\
\stackrel{\varepsilon}{=} & \{(\overline{\hp},h_0,h_x) ~|~ h_0\rin\GG;~ h_x\rin\GG; \forall i\in\{0,\dots,x\}:~ \hk_i\ralgout\HKGen(\Laux); \\
& ~~\hp_i\algout\PKGen(\hk_i,\Laux)\}.
\end{align*}

\begin{theorem}[Extended SPHF Smoothness]\label{theo:smoothnessc}
The extended \ac{SPHF} construction from Definition \ref{def:symgensphf} on cyclic groups is statistically smooth.
\end{theorem}

\begin{proof}
We show that the logarithm of the projection keys $\overline{\hp}$ and the logarithm of the hash values $h_0$ and $h_x$ are defined by linearly independent equations and thus $h_0$ and $h_x$ are uniform in $\GG$, given $\overline{\hp}$.
In addition to this general proof we give an extended proof of extended \ac{SPHF} smoothness instantiated with labelled Cramer-Shoup encryption for better understanding in Section \ref{sec:excssmoothness}.
To show that $(\overline{\hp},h_0,h_x)$ is uniformly distributed in $\GG^{k+2}$ for $C\not\in\Laux$, \ie $\varepsilon$-close to $(\overline{\hp},g_0,g_x)$ for random $g_0,g_x\rin\GG$, we consider a word $C=(C_0,C_1,\dots,C_x)\not\in \Laux$ and a projection key $\hp_j=\Gamma \odot \hk_j$ such that one $C_j$ does not fulfil the property $C_j\in L_{\aux_j}$, \ie $\exists j\in\{0,\dots,x\},\forall\lambda^j\in\ZZ_p^{1\times k}:~\Theta_{\aux_j}(C_j)\not=\bm{\lambda}^j\odot \Gamma$.
From \cite[Appendix D.3]{cryptoeprint:2013:034} it follows directly that $\Theta_{\aux_j}(C_j)\odot\hk_j$ is a uniformly distributed element in $\GG$, and thus $\Theta^x_{\aux}(C_1,\dots,C_x)\odot\hk_0$ and $\prod^{x}_{i=1}(\Theta_{\aux}^0(C_0)\odot\hk_i)$ are uniformly distributed in $\GG$.
The projection key $\overline{\hp}$ is uniformly at random in $\GG^{k}$ anyway, given the randomness of all $\hk_i$.
Note that any violation of $\Dec'_{\pi}(C_0)=\Dec'_{\pi}(g(C_1,\dots,C_x))$ implies the existence of an index $j$ such that $C_j\not\in L_{\aux_j}$.
\end{proof}

\noindent
While smoothness is the foremost property of (extended) smooth projective hash functions, cases like \ac{PAKE} require pseudorandomness of the produced hash values.
Let $\overline{\hp}$ denote the vector of projection keys $\hp_i$ for $i=0,\dots,x$.
An extended \ac{SPHF} is pseudorandom if its hash values are computationally indistinguishable from random without knowledge of the uniformly chosen hash keys $\overline{\hk}$ or the witnesses $\overline{w}$, \ie for all $C=(C_0,\dots,C_x)\in\Laux$ the following distributions are computationally $\varepsilon$-close:

\begin{align*}
& \{(\overline{\hp},C,h_0,h_x) ~|~ \forall i\in\{0,\dots,x\}:~ \hk_i\ralgout\HKGen(\Laux); \hp_i\algout\PKGen(\hk_i,\Laux); \\
& ~~h_0\algout\Hash_0(\hk_1,\dots,\hk_x,\Laux,C_0);~ h_x\algout\Hash_x(\hk_0,\Laux,C_1,\dots,C_x)\} \\
\stackrel{\varepsilon}{=}~ & \{(\overline{\hp},C,h_0,h_x) ~|~ \forall i\in\{0,\dots,x\}:~ \hk_i\ralgout\HKGen(\Laux); \hp_i\algout\PKGen(\hk_i,\Laux); \\
& ~~ h_0\rin\GG; h_x\rin\GG\}
\end{align*}

\noindent
To prove pseudorandomness of an extended \ac{SPHF} we use modified experiments from \cite{Gennaro2003} given in Definition \ref{def:prplus}.

\begin{definition}[Extended SPHF Pseudorandomness]\label{def:prplus}
An extended \ac{SPHF} $\Pi$ is pseudorandom if for all \ac{PPT} algorithms $\cA$ there exists a negligible function $\varepsilon(\cdot)$ such that
\[\Adv_{\Pi,\cA}^{\Pr}=\left|\Pr[\Exp_{\Pi,\cA}^{\Pr}=1] - \frac12 \right|\leq \varepsilon(\secpar)\]

\noindent
$\Exp_{\Pi,\cA}^{\Pr}(\secpar):$ \\
\hspace*{2em} $b\rin\bits$\\
\hspace*{2em} $\hp_i\gets\PKGen(\hk_i,\Laux,C_i)$ and $\hk_i\gets\HKGen(\Laux)$ for all $i\in 0,\dots,x$\\
\hspace*{2em} $b'\gets\cA^{\Omega^{\cL}_\pk(\cdot)}(\secpar,\hp_0,\dots,\hp_x)$ \\
\hspace*{2em} return $b=b'$

\begin{description}
	\item $\Omega^{\cL}_\pk(\ell,\aux)$ returns elements $C=(C_0,\dots,C_x)\in\Laux$ with $C_0\gets\Enc^\cL_\pk(\ell_0,\allowbreak\aux';r_0)$ and $C_i\gets\Enc^\cL_\pk\allowbreak(\ell_i,\aux'_i;r_i)$ for all $i\in1,\dots,x$ and $\pk\in\aux$ using encryption scheme \cL and according labels $\ell_i$.
	It additionally returns $\Hash_0(\hk_1,\dots,\allowbreak\hk_x,\Laux,C_0),\Hash_x(\hk_0,\allowbreak \Laux,C_1,\dots,C_x)$ if $b=0$ or $h_0,h_x$ $\rin\GG$ if $b=1$. \eod
\end{description}
\end{definition}

\noindent
The following theorem shows pseudorandomness of hash values in extended \acp{SPHF}.

\begin{theorem}[Extended SPHF Pseudorandomness]\label{theo:prsphff}
The extended \ac{SPHF} construction from Definition \ref{def:symgensphf} on cyclic groups is pseudorandom if $\cL$ is a \ac{CCA}-secure labelled encryption scheme.
\end{theorem}

\begin{proof}
Pseudorandomness of extended \ac{SPHF} follows immediately from its smoothness and the \ac{CCA}-security of the used encryption scheme.
First we change $\Omega^\cL_\pk$ such that it returns the encryption of $0$ for a random $i\in0,\dots,x$.
This change is not noticeable by the adversary due to the \ac{CCA}-security of the encryption scheme.
Assuming $0$ is not a valid message, \ie $\aux'\not=0$ and $\aux_i\not=0$ for all $i\in1,\dots,x$, the pseudorandomness of extended \ac{SPHF} follows from its smoothness.
\end{proof}

\noindent
\citet{Katz2011} highlight that this definition of pseudorandomness is not enough when used in PAKE protocols if the hash values are not bound to a specific session by signatures or \acp{MAC}.
Therefore, they prove pseudorandomness under re-use of hash keys and ciphertexts.
Taking into account re-use of \SPHFF values such as ciphertexts and keys we formalise the notion of concurrent pseudorandomness for extended \ac{SPHF} following the approach from \citet{Katz2011}.
Let $\overline{\hp}$ denote the vector of projection keys $\hp_i$ for $i=0,\dots,x$.
An extended \ac{SPHF} is pseudorandom in concurrent execution if the hash values are computationally indistinguishable from random without knowledge of the uniformly chosen hash keys or the witnesses, \ie for fixed $l=l(\secpar)$ the following distributions are computationally $\varepsilon$-close:
\begin{align*}
& \{(\overline{\hp}_1,\dots,\overline{\hp}_l,C_1,\dots,C_l,h_{0,1},\dots, h_{0,l},h_{x,1},\dots, h_{x,l}) ~|~ \\
&  \forall i\in\{0,\dots, x\}, j\in\{1,\dots,l\}: \hk_{i,j}\ralgout\HKGen(\Laux);~ \hp_{i,j}\algout\PKGen(\hk_i,\Laux); \\
&  \forall j\in\{1,\dots,l\}: h_{0,j}\algout\Hash_0(\hk_{1,j},\dots,\hk_{x,j},\Laux,C_{0,j}); \\
&  h_{x,j}\algout\Hash_x(\hk_{0,j}, \Laux,C_{1,j},\dots,C_{x,j})\} \\
\stackrel{\varepsilon}{=}~ & \{(\overline{\hp}_1,\dots,\overline{\hp}_l,C_1,\dots,C_l,h_{0,1},\dots, h_{0,l},h_{x,1},\dots, h_{x,l}) ~|~ \\
&  \forall i\in\{0,\dots, x\}, j\in\{1,\dots,l\}: \hk_{i,j}\ralgout\HKGen(\Laux); \hp_{i,j}\algout\PKGen(\hk_i,\Laux); \\
&  \forall j\in\{1,\dots,l\}: h_{0,j}\rin\GG; h_{x,j}\rin\GG\}
\end{align*}

\noindent
We extend Definition \ref{def:prplus} to capture re-use of hash keys and ciphertexts.
The corresponding experiment in Definition \ref{def:prplusc}
generates $l$ hash values to each ciphertext, one for each hash key.

\begin{definition}[Extended SPHF Concurrent Pseudorandomness]\label{def:prplusc}
An extended \ac{SPHF} $\Pi$ offers concurrent pseudorandomness if for all \ac{PPT} algorithms $\cA$ and polynomials $l$ there exists a negligible function $\varepsilon(\cdot)$ such that
\[\Adv_{\Pi,\cA}^{\Pr}=\left|\Pr[\Exp_{\Pi,\cA}^{\Pr}=1] - \frac12 \right|\leq \varepsilon(\secpar)\]

\noindent
$\Exp_{\Pi,\cA}^{\Pr}(\secpar):$ \\
\hspace*{2em} $b\rin\bits$, $\overline{\hp_j}=(\hp_0,\dots,\hp_x)$ with \\
\hspace*{2em} $\hp_i\gets\PKGen(\hk_i,\Laux,C_i)$ and $\hk_i\gets\HKGen(\Laux)$ \\
\hspace*{4em} for all $i\in 0,\dots,x$ and $j\in 1,\dots,l$ \\
\hspace*{2em} $b'\gets\cA^{\Omega^{\cL}_\pk(\cdot)}(\secpar,\overline{\hp_1},\dots,\overline{\hp_l})$ \\
\hspace*{2em} return $b=b'$

\begin{description}
	\item[$\Omega^{\cL}_\pk(\ell,\aux)$] returns elements $C=(C_0,\dots,C_x)\in\Laux$ with $C_0\gets\Enc^\cL_\pk(\ell_0,\allowbreak\aux';r_0)$ and $C_i\gets\Enc^\cL_\pk\allowbreak(\ell_i,\aux_i;r_i)$ for all $i\in1,\dots,x$ and $\pk\in\aux$ using encryption algorithm \cL and according labels $\ell_i$.
	It additionally returns $\Hash_{0,j}(\hk_{1,j},\dots,\hk_{x,j},\Laux,$ $C_0)$, $\Hash_{x,j}(\hk_{0,j},\allowbreak \Laux,C_1,\dots,C_x)$ if $b=0$ or $h_{0,j},\allowbreak h_{x,j}\in\GG$ if $b=1$ for all $j\in 1,\dots, l$. \eod
\end{description}
\end{definition}

\noindent
The following theorem give concurrent pseudorandomness of the proposed construction for extended \acp{SPHF}.

\begin{lemma}[Concurrent Pseudorandomness of Extended SPHF]\label{cor:pr}
The extended \ac{SPHF} construction from Definition \ref{def:symgensphf} on cyclic groups is pseudorandom on re-use of hash and ciphertext values if $\cL$ is a \ac{CCA}-secure labelled encryption scheme.
\end{lemma}

\begin{proof}
Using a hybrid argument it is enough to show that the adversary can not distinguish between experiment $\Exp_1$ where $\Omega$ returns random elements for the first $i$ hash values of the $j$-th query and all queries $<j$ and correct hashes for all subsequent queries and indices $>i$, and $\Exp_2$ where $\Omega$ returns random elements for the first $i+1$ hash values of the $j$-th query and all queries $<j$ and correct hashes for all subsequent queries and indices $>i+1$.
Having this in mind the proof follows the same argument as the one for extended \ac{SPHF} pseudorandomness.
We briefly recall the argument there.
We modify $\Exp_1$ to $\Exp'_1$ and $\Exp_2$ to $\Exp'_2$ such that $\Omega$ returns an encryption of $0$ instead of correct encryptions for $C_j$.
Note that we assume $0$ is not a valid message such that $C_j\not\in\Laux$ in $\Exp'_1$.
Due to \ac{CCA}-security of \cL this step is not observable by the adversary.
Changing $\Exp'_1$ to $\Exp'_2$ the smoothness of extended \ac{SPHF} ensures that $\cA$ can not distinguish between the two experiments, which proves the lemma.
\end{proof}

\subsection{Distributed Smooth Projective Hashing}\label{sec:dsphf}
Using extended \acp{SPHF} only makes sense in a distributed setting.
We therefore consider $n=x+1$ entities participating in the distributed computation of extended \ac{SPHF} hash values $h_0,h_x$.
Let $P_i$ for $i\in\{1,\dots,x\}$ denote parties, each knowing $\aux_i$ and computing the according ciphertext $C_i$ and projection key $\hp_i$.
Furthermore, let $P_0$ denote the participant knowing $\aux$ and computing $C_0$ and $\hp_0$.
We define protocols in this setting with the purpose that both $P_0$ and $P_1$ can eventually compute $h_0$ and $h_x$.

While $P_0$ can compute $\ProjHash_0$ and $\Hash_x$ after receiving all $C_i$ and $\hp_i$, computation of $\Hash_0$ and $\ProjHash_x$ can not be performed solely by the previously described algorithms in this setting, without disclosing witnesses or the hashing keys.
To compute $\ProjHash_x$ and $\Hash_0$, parties $P_1,\dots,P_x$ have to collaborate since they know only part of the input parameters.
\ac{D-SPHF} defines protocols that allow secure calculation of $h_0$ and $h_x$.
Intuitively \ac{D-SPHF} reaches the same security properties as extended \ac{SPHF}, namely smoothness and pseudorandomness in presence of a passive adversary, by additionally ensuring that no protocol participant alone is able to compute the hash values.
Note that while we assume each $P_i$ for $i>0$ holds a key-pair and knows public keys of all other $P_i$ such that all communication between two $P_i$ is secured by the receivers public key, those keys are not authenticated, \ie we do not assume a \ac{PKI}, neither does $P_0$ require a key.

A \ac{D-SPHF} protocol between $n$ participants $P_0,\dots,P_x$ computing $h_x$ and $h_0$ consists of three interactive protocols \Setup, $\ProjHash_x^D$ and $\Hash_0^D$.
Let $\Pi$ denote the extended \ac{SPHF} algorithm that is being distributed.
\begin{itemize}
	\item $\Setup(\aux,P_0,\dots,P_x)$ initialises a new instance for each participant with $(\aux$, $P_0$, $P_1$, $\dots$, $P_x)$ for $P_0$ and $(\aux_i,P_i,P_0,\dots,P_x)$ for $P_i$, $i\in\{1,\dots,x\}$.
	Eventually, all participants compute and broadcast projection keys $\hp_i$ and encryptions $C_i\gets\Enc_{\pk}^{\cL}(\ell_i,\aux'_i;r_i)$ of their secret $\aux'_i$ using $\Pi.\HKGen$, $\Pi.\PKGen$ and the associated encryption scheme $\cL$.
	Participants store incoming $(\hp_i,C_i)$ for later use.
	After receiving $(\hp_1,C_1,\dots,\hp_x,C_x)$, $P_0$ computes $h_0\gets\Pi.\ProjHash_0(\hp_1,\dots,\hp_x,\Laux,C_0,r_0)$ and $h_x\gets\Pi.\Hash_x(\hk_0,\Laux$, $C_1$, $\dots,C_x)$.

	\item $\ProjHash_x^D$ is executed between parties $P_1,\dots,P_x$.
	Each $P_i$ performs $\ProjHash_x^D$ on input $(\hp_0,\aux_i,\allowbreak C_1,\dots,C_x,r_i)$ such that $P_1$ eventually holds $h_x$ while all $P_i$ for $i>1$ do not learn anything about $h_x$.
	
	\item $\Hash_0^D$ is executed between parties $P_1,\dots,P_x$.
	Each $P_i$ performs $\Hash_0^D$ on input $(\aux'_i,\hk_i,\allowbreak C_0,\dots,C_x)$ such that $P_1$ eventually holds $h_0$ and all $P_i$ for $i>1$ do not learn anything about $h_0$.
\end{itemize}

\noindent
A \ac{D-SPHF} is said to be correct if $\ProjHash_x^D=\ProjHash_x$ and $\Hash_0^D=\Hash_0$ assuming that all messages are honestly computed and transmitted.
The security of a \ac{D-SPHF} in presence of a passive adversary follows immediately from smoothness and pseudorandomness of the extended \ac{SPHF} algorithms.

\begin{remark}
Note that we focus on asymmetric distribution here such that only $P_1$ computes the hash values.
Building symmetric distribution protocols where all parties $P_i$ compute the same hash values from this is straightforward but requires a different security model.
Likewise, it is possible to build asymmetric distribution protocols where \emph{all} $P_i$ compute \emph{different} hash values (we will see an example of that later).
\end{remark}

\subsection{Security Model}
Smooth projective hashing has not been used in a distributed manner before such that it was not necessary to consider active adversaries.
By introducing distributed computation of hash values $\Hash_0^D$ and $\ProjHash_x^D$ protocols are exposed to active attacks.
However, the adversary must still not be able to distinguish real hash values from random elements, \ie smoothness and pseudorandomness must hold.
Therefore we introduce a security model for \ac{D-SPHF} smoothness and pseudorandomness, capturing active attacks in a multi-user and multi-instance setting.
Let $\{(P^j_0,P^k_1,\dots,P^l_x)\}_{P_0^j\in\cP_0,P^k_i\in\cP~i\in\{1,\dots,x\}}$ denote all tuples $(P^j_0,P^k_1,\dots,P^l_x)$ such that $P^j_0\in\cP_0$ knows \aux and $P^k_1,\dots,P^l_x\in\cP$ each know according $\aux_i$.
We say $P_0$ is \emph{registered} with $(P_1,\dots,P_x)$.
The additional indices $j,k,l$ denote the instance of the respective participant.

\begin{definition}[D-SPHF Security]\label{def:activesphff}
A \ac{D-SPHF} protocol $\Pi$ is secure (offers adaptive smoothness and concurrent pseudorandomness) if for all \ac{PPT} adversaries $\cA$ there exists a negligible function $\varepsilon(\cdot)$ such that :
\[\Adv_{\Pi,\cA}^{\DSPHF}(\secpar)=\left|\Pr[\Exp^{\DSPHF}_{\Pi,\cA}(\secpar)=1]-\frac12\right|\leq\varepsilon(\secpar)\]

\noindent
$\Exp^{\DSPHF}_{\Pi,\cA}(\secpar):$ \\
\hspace*{2em} $b\rin\bits$ \\
\hspace*{2em} $b'\gets\cA^{\Setup(\cdot),\Send(\cdot),\Test(\cdot)}(\secpar,\aux_2,\dots,\allowbreak \aux_x,\cL,\crs)$ \\
\hspace*{2em} return $b=b'$

\begin{itemize}
	\item $\Setup(P_0,\dots,P_x)$ initialises new instances with $(\aux,P_1,\dots,P_x)$ for $P_0$ registered with $(P_1,\dots,P_x)$ and $(\aux_1,P_1,P_0,\dots,P_x)$ for $P_1$ and returns $((\hp_0,\allowbreak C_0),(\hp_1,C_1))$ with $C_i\gets\Enc_{\pk}^{\cL}(\ell,\aux'_i;r_i)$ and $\hk_i\gets\Pi.\HKGen(\Laux)$, $\hp_i\gets\Pi.\PKGen(\hk_i,\Laux)$
		
	\item $\Send(P_b,P_a,m)$ sends message $m$ with alleged originator $P_a$ to $P_b$ and returns $P_b$'s resulting message $m'$ if any.
	
	\item $\Test(P_i^j)$ for $i\in\bits$ returns two hash values $(h_0,h_x)$. If the global bit $b$ is $0$, the hash values are chosen uniformly at random from $\GG$, otherwise the hash values are computed according to protocol specification $\Pi$. \eod
\end{itemize}
\end{definition}

\noindent
Note that we assume without loss of generality that all participants $P_2,\dots,P_x$ are corrupted by the adversary, who thus knows their secrets.
Furthermore, note that $\cA$ can query the $\Test$ oracle only once.

The active security notion for \ac{D-SPHF} computation covers smoothness and pseudorandomess as defined before.
The experiment is equivalent to the computational smoothness definition when $\cA$ computes and forwards all messages honestly but changes at least one $\aux_i$.
Note that this is actually a stronger notion than smoothness as we require pseudorandomness of hash values output by the projection function on a word not in the language.
This is usually not included in the smoothness definition, which is defined over the hash function.
Further, Definition \ref{def:activesphff} is equivalent to Definition \ref{def:prplusc} when $\cA$ computes and forwards all messages honestly and does \emph{not} change any $\aux_i$.

\subsection{Instantiation -- Cramer-Shoup D-SPHF}\label{sec:dcssphf}
We exemplify the \ac{D-SPHF} definition on labelled Cramer-Shoup ciphertexts.
The ciphertexts are created as $C_i=(u_{1,i},u_{2,i},e_{i},v_i)\gets\Enc^{\CS}_{\pk}(\ell_i,\aux'_i;r_i)$ for all $i=1,\dots,x$ with $\aux'_i=h(\aux')[i]$ and $C_0=(u_{1,0},u_{2,0},\allowbreak e_{0},v_0)\gets\Enc^{\CS}_{\pk}(\ell_0,\allowbreak \aux'_0;r_0)$, where $\ell_i$ consists of participating parties and the party's projection key.
We define modified decryption as $\Dec'_\pi(C)=e\cdot u_1^{-z}$.
The combining function $g$ uses the homomorphic property of $u_1$ and $e$ of the CS ciphertext such that $g(C_1,\dots,C_x)=(\prod^x_{i=1}u_{1,i},\prod^x_{i=1}e_i)$ and $\aux'=\sum_{i=1}^{x}\aux'_i$.
The following variables then define Cramer-Shoup \ac{D-SPHF}:

\begin{align*}
& \Gamma=
		\begin{pmatrix}
			g_1 & 1 & g_2 & h & c \\
			1 & g_1 & 1 & 1 & d
		\end{pmatrix} \in \GG^{2\times 5},~~
		{\bm \lambda}=(r, r\xi)\in\ZZ_p^{1\times 2} \\
& \Theta^0_{\aux}(C_0)=(u_1,u_1^\xi,u_2,e/\aux',v)\in\GG^{1\times 5} \\
& \Theta^x_{\aux}(C_1,\dots,C_x)=(\prod^x_{i=1}u_{1,i},\prod^x_{i=1}u_{1,i}^{\xi_i},\prod^x_{i=1}u_{2,i},\prod^x_{i=1}e_i/\aux',\prod^x_{i=1}v_i)\in\GG^{1\times 5}
\end{align*}

\noindent
Using them in the extended \ac{SPHF} Definition \ref{def:symgensphf} yields the extended Cramer-Shoup \ac{SPHF}.
% For a detailed description of the resulting \SPHFF see Appendix \ref{app:cssphff}.
Instead of aiming for absolute generality we describe the Cramer-Shoup \ac{D-SPHF} for $x=2$ such that both participants $P_1$ and $P_2$ compute and broadcast $(\hp_i,C_i)$, while $P_0$ computes and broadcasts $(\hp_0,C_0)$, \ie for the \ac{2PAKE} setting.
Let $\times$ denote element wise multiplication, \eg for El-Gamal ciphertexts $C_1=(u_1,e_1), C_2=(u_2,e_2)$, $C_1\times C_2$ is defined as $(u_1u_2,e_1e_2)$.
$\ProjHash_x^D$ and $\Hash_0^D$ protocols are defined as follows:
%(Figure \ref{fig:dsphf} depicts the entire \ac{D-SPHF} execution)
\begin{itemize}
	\item $\ProjHash_x^D$ is executed between $P_1$ and $P_2$.
	$P_2$ computes $h_{x,2}=\lambda\odot\hp_0=(\hp_0[1]\cdot \hp_0[2]^{\xi_2})^{r_2}$ and sends it to $P_1$.
	Eventually, $P_1$ holds $h_x=h_{x,2}\cdot (\lambda\odot\hp_0)=\hp_0[1]^{r_1+r_2}\cdot \hp_0[2]^{\xi_1\cdot r_1+\xi_2\cdot r_2}$.
	Note that $P_1$ always performs checks that $\hp_0\in\GG$ and $\GG\ni h_2^x\not=0$.
	
	\item $\Hash_0^D$ is executed between $P_1$ and $P_2$ such that $P_1$ eventually holds $h_0$.
	Let $P_i$ for $i\in\{1,2\}$ denote the participating party knowing $(\aux_i$, $\sk_i$, $\hk_i=(\eta_1$, $\eta_2$, $\theta$, $\mu$, $\nu)$, $\pk_1$, $\pk_2$, $C_0=(u_1$, $u_2$, $e$, $v$, $\xi))$.
	\begin{itemize}
		\item $P_1$ computes $m_0\gets\Enc_{\pk_1}^{\EG}(g_1^{-\mu};r)$ and $c'_1\gets\Enc_{\pk_1}^{\EG}(g_1^{\aux'_1};r')$, and sends $(m_0,c'_1)$ to $P_2$.
	
		\item Receiving $(m_0,c'_1)$ from $P_1$, $P_2$ computes
				$$m_1\gets (m_0)^{\aux'_2}\times (c'_1)^{-\mu}\times \Enc_{\pk_1}^{\EG}(g_{1}^{-\mu\cdot \aux'_2}\cdot u_1^{\eta_1+\xi\eta_2}\cdot u_2^{\theta}\cdot e^{\mu}\cdot v^{\nu};r'')$$
				and sends it to $P_1$.
	
		\item Receiving $m_1$, $P_1$ computes the hash value
				$$h_0=g_1^{-\mu\cdot\aux'_1}\cdot\Dec_{\sk_1}^{\EG}(m_1)\cdot u_1^{\eta_1+\xi\eta_2}\cdot u_2^{\theta}\cdot e^{\mu}\cdot v^{\nu}.$$
	\end{itemize}
\end{itemize}

% \begin{figure}
% \centering
% \begin{tikzpicture}[scale=0.55, every node/.style={scale=0.55}]
% \matrix (m)[matrix of nodes, column  sep=.6cm,row  sep=1mm,
% 		nodes={draw=none, anchor=center,text depth=1pt},
% 		column 2/.style={nodes={minimum width=4em}},
% 		column 3/.style={nodes={minimum width=15em}},
% 		column 4/.style={nodes={minimum width=4em}},
% 		column 5/.style={nodes={minimum width=15em}}]{
% 	{$\bm{P_0}$} & & {$\bm{P_1}$} & & {$\bm{P_2}$}\\ [-1mm]
% 	
% 	$\aux$ & & $\aux_1$ & & $\aux_2$\\ [2mm]
% 	
% 	$\hk_0\gets \HKGen$& & $\hk_1\gets \HKGen$ & & $\hk_2\gets \HKGen$ \\
% 	
% 	$\hp_0\gets \PKGen(\hk_0)$ & &
% 	$\hp_1\gets \PKGen(\hk_1)$ & &
% 	$\hp_2\gets \PKGen(\hk_2)$ \\
% 	
% 	$\ell=(P_0,P_1,P_2)$ & & $\ell_1=(P_1,P_0,P_2)$ & & $\ell_2=(P_2,P_0,P_1)$ \\
% 	
% 	$C_0\gets\Enc_{\pk}^{\CS}(\ell,\aux';r)$ &
% 		$C_0,\hp_0$ &
% 		$C_1\gets\Enc_{\pk}^{\CS}(\ell_1,\aux'_1;r_1)$ &
% 		$C_2,\hp_2$ &
% 		$C_2\gets\Enc_{\pk}^{\CS}(\ell_2,\aux'_2;r_2)$ \\
% 	
% 	& $C_1,\hp_1$ &
% 		$m_1\gets\Enc_{\pk_1}^{\EG}(g^{-\mu_1})$ & & \hspace*{1em}\\
% 	
% 	& & $c'_1\gets\Enc_{\pk_1}^{\EG}(g^{\aux'_1})$ &
% 	& $x\gets g_1^{-\mu_2\cdot\aux'_2}u_{1,0}^{\eta_{1,2}+\xi_0\eta_{2,2}}u_{2,0}^{\theta_2}e_0^{\mu_2}v_0^{\nu_2}$ \\
% 	
% 	& & $t\gets u_{1,0}^{\eta_{1,1}+\xi_0\eta_{2,1}}u_{2,0}^{\theta_1}e_0^{\mu_1}v_0^{\nu_1}$&
% 	 	$m_1,c'_1$ &
% 	 	$t\gets\Enc_{\pk_1}^{\EG}(x)$\\
% 	
% 	$h_x\gets\Hash_x(\hk_0,\Laux,C_1,C_2)$ &
% 	& $h_0=g_1^{-\mu_1\cdot\aux'_1}\cdot\Dec_{\sk_1}^{\EG}(m_2)\cdot t$ &
% 	 	$m_2$ &
% 	 	$m_2\gets m_1^{\aux'_2}\times (c'_1)^{-\mu_2}\times t$ \\
% 	
% 	$h_0\gets\ProjHash_0(\hp_1,\hp_2,\Laux,C_0,r)$ & &
% 		$h_x=(\hp[1]\cdot \hp[2]^{\xi_1})^{r_1}\cdot h_{x,2}$ &
% 		$h_{x,2}$ &
% 		$h_{x,2}=(\hp[1]\cdot h_p[2]^{\xi_2})^{r_2}$ \\
% };
%
% \draw[->,dashed] (m-6-2.south west)--(m-6-2.south east);
% \draw[->,dashed] (m-7-2.south east)--(m-7-2.south west);
% \draw[->,dashed] (m-6-4.south east)--(m-6-4.south west);
% \draw[->] (m-9-4.south west)--(m-9-4.south east);
% \draw[->] (m-10-4.south east)--(m-10-4.south west);
% \draw[->] (m-11-4.south east)--(m-11-4.south west);
%
% \draw ([xshift=0pt,yshift=-3.5pt] m-10-3.south west) rectangle ([xshift=0pt,yshift=5pt] m-7-5.north east);
% \coordinate[label=right:$\Hash_0^D$] (hx) at (m-9-5.north east);
% \draw ([xshift=0pt,yshift=-5pt] m-11-3.south west) rectangle (m-11-5.north east);
% \coordinate[label=right:$\ProjHash_x^D$] (hx) at (m-11-5.mid east);
% \end{tikzpicture}
% \caption[Cramer-Shoup D-SPHF]{Cramer-Shoup D-SPHF\\{\small Dashed lines denote broadcast messages.}}
% \label{fig:dsphf}
% \end{figure}

\subsubsection{Smoothness of extended Cramer-Shoup SPHF}\label{sec:excssmoothness}
We want to discuss the statistical smoothness of extended \ac{SPHF} from Theorem \ref{theo:smoothnessc} in this section.
While the intuition and actual proof has been given in Section \ref{sec:sphff}, we want to formulate what actually happens there.
Therefore, we use the instantiation of extended Cramer-Shoup \ac{SPHF} and limit $x=2$.
Recall that we thus want to show that $(\hp_0,\hp_1,\hp_2,h_0,h_x)$ is uniformly distributed in $\GG^{k+2}$ for all $C\not\in\Laux$.
Actually, we do not have to bother with the projection keys $\hp_0,\hp_1,\hp_2$, as they are uniformly at random in $\GG^{k}$ anyway, given the randomness of all $\hk_i$.
What we want to show is that given $\hp_0,\hp_1,\hp_2$, the hash values $h_0$ and $h_x$ are uniformly distributed in $\GG$.
More precisely, we show that for all $C=(C_0,C_1,C_2)\not\in \Laux$ the projection keys $\hp_0,\hp_1,\hp_2$ are defined by functions that are linearly independent from the functions used in $\Hash_0$ and $\Hash_x$, such that the resulting hash values $h_0\gets\Hash_0$ and $h_x\gets\Hash_x$ are uniformly distributed in $\GG$.
Computing the discrete logarithm in base $g_1$ of $h_x,h_0$ and the projection keys $\hp_1,\hp_1$ and $\hp_2$ with $m=g^{\pwd}$ and $m'=g^{\pwd'}$ such that $\Enc^{\CS}_\pk(m')\not\in\Laux$ we get the following equations:
\begin{align}
	\log_{g_1}(h_0) =& \log_{g_1}(\hp_1[0])\cdot r_0 + \log_{g_1}(\hp_1[1])\cdot \xi_0r_0  \notag\\
					 & + \log_{g_1}(\hp_2[0])\cdot r_0 + \log_{g_2}(\hp_2[1])\cdot \xi_0r_0 + \log_{g_1}(m'/m)\cdot (\nu_1+\nu_2)  \notag\\
					=&~ r_0(\eta_{1,1}+\eta_{1,2}) + \xi_0r_0(\eta_{2,1}+\eta_{2,2}) + \log_{g_1}(g_2)\cdot r_0(\theta_1+\theta_2) + z\cdot r_0(\nu_1+\nu_2)  \label{eq:1}\\
					 & + \log_{g_1}(c)\cdot r_0(\nu_1+\nu_2) + \log_{g_1}(d)\cdot\xi_0r_0(\nu_1+\nu_2) + \log_{g_1}(m'/m)\cdot (\nu_1+\nu_2) \notag
\end{align}
\begin{align}
	\log_{g_1}(h_x) =& \log_{g_1}(\hp_0[0])\cdot(r_1+r_2) + \log_{g_1}(\hp_0[1])\cdot(\xi_1r_1+\xi_2r_2) + \log_{g_1}(m'/m)\cdot \nu_0  \notag\\
					=&~ (r_1+r_2)\eta_{1,0} + (\xi_1r_1+\xi_2r_2)\eta_{2,0} + \log_{g_1}(g_2)\cdot (r_1+r_2)\theta_0 + z\cdot (r_1+r_2)\nu_0 \label{eq:2}\\
					 & + \log_{g_1}(c)\cdot (r_1+r_2)\nu_0 + \log_{g_1}(d)\cdot (\xi_1r_1+\xi_2r_2)\nu_0 + \log_{g_1}(m'/m)\cdot \nu_0 \notag
\end{align}
\begin{align}
	\log_{g_1}(\hp_0[0]) =&~ \eta_{1,0} + \log_{g_1}(g_2)\cdot\theta_{0} + \log_{g_1}(h)\cdot\mu_{0} + \log_{g_1}(v)\cdot\nu_{0} \label{eq:3}\\
	\log_{g_1}(\hp_0[1]) =&~ \eta_{2,0} + \log_{g_1}(d)\cdot\nu_{0} \label{eq:4}\\
	\log_{g_1}(\hp_1[0]) =&~ \eta_{1,1} + \log_{g_1}(g_2)\cdot\theta_{1} + z\cdot\mu_{1} + \log_{g_1}(v)\cdot\nu_{1} \label{eq:5}\\
	\log_{g_1}(\hp_1[1]) =&~ \eta_{2,1} + \log_{g_1}(d)\cdot\nu_{1} \label{eq:6}\\
	\log_{g_1}(\hp_2[0]) =&~ \eta_{1,2} + \log_{g_1}(g_2)\cdot\theta_{2} + z\cdot\mu_{2} + \log_{g_1}(v)\cdot\nu_{2} \label{eq:7}\\
	\log_{g_1}(\hp_2[1]) =&~ \eta_{2,2} + \log_{g_1}(d)\cdot\nu_{2}. \label{eq:8}
\end{align}

\noindent
Since $C\not\in\Laux$ we know that $m\not=m'$ and thus $m'/m\not=1$.
Therefore, the probability for $g_0=\log_{g_1}(m'/m)\cdot (\nu_1+\nu_2)$ and $g_x=\log_{g_1}(m'/m)\cdot \nu_0$ for any $g_0,g_x\in\GG$ is $1/|\GG|$ even knowing the projection keys $\hp_0,\hp_1,\hp_2$.
Note that these equations for $g_0$ and $g_x$ are linearly independent from Equations \ref{eq:3} - \ref{eq:8} such that every element from $\GG$ is equally likely to be the result.
Equations \ref{eq:1} and \ref{eq:2} are fully determined by public information $C_0,C_1,C_2$ and $\hp_0,\hp_1,\hp_2$ such that their result is uniformly distributed in $\GG$ given the randomness of $g_0$ and $g_x$.

\subsubsection{Security Analysis}
We show now that the proposed Cramer-Shoup \ac{D-SPHF} is secure.
The intuition behind the proof is that the pseudorandomness of $h_x$ can be reduced directly to the \ac{DDH} problem in $\GG$ and \ac{CCA} security of \ac{CS} encryption, while pseudorandomness of $h_0$ follows from the smoothness and pseudorandomness of the underlying extended \ac{SPHF} scheme.

\begin{theorem}[Cramer-Shoup \ac{D-SPHF} Security]\label{theo:cssphff}
The Cramer-Shoup \ac{D-SPHF} instantiation is secure against active adversaries according to Definition \ref{def:activesphff} when the \ac{DDH} assumption in the used group \GG holds and $\cL=\CS$ is \ac{CCA}-secure.
\end{theorem}
\begin{proof}
First, note that the theorem follows immediately from smoothness and pseudorandomness in the passive case if the adversary queries $\Test(P_0)$.
We therefore focus on $\Test(P_1)$ queries.
We start with the pseudorandomness of $h_x$, \ie for all $g$ it holds that $\Pr[h_x=g]=1/|\GG|$.
Consider an attacker $\cA$ on input $(\secpar,\aux_2,\cL,\crs)$ and let $\Exp_0$ denote the original \ac{D-SPHF} experiment.\\

\noindent$\Exp_1:$
We change $\Test$ such that a uniformly at random chosen element $g_x\rin\GG$ is returned for $h_x$.

\begin{claim}
$\left|\Adv_{\Pi,\cA}^{\Exp_0}-\Adv_{\Pi,\cA}^{\Exp_1}\right|\leq\varepsilon(\secpar)$
\end{claim}

\begin{proof}
The hash value $h_x$ in $\Exp_0$ is computed as $h_x=(\hp'_0[1]\cdot\hp'_0[2]^{\xi_1})^{r_1}\cdot h_{x,2}$ with adversarially generated $h_{x,2}$ and $\hp'_0$.
Indistinguishability of $h_x$ and $g_x$, and thus the claim, follows immediately as long as the \ac{DDH} assumption in $\GG$ holds (using \ac{DDH} triple $(\hp'_0[1]\cdot\hp'_0[2]^{\xi_1},g^{r_1},h_x)$ and $(\hp'_0[1]\cdot\hp'_0[2]^{\xi_1},g^{r_1},g_x)$).
Note that $P_1$ aborts if either $h_{x,2}\not\in\GG$ or $\hp_0'\not\in\GG^2$.
\end{proof}

\noindent
To show the security (concurrent pseudorandomness and adaptive smoothness) of $h_0$ we define two $\Send$ queries that allow execution of the protocol:
$(m_1,c'_1)\gets\Send_1(P_1$, $P_2$, $(\hp'_0$, $C'_0$, $\hp'_2$, $C'_2))$ starts the protocol execution between $P_1$ and $P_2$ and provides the attacker with $(m_1,c'_1)$.
Using these messages the adversary ($P_2$) computes a message $m_2$ and sends it to $P_1$ with $\Send_2(P_1,P_2,m_2)$.
This reflects the execution of a single protocol run of $\Hash_0^D$ such that $P_1$ eventually computes $h_0$.
In contrast to the passive and classical SPHF proofs we can not replace the ciphertexts with encryptions of words not in the language.
However, this is not necessary as $t$ is in fact the \Hash computation of the classical Cramer-Shoup SPHF without cancelling the message, \ie $t=h\cdot m^\mu$.\\

\noindent$\Exp_2:$
We change $\Test$ such that a uniformly at random chosen element $g_0\rin\GG$ is returned for $h_0$.

\begin{claim}
$\left|\Adv_{\Pi,\cA}^{\Exp_1}-\Adv_{\Pi,\cA}^{\Exp_2}\right|\leq\varepsilon(\secpar)$
\end{claim}

\begin{proof}
The hash value $h_0$ in $\Exp_1$ is computed as $h_0=g^{-\mu_1\cdot\aux'_1}\cdot\Dec_{\sk_1}^{\EG}(m_2)\cdot t$ with $t=u_{1,0}^{\eta_{1,1}+\xi_0\eta_{2,1}}u_{2,0}^{\theta_1}\allowbreak e_0^{\mu_1}v_0^{\nu_1}$ where $m_2$ and $C'_0=(u_{1,0}, u_{2,0},\allowbreak e_0, v_0)$ may be adversarially generated.
The value $t$ is actually the $\Hash$ value of the classical Cramer-Shoup SPHF without cancelled message, or in other words $t$ is the result of a SPHF \Hash computation for language $L_{(\crs,0)}$ such that any $C'_0$, encrypting some correct $\aux'\not=0$, is not in this language.
Due to smoothness of the \Hash function \cite{Benhamouda2013} $t$ is indistinguishable from a uniformly at random chosen element.
If the adversary encrypted $0$ in $C'_0$, pseudorandomness of \Hash takes effect.
Therefore $h_0=d\cdot t$ is indistinguishable from a random group element for all $d\in\GG$.
\end{proof}

\noindent
In $\Exp_2$ the adversary always gets random group elements in answer to his $\Test$ query.
Therefore, he can not do better than guessing bit $b$.
\end{proof}

\subsection{Instantiation -- ElGamal D-SPHF}\label{app:elgamalsphff}
Similar to Cramer-Shoup \ac{D-SPHF} we can instantiate ElGamal \ac{D-SPHF}.
We use $m$ for the encrypted message, that is part of $\aux'$ and $\pk$ for the ElGamal public key $h=g^z$ from the \crs.
The ciphertexts are created as $C_i=(u,e)\gets\Enc^{\EG}_{\pk}(m_i;r_i)$ for all $i=1,\dots,x$ with $m_i=h(m)[i]$ and $C_0=(u,e)\gets\Enc^{\EG}_{\pk}(m;r_0)$.
The decryption follows the ElGamal decryption such that $\Dec'_\pi=\Dec^{\EG}_z$.
The combining function $g$ uses the homomorphic property of $u$ and $e$ such that $g(C_1,\dots,C_x)=(\prod^x_{i=1}u_{i},\prod^x_{i=1}e_i)$.
To use the \ac{SPHF} framework we also need the following variables and functions:

\[ \Gamma(C)= (g,h)^T \in \GG^{2\times 1},~~ {\lambda}=r\in\ZZ_p,~~ \Theta^0_{\aux}(C)=(u,e/m)\in\GG^{1\times 2} \]
\[ \Theta^x_{\aux}(C_1,\dots,C_x)=(\prod^x_{i=1}u_{i},\prod^x_{i=1}e_i/m)\in\GG^{1\times 2} \]

\noindent
Using them in the extended \ac{SPHF} Definition \ref{def:symgensphf} yields the following extended ElGamal \ac{SPHF} (building the according \ac{D-SPHF} is then straightforward):
\begin{itemize}
	\item $h_0\algout\Hash_0(\hk_1,\dots,\hk_x,\Laux,C_0):$
	 $$h_0=\prod_{i=1}^x\Theta_\aux^0(C) \odot \hk_i = \prod_{i=1}^x [(u_0,e_0/m) \odot (\eta,\theta)] =\prod_{i=1}^x [u_0^{\eta_i}(e_0/m)^{\theta_i}]\in\GG$$
	
	\item $h_x\algout\ProjHash_x(\hp_0,\Laux,C_1,\dots,C_x,r_1,\dots,r_x):$
	$$h_x = \prod_{i=1}^x (\lambda^i \odot \hp_0) = \prod_{i=1}^x (r_i \odot g^{\eta_0}h^{\theta_0}) = \prod_{i=1}^x (g^{\eta_0}h^{\theta_0})^{r_i} \in\GG$$
\end{itemize}


% ************************** 2PAKE Framework from D-SPHF

\section{Two-Server PAKE from Distributed SPHF}\label{sec:twoserverpake}
Using \ac{D-SPHF} it is easy to build \ac{2PAKE} protocols.
In this section we present a new \ac{2PAKE} framework based on \acp{D-SPHF}.
Moreover, we show that the \ac{2PAKE} protocol by \citet{Katz2012a} can be considered a variant of the proposed framework using a mix of \acp{D-SPHF} for Cramer-Shoup and El-Gamal ciphertexts.
We consider the same setting as \citet{Katz2012a} here, in which the client computes two independent session keys with the two servers.

% With a single server storing the password, password authenticated key exchange (PAKE) protocols have an intrinsic single point of failure.
% As soon as the server's database, storing the client's secrets, gets compromised the attacker can impersonate the client to this server, and most likely also to others considering that users tend to reuse their passwords across multiple services.
% Mechanisms have been proposed to solve the problem of server compromise \cite{Gentry2006,rfc2945}.
% However, as long as only one server is used, PAKE protocols are prone to offline dictionary attacks on the server side.
% Two-server PAKE (2PAKE) protocols can solve this problem by splitting the password in two parts such that a malicious or compromised server can be used to recover only one part of the password.
% Raimondo and Gennaro \cite{Raimondo_Gennaro_2003} proposed a $t$-out-of-$n$ threshold PAKE, which is not suitable for the 2PAKE setting as it requires $t<n/3$.
% Another $t$-out-of-$n$ threshold PAKE was proposed in a PKI-based setting with random oracles \cite{MacKenzie_Shrimpton_Jakobsson_2002}.
% Brainard and Juels \cite{Brainard_Juel\Server_2003} proposed two-server password based authentication without security proof.
% Szydlo and Kaliski \cite{Szydlo_Kaliski_2005} later modified constuctions from \cite{Brainard_Juel\Server_2003} and proved their security in a simulation-based model.
% The first two-server PAKE in the password-only setting, \ie without a PKI, is due to Katz et al. \cite{Katz2012a}, based on the KOY protocol from \cite{Katz_Ostrovsky_Yung_2001}.

\subsection{A Two-Server PAKE Framework}
% Using \ac{D-SPHF} we can build efficient \ac{2PAKE} protocols.
Considering the setting from \citet{Katz2012a} means that a client negotiates independent session keys with both servers that hold $\share_1+\share_2=\pwd$.
We omit the second server in the description of the protocol as the framework is symmetric in the sense that the second server $\Server_2$ performs like $\Server_1$.
% in Figure \ref{fig:twopake}
The framework follows the same principle as the latest \ac{PAKE} frameworks from \acp{SPHF}.
In particular it can be seen as a two-server variant of the \ac{PAKE} protocol from \citet{Katz2011}.

You can think of the two-server protocol as the execution of two \ac{D-SPHF} protocols, one between $(\Client,\Server_1,\Server_2)$ and one between $(\Client,\Server_2,\Server_1)$ where servers $\Server_2$ and $\Server_1$ swap roles, such that $(\Client,\Server_1)$ and $(\Client,\Server_2)$ eventually hold common hash values that can be used to generate a shared session key $\sk_1$ and $\sk_2$.
The only overlap between the two \ac{D-SPHF} executions is the $\Hash_x$ computation.
The reuse of ciphertexts $C_1$ and $C_2$ in $\Hash_x$ functions across the two servers is covered by the concurrent pseudorandomness.

\subsubsection{The Framework}
The servers encrypt their password shares under a public key $\pk$ stored in the \crs using a \ac{CCA}-secure labelled encryption scheme and distribute this ciphertext together with two appropriate projection keys for a secure \ac{D-SPHF}, $(\hp_{1,1},\hp_{1,2},C_1)$ and $(\hp_{2,1},\hp_{2,2},C_2)$.
The client computes two independent encryptions of the password and generates two independent according projection keys $(\hp_{0,1},C_{0,1},\hp_{0,2},C_{0,2})$.
The previously described \ac{D-SPHF} allows us to send all $\hp_i,C_i$ in one round and therefore reach optimality for this step.
Using these values, the client can compute session keys as product of the two hash values $h_{0,1},h_{x,1}$ for $\sk_1$, which is shared with $\Server_1$ and from $h_{0,2},h_{x,2}$ for $\sk_2$ that is shared with $\Server_2$.

Subsequently, the two servers perform the $\Hash_0^D$ and $\ProjHash_x^D$ protocols such that $\Server_1$ holds hash values $h_{0,1}$ and $h_{x,1}$ to compute $\sk_1$ and $\Server_2$ holds $h_{0,2}$ and $h_{x,2}$ to compute $\sk_2$.
Eventually, \Client holds $\sk_1=h_{0,1}\cdot h_{x,1}$ and $\sk_2=h_{0,2}\cdot h_{x,2}$, $\Server_1$ holds $\sk_1=h_{0,1}\cdot h_{x,1}$ and $\Server_2$ holds $\sk_2=h_{0,2}\cdot h_{x,2}$.
An instantiation of the framework using labelled Cramer-Shoup encryption and the aforementioned \ac{D-SPHF} yields a secure \ac{2PAKE} protocol.
Note that this actually requires two \ac{D-SPHF} executions.

% \begin{figure}[!t]
% \centering
% \begin{tikzpicture}[scale=0.5, every node/.style={scale=0.5}]
% \matrix (m)[matrix of nodes, column  sep=.6cm,row  sep=1mm,
% 		nodes={draw=none, anchor=center,text depth=1pt},
% 		column 2/.style={nodes={minimum width=10em}},
% 		column 3/.style={nodes={minimum width=20em}},
% 		column 4/.style={nodes={minimum width=5em}}]{
% 	{\Client} & & {$\Server_1$} \\ [-1mm]
% 	
% 	$\pk,\pwd$ & & $\pk,\share_1,\sk_1,\pk_2$ & \\ [2mm]
% 	
% 	$\hk_{0,1}\gets \HKGen(\Laux), \hk_{0,2}\gets \HKGen(\Laux)$ & &
% 	$\hk_{1,1}\gets \HKGen(\Laux), \hk_{1,2}\gets \HKGen(\Laux)$ & \\
% 	
% 	$\hp_{0,1}\gets \PKGen(\hk_{0,1}), \hp_{0,2}\gets \PKGen(\hk_{0,2})$ & &
% 	$\hp_{1,1}\gets \PKGen(\hk_{1,1}), \hp_{1,2}\gets \PKGen(\hk_{1,2})$ & \\
% 	
% 	$\ell_{0,1}=(C,\Server_1,\Server_2), \ell_{0,2}=(C,\Server_2,\Server_1)$ & & $\ell_1=(\Server_1,C,\Server_2)$ & \\
% 	
% 	$C_{0,1}\gets\Enc_{\pk}^\cL(\ell_{0,1},\pwd;r_{0,1}), C_{0,2}\gets\Enc_{\pk}^\cL(\ell_{0,2},\pwd;r_{0,2})$ &
% 	$\hp_{0,1},C_{0,1},\hp_{0,2},C_{0,2}$ &
% 	$C_1\gets\Enc_\pk^\cL(\ell_1,\share_1,r_1)$ &
% 	$\hp_{2,1},\hp_{2,2},C_2$ \\
% 	
% 	 &
% 	$\hp_{1,1},\hp_{1,2},C_1$ &
% 	 & \\
% 	
% 	$h_{0,1}\gets\ProjHash_0(\hp_1,\hp_2,\Laux,C_{0,1},r_{0,1})$ & &
% 	$h_{0,1}\gets\Hash_0^{D}(C_{0,1},\hk_{1,1},\share_1,\sk_1,\pk_2)$ & \hspace*{5em} \\
%
% 	$h_{0,2}\gets\ProjHash_0(\hp_1,\hp_2,\Laux,C_{0,2},r_{0,2})$ & &
% 	$h_{x,1}\gets\ProjHash_x^{D}(\hp_{1,1},C_1,r_1)$ & \hspace*{5em} \\
% 	
% 	$h_{x,1}\gets\Hash_x(\hk_{0,1},\Laux,C_1,C_2)$ & &
% 	$\Hash_0^{D}(C_{0,2},\hk_{1,2},\share_1,\sk_1,\pk_2)$ & \hspace*{5em} \\
% 	
% 	$h_{x,2}\gets\Hash_x(\hk_{0,2},\Laux,C_1,C_2)$ & &
% 	$\ProjHash_x^{D}(\hp_{1,2},C_1,r_1)$ & \hspace*{5em} \\
% 	
% 	$\sk_1=h_{0,1} h_{x,1},~\sk_2=h_{0,2} h_{x,2}$ &
% 	& $\sk_1=h_{0,1} h_{x,1}$ & \hspace*{5em} \\
% };
%
% \draw[->,dashed] (m-6-2.south west)--(m-6-2.south east);
% \draw[<-,dashed] (m-7-2.south west)--(m-7-2.south east);
% \draw[->,dashed] (m-6-4.south east)--(m-6-4.south west);
%
% \draw[<->] (m-8-4.west)--(m-8-4.east);
% \draw[<->] (m-9-4.west)--(m-9-4.east);
% \draw[<->] (m-10-4.west)--(m-10-4.east);
% \draw[<->] (m-11-4.west)--(m-11-4.east);
%
% \draw[dashed] ([xshift=0pt,yshift=5pt] m-8-3.north west) -- ([xshift=0pt,yshift=9pt] m-8-4.north east);
% \draw[dashed] ([xshift=0pt,yshift=0pt] m-8-3.north west) -- ([xshift=0pt,yshift=0pt] m-11-3.south west);
% \draw[dashed] ([xshift=0pt,yshift=0pt] m-8-3.south west) -- ([xshift=0pt,yshift=-4pt] m-8-4.south east);
% \draw[dashed] ([xshift=0pt,yshift=0pt] m-9-3.south west) -- ([xshift=0pt,yshift=-4pt] m-9-4.south east);
% \draw[dashed] ([xshift=0pt,yshift=0pt] m-10-3.south west) -- ([xshift=0pt,yshift=-4pt] m-10-4.south east);
% \draw[dashed] ([xshift=0pt,yshift=0pt] m-11-3.south west) -- ([xshift=0pt,yshift=-4pt] m-11-4.south east);
%
% \end{tikzpicture}
% \caption[Two-Server PAKE framework using D-SPHF]{Two-Server PAKE framework using D-SPHF
% \\{\small Dashed lines denote broadcast messages.}}
% \label{fig:twopake}
% \end{figure}

\subsubsection{Security Analysis}
We use the well-known game based \ac{PAKE} model first introduced by \citet{Bellare2000} in it's two-server variant by \citet{Katz2012a} (cf. Section \ref{sec:2pake-model} Chapter \ref{ch:prelims}).
% For a formal description of the model we refer to \cite{Katz2012a}.
Security of the \ac{2PAKE} framework follows directly from the \ac{CCA}-security of the used encryption scheme and the security of the used \ac{D-SPHF}.

\begin{theorem}\label{theo:twopake}
Let $(\HKGen,\allowbreak\PKGen,\allowbreak\ProjHash_0,\allowbreak\Hash_x,\allowbreak\Hash_0^D,\allowbreak\ProjHash_x^D)$ be a secure \ac{D-SPHF} and $(\KGen$, $\Enc$, $\Dec)$ a \ac{CCA}-secure labelled encryption scheme, then the proposed framework is a secure \ac{2PAKE} protocol.
% in Figure \ref{fig:twopake}
\end{theorem}

\begin{proof}[Proof sketch]
Let $\Pi$ denote a secure instantiation of the \ac{2PAKE} framework.
To prove security of $\Pi$ we introduce three experiments such that the adversary in the last experiment $\Exp_3$ can not do better than guessing the password as all messages are password independent, \ie $\Adv_{\Pi,\cA}^{\Exp_3}\leq q/|\cD|$ for $q$ active attacks.
We initially focus on the \ac{AKE}-security of $\sk_1$.

$\Exp_1$ is identical to the two-server \ac{AKE}-security experiment except that the simulator knows $\pi$, the decryption key to $\pk$ in the $\crs$ (only a syntactical change) and the following changes:
If $C_{0,1}$ or $C_{1}$, handed to $\Server_1$ or \Client are adversarially generated and encrypt the correct password(share), the simulator stops and $\cA$ wins the experiment.
If $C_{0,1}$, $C_1$ or $C_2$, handed to $\Server_1$ or \Client encrypt a wrong password(share), the key for that session is drawn uniformly at random from $\GG$.
The first change only increases the adversarial advantage and the second one introduces a negligible gap according to the adaptive smoothness of the used \ac{D-SPHF}.

$\Exp_2$ performs like $\Exp_1$ except that it draws the session key at random from $\GG$ if all $C_i$ handed to \Client and $\Server_1$ are oracle generated or encrypt the correct password and no session key has been chosen for the partner in that session (otherwise that previously drawn key is used).
This introduces a negligible gap between advantages in $\Exp_1$ and $\Exp_2$ due to the concurrent pseudorandomness of the used \ac{D-SPHF}.

$\Exp_3$ acts like $\Exp_2$ except that it returns encryptions of $0$ for $C_{0,1}$ and $C_1$ (note that $0$ is not a valid password).
This step is covered by the \ac{CCA}-security of the used encryption scheme.

\ac{AKE}-security of $\sk_1$ follows as all messages are password independent in $\Exp_3$ unless the adversary guesses the correct password.
Using the same sequence of experiments but considering \Client and $\Server_2$ instead of \Client and $\Server_1$, \ac{AKE}-security of $\sk_2$ follows.
\end{proof}

\subsection{The 2-Server KOY Protocol}\label{sec:twokoy}
Using \ac{D-SPHF} we can ``explain'' the 2KOY proposed by \citet{Katz2012a}; similar to \citet{Gennaro2003} who ``explained'' the original KOY protocol from \citet{KatzOY01}.
We define encryption schemes and \ac{D-SPHF} used in 2KOY, highlight changes to our framework and discuss implications of this on the security of 2KOY.

The \crs contains a public key $\pk$ for Cramer-Shoup encryption as well as a public key $g_3$ for El-Gamal encryption.
Since 2KOY uses El-Gamal encryptions on the server side, we have to use a combination of Cramer-Shoup and ElGamal based \ac{D-SPHF} here.
Instead of using Cramer-Shoup encryptions and \ac{D-SPHF}, the client computes projection keys for an ElGamal \ac{D-SPHF}.

Likewise, the servers compute projection keys for a Cramer-Shoup distributed GL-D-SPHF and El-Gamal encryptions of their password shares.\footnote{Note that an additional signature on the session transcript in round three ensures ``non-malleability'' of these ciphertexts.}
% We describe the original GL-\ac{SPHF} on Cramer-Shoup ciphertexts in Appendix \ref{app:glsphf}
Recall that \citet{Gennaro2003} formally introduced the first use of \ac{SPHF} in the \ac{PAKE} setting, denoted by GL-\ac{SPHF} here.
To describe GL-\ac{SPHF} on labelled Cramer-Shoup ciphertexts in the framework from \cite{cryptoeprint:2013:034} it is sufficient to define the following variables:
\[
\Gamma(C)=
\begin{pmatrix}
g_1 & g_2 & h & c \\
1 & 1 & 1 &  d^\xi
\end{pmatrix}^T \in \GG^{4\times 2}
\]
\[
{\bm \lambda}=r\in\ZZ_p \text{ and } \Theta_\aux(C)=(u_1,u_2,e/m,v)\in\GG^{1\times 4}
\]
% GL-\ac{SPHF} then is defined as follows:
% \begin{itemize}
% 	\item $\hk\ralgout\HKGen(\Laux):$ $\hk=(\eta,\theta,\mu,\nu)\rin\ZZ_p^{1\times 4}$
% 	
% 	\item $\hp\algout\PKGen(\hk,\Laux,C):$ $$\hp = \Gamma \odot \hk =
% 		\begin{pmatrix}
% 		g_1 & g_2 & h & c \\
% 		1 & 1 & 1 &  d^\xi
% 		\end{pmatrix}^T \odot (\eta,\theta,\mu,\nu) = g_1^{\eta}g_2^\theta h^\mu c^\nu  d^{\xi\nu} \in\GG$$
% 	
% 	\item $h\algout\Hash(\hk,\Laux,C):$ $$h=\Theta_\aux(C) \odot \hk = (u_1,u_2,e/m,v) \odot (\eta,\theta,\mu,\nu)
% 			= u_1^{\eta}u_2^{\theta}(e/m)^\mu v^\nu\in\GG$$
% 	
% 	\item $h\algout\ProjHash(\hp,\Laux,C,r):$
% 		$$h = \lambda \odot \hp = r \odot g_1^{\eta}g_2^\theta h^\mu c^\nu  d^{\xi\nu} = (g_1^{\eta}g_2^\theta h^\mu c^\nu  d^{\xi\nu})^r \in\GG$$
% \end{itemize}

% \noindent
The client sends the projection keys in a third round together with a one-time signature on the session transcript to the servers.
The protocol is sketched in Figure \ref{fig:twokoy} using our notation for \acp{D-SPHF}.
Note that we have to move and rename several computations but do not modify the protocol.
%$K^r$ into a separate encryption compared to the original protocol.
The ElGamal encryption of the password of server $\Server_i$, $\hat{C}_{i}\gets\Enc^\EG_{\pk_i}(g^{\pwd_i};r_i)$ is precomputed and stored on $\Server_{j},j\not=i,j\in\{1,2\}$.
Eventually, the client computes hash values using the $\ProjHash_0$ function of the GL-\ac{D-SPHF} scheme on CS ciphertexts and the $\Hash_x$ function of the \ac{D-SPHF} scheme on ElGamal ciphertexts.
Further, the servers execute the $\Hash_0^D$ protocol of the distributed GL-\ac{D-SPHF} scheme on Cramer-Shoup ciphertexts and the $\ProjHash_x^D$ protocol of the \ac{D-SPHF} scheme on ElGamal ciphertexts.

% \begin{figure}[tbhp]
% \centering
% \begin{tikzpicture}[scale=0.5, every node/.style={scale=0.5}]
% \matrix (m)[matrix of nodes, column  sep=.6cm,row  sep=1mm,
% 		nodes={draw=none, anchor=center,text depth=1pt},
% 		column 2/.style={nodes={minimum width=8em}},
% 		column 3/.style={nodes={minimum width=20em}},
% 		column 4/.style={nodes={minimum width=8em}}]{
% 	{\Client} & & {$\Server_1$} \\ [-1mm]
% 	
% 	$\crs,\pwd$ & & $\crs,\share_1,\hat{C}_2$ & \\ [2mm]
% 	
% 	$(\vk,\sk)\gets\Gen,~\ell=(C,\vk)$ & &
% 	$(\hk_1,\hk'_1)\gets \HKGen^\CS(\Laux)$ & \\
% 	
% 	$C_{1,0}\gets\Enc_{\pk}^{\CS}(\ell,\pwd;r_1)$ & &
% 	 & \\
% 	
% 	$C_{2,0}\gets\Enc_{\pk}^{\CS}(\ell,\pwd;r_2)$ &
% 	$C_{1,0},C_{2,0},\vk$ &
% 	$(\hp_1,\hp'_1)\gets \PKGen^\CS(\hk_1,\hk'_1, C_{1,0})$ & \\
% 	
% 	$(\hk_{1,0},\hk_{2,0})\ralgout\HKGen^\EG(\Laux)$  & & & \\
% 	
% 	$(\hp_{1,0},\hp_{2,0})\gets \PKGen^\EG(\hk_{1,0},\hk_{2,0},\Laux)$
% 	& $C_1,\hp_1,\hp'_1$ &
% 		$C_1\gets\Enc^{\EG}_{g_3}(g_1^{\share_1};r_1)$
% 	& $C_2,\hp_2,\hp'_2$ \\
% 	
% 	$\sigma\gets\Sign(\trans,\hp_1,\hp_2)$
% 	& & & \\
% 	
% 	$h_{0,1}\gets\ProjHash_0^\CS(\hp_1,\hp_2,\Laux,C_{1,0},r_1)$
% 	& $\sigma,\hp_{1,0},\hp_{2,0}$
% 	& check $C,G_2,\trans$ & \\
% 	
% 	$h_{0,2}\gets\ProjHash_0^\CS(\hp'_1,\hp'_2,\Laux,C_{2,0},r_2)$ & & $h_{0,1}\gets\Hash_0^{D-CS}$ & \hspace*{8em} \\
%
% 	$h_{x,1}\gets\Hash_x^\EG(\hk_{1,0},\Laux,C_1,C_2)$ & & \hspace*{10em} & \hspace*{8em} \\
% 	
% 	$h_{x,2}\gets\Hash_x^\EG(\hk_{2,0},\Laux,C_1,C_2)$ & & $h_{x,1}\gets\ProjHash_x^{D-\EG}$ & \hspace*{8em} \\
% 	
% 	$\sk_1=h_{0,1} h_{x,1}, \sk_2=h_{0,2} h_{x,2}$ &
% 	& $\sk_1=h_{0,1} h_{x,1}$ & \hspace*{8em} \\
% };
%
% \draw[->,dashed] (m-5-2.south west)--(m-5-2.south east);
% \draw[->,dashed] (m-7-2.south east)--(m-7-2.south west);
% \draw[->,dashed] (m-7-4.south east)--(m-7-4.south west);
% \draw[->,dashed] (m-9-2.south west)--(m-9-2.south east);
%
% \draw[dashed] ([xshift=0pt,yshift=0pt] m-10-3.north west) -- ([xshift=0pt,yshift=4pt] m-10-4.north east);
% \draw[dashed] ([xshift=0pt,yshift=0pt] m-10-3.north west) -- ([xshift=0pt,yshift=0pt] m-12-3.south west);
% \draw[dashed] ([xshift=0pt,yshift=5pt] m-11-3.south west) -- ([xshift=0pt,yshift=5pt] m-11-4.south east);
% \draw[dashed] ([xshift=0pt,yshift=0pt] m-12-3.south west) -- ([xshift=0pt,yshift=-4pt] m-12-4.south east);
%
% %\draw[dashed] ([xshift=0pt,yshift=0pt] m-11-3.south west) -- ([xshift=0pt,yshift=0pt] m-12-3.south west);
%
% \end{tikzpicture}
% \caption[Two-Server KOY using D-SPHF]{Two-Server KOY \cite{Katz2012a} using D-SPHF
% \\{\small Dashed lines denote broadcast messages.}}
% \label{fig:twokoy}
% \end{figure}

\begin{figure*}[t]
\begin{center}
\scalebox{0.73}{
\begin{tabular}{ l c l c}
\toprule
{\bf Client \Client} & & {\bf Server $\Server_1$} \\
Input: $\crs, \pwd, \Server_1, \Server_2$ & & Input: $\crs, \share_1, \Client, \Server_2, \hat{C}_2$ \\
\midrule
  $(\vk,\sk)\gets\Gen,~\ell=(C,\vk)$ & & $(\hk_1,\hk'_1)\gets \HKGen^\CS(\Laux)$ & \\
  $C_{1,0}\gets\Enc_{\pk}^{\CS}(\ell,\pwd;r_1)$ & & & \\
  $C_{2,0}\gets\Enc_{\pk}^{\CS}(\ell,\pwd;r_2)$ &
	  $\xrightarrow{\makebox[2cm]{$C_{1,0},C_{2,0},\vk$}}$ &
	  $(\hp_1,\hp'_1)\gets \PKGen^\CS(\hk_1,\hk'_1, C_{1,0})$ \\
	$(\hk_{1,0},\hk_{2,0})\ralgout\HKGen^\EG(\Laux)$  & & & \\
	$(\hp_{1,0},\hp_{2,0})\gets \PKGen^\EG(\hk_{1,0},\hk_{2,0},\Laux)$
	  & $\xleftarrow{\makebox[2cm]{$C_1,\hp_1,\hp'_1$}}$ &
		  $C_1\gets\Enc^{\EG}_{g_3}(g_1^{\share_1};r_1)$
	  & $\xleftarrow{\makebox[2cm]{$C_2,\hp_2,\hp'_2$}}$ \\
	$\sigma\gets\Sign(\trans,\hp_1,\hp_2)$ & & & \\
	$h_{0,1}\gets\ProjHash_0^\CS(\hp_1,\hp_2,\Laux,C_{1,0},r_1)$
	  & $\xrightarrow{\makebox[2cm]{$\sigma,\hp_{1,0},\hp_{2,0}$}}$
	  & check $C,G_2,\trans$ & \\
	$h_{0,2}\gets\ProjHash_0^\CS(\hp'_1,\hp'_2,\Laux,C_{2,0},r_2)$ & & $h_{0,1}\gets\Hash_0^{D-CS}$ & \hspace*{8em} \\
	$h_{x,1}\gets\Hash_x^\EG(\hk_{1,0},\Laux,C_1,C_2)$ & & \hspace*{10em} & \hspace*{8em} \\
	$h_{x,2}\gets\Hash_x^\EG(\hk_{2,0},\Laux,C_1,C_2)$ & & $h_{x,1}\gets\ProjHash_x^{D-\EG}$ & \hspace*{8em} \\
	$\sk_1=h_{0,1} h_{x,1}, \sk_2=h_{0,2} h_{x,2}$ &
	  & $\sk_1=h_{0,1} h_{x,1}$ & \hspace*{8em} \\
\bottomrule
\end{tabular}}
\end{center}
\caption[Two-Server KOY using D-SPHF]{Two-Server KOY \cite{Katz2012a} using D-SPHF}
\label{fig:twokoy}
\end{figure*}

\paragraph{Security Analysis}
Security of the 2KOY protocol against passive adversaries follows immediately from \cite[Theorem 1]{Katz2012a} as we do not change the protocol in Figure \ref{fig:twokoy}
However, \citet{Katz2012a} need additional mechanisms to prove their protocol secure against an active adversary.
They add witness-indistinguishable $\Sigma$-protocols to the $\ProjHash_x^{D}$ and $\Hash_0^{D}$ protocols that prove correctness of their messages.
Without giving a proof it seems that Theorem \ref{theo:twopake} also holds for the two-server KOY instantiation \emph{without} additional mechanisms.
Examining the proof of \cite[Theorem 2]{Katz2012a} shows that the additional steps are only necessary to conduct the proof without actually giving additional security.
This shows the power of \ac{D-SPHF} as they allow for much simpler proofs of multi-party protocols.
Furthermore, with the proposed framework the protocol becomes more efficient than the two-server KOY protocol as it needs only two rounds instead of three and does not need correctness proofs in the distributed hash and projection protocols.

% \mynote{DSPHF ACNS'14 \cite{KieferM14b}}

\section{UC Security for Two-Server PAKE}
\mynote{move this to the beginning of 2PAKE stuff and leave only UC stuff here (maybe)}
Password Authenticated Key Exchange (PAKE) protocols have been extensively researched over the last twenty years.
They allow two protocol participants, holding a low-entropy secret (password) each, to negotiate an authenticated session key.
Several security models have been developed including the well-known game-based notion from Bellare, Pointcheval and Rogaway \cite{Bellare2000,Abdalla2005a} and a notion in the universal composability (UC) framework \cite{Canetti2005}.
While PAKE protocols can be executed between two humans holding the same password, they are usually considered in a client, server scenario where the client registers with a server that then stores the password and uses it in subsequent sessions to authenticate the client.
This approach however leads to an intrinsic weakness regarding server compromise.
As soon as a server, storing client passwords is compromised the attacker learns the passwords.
This allows the attacker to log into the client's account on the server and most likely also on others if the client re-used the password across multiple servers.
Mechanisms have been proposed to solve the problem of server compromise.
In particular verifier-based PAKE \cite{Gentry2006,rfc2945,BenhamoudaP13}, also known as augmented PAKE \cite{BellovinM93}, considers an asymmetric setting in which the server uses a function of the password (verifier) to verify a client holding the corresponding password.
However, as long as only one server is used, PAKE protocols are prone to offline dictionary attacks on the server side, i.e. server attacks still leak password verifiers that allow to recover the password, which is rather efficiently with current methods \cite{hashcat,johntheripper}.

Two-server PAKE (2PAKE) protocols solve this problem by splitting the password in two parts such that a malicious or compromised server can only recover a password share that does not allow to recover the password.
In contrast to PAKE protocols 2PAKE protocols are less well studied.
2PAKE can be seen as a special case for $t=n=2$ of threshold PAKE where $t$ out of $n$ servers, participating in the protocol and holding a password share, must be honest.
Raimondo and Gennaro \cite{Raimondo_Gennaro_2003} and MacKenzie, Shrimpton and Jakobson \cite{MacKenzie_Shrimpton_Jakobsson_2002} were among the firsts to propose t-out-of-n password authenticated key exchange protocols.
The former is not suitable for 2PAKE as it needs $t<n/3$ while the latter requires a PKI in addition to the password.
The first real two-server PAKE protocol is due to Brainard and Jules \cite{Brainard_Juels_2003}, which was proven secure by Szydlo and Kaliski \cite{SzydloK05} in a modified version.
The first two server PAKE with thorough security model based on the classical game-based BPR model is due to Katz, MacKenzie,  Taban and Gligor \cite{KatzMTB05}, which was recently generalised to a two-server PAKE framework by Kiefer and Manulis \cite{KieferM14a}.
Threshold and two-server password authenticated key exchange is closely related to password authenticated secret sharing (PASS).
Password authenticated secret sharing was first proposed as password protected secret sharing by Bagherzandi, Jarecki, Saxena and Lu in 2011 \cite{Bagherzandi2011} and gained a lot of attention since then \cite{Camenisch_Lysyanskaya_Neven_2012,JareckiKK14,CamenishEN15}.
As shown in \cite{JareckiKK14} PASS can also be used to implement efficient threshold PAKE protocols.
While using the universally composability framework to prove security of the PASS primitive, none of these papers give a threshold or two-server PAKE protocol secure in UC.

In this work we propose the first notion of UC security for two-server PAKE protocols and give an efficient protocol in the standard model with common reference string and static corruptions.
To this end we leverage recent advancements of smooth projective hashing in the two-server setting from \cite{KieferM14a} and in efficiency from \cite{Benhamouda2013}.

\subsection{Two-Server Password Authenticated Key Exchange}
Before diving into technical preliminaries in the next section we want to discuss two-server PAKE and security guarantees it should provide.
We discuss informally several security requirements for 2PAKE protocols, which is the basis for the ideal functionality \FTWOPAKE defined later.
% Depending on the 2PAKE protocol not all of them can be achieved.
First, note that the adversary has full control over the communication channel between the client and the servers as usual.
% \emph{Security against offline dictionary attacks}
The first and foremost security requirement of password protocols is security against \emph{offline dictionary attacks} and therefore has to be fulfilled by 2PAKE protocols as well.
In particular, no eavesdropping adversary must be able to perform an offline dictionary attack on the exchanged messages.
% \emph{Security against impersonation attacks}
Further, a malicious server must not be able to \emph{impersonate a registered client} in a 2PAKE execution.
An attacker, even with knowledge and capabilities of one of the two servers, must have success probability that is not significantly better than the success probability of a brute-force attacker when running a 2PAKE protocol on behalf of a registered client.
% \emph{Security against malicious server}
The BPR game-based security notion for PAKE and two-server PAKE, which is derived from the AKE security notion, captures security by testing whether an attacker is able to distinguish between a real session key generated by a (two-server) PAKE protocol, and a randomly chosen session key.
This implies in particular that the client agrees on two independent session keys with the two servers in the two-server PAKE setting.
UC-security in contrast requires simulatability of a (two-server) PAKE protocol and is therefore harder to instantiate because the protocol has to be simulated in the case the adversary is able to guess the correct password (in which case the game-based model simply aborts the protocol execution and declares the adversary won the game).

% The strongest requirement on a 2PAKE protocol is the security against a malicious server.
% Here, a malicious server, or attacker controlling the server, must not be able to distinguish the session key between the client and the other server.
% The malicious server may be active or passive in its activities.
% Note again that not all server necessarily compute keys.

We want to further elaborate the difference between 2PAKE protocols where both servers compute the same key, compared to a 2PAKE protocol where the servers compute different keys.
Note that asymmetric key generation may also imply that only one server calculates a key while the second server only assists in the computation.
In the symmetric setting both servers calculate the same session key as result of the 2PAKE protocol.
Even though it is the first natural extension to the single server PAKE scenario, to the best of our knowledge no such protocol has been proposed yet.
This may be due to the fact that corruption of a single server compromises the session key of any execution of a 2PAKE protocol that involves this server.
In the asymmetric setting both servers generate different session keys as result of the 2PAKE protocol, possibly none.
Katz et al. proposed the first password-only 2PAKE provably secure in the standard model, based on the Katz-Ostrovsky-Yung (KOY) protocol \cite{Katz_Ostrovsky_Yung_2001} in 2005 \cite{KatzMTB05}.
While their protocol is symmetric in its execution the client computes two independent session keys, one with each server.
Other asymmetric 2PAKE protocols have been proposed \cite{Yang_Deng_Bao_2006,Jin_Wong_Xu_2007} where the server interacts with only one server.
% In this setting the second server may not even interact with the client or compute a session key.
% Before discussing these possible 2PAKE classes and their security properties we give an informal overview on security requirements on 2PAKE protocols.
% \emph{Symmetric Key Generation}
% Key generation in 2PAKE protocols however can have different forms.
% We distinguish here between \emph{symmetric} and \emph{asymmetric} key generation.
% In this case 2PAKE offers resistance against offline dictionary attacks by the server and server compromise attacks of a single server that leak the server's password database.
% However, an attacker controlling one of the two servers calculates the same session key as the honest server and the client.
% A 2PAKE protocol secure in this setting can only protect against impersonation attacks, and password database leaks while they can not prevent malicious servers from attacking it.
% \emph{Asymmetric Key Generation}
In this work we stick to the asymmetric setting where only one server computes a session key.
This however can be easily extended to a 2PAKE protocol that computes an independent session keys for each server.
% However, instead of using a game-based model such as the BPR derivatives, we propose an ideal functionality for the UC framework to show security of our 2PAKE protocol.
To this end we give the first definition of a 2PAKE UC functionality.
It brings all benefits UC-security carries in the PAKE setting such as universal composability, security holds with arbitrary (related) password choices, security on execution with non-matching password (shares), etc.
For a comprehensible overview of advantages using UC in the PAKE setting we refer to \cite{Canetti2005}.


%==============================================================================
% Trapdoor D-SPHF
%==============================================================================


\subsection{Trapdoor D-SPHF}\label{sec:tdsphf}
The 2PAKE protocol we propose in the following section makes use of trapdoor smooth projective hashing in order to allow efficient instantiation.
Trapdoor SPHFs enable single round PAKE protocols because they achieve simulatability even in the case where the attacker guessed the correct password.
To use \TSPHF in the two-server setting we extend the notion of trapdoor SPHFs (cf. Section \ref{sec:tsphf}) to the distributed setting of D-SPHFs (cf. Section \ref{sec:dsphf}).
While our description is specific to the two-server PAKE setting with languages over Cramer-Shoup ciphertexts an extension to more servers and other languages is straightforward along the lines of \DSPHF.

\begin{definition}[TD-SPHF]\label{def:tdsphf}
Let $L_{\widehat{\pwd}}$ denote a language such that $C=(C_0,C_1,C_2)\in L_{\widehat{\pwd}}$ if there exists a witness $r=(r_0,r_1,r_2)$ proving so, $\pwd=\pwd_1+\pwd_2$ and there exists a function $\Dec'$ such that $\Dec'(C_1 C_2)=\Dec'(C_0)$.
% and $g:\GG^l\mapsto\GG^{l'}$ as described above.
A trapdoor distributed smooth projective hash function for language $L_{\widehat{\pwd}}$ consists of the following ten algorithms:

\begin{itemize}
	\item $(\crs',\tau')\ralgout\TSetup(\crs)$ generates $\crs'$ with trapdoor $\tau'$ from \crs

	\item \HKGen, \PKGen, $\Hash_x$, $\PHash_x$, $\Hash_0$, $\PHash_0$ behave as for D-SPHF
	
	\item $b\algout\VerHp(\hp,L_{\widehat{\pwd}})$ returns $b=1$ iff \hp is a valid projection key and $b=0$ otherwise
	
	\item $h_x\algout\THash_x(\hp_0,L_{\widehat{\pwd}},C_1,C_2,\tau')$ computes hash value $h_x$ of ciphertexts $C_1$ and $C_2$ using projection key $\hp_0$ and trapdoor $\tau'$
	
	\item $h_0\algout\THash_0(\hp_1,\hp_2,L_{\widehat{\pwd}},C_0,\tau')$ computes hash value $h_0$ of $C_0$ using projection keys $\hp_1$ and $\hp_2$, and trapdoor $\tau'$
\end{itemize}
\end{definition}


% \subsubsection{TD-SPHF Security}\label{sec:tdsphf-security}
\noindent
Security of \TDSPHF can be derived from \DSPHF security and the extensions made on SPHFs for T-SPHFs.
However, we do not consider security of \TDSPHF on its own but rather incorporate it in the security proof of the 2PAKE protocol in the following section.
This is due to the fact that description of \TDSPHF is done only for this specific application such that a separate security definition is more distracting than giving any benefit.
However, we define correctness and soundness of \TDSPHF since they differ from that of \DSPHF.
In particular, \emph{correctness} of TD-SPHFs extends correctness of D-SPHFs by the statement that for every  valid ciphertext triple $(C_0,C_1,C_2)$, generated by \cL, and honestly generated keys $(\hk_0,\hk_1,\hk_2)$ and $(\hp_0,\hp_1,\hp_2)$, it holds not only that 
\begin{align*}  
&\Hash_0(\hk_1, \hk_2, L_{\widehat{\pwd}}, C_0)=\PHash_0(\hp_1, \hp_2, L_{\pwd, \pwd_1, \pwd_2}, C_0, r_0) \text{ and} \\
&\Hash_x(\hk_0, L_{\widehat{\pwd}}, C_1, C_2)=\PHash_x(\hp_0, L_{\pwd, \pwd_1, \pwd_2}, C_1, C_2, r_1, r_2)
\end{align*}
but also that $\VerHp(\hp_i, L_{\widehat{\pwd}})=1$ for $i\in\{0,1,2\}$ and 
\begin{align*}
&\Hash_0(\hk_1, \hk_2, L_{\widehat{\pwd}}, C_0)=\THash_0(\hp_1, \hp_2, L_{\pwd, \pwd_1, \pwd_2}, C_0, \tau') \text{ and} \\
&\Hash_x(\hk_0, L_{\widehat{\pwd}}, C_1, C_2)=\THash_x(\hp_0, L_{\pwd, \pwd_1, \pwd_2}, C_1, C_2, \tau').
\end{align*}
To capture soundness of TD-SPHFs we define \emph{$(t,\varepsilon)$-soundness}, complementing the previous correctness extension, as follows.

\begin{definition}[TD-SPHF $(t,\varepsilon)$-soundness]\label{def:tdsphf-soundness}
Given \crs, $\crs'$ and $\tau$, no adversary running in time at most $t$ can produce a projection key \hp, a password \pwd with shares $\pwd_1$ and $\pwd_2$, a word $(C_0,C_1,C_2)$, and valid witness $(r_0,r_1,r_2)$, such that $(\hp_0,\hp_1,\hp_2)$ are valid, i.e. $\VerHp(\hp_i,L_{\widehat{\pwd}})=1$ for $i\in\{0,1,2\}$, but 
\begin{align*}
&\THash_x(\hp_0,L_{\widehat{\pwd}},C_1, C_2, \tau')\not=\PHash_x(\hp_0,L_{\widehat{\pwd}}, C_1, C_2, r_1, r_2) \text{ or} \\
&\THash_0(\hp_1, \hp_2, L_{\widehat{\pwd}}, C_0, \tau')\not=\PHash_0(\hp_1, \hp_2, L_{\widehat{\pwd}}, C_0, r_0)
\end{align*}
with probability at least $\varepsilon(\secpar)$.
The perfect soundness states that the property holds for any $t$ and any $\varepsilon(\secpar)>0$.
\end{definition} 

\subsubsection{Cramer-Shoup TD-SPHF}\label{sec:cs-tdsphf}
Extending the Cramer-Shoup \DSPHF (cf. Appendix \ref{app:dsphf}) to a \TDSPHF is straight-forward combining it with the Cramer-Shoup \TSPHF.
This \TDSPHF is, like the Cramer-Shoup \TSPHF (cf. Appendix \ref{app:tsphf}), defined over groups with bilinear pairings (cf. Section \ref{sec:preliminaries}).
Let $C=(\ell, u_1, u_2, e, v)$ denote a Carmer-Shoup ciphertext as defined in Section \ref{sec:csencryption}.
% We only describe the differences to the previous CS \DSPHF.
% $\pk=(p,\GG_1,\GG_2,\GG_T,e,g_{11},g_{12},c,f,h,H_k)$ and $c=g_{11}^{x_1}g_{12}^{x_2}, d=g_{11}^{y_1}g_{12}^{y_2}, h=g_{11}^z$ s.t. $dk=(x_1,x_2,y_1,y_2,z)$

\begin{itemize}
	\item $\TSetup(\crs)$ draws a random $\tau'\rin\ZZ_q$ and computes $\crs'=\zeta=g_2^{\tau'}$
	
	\item $\HKGen(L_{\widehat{\pwd}})$ returns $\hk_i=(\eta_{1,i},\eta_{2,i},\theta_i,\mu_i,\nu_i)\rin\ZZ_p^{1\times 5}$ for $i\in\{0,1,2\}$
	
  \item $\PKGen(\hk_i, L_{\widehat{\pwd}})$ generates 
      $\hp_i=(\hp_{1,i}=g_{1,1}^{\eta_{1,i}} g_{1,2}^{\theta_i} h^{\mu_i} c^{\nu_i}, \hp_{2,i}=g_{1,1}^{\eta_{2,i}} d^{\nu_i}, \hp_{3,i})$ with
      $\hp_{3,i}=(\chi_{1,1,i}, \chi_{1,2,i}, \chi_{2,i}$, $\chi_{3,i}, \chi_{4,i})$ and 
      $\chi_{1,1,i}={\zeta}^{\eta_{1,i}}, \chi_{1,2,i}={\zeta}^{\eta_{2,i}}, \chi_{2,i}={\zeta}^{\theta_i}, \chi_{3,i}={\zeta}^{\mu_i}, \chi_{4,i}={\zeta}^{\nu_i}$ for $i\in\{0,1,2\}$
  
  \item $\Hash_x(\hk_0, L_{\widehat{\pwd}}, C_1, C_2)$ computes 
      $h'_x=(u_{1,1}\cdot u_{1,2})^{\eta_{1,0}+(\xi_1+\xi_2)\eta_{2,0}} (u_{2,1}\cdot u_{2,2})^{\theta_0} ((e_1\cdot e_2)/g_{1,1}^{\pwd})^{\mu_0} (v_1\cdot v_2)^{\nu_0}$
      and returns $h_x=e(h'_x,g_2)$
  \item $\PHash_x(\hp_0, L_{\widehat{\pwd}}, C_1, C_2, r_1, r_2)$ computes 
      $h'_x=\hp_{1,0}^{r_1+r_2} \hp_{2,0}^{\xi_1 r_1+\xi_2 r_2}$
      and outputs $h_x=e(h'_x,g_2)$
  \item $\Hash_0(\hk_1, \hk_2, L_{\widehat{\pwd}}, C_0)$ computes 
      $h'_0=u_{1,0}^{\eta_{1,1}+\eta_{1,2}+\xi_0(\eta_{2,1}+\eta_{2,2})} u_{2,0}^{\theta_1+\theta_2} (e_0/g_{1,1}^{\pwd})^{\mu_1+\mu_2} v_0^{\nu_1+\nu_2}$
      and outputs $h_0=e(h'_0,g_2)$
  \item $\PHash_0(\hp_1, \hp_2, L_{\widehat{\pwd}}, C_0, r_0)$ computes $h'_0=(\hp_{1,1}\hp_{1,2})^{r_0} (\hp_{2,1}\hp_{2,2})^{r_0\xi_0}$
      and outputs $h_0=e(h'_0,g_2)$
	
  \item $\VerHp(\hp_i, L_{\widehat{\pwd}})$ verifies that 
      $e(\hp_{1,i}, \crs')\stackrel{?}{=} e(g_{1,1}, \chi_{1,1,i})\cdot e(g_{1,2}, \chi_{2,i})\cdot e(h, \chi_{3,i})\cdot e(c, \chi_{4,i})$ and
      $e(\hp_{2,i}, \crs')$ $\stackrel{?}{=}$ $e(g_{1,1}$, $\chi_{1,2,i})\cdot e(d, \chi_{4,i})$ for $i\in\{0,1,2\}$

  \item $\THash_0(\hp_1, \hp_2, L_{\widehat{\pwd}}, C_0, \tau')$ computes 
      \[h_0=\left[ e(u_{1,0}, \chi_{1,1,1}\chi_{1,1,2}(\chi_{1,2,1}\chi_{1,2,2})^{\xi_0})\cdot e(u_{2,0}, \chi_{2,1}\chi_{2,2})\cdot e(e_0/g_{1,1}^{\pwd}, \chi_{3,1}\chi_{3,2})\cdot e(v_0, \chi_{4,1}\chi_{4,2}) \right]^{1/\tau'}\]
			
	\item $\THash_x(\hp_0, L_{\widehat{\pwd}}, C_1, C_2,\tau')$ computes 
    	\[h_x=\left[ e(u_{1,1} u_{1,2}, \chi_{1,1,0}\chi_{1,2,0}^{\xi_1+\xi_2}) \cdot e(u_{2,1} u_{2,2}, \chi_{2,0}) \cdot e( (e_1 e_2) / g_{1,1}^{\pwd}, \chi_{3,0}) \cdot e( v_{1} v_2, \chi_{4,0})\right]^{1/\tau'}\]
\end{itemize}

\noindent
Distributed computation of $\PHash_x$ and $\Hash_0$ is done as in \DSPHF with additional proofs for correctness and adding the pairing computation at the end to lift the hash value into $\GG_T$.
We formalise execution of the Cramer-Shoup \TDSPHF in Figure \ref{fig:cs-tdsphf}.
Necessary zero-knowledge proofs are described in the following two paragraphs and only referenced in Figure \ref{fig:cs-tdsphf}.
We describe the $\Sigma$ protocol here, which is then used in the committed version (cf. Section \ref{sec:zk}).
While the constructions are $\Sigma$ proofs (and therefore honest verifier zero-knowledge proofs of knowledge) we regard them as zero-knowledge proofs (without knowledge extractor) in order to avoid the necessity of rewinding the prover.
% Note that we merge \crs and $\crs'$ here for readability.
Protocol participants are denoted $C$, $S_1$ and $S_2$ if their role is specified, or $P$, $Q$ and $R$ otherwise.
Let further $0$ denote the client's index and $1$, respectively $2$, denote the indices of server $S_1$, $S_2$ respectively.
The session ID is given by $\sid=C||S_1||S_2$ and the unique query identifier \qid is agreed upon start using \Finit.

All \TDSPHF participants have $\crs=(q,g_{1,1},g_{1,2},h,c,d,\GG_1,g_{2},\zeta,\GG_2,\GG_T,e,H_k)$ as common input where $\tau=(x_1,x_2,y_1,y_2,z)$ is the \crs trapdoor, i.e. the according Cramer-Shoup secret key, and $\tau'$ the trapdoor, i.e. discrete logarithm to base $g_2$, of $\crs'=\zeta$.
Each server holds an ElGamal key pair $(\pk_1, \sk_1)$ and $(\pk_2, \sk_2)$ respectively such that $\pk_1$ is registered with the \CA for $S_1$ and $\pk_2$ for $S_2$ and thus available to all parties (using \Fca).
An, otherwise unspecified, protocol participant $P$ is initiated with $(\NS, \sid, \qid, P, x)$.
We further define $\pwd_0=\pwd$.
%use $P_i$ for party $i\in[1,2,3]$ and
% When $P$ is activated with a NewSession query $(\NS, \sid, \qid, P, x)$ it proceeds as follows.


\begin{figure}[htbp]
\begin{mdframed}[innertopmargin=10pt]
\begin{center}
\caption{Cramer-Shoup \TDSPHF}
\label{fig:cs-tdsphf}
% {\bf Cramer-Shoup \TDSPHF}\smallskip
\end{center}

% \begin{description}\itemsep5pt
	
% 	\item[NewSession:] When $P_i$ is activated with $(\NS, \sid, \qid, P_i, x, \role)$ it proceeds as follows.\smallskip
	  \begin{enumerate}
	    \item Generate \TDSPHF keys $\hk_i\rin\ZZ_q^{5}$ and 
	      $\hp_i=(\hp_{1,i}=g_{1,1}^{\eta_{1,i}} g_{1,2}^{\theta_i} h^{\mu_i} c^{\nu_i}, \hp_{2,i}=g_{1,1}^{\eta_{2,i}} d^{\nu_i}, \chi_{1,1,i}={\zeta}^{\eta_{1,i}}, \chi_{1,2,i}={\zeta}^{\eta_{2,i}}, \chi_{2,i}={\zeta}^{\theta_i}, \chi_{3,i}={\zeta}^{\mu_i}, \chi_{4,i}={\zeta}^{\nu_i})$. 
	    Encrypt $\pwd_i$ to $C=(\ell_i,u_{1,i},u_{2,i},e_i,v_i)\gets(\ell, g_{1,1}^{r_i}, g_{1,2}^{r_i}, h^{r_r} g_{1,1}^{\pwd_i}, (cd^{\xi_i})^{r_i})$ with $\xi_i=H_k(\ell_i, u_{1,i}, u_{2,i}, e_i)$ for $\ell_i=\sid || \qid || \hp_i$ and $r_i\rin\ZZ_q$.
	    If $P=S_1$, set $h_0=h_x=\NULL$.
	    Output $(\sid,\qid,0,P,C_i,\hp_i)$ to $Q$ and $R$.
	    
	    \item When $P$, waiting for the initial messages, is receiving a message $(\sid,\qid,0,Q,C_1,\hp_1)$ and $(\sid,\qid,0,R,C_2,\hp_2)$ it proceeds as follows.
	      $P$ proceeds only if the projection keys $\hp_1$ and $\hp_2$ are correct, i.e. $\VerHp(\hp_1, L_{\widehat{\pwd}})=1$ and $\VerHp(\hp_2, L_{\widehat{\pwd}})=1$.
	          If the verification fails, $P$ outputs $(\sid, \qid, \bot, \bot)$ and aborts the protocol.
	      \begin{enumerate}
	        \item If $P=C$, compute \\
	          $ h_x=e\left((u_{1,1}\cdot u_{1,2})^{\eta_{1,0}+(\xi_1+\xi_2)\eta_{2,0}} (u_{2,1}\cdot u_{2,2})^{\theta_0} ((e_1\cdot e_2)/g_{1,1}^{\pwd})^{\mu_0} (v_1\cdot v_2)^{\nu_0}, g_2\right) $ and \\
	          $ h_0=e\left( (\hp_{1,1}\hp_{1,2})^{r_0} (\hp_{2,1}\hp_{2,2})^{r_0\xi_0}, g_2 \right)$,
	          and outputs $(\sid, \qid, h_0, h_x)$.\\
	        \item If $P=S_2$, compute
	          $h_{x,2} = (\hp_{1,0} \cdot \hp_{2,0}^{\xi_2})^{r_2}$ and 
	          $C_{h_{x,2}}=g_{1,1}^{H(h_{x,2}, \Comm_{1})} h^{r_{c1}}$ with $r_{c1}\rin\ZZ_q$
% 	          $t_1=(\hp_{0,1}\hp_{0,2}^{\xi_2})^{k_1}$ and
% 	          $t_2=(cd^{\xi_2})^{k_1}$,
	          and send $(\sid,\qid,\PHash_x,0,S_2,C_{h_{x,2}})$ to $S_1$.
	        \item If $P=S_1$, compute
	          $m_{0} = \Enc_{\pk_1}^\EG(g_{1,1}^{-\mu_1}; r)$ and
	          $c_{0} = \Enc_{\pk_1}^\EG(g_{1,1}^{\pwd_1}; r')$
	          with $r,r'\rin\ZZ_q$, and send $(\sid,\qid,\Hash_0,0,S_1,m_0,c_0)$ to $S_2$.
	      \end{enumerate}
	      
% 	  \end{enumerate}
%   \item[\hspace*{1em}] {\it Computation of $\PHash_x$ between $S_1$ and $S_2$.}\smallskip
%     \begin{enumerate} \setcounter{enumi}{2}
    
	    \item On input $(\sid,\qid,\PHash_x,0,S_2,C_{h_{x,2}})$ $S_1$ in the correct state draws challenge $\fc\rin\ZZ_q$ and returns $(\sid,\qid,\PHash_x,1,S_1,\fc)$ to $S_2$.
	    
	    \item On input $(\sid,\qid,\PHash_x,1,S_1,\fc)$ $S_2$ in the correct state computes
	    $C_{s_{h_{x,2}}}=g_{1,1}^{H(\Res_{1})} h^{r_{c2}}$ with $r_{c2}\rin\ZZ_q$
% 	    $s_{h_{x,2}}=k_1-c r_{2}$ 
	    and sends $(\sid,\qid,\PHash_x,2,S_2,C_{s_{h_{x,2}}})$ to $S_1$.
	    Subsequently, it sends $(\sid,\qid,\PHash_x,3,S_2,h_{x,2}, \Comm_1, \Res_1, r_{c1}, r_{c2})$ to $S_1$.
	    %t_1, t_2, s_{h_{x,2}}
	    
	    \item On input $(\sid,\qid,\PHash_x,2,S_2,C_{s_{h_{x,2}}})$ $S_1$ in the correct state stores it and waits for the final $\PHash_x$ message.
	    
	    \item On input $(\sid,\qid,\PHash_x,3,S_2,h_{x,2}, \Comm_1, \Res_1, r_{c1}, r_{c2})$ $S_1$ in the correct state parses $\Comm_1$ as $(t_1,t_2)$ and $\Res_2$ as $s_{h_{x,2}}$ and verifies correctness of commitments and the \ZKP
% 	    $C_{h_{x,2}}\stackrel{?}{=}g_{1,1}^{H(h_{x,2}, t_1, t_2)} h^{r_{c1}}$,
% 	    $C_{s_{h_{x,2}}}\stackrel{?}{=}g_{1,1}^{H(s_{h_{x,2}})} h^{r_{c2}}$,
% 	    $t_1\stackrel{?}{=}h_{x,2}^{c} (\hp_{0,1}\hp_{0,2}^{\xi_2})^{s_{h_{x,2}}}$, and
% 	    $t_2\stackrel{?}{=}v_2^{c} (cd^{\xi_2})^{s_{h_{x,2}}}$,
	     and computes
 	    $h_x = e\left( h_{x,2}\cdot (\hp_{0,1} \cdot \hp_{0,2}^{\xi_1})^{r_1}, g_2 \right)$
 	    if the verifications are successful, $h_x\not=\bot$ and $h_0\not=\bot$, or sets $h_0=\bot$ and $h_x=\bot$ otherwise.
	    
% 	  \end{enumerate}
%   \item[\hspace*{1em}] {\it Computation of $\Hash_0$ between $S_1$ and $S_2$.}\smallskip
%     \begin{enumerate} \setcounter{enumi}{6}
	    
	    \item On input $(\sid,\qid,\Hash_0,0,S_1,m_0,c_0)$ $S_2$ in the correct state retrieves $\pk_1$ from \Fca and computes
	      $C_{\Hash_{0,1}}=g_{1,1}^{H(m_{1},m_{2},\Comm_2)} h^{r_{c3}}$
	      with $r_{c3}\rin\ZZ_q$,
	      $m_{1}\gets m_0^{\pwd_2} \times c_0^{-\mu_2} \times \Enc_{\pk_1}^\EG(g_{1,1}^{-\mu_2\cdot {\pwd_2}} \cdot u_{1,0}^{\eta_{1,2}+\xi_0\eta_{2,2}} \cdot u_{2,0}^{\theta_2} \cdot e_0^{\mu_2} \cdot v_0^{\nu_2}; r'')$, and $m_{2}\gets \Enc_{\pk_1}^\EG(g_{1,1}^{-\mu_2}; r''')$ with $r'',r'''\in\ZZ_q$,
	      and sends $(\sid,\qid,\Hash_{0,1},S_2,C_{\Hash_{0,1}})$ back to $S_1$.
	      
	    \item On input $(\sid,\qid,\Hash_{0,1},S_2,C_{\Hash_{0,1}})$ $S_1$ in the correct state draws challenge $\fc\rin\ZZ_q$ and returns $(\sid,\qid,\Hash_{0,2},S_1,\fc)$ to $S_2$. 
	    
	    \item On input $(\sid,\qid,\Hash_{0,2},S_1,\fc)$ $S_2$ in the correct state computes
	    $C_{\Res2}=g_{1,1}^{H(\Res_{2})} h^{r_{c4}}$ with $r_{c4}\rin\ZZ_q$
	    and sends $(\sid,\qid,\Hash_{0,3},S_2,C_{\Res2})$ to $S_1$.
	    Subsequently, it sends $(\sid,\qid,\Hash_{0,4},S_2, m_1, m_2, \Comm_2, \Res_2, r_{c3}, r_{c4})$ to $S_1$.
	    
	    \item On input $(\sid,\qid,\Hash_{0,4},S_2, m_1, m_2, \Comm_2, \Res_2, r_{c3}, r_{c4})$ $S_1$ in the correct state parses $\Comm_{2}$ as $(t_{\overline{m}1}, t_{\overline{m}2}, t_{e2}, t_{v2}, t_{\hp12}, t_{\hp22})$ and $\Res_{2}$ as $(s_{\pwd_2}, s_{\mu2}, s_{\eta12}, s_{\eta22}, s_{\theta2}, s_{\nu2}, s_{r2})$, verifies correctness of commitments and \ZKP , and computes
	      $h_0 = e\left( g_{1,1}^{-\mu_1\cdot \pwd_1} \cdot \Dec_{\sk_1}^\EG(m_1) \cdot u_{1,0}^{\eta_{1,1}+\xi_0\eta_{2,1}} \cdot u_{2,0}^{\theta_1} \cdot e_0^{\mu_1} \cdot v_0^{\nu_1}, g_2 \right)$ if the verifications are successful, $h_x\not=\bot$ and $h_0\not=\bot$, or sets $h_0=\bot$ and $h_x=\bot$.
	    
	    \item Eventually $S_1$ outputs $(\sid, \qid, h_0, h_x)$ if $h_0\not=\NULL$ and $h_x\not=\NULL$.
	  \end{enumerate}
	  	
% \end{description}
\end{mdframed}
\end{figure}

\paragraph{ZK Proof for $\PHash_x$ Correctness}
In order to ensure correct computation of $h_x$ on $S_1$ server $S_2$ has to prove correctness of his computations.
To this end $S_2$ sends, in addition to the $\PHash_x$ message $h_{x,2}$ the following zero-knowledge proof.
\begin{equation}
    \ZKP\big\{ (r_2):~~ h_{x,2} = (\hp_{1,0} \hp_{2,0}^{\xi_2})^{r_2} ~ \wedge ~ v_2=(cd^{\xi_2})^{r_2} \big\}
\end{equation}
where $r_2$ is the randomness used to create $C_2$, $\xi_2$ and $v_2$ are part of $C_2$, $\hp_{1,0},\hp_{2,0}$ are part of $C$'s projection key, and $c,d$ are from the \crs.
The construction of the according zero-knowledge proof is straight-forward. 
The prover computes commitments
\[ t_{hx2} = (\hp_{1,0} \hp_{2,0}^{\xi_2})^{k_{hx2}}; ~~ t_{v2} = (cd^{\xi_2})^{k_{hx2}} \]
with fresh randomness $k_{hx2}\rin\ZZ_q$, and response $s_{r2} = k_{hx2} - \fc r_2$ for verifier provided challenge $\fc$.
This allows the verifier to check
\[ t_{hx2} \stackrel{?}{=} h_{x,2}^\fc (\hp_{1,0} \hp_{2,0}^{\xi_2})^{s_{hx2}}; ~~ t_{v2} \stackrel{?}{=} v_2^\fc (cd^{\xi_2})^{s_{hx2}}. \]
It is easy to see that this zero-knowledge proof is correct, sound and (honest-verifier) simulatable.
In Figure \ref{fig:cs-tdsphf} we refer to the messages as $\Comm_{1}=(t_{hx2}, t_{v2})$, $\Res_{1}=s_{r2}$, and $\Ch_{1}=\fc$.

\paragraph{ZK Proof for $\Hash_0$ Correctness}
Let $\overline{m}_1$ and $\overline{m}_2$ denote the messages encrypted in $m_1$ and $m_2$ respectively and $m_{0,1}$ and $c_{0,1}$ the second part ($e$) of the ElGamal ciphertext $m_0$, $c_1$ respectively.
In order to ensure correct computation of $h_0$ on $S_1$ server $S_2$ has to prove correctness of his computations.
To this end $S_2$ sends, additionally to the $\Hash_0$ messages $\overline{m}_1$ and $\overline{m}_2$ the following zero-knowledge proof 
\begin{equation}
\begin{split}
    \ZKP\big\{ (x, \eta_{1,2}, \eta_{2,2}, \theta_2, \mu_2, \nu_2, r_2): ~ &  
      \overline{m}_1 = m_{0,1}^{\pwd_2} c_{0,1}^{-\mu_2} g_{1,1}^{-\mu_2 x} u_{1,0}^{\eta_{1,2} + \xi_0 \eta_{2,2}} u_{2,0}^{\theta_2} e_0^{\mu_2} v_0^{\nu_2} \\
      % ~ \wedge ~ m_2 = g_{1,1}^x
      & \wedge ~ \overline{m}_2 = g_{1,1}^{-\mu_2} ~ \wedge ~ e_2 = h^{r_2}g_{1,1}^{\pwd_2} ~ \wedge ~ v_2 = (cd^{\xi_2})^{r_2} \\
      & \wedge ~ \hp_{1,2} = g_{1,1}^{\eta_{1,2}} g_{1,2}^{\theta_2} h^{\mu_2} c^{\nu_2} ~ \wedge ~ \hp_{2,2} = g_{1,1}^{\eta_{2,2}} d^{\nu_2}
      \big\},
\end{split}
\end{equation}
where $r_2$ is the randomness used to create $C_2$, $\xi_2$ and $v_2$ are part of $C_2$, $\xi_0$ is part of $C_0$, $(\mu_2, \eta_{1,2}, \eta_{2,2}, \theta_2, \nu_2)$ is $S_2$'s hashing key, $\pwd_2$ $S_2$'s password share, and $c,d$ are from the \crs.
%variables as defined in Figure \ref{fig:cs-tdsphf}.
The construction of the according $\Sigma$ proof is straight-forward.
The prover computes commitments
\begin{align*}
  & t_{\overline{m}1} = m_{0,1}^{{\pwd_2}} c_{0,1}^{k_{\mu2}} \overline{m}_{2}^{k_{x}} u_{1,0}^{k_{\eta12} + \xi_0 k_{\eta22}} u_{2,0}^{k_{\theta2}} e_0^{-k_{\mu2}} v_0^{k_{\nu2}}; ~~
    t_{\overline{m}2} = g_{1,1}^{k_{\mu2}}; ~~  t_{e2} = h^{k_{r2}}g_{1,1}^{{\pwd_2}}; ~~ t_{v2} = (cd^{\xi_2})^{k_{r2}}; \\
  & t_{\hp12} = g_{1,1}^{k_{\eta12}} g_{1,2}^{k_{\theta2}} h^{k_{\mu2}} c^{k_{\nu2}}; ~~ t_{\hp22} = g_{1,1}^{k_{\eta22}} d^{k_{\nu2}}
    \text{ ~for~ } k_{\pwd_2}, k_{\mu2}, k_{\eta12}, k_{\eta22}, k_{\theta2}, k_{\nu2} \rin \ZZ_q
\end{align*}
and responses
\begin{align*}
  & s_{\pwd_2} = k_{\pwd_2} - \fc\pwd_2; ~~ s_{\mu2} = k_{\mu2} + \fc\mu_2; ~~ s_{\eta12} = k_{\eta12} - \fc\eta_{1,2}; ~~ s_{\eta22} = k_{\eta22} - \fc\eta_{2,2}; \\
  & s_{\theta2} = k_{\theta2} - \fc\theta_2; ~~ s_{\nu2} = k_{\nu2} - \fc\nu_2; ~~ s_{r2} = k_{r2} - \fc r_2
\end{align*}
for verifier provided challenge $\fc$.
This allows the verifier to check
\begin{align*}
  & t_{\overline{m}1} \stackrel{?}{=} \overline{m}_1^\fc m_{0,1}^{s_{\pwd_2}} c_{0,1}^{s_{\mu2}} \overline{m}_{2}^{s_{\pwd_2}} u_{1,0}^{s_{\eta12} + \xi_0 s_{\eta22}} u_{2,0}^{s_{\theta2}} e_0^{s_{\mu2}} v_0^{s_{\nu2}}; ~~ 
    t_{\overline{m}2} \stackrel{?}{=} \overline{m}_2^\fc g_{1,1}^{s_{\mu2}}; ~~  t_{e2} \stackrel{?}{=} e_2^\fc h^{s_{r2}}g_{1,1}^{s_{\pwd_2}}; \\
  & t_{v2} \stackrel{?}{=} v_2^\fc (cd^{\xi_2})^{s_{r2}}; ~~
    t_{\hp12} \stackrel{?}{=} \hp_{1,2}^\fc g_{1,1}^{s_{\eta12}} g_{1,2}^{s_{\theta2}} h^{s_{\mu2}} c^{s_{\nu2}}; ~~ 
    t_{\hp22} \stackrel{?}{=} \hp_{2,2}^\fc g_{1,1}^{s_{\eta22}} d^{s_{\nu2}} .
\end{align*}
While this is mainly a standard zero-knowledge proof $t_{\overline{m}1}$ uses $\overline{m}_2$ instead of $g_{1,1}$ as base for the third factor and $k_{\pwd_2}$ as exponent ($s_{\pwd_2}$ in the verification).
This is necessary due to the fact that the exponent $-\mu_2 {\pwd_2}$ of the third factor in $\overline{m}_1$ is a product of two values that have to be proven correct.
The ZK proof uses the auxiliary message $\overline{m}_2$ to prove that $\log_{g_{1,1}}(\overline{m}_2)=-\mu_2$ such that it is sufficient to prove $\log_{\overline{m}_2}(\overline{m}_2^{{\pwd_2}})={\pwd_2}$.
In Figure \ref{fig:cs-tdsphf} we refer to the messages as $\Comm_{2}=(t_{\overline{m}1}, t_{\overline{m}2}, t_{e2}, t_{v2}, t_{\hp12}, t_{\hp22})$, $\Res_{2}=(s_{\pwd_2}, s_{\mu2}, s_{\eta12}, s_{\eta22}, s_{\theta2}, s_{\nu2}, s_{r2})$, and $\Ch_{2}=\fc$.


%==============================================================================
% 2-Server PAKE in UC
%==============================================================================
\subsection{UC-secure Two Server PAKE}\label{sec:2pake}
With \TDSPHF it is straight forward to build a 2PAKE protocol.
We follow the general framework described in \cite{KieferM14a} to build 2PAKE protocols from distributed smooth projective hash functions.
However, instead of aiming for key generation, where the client establishes a key with each of the two servers, we focus on a protocol that establishes a single key with one server, w.l.o.g. the first server.
By running the protocol twice, keys can be exchanged between the client and the second sever.
Note that UC security allows concurrent execution of the protocol such that round complexity is not increased by establishing two keys.

\subsubsection{The Protocol}\label{sec:2pakeprotocol}
% \fk{add figure}
Building a 2PAKE protocol from the \TDSPHF defined in Figure \ref{fig:cs-tdsphf} is straight forward following \cite{KieferM14a}.
Client $C$, Server $S_1$ and Server $S_2$ execute a \TDSPHF protocol as described in Section \ref{sec:tdsphf}.
This provides $C$ and $S_1$ with two hash values $h_0$ and $h_x$ each (if all protocol participants are honest and all messages reach their destination unaltered).
A session key can then be derived by simply multiplying $h_0$ and $h_x$ to $\sk=h_0\cdot h_x$.
If the protocol is not terminated prematurely and a session key is computed on $C$ and $S_1$, this ensures that $C$ and $S_1$ share a unique random session key $\sk$ after finishing an honest and correct protocol run and two independent random session keys $\sk$ in case one of the parties inputs a wrong password (share) or the traffic is altered during transport.

% \begin{tikzpicture}
% \matrix (m)[matrix of nodes, column  sep=1cm,row  sep=4mm, nodes={draw=none, anchor=center,text depth=0pt} ]{
% Client & & Server 1 & & Server 2\\ %[-4mm]
% a) & & a) && a) \\
%  & $(C_0,\hp_0)$ & & $(C_1,\hp_1)$ &  \\ [-4mm]
%  & $(C_1,\hp_1)$ & & $(C_2,\hp_2)$ &  \\ [-4mm]
%  & \hfill & $(C_0,\hp_0)$ & \hfill &  \\ [-4mm]
%  & \hfill & $(C_2,\hp_2)$ & \hfill &  \\
% b) i) & & b) iii) & & b) ii) \\
%  & & & $C_{h_{x,2}}$ &  \\
%  & & & $(m_0, c_0)$ &  \\
%  & & c) & & g)\\
%  & & & $c$ &  \\
%  & & & $C_{\Hash_{0,1}}$ &  \\
%  & & h) & & d)\\
%  & & & $c$ &  \\
%  & & & $C_{s_{h_{x,2}}}$ &  \\
% };
% 
% \draw ($(m-2-1.south west)-(0.1,0.2)$) rectangle ($(m-2-1.north east)+(0,0.1)$);
% \draw ($(m-2-3.south west)-(0.1,0.2)$) rectangle ($(m-2-3.north east)+(0,0.1)$);
% \draw ($(m-2-5.south west)-(0.1,0.2)$) rectangle ($(m-2-5.north east)+(0,0.1)$);
% % \draw[shorten <=-1.5cm,shorten >=-1.5cm] (m-1-1.south east)--(m-1-1.south west);
% % \draw[shorten <=-1.5cm,shorten >=-1.5cm] (m-1-3.south east)--(m-1-3.south west);
% % \draw[shorten <=-1.5cm,shorten >=-1.5cm] (m-1-5.south east)--(m-1-5.south west);
% \draw[shorten <=-1cm,shorten >=-1cm,-latex] (m-3-2.south west)--(m-3-2.south east);
% \draw[shorten <=-1cm,shorten >=-1cm,-latex] (m-3-4.south west)--(m-3-4.south east);
% \draw[shorten <=-1cm,shorten >=-1cm,-latex] (m-4-2.south east)--(m-4-2.south west);
% \draw[shorten <=-1cm,shorten >=-1cm,-latex] (m-4-4.south east)--(m-4-4.south west);
% \draw[shorten <=-1cm,shorten >=-1cm,-latex] ($(m-5-2.south west)-(0.7,0.12)$)--($(m-5-4.south east)-(-0.7,0.12)$);
% \draw[shorten <=-1cm,shorten >=-1cm,-latex] ($(m-6-4.south east)-(-0.7,0.12)$)--($(m-6-2.south west)-(0.7,0.12)$);
% \draw[shorten <=-1cm,shorten >=-1cm,-latex] (m-8-4.south east)--(m-8-4.south west);
% \draw[shorten <=-1cm,shorten >=-1cm,-latex] (m-9-4.south west)--(m-9-4.south east);
% \draw[shorten <=-1cm,shorten >=-1cm,-latex] (m-11-4.south west)--(m-11-4.south east);
% \draw[shorten <=-1cm,shorten >=-1cm,-latex] (m-12-4.south east)--(m-12-4.south west);
% \draw[shorten <=-1cm,shorten >=-1cm,-latex] (m-14-4.south west)--(m-14-4.south east);
% \draw[shorten <=-1cm,shorten >=-1cm,-latex] (m-15-4.south east)--(m-15-4.south west);
% 
% % \draw[shorten <=-1cm,shorten >=-1cm,-latex] (m-5-2.south east)--(m-5-2.south west);
% \end{tikzpicture}

\subsubsection{UC Security Model for Two Server PAKE}\label{sec:2pakesecurity}
We describe the ideal functionality for 2PAKE in the following, which can be seen as an extension of the known ideal functionality for PAKE protocols from Canetti et al. \cite{Canetti2005}.
The ideal functionality \FTWOPAKE is given in Figure \ref{fig:2pakef}.
\FTWOPAKE is very similar to the original two-party PAKE functionality but requires some additional functionality.
We recall the PAKE functionality in Appendix \ref{app:pake} for convenience.
In contrast to other password-related ideal functionalities such as VPAKE \cite{Gentry2006} (a.k.a. augmented PAKE) and 2PASS \cite{Camenisch_Lysyanskaya_Neven_2012} (two-server password authenticated secret sharing) we consider only the key-exchange functionality without explicit client authentication.
While it is compelling to model 2PAKE (as well as PAKE) similar to \cite{Camenisch_Lysyanskaya_Neven_2012} and allow for throttling on wrong password guesses,\footnotemark\ this is only possible if the servers are able to verify validity of the client's password, i.e. explicit authentication of the client.
We want to keep the 2PAKE functionality simple and it is known that implicit authentication is sufficient for secure channels \cite{CanettiK01} and therefore stick with implicit authentication.

\footnotetext{Servers often lock accounts or require additional authentication after a certain number of failed log-in attempts. This prevents automated online dictionary attacks, which are always possible in the password authenticated key exchange setting.}

\paragraph{A Note on Corruption}
We consider static corruption, s.t. the corrupted parties (clients and servers) are chosen by the adversary in advance.
If a server $S_b$ is corrupted, the adversary only learns $S_b$'s password share. 
The password stays secret as long as the attacker does not corrupt both servers.
while there exist two-party PAKE protocols in literature that are UC-secure with adaptive corruptions \cite{AbdallaBBCP13,AbdallaBP14a,AbdallaCCP09,AbdallaCP09}, we consider only static corruptions in this work.
Static corruption in the UC model implies security in the BPR model with adaptive corruptions \cite{Canetti2005} for PAKE protocols.
However, adaptive corruptions are obviously a stronger security notion that allow an attacker not only to execute a protocol on behalf of an honest participant but interfere with the execution of an honest party while executing the protocol.
The only PAKE constructions known today that are secure against adaptive corruptions in the UC model require more complex SPHF constructions that are not translatable to the \DSPHF approach.
It therefore seems reasonable to settle with static corruptions for two-server PAKE UC security for now.
We discuss the relation between our new UC formalisation of 2PAKE and the known BPR-based security model in Appendix \ref{app:relations}.

% FIXME: \fk{currently both servers have to be corrupted to set \sk, see if we want this or if corruption of $S_1$ should be enough}

\paragraph{2PAKE Functionality}


\begin{figure}[tbp]
\begin{mdframed}[innertopmargin=10pt]
\begin{center}
{\bf Functionality $\cF_{\mathrm{2PAKE}}$}
\end{center}
The functionality $\cF_{\mathrm{2PAKE}}$ is parameterised by a security parameter $\secpar$.
It interacts with an adversary, a client $C$ and two servers $S_1$ and $S_2$ via the following interfaces.
Without loss of generality the key is exchanged between $C$ and $S_1$.

\begin{description}

  \item[KEX Init$_C$:] Upon input $(\KEXinit, \sid, \qid, \pwd)$ from client $C$, check that \sid is $(C,S_1, S_2)$ and that \qid is unique (entries $(\KEX, \sid, \qid, S_{1}, \alpha_{1})$ or $(\KEX, \sid, \qid, S_{2}, \alpha_{2})$ may exist) and send $(\KEX, \sid, \qid, C)$ to \SIM.
      If this is a valid request, create a \emph{fresh} record $(\KEX, \sid, \qid, C, \pwd)$.
  
  \item[KEX Init$_S$:] Upon input $(\KEXinit, \sid, \qid, \alpha_b)$ from server $S_b$, $b\in\{1,2\}$, check that \sid is $(C,S_1, S_2)$ and that \qid is unique (entries $(\KEX, \sid, \qid, C, \pwd)$ or $(\KEX, \sid, \qid, S_{3-b}, \alpha_{3-b})$ may exist) and send $(\KEX, \sid, \qid, S_b)$ to \SIM.
      If this is a valid request, create a fresh record $(\KEX, \sid, \qid, S_b, \alpha_b)$.
    
  \item[TestPwd:] Upon input $(\TP, \sid, \qid, \pwd')$ from \SIM check that a fresh record $(\KEX, \sid, \qid, C, \pwd)$ exists. 
      If this is the case, mark $(\KEX, \sid, \qid, S_1, \alpha_1)$ as \compromised and reply with ``correct guess'' if $\pwd=\pwd'$, and mark it as \interrupted and reply with ``wrong guess'' if $\pwd\not=\pwd'$.
    
%   \item[TestShares:] Upon input $(\TS, \sid, \qid, \alpha'_1, \alpha'_2)$ from \SIM check that a fresh records $(\KEX, \sid, \qid, S_1, \alpha_1)$, $(\KEX, \sid, \qid, S_2, \alpha_2)$ and $(\KEX, \sid, \qid, C, \pwd)$ exist. 
%       If this is the case, mark  $(\KEX, \sid, \qid, C, \pwd)$ as \compromised and reply with ``correct guess'' if $\alpha'_1=\alpha_1 ~ \wedge ~ \alpha'_2=\alpha_2$ and as \interrupted and reply with ``wrong guess'' if $\alpha'_1\not=\alpha_1 ~ \vee ~ \alpha'_2\not=\alpha_2$.
      
  \item[Failed:] Upon input $(\FA, \sid, \qid)$ from \SIM check that records $(\KEX, \sid, \qid, C, \pwd)$ and $(\KEX, \sid, \qid, S_1, \alpha_1)$ exist that are not marked \completed. 
      If this is the case, mark both as \failed.
	
	\item[NewKey:] Upon input $(\NK, \sid,\qid, P, \sk')$ from \SIM with $P\in\{C,S_1\}$, check that a respective $(\KEX, \sid, \qid, C, \pwd)$ or $(\KEX, \sid, \qid, S, \alpha_1)$ record exists, $\sid=(C,S_1,S_2)$, $|\sk'|=\secpar$, then: %and the session is not marked for $P$ yet,
	\begin{itemize}
		\item If the session is \compromised, or either $C$ or $S_1$ and $S_2$ are corrupted, then output $(\NK, \sid,\qid,\sk')$ to $P$; else
		
		\item if the session is \emph{fresh} and a key $\sk$ was sent to $P'$ with $\sid=(P, P', S_2)$ or $\sid=(P', P, S_2)$ while $(\KEX, \sid, \qid, P', \cdot)$ was fresh, then output $(\NK, \sid, \qid, \sk)$ to $P$.
		
		\item In any other case, pick a new random key $\sk$ of length $\secpar$, and send $(\NK, \sid, \qid, \sk)$ to $P$.
	\end{itemize}
	In any case, mark \qid as \completed for $P$.
	
\end{description}
\end{mdframed}
\caption{Ideal Functionality $\cF_{\mathrm{2PAKE}}$}
\label{fig:2pakef}
\end{figure}

\noindent
The ideal functionality for 2PAKE is very similar to the PAKE functionality but considers two servers from which only one generates a session key.
The main difference is therefore the explicit modelling of participants (in contrast to symmetric parties in the two-party case).
We specify two initialisation interfaces \textbf{KEX Init}, one for the client and one for the servers.
A client is initialised with a password \pwd while a server gets a password share $\alpha_b$.
The \textbf{TestPwd} interface allows the ideal world adversary to test client passwords.
A tested session is marked \interrupted if the guess is wrong, i.e. client and server in this session receive randomly chosen, independent session keys, or marked as \compromised if the password guess is correct, i.e. the attacker is now allowed to set the session key.
The attacker can only test client passwords but not password shares of the servers.
Without knowledge of the password or any password share, a share is a uniformly at random chosen element and therefore not efficiently guessable.
If the adversary corrupted server $S_2$, retrieving the second password share $\alpha_1$ from $S_1$ is equivalent to guessing the password.
Complementing the TestPwd interface is a \textbf{Failed} interface that allows the adversary to let sessions fail.
This allows the attacker to prevent protocol participants from computing any session, i.e. failed parties do not compute a session key.
Eventually the \textbf{NewKey} interface generates session keys for client $C$ and server $S_1$.
NewKey calls for $S_2$ are ignored.
If client $C$ or server $S_1$ and $S_2$ are corrupted, or the attacker guessed the correct password, the adversary chooses the session key.
If a session key was chosen for the partnered party and the session was fresh at that time, i.e. not \compromised or \interrupted, the same session key is used again.
In any other case a new random session key is drawn.

Instead of using a single session identifier \sid we use \sid and \qid.
%(as done in \cite{Camenisch_Lysyanskaya_Neven_2012})
The session identifier \sid is composed of the three participants $(C,S_1,S_2)$ (note that we use the client $C$ also as ``username'' that identifies its account on the servers) and therefore human memorable and unique.
To handle multiple, concurrent 2PAKE executions of one \sid, we use a query identifier \qid that is unique within \sid and can be established with \Finit.
In the multi-session extension \FTWOPAKEM the \sid becomes \ssid and \sid is a globally unique identifier for the used universe, i.e. server public keys (\CA) and \crs.


\subsubsection{Security}\label{sec:2pakeproof}
The following theorem formalises the security of the previously defined 2PAKE protocol.
Note that we do not rely on any security of the \TDSPHF.
Instead we reduce the security of our 2PAKE protocol directly to the underlying problem (SXDH).
Thereby we give an indirect security proof of the proposed \TDSPHF.
% For details on the security of the \TDSPHF we refer to the full version.
 
\begin{theorem}\label{theo:uc2pake}
  The 2PAKE protocol from Section \ref{sec:2pakeprotocol} securely realises \FTWOPAKEM with static corruptions in the \Fcrs-\Fca-hybrid model if the DDH assumption holds in $\GG_1$ and $\GG_2$ and $H_k$ is a universal one-way hash function, i.e. Cramer-Shoup encryption is secure in $\GG_1$ and the DDH assumption holds in $\GG_2$.
\end{theorem}

\paragraph{Sequence of Games}
We start the proof of Theorem \ref{theo:uc2pake} by giving a sequence of games with \G{1} equal to the real-world execution with honest participants following the protocol description and the real-world adversary \cA that may have control over a set of participants, and \G{17} equal to the ideal-world execution where the protocol is replaced with the ideal functionality \FDSPHF acting on behalf of all honest protocol participants and the ideal-world adversary \SIM, detailed later.
Let $\view_i$ denote the view of environment \cZ when interacting with game \G{i}.
Note that \view is implicitly parametrised with \sid and the security parameter \secpar.
Security then follows from showing that each $\view_i$ is computationally indistinguishable from the subsequent $\view_{i+1}$, such that we can eventually follow by an hybrid argument that $\view_1$ and $\view_{17}$ are computationally indistinguishable and the protocol therefore securely realises the ideal functionality \FTWOPAKE.
All participants in the games are operated by the challenger \cC (receiving the participants input from environment \cZ), which we modify from game to game.
Every session for an $\sid=(C,S_1,S_2)$ is started with a KexInit call for each participant, defining secrets, roles, and the used query identifier.
Invalid messages, i.e. messages that do not pass the usual tests such as group membership, are discarded by the challenger.
Note that we usually only give the actual payload of messages and omit additional parts such as \sid, \qid etc.

\Gh Game $1$ is the real-world experiment in which \cZ interacts with real participants that follow, if honest, the protocol description, and the real-world adversary \cA controlling the corrupted parties.
All participants are honestly simulated by challenger \cC that knows all their inputs.

\Gh This game is identical to \G{\theoldgame}, except that the \crs is generated by \cC such that it knows the trapdoor $\tau$.
Note that the second trapdoor $\tau'$ for $\zeta$ is \emph{not} controlled by \cC yet as this would destroy any security.
Knowledge of $\tau$ allows \cC to decrypt ciphertexts $C_i$ and retrieve the used message.
This does not change anything and is therefore perfectly indistinguishable from \G{\theoldgame}.

\Gh When \cC, on behalf of $S_1$, receives first messages $(C_0, \hp_0)$ and $(C_2, \hp_2)$, it decrypts $C_0$ to $\pwd'$ and checks if this is the correct password, i.e. $\pwd'=\pwd$.
If this is not the case, $\pwd'\not=\pwd$, \cC chooses a random $h'_0\rin\GG_T$ if the subsequent $\Hash_0$ computation with $S_2$ is successful, i.e. all zero-knowledge proofs can be verified, and aborts $S_1$ otherwise.
We claim that $\view_{\theoldgame}$ is computationally indistinguishable from $\view_{\thegame}$.
The probability to distinguish the two games is bounded by the negligible probability to notice that $h_0$ is now chosen uniformly at random.
Since $C=(C_0, C_1, C_2)$ is not in $L_{\pwd, \pwd_1, \pwd_2}$ the computation of $\Hash_0$ between $S_1$ and $S_2$ yields a uniformly at random distributed hash value $h_0$.
This can be either deduced from the smoothness proven for the generic (not distributed) \TSPHF in \cite{Benhamouda_Pointcheval_2013} or by the following simplified argument.
% The hash value for $S_1$ is computed as
% \begin{align*}
%   h_0 &= e\left( g_{1,1}^{-\mu_1 \pwd_1} \cdot \left( g_{1,1}^{-\mu_1 \pwd_2} \cdot g_{1,1}^{-\mu_2 \pwd_1} \cdot g_{1,1}^{-\mu_2 \pwd_2} \cdot u_{1,0}^{\eta_{1,2}+\xi_0 \eta_{2,2}} \cdot u_{2,0}^{\theta_2} \cdot e_0^{\mu_2} \cdot v_0^{\nu_2} \right) \cdot u_{1,0}^{\eta_{1,1}+\xi_0\eta_{2,1}} \cdot u_{2,0}^{\theta_1} \cdot e_0^{\mu_1} \cdot v_0^{\nu_1}, g_2 \right) \\
% %   &= e\left( g_{1,1}^{-\mu_1 \pwd_1 -\mu_1 \pwd_2 -\mu_2 \pwd_1 -\mu_2 \pwd_2} \cdot u_{1,0}^{\eta_{1,2}+\xi_0 \eta_{2,2} + \eta_{1,1}+\xi_0\eta_{2,1}} \cdot u_{2,0}^{\theta_2 + \theta_1} \cdot e_0^{\mu_1 + \mu_2} \cdot v_0^{\nu_1 + \nu_2}, g_2 \right) \\
%   &= e\left( g_{1,1}^{(\pwd - \pwd')(\mu_1 + \mu_2)} \cdot u_{1,0}^{\eta_{1,2}+\xi_0 \eta_{2,2} + \eta_{1,1}+\xi_0\eta_{2,1}} \cdot u_{2,0}^{\theta_2 + \theta_1} \cdot h^{r_0(\mu_1 + \mu_2)} \cdot v_0^{\nu_1 + \nu_2}, g_2 \right).
% \end{align*}
As long as $C\not\in L_{\pwd, \pwd_1, \pwd_2}$ the same argument as used for SPHF and \DSPHF can be used, namely that $h_0$ is linearly independent from the adversarially known values and therefore indistinguishable from a random one.
However, this is not sufficient in this case as the attacker has the possibility to distinguish real $h_0$ values from random ones with use of the third projection keys $\hp_{3,i}$.
To show that this is not possible we show how to break the DDH assumption in $\GG_2$ if there exists a distinguisher that can distinguish real $h_0$ from random ones.
To this end we build a DDH triple $(\zeta, \fa, \fb)$ with $\crs'=\zeta=g_2^{\tau'}$ as follows.
% The first triple, $(\zeta, \fa, \fb)$, is \emph{not} a DDH triple with random elements in $\GG_2$, $\fa,\fb\rin\GG_2$.
% We construct $(\zeta, \fa, \fb)$ as DDH triple as follows.
Let $\fa=\zeta^\alpha$ and $\fb=g_2^\alpha$, then $(\zeta, \fa, \fb)$ is obviously a DDH triple.
To link this to the \TDSPHF we set $\alpha=\hk_{i,j}$, then $\fa=\hp_{3,i,j}=\zeta^{\hk_{i,j}}$ such that $\fb=g_2^{\hk_{i,j}}$. 
To build a non-DDH triple $(\zeta, \fa, \fb)$ we choose random $\bm \alpha$ and set $\fa=\hp_{3,i,j}=\zeta^{\hk_{i,j}}$ and $\fb=g_2^{\alpha_{j}}$.
To guarantee correctness we have to choose $\alpha$ such that $\alpha_j=\hk_{j,i}+\beta_j$ for $\beta\in\ker\begin{pmatrix}
  g_{1,1} & 1 & g_{1,2} & h & c \\
  1 & g_{1,1} & 1 & 1 & d
\end{pmatrix}$ for $j\in[1,5]$.
Note that this is possible because we know $\tau$, which contains the secret Cramer-Shoup key.
If we can build a distinguisher on $h_0$, we can now decide whether $(\zeta, \fa, \fb)$ is a valid DDH triple or not.

% The idea here is to build a second hashing key $\widehat{\hk}_i$ that produces the same projection key $\hp_{1,i}, \hp_{2,i}$ but a different $\widehat{\hp}_{3,i}\not=\hp_{3,i}$.
% This is possible because we know $\tau$, which contains the secret Cramer-Shoup key $(x_1, x_2, y_1, y_2, z)$ for public key $(c=g_{1,1}^{x_1}g_{1,2}^{x_2}, d=g_{1,1}^{y_1}g_{1,2}^{y_2}, h=g_{1,1}^z)$ in the \crs such that we are able to compute $\widehat{\hp}_{3,i}=(g_2^{\alpha_1}, g_2^{\alpha_2}, g_2^{\alpha_3}, g_2^{\alpha_4}, g_2^{\alpha_5})$ with $\alpha_j=\hk_{j,i}+\beta_j$ for $\beta\in\ker\begin{pmatrix}
%   g_{1,1} & 1 & g_{1,2} & h & c \\
%   1 & g_{1,1} & 1 & 1 & d
% \end{pmatrix}$ for $j\in[1,5]$.

% We therefore build $\widehat{\hk}_i=(\widehat{\eta}_{1,i}, \widehat{\eta}_{2,i}, \widehat{\theta}_{i}, \widehat{\mu}_{i}, \widehat{\nu}_{i})$ such that 
% $\hp_{1,i}=g_{1,1}^{\widehat{\eta}_{1,i}} g_{1,2}^{\widehat{\theta}_{i}} h^{\widehat{\mu}_{i}} c^{\widehat{\nu}_{i}}$ and 
% $\hp_{2,i}=g_{1,1}^{\widehat{\eta}_{2,i}} d^{\widehat{\nu}_{i}}$.

% \smallskip
% In order to solve this with our attacker we have to define $\widehat{\hk}_i=()$ .
% We can build $(\zeta, \hp_{1,0}, \hp_{2,0}, \hp_{3,0}, \Gamma)$ with $\Gamma=(e(g_{1,1}, g_2^{\eta_{1,0} + \xi_0\eta_{2,0}}), e(g_{1,1}, g_2^{\theta_0}), e(g_{1,1}, g_2^{\mu_0}), e(g_{1,1}, g_2^{\nu_0}))$.

\Gh In this game we choose $\sk\rin\GG_T$ at random in case we choose $h_0$ at random (the setting described in \G{\theoldgame}) and computation of \sk on $S_1$ is successful.
% In case \theoldgame\xspace the password is not the correct one, and the session key computed on $S_1$ (if computed) is therefore supposed to be uniformly at random distributed.
Since $h_0$ on $S_1$ is uniformly at random already and $\sk=h_0h_x$, $\view_{\thegame}$ is perfectly indistinguishable from $\view_{\theoldgame}$.

\Gh Receiving an adversarially generated or modified $C_1$ or $C_2$ on behalf of client $C$, challenger \cC chooses $h_x\rin\GG_T$ uniformly at random instead of computing it with $\Hash_x$ if $C_1$ or $C_2$ do not encrypt the correct password share $\pwd_1$ or $\pwd_2$ respectively.
% Let this denote case \thegame.
We claim that $\view_{\thegame}$ is computationally indistinguishable from $\view_{\theoldgame}$.
In this case we have $(C_0,C_1,C_1)\not\in L_{\widehat{\pwd}}$ with overwhelming probability.
The claim therefore follows by a similar argument as in Game 3, i.e. from the DDH assumption in $\GG_2$.

\Gh In this game we choose $\sk\rin\GG_T$ at random in case we choose $h_x$ at random (the setting described in \G{\theoldgame}) and computation of \sk on $C$ is successful (projection keys $\hp_1$ and $\hp_2$ are correct).
% In case \theoldgame\xspace the password shares encrypted in $C_1$ and $C_2$ do not reconstruct to \pwd, and the session key computed on $C$ (if computed) is therefore supposed to be uniformly at random distributed.
Since $h_x$ on $C$ is uniformly at random already and $\sk=h_0h_x$, $\view_{\thegame}$ is perfectly indistinguishable from $\view_{\theoldgame}$.

\Gh In this game we replace computation of hash values $h_0$ and $h_x$ 
with a lookup table with index $(\hk_1, \hk_2, L_{\pwd, \pwd_2, \pwd_2}, C_0)$ for $h_0$ and $(\hk_0, L_{\pwd, \pwd_2, \pwd_2}, C_1, C_2)$ for $h_x$.
If no such value exists, it is computed with the appropriate \Hash or \PHash function and stored in the lookup table.
Due to the correctness of the used Cramer-Shoup \TDSPHF $\view_{\thegame}$ is perfectly indistinguishable from $\view_{\theoldgame}$.

\Gh Instead of computing $\Hash_0$ for $S_1$ in case $\pwd'$ decrypted from $C_0$ is the same as $\pwd$, \cC draws a random $h_0\rin\GG_T$.
% We denote this case 2.
That is, in this game $h_0$ for $S_1$ is always chosen uniformly at random instead of computing it with $\Hash_0$.
We claim that $\view_{\thegame}$ is computationally indistinguishable from $\view_{\theoldgame}$.
The claim follows from the CCA-security of the labelled Cramer-Shoup encryption and the same argument as in Game 3, i.e. from SXDH.
In particular, we define $\G{\theoldgame}'$ and $\G{\theoldgame}''$ with computationally indistinguishable views from \G{\theoldgame} as intermediate games before \G{\thegame} such that the claim follows.
Note that the following games modify the experiment only in the previously defined case.
In $\G{\theoldgame}'$ challenger \cC computes $C_1$ for $S_1$ on a random value $\pwd'_1\rin\ZZ_q$, $\pwd'_1\not=\pwd_1$.
The CCA-security of the encryption scheme ensure that $\view_{\theoldgame'}$ is computationally indistinguishable from $\view_{\theoldgame}$.
In $\G{\theoldgame}''$ we choose a random $h_0\rin\GG_T$ instead of using the distributed $\Hash_0$ computation (the protocol is still performed but the values are not used).
Using the same argument as in \G{3}, $\view_{\theoldgame''}$ is computationally indistinguishable from $\view_{\theoldgame'}$.
The only difference between $\G{\theoldgame}''$ and \G{\thegame} now is that \cC encrypts a random value instead of $\pwd_1$ in $C_1$ in $\G{\theoldgame}''$.
The claim now follows by observing again that $\view_{\theoldgame''}$ and $\view_{\thegame}$ are computationally indistinguishable considering the CCA-security of the labelled Cramer-Shoup encryption scheme.

\Gh In this game we choose $\sk\rin\GG_T$ at random in case we choose $h_0$ at random (the setting described in \G{\theoldgame}) and computation of \sk on $S_1$ is successful.
Since $h_0$ on $S_1$ is uniformly at random and $\sk=h_0h_x$, $\view_{\thegame}$ is perfectly indistinguishable from $\view_{\theoldgame}$.

\Gh Receiving correct $C_1$ or $C_2$, i.e. encrypting $\pwd_1$ and $\pwd_2$ respectively, on behalf of client $C$, challenger \cC chooses $h_x\rin\GG_T$ uniformly at random instead of computing it with $\Hash_x$.
We claim that $\view_{\thegame}$ is computationally indistinguishable from $\view_{\theoldgame}$.
Since we have $(C_0,C_1,C_1)\in L_{\widehat{\pwd}}$ in this case, the claim follows by a similar argument as in Game 8, i.e. from the SXDH assumption.

\Gh In this game we choose $\sk\rin\GG_T$ at random in case we choose $h_0$ at random (the setting described in \G{\theoldgame}) and computation of \sk on $C$ is successful (projection keys $\hp_1$ and $\hp_2$ are correct).
% In case \theoldgame\xspace the password shares encrypted in $C_1$ and $C_2$ do not reconstruct to \pwd, and the session key computed on $C$ (if computed) is therefore supposed to be uniformly at random distributed.
Since $h_x$ on $C$ is uniformly at random already and $\sk=h_0h_x$, $\view_{\thegame}$ is perfectly indistinguishable from $\view_{\theoldgame}$.

% \Gh In this game we choose $\sk\rin\GG_T$ at random if $\Dec(C_0)=\Dec(C_1)+\Dec(C_2)$, computation of \sk is successful and no session key has been chosen for this session yet.
% If a session key $\sk$ has been chosen for this session already, this key is used.
% In game 9 and 10 $h_x$ for client $C$ and $h_0$ are chosen uniformly at random such that $\view_{\thegame}$ is perfectly indistinguishable from $\view_{\theoldgame}$.

\Gh The entire \crs including $\zeta$ is chosen by challenger \cC in this experiment.
The $\view_{\thegame}$ is perfectly indistinguishable from $\view_{\theoldgame}$ since this does not change anything else.

\Gh Upon receiving $C_1$ and $C_2$, encrypting correct password shares, \cC uses $\THash_0$ to compute $h_0$ on client $C$ instead of $\PHash_0$.
This is possible because \cC now knows trapdoor $\tau'$.
Due to \TDSPHF soundness, $\view_{\thegame}$ is perfectly indistinguishable from $\view_{\theoldgame}$.

\Gh Upon receiving $C_0$, encrypting correct password, \cC uses $\THash_x$ to compute $h_x$ on server $S_1$ instead of $\PHash_x$.
This is again possible because \cC now knows trapdoor $\tau'$.
Due to \TDSPHF soundness, $\view_{\thegame}$ is perfectly indistinguishable from $\view_{\theoldgame}$.

\Gh Instead of encrypting the correct password \pwd in $C_0$ on behalf of client $C$, \cC encrypts $0$ (which is not a valid password).
We claim that $\view_{\thegame}$ is computationally indistinguishable from $\view_{\theoldgame}$ under the DDH assumption in $\GG_1$, i.e. the CCA-security of the Cramer-Shoup encryption.
Note that encryption randomness $r$ is not used in the computation of $h_0$ anymore such that the claim follows from the Cramer-Shoup CCA-security.

\Gh Instead of encrypting the correct password share $\pwd_i$ in $C_i$ on behalf of server $S_i$ with $i\in[1,2]$, \cC encrypts a random element $\pwd'_i\rin\ZZ_q$.
We claim that $\view_{\thegame}$ is computationally indistinguishable from $\view_{\theoldgame}$ under the DDH assumption in $\GG_1$, i.e. the CCA-security of the Cramer-Shoup encryption.
Note that the probability for $\pwd'_i=\pwd_i$ is negligible such that the claim follows from the Cramer-Shoup CCA-security.

\Gh Instead of the challenger \cC simulating the protocol execution the ideal functionality \FTWOPAKE is used to interact with the ideal-world adversary \SIM.
While this game is structurally different from \G{\theoldgame} its execution is indistinguishable from the latter.
This combined with the following description of the ideal world adversary \SIM concludes the proof.


\paragraph{Simulator}
We now describe the simulator \SIM that is used in the last experiment and acts as an attacker in the ideal world against the ideal functionality \FTWOPAKE, interacting with the real world adversary \cA.
It uses a real-world adversary \cA in a way that the environment \cZ cannot  distinguish whether it is interacting with \cA and honest protocol participants in the real world, or with \SIM and dummy protocol participants (simulated by \FTWOPAKE) in the ideal world.
We describe \SIM for a single session $\sid=(C,S_1,S_2)$.
The security then follows from the UC composition theorem \cite{Canetti2001a}, covering multiple sessions of the protocol, and joint-state UC composition theorem \cite{CanettiR03}, covering the fact that \Fca and \Fcrs create a joint state for all sessions and participants.
As before, we assume that $0$ is not a valid password.

First, \SIM generates $\crs=(q,g_{1,1},g_{1,2},h,c,d,\GG_1,g_{2},\zeta,\GG_2,\GG_T,e,H_k)$ with Cramer-Shoup secret key as trapdoor $\tau=(x_1,x_2,y_1,y_2,z)$ and second trapdoor $\tau'$ for $\zeta=g_2^{\tau'}$ to answer all \Fcrs queries with \crs.
Further, \SIM generates ElGamal key pairs $(g^{z_1},z_1)$ and $(g^{z_2},z_2)$, and responds to $\Retrieve(S_i)$ queries to \Fca from $S_i$ with $(\Retrieve, S_i, (g^{z_i},z_i))$ for $i\in\{1,2\}$ and with $(\Retrieve, S_i, g^{z_i})$ to all other request.
We describe different scenarios in which the simulator operates.
First we describe simulation of the initial KEXInit call before showing the way \SIM handles different input messages and the key generation.
The simulator essentially has to ensure that the functionality chooses random, correct session keys if the execution is correct and random, independent ones in case of an error during the execution.

% \hfill\\\noindent
% {\bf A) Honest Client:}
% Consider that client $C$ is honest.
% We describe the four possible cases with i) two honest servers, ii) server $S_2$ is corrupted, iii) server $S_1$ is corrupted and iv) both servers are corrupted.
% The first case describes the entire default simulation while for all other cases we give only differences to the first case.
% 
% \begin{enumerate}[i)]
%   \item In this setting all parties $C, S_1, S_2$ are honest such that the entire protocol is simulated by \SIM and the attacker \cA has control over the network in between.
        When receiving $(\KEX, \sid, \qid, P)$ with $\sid=(C,S_1,S_2)$ and $P\in\{C,S_1,S_2\}$ from \FTWOPAKE, \SIM starts simulation of the protocol for protocol participant $P$ by computing ciphertext, projection key pair $M_i=(C_i, \hp_i)$ for $i\in\{0,1,2\}$, encrypting a dummy value ($0$ for $P=C$ and a random value $\alpha'_i\rin\ZZ_q$ for $P=S_i$, $i\in\{1,2\}$).
        \SIM outputs the computed $(C_i, \hp_i)$ to \cA.
        The first round of messages is handled as follows.
        
\begin{enumerate}
  \item When any party receives an adversarially generated but well formed first message $M_i$, $i\in\{1,2\}$ from uncorrupted $S_i$, i.e. $\VerHp$ on the projection key $\hp_i$ is $1$, \SIM queries $(\FA, \sid, \qid)$, which marks the session \failed for the receiving party and thus ensures that the party receives an independent, random session key (if any) on a NewKey query.
        
  \item When any party receives an adversarially generated but well formed first message $M_2$ from a corrupted $S_2$ while $S_1$ is not corrupted, \SIM decrypts $C_2$ to $\alpha'_2$.
        If this value is not correct, $\alpha'_2\not=\alpha_2$ (the party is corrupted such that \SIM knows the correct value), \SIM queries $(\FA, \sid, \qid)$ to ensure independent session keys on NewKey queries.    
        
  \item When client $C$ receives an adversarially generated but well formed first message $M_1$ from a corrupted $S_1$ while $S_2$ is not corrupted, \SIM decrypts $C_1$ to $\alpha'_1$.
        If this value is \emph{not} correct, $\alpha'_1\not=\alpha_1$, \SIM queries $(\FA, \sid, \qid)$ to ensure independent session keys on NewKey queries.    
        
  \item When any party receives adversarially generated but well formed first messages $M_1, M_2$ from corrupted $S_1, S_2$, \SIM decrypts $C_1$ and $C_2$ to $\alpha'_1$, $\alpha'_2$ respectively, and verifies their correctness against $\alpha_1$ and $\alpha_2$.
        If they are correct, \SIM computes $h_0\gets\THash_0(\hp_1, \hp_2,  L_{\pwd, \pwd_1, \pwd_2}, C_0, \tau')$, $h_x\gets\Hash_x(\hp_0, L_{\widehat{\pwd}}$, $C_1, C_2)$, and $\sk_C=h_0\cdot h_x$.
        Otherwise choose a random $\sk_C\in\GG_T$.
         
  \item When an honest $S_1$ or $S_2$ receives an adversarially generated but well formed first message $M_0$, i.e. $\VerHp$ on $\hp_0$ is $\true$, \SIM extracts $\pwd'$ from $C_0$ and sends $(\TP, \sid, \qid, C, \pwd')$ to \FTWOPAKE.
        If the functionality replies with ``correct guess'', \SIM uses $\pwd'$, \crs and $\tau'$ to compute $h_x\gets\THash_x(\hp_0, L_{\widehat{\pwd}}, C_1, C_2, \tau')$, $h_0\gets\Hash_0(\hk_1, \hk_2, L_{\pwd, \pwd_1, \pwd_2}, C_0)$, and $\sk_S=h_0\cdot h_x$.
        
  \item If verification of any $\hp_i$ fails at a recipient, \SIM aborts the session for the receiving participant.
\end{enumerate}  

\noindent              
        If a party does not abort, it proceeds as follows.
        After $C$ received all ciphertext, projection key pair messages and the previously described checks were performed \SIM sends $(\NK, \sid, \qid, C, \sk_C)$ to \FTWOPAKE if an $\sk_C$ for this session exists, or $(\NK, \sid, \qid, C, \bot)$ otherwise.
        %, which triggers \FTWOPAKE to either choose a random session key $\sk$, or use $\sk$ if $M_1$ and $M_2$ are unchanged and $sk$ was sent to $S_1$ earlier, and output $(\NK, \sid, \qid, \sk)$ to $C$.       $(\sid, \qid, 0, S_1, C_1, \hp_1)$ and $(\sid, \qid, 0, S_2, C_2, \hp_2)$
        After $S_1$ and $S_2$ received all ciphertext, projection key pair messages and the previously described checks were performed, \SIM simulates all further messages for honest parties, i.e. $\PHash_x$ and $\Hash_0$ computation between $S_1$ and $S_2$, with random elements and simulated zero-knowledge proofs.
        If all messages received by $S_1$ are oracle generated, send $(\NK, \sid, \qid, S_1, \sk_S)$ to \FTWOPAKE if this session is \compromised and $(\NK, \sid, \qid, S_1, \bot)$ if not.
        %, which triggers \FTWOPAKE to either choose a random session key $\sk$, or uses $\sk$, sent to $C$ earlier, and output $(\NK, \sid, \qid, \sk)$ to $S_1$.
        If any $\PHash_x$ or $\Hash_0$ message received by $S_1$ can not be verified, i.e. validation of the zero-knowledge proof fails, \SIM does nothing and aborts the session for $S_1$.
        
        

\subsection{\FTWOPAKE Discussion}\label{app:relations}
In this section we discuss some additional points of the \FTWOPAKE functionality and investigate relations to other 2PAKE security models and UC models in the password setting.

\subsubsection{\FTWOPAKE and the BPR 2PAKE Model}
While other security models for 2PAKE protocols where proposed \cite{SzydloK05}, the BPR-like security model from \cite{KatzMTB05} is the most comprehensible and (in its two-party version) established model.
We therefore discuss relation between the proposed 2PAKE UC-security using \FTWOPAKE and the BPR-like security model from \cite{KatzMTB05}.
To compare security of a 2PAKE protocol $\Pi$ in a game-based and UC setting we have to ensure that it supports session ids (necessary in the UC framework).
We therefore assume that $\Pi$ already uses UC compliant session ids.
Note that it is easy to transform any 2PAKE protocol into a 2PAKE protocol with such session ids.
Before looking into relation between the full game-based model for 2PAKE and \FTWOPAKE we want to point out that $\Pi$, securely realising \FTWOPAKE, offers ``forward secrecy'', i.e. even an adversary that knows the correct password is not able to attack an execution of $\Pi$ without actively taking part in the execution.
With this in mind it is easy to see that $\Pi$, securely realising \FTWOPAKE, is secure in the BPR-like model from \cite{KatzMTB05}.
This is because the attacker is either passive, which is covered by the previous observation, or is active and is therefore able tests one password.
Those password tests (\TestPwd in \FTWOPAKE and \Send in the game based model) give the attacker a success probability of $q/|\cD|$, with $q$ the number of active sessions and $|\cD|$ the dictionary size, when considering a uniform distribution of passwords inside the dictionary \cD.
Note that while the attacker may have knowledge of a password share, this does not increase this probability.
Security on the model from \cite{KatzMTB05} follows.

\subsubsection{\FTWOPAKE and \FPAKE}
While \FPAKE and \FTWOPAKE are very similar they contain some significant difference we want to point out here.
First, the key-exchange is performed between all three participants, but only $C$ and, w.l.o.g., $S_1$ agree on a common session key.
The \role is a technical necessity in \FPAKE for correct execution.
Since we have explicit roles in \FTWOPAKE this is not necessary here.
Due to the asymmetry in \FTWOPAKE (a client negotiates with two servers) we assume that the client is always the invoking party.
While this is the case in \FPAKE as well when considering a real world scenario, the roles might be different there such that any of the two participating parties can start the protocol execution.
The asymmetric setting in \FTWOPAKE further restricts \TestPwd queries to the client since the servers hold high entropy password shares.
While it is enough for the attacker to corrupt one party in \FPAKE to control the session key, in \FTWOPAKE he has to either corrupt or compromise the client, or corrupt both servers.
As long as only one server is corrupted, the adversary has no control over the session 	keys and the parties receive uniformly at random chosen session keys
In \FTWOPAKE session ids are human memorisable, consisting of all three involved parties $(C,S_1,S_2)$, and unique query identifier is used to distinguish between different (possibly concurrent) protocol runs of one account (\sid).
This is a rather technical difference to \FPAKE that uses only session identifiers.

% This allows both servers to negotiate a session key with $C$ by running the protocol twice with swapped roles.
	
% 	\item \FPAKE does not allow for any throttling mechanisms to avoid online dictionary attacks. 
% 	  This is an inherent shortcoming of the current \FPAKE definition and should be fixed, but is out of scope of this work.
% 	  However, we design our \FTWOPAKE functionality such that servers can refuse authentication queries from client according to a policy we do not further describe.
% 	  A general policy could be for example to refuse login attempts after three failed attempts and require additional out of band authentication thereafter to reset this counter.
% 	  Note that this does not require the protocol to provide mutual authentication as the success of a protocol can be established by the servers using other means, e.g., subsequent traffic.


\subsubsection{Corruptions}
The two-server extension of the BPR 2PAKE model used in \cite{KatzMTB05} does not consider corruptions at all.
While parties can be malicious in the model (static corruption), the attacker is not allowed to query a corrupt oracle to retrieve passwords or internal state of participants.
In our model the attacker is allowed to corrupt parties before execution.
This however implies security in the model from \cite{KatzMTB05} even if the attacker is allowed to corrupt clients to retrieve their passwords.
This is because the environment can provide the BPR attacker with the password.
However, this does not increase his success probability.
Dynamic corruptions in \FTWOPAKE on the other hand are much more intricate.
While UC-secure two party PAKE protocols with dynamic corruptions exist their approaches are not translatable to the 2PAKE setting.
The challenge of dynamic corruptions is that the simulation has to be correct even if the attacker corrupts one party \emph{after} the protocol execution has started.
This is left open for future work.

% \mynote{2PAKE UC ???(ePrint) \cite{KieferM15b}}

\section{Conclusion}
\mynote{2BPR and 2PAKE conclusion}
