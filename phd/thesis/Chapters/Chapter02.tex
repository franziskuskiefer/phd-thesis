%*****************************************
\chapter{Prerequisites}\label{ch:background}
%*****************************************
In this chapter we recall some definitions useful for following this work.
We start with some general mathematical fundamentals before giving generally useful definition in our setting.
While these definitions are all well known and do not need reference, most of them can be found (in maybe slightly different version) in \cite{katz2008introduction}.
%\graffito{You might get unexpected results using math in chapter or section heads. Consider the \texttt{pdfspacing} option.}

\section{Fundamentals of Mathematics}
We will not give an entire mathematical introduction but recall the most important definitions needed in our context.

\subsection{Groups}
Groups are an important mathematical structure in cryptography.
In the following we give some useful definitions.

\begin{definition}[Groups]\label{def:groups}
Let \GG denote a set and $\circ$ a binary operation on two elements from \GG.
\GG is a group if it has \emph{closure}, a \emph{neutral} element, every element has an \emph{inverse} element and it is \emph{associative}:
\begin{itemize}
	\item For all $g,h\in\GG$, $g\circ h \in\GG$
	\item There exists an element $e\in\GG$, called \emph{identity}, such that for all $g\in\GG$, $e\circ g=g=g\circ e$.
	\item For all $g\in\GG$ there exists an \emph{inverse} element $h\in\GG$ such that $g\circ h=e=h\circ g$.
	\item For all $g,h,k\in\GG$, $(g\circ h)\circ k=g\circ (h \circ k)$.
\end{itemize}
\eod
\end{definition}

\noindent
The \emph{order} of a finite group \GG is denoted by $|\GG|$ and is defined as the number of elements in \GG.
An \emph{abelian} group additionally is commutative, \ie for all $g,h\in\GG$, $g\circ h=h\circ g$.
In this work, and cryptography in general, we mainly use cyclic groups.

\begin{definition}[Cyclic Groups]\label{def:cyclicgroups}
Let \GG denote a finite group of order $p$.
\GG is \emph{cyclic} if there exists a \emph{generator} $g\in\GG$ such that $\{g^0,g^1,\dots,g^{p-1}\}=\GG$.
\eod
\end{definition}

\noindent
Since the computational assumptions on groups used in cryptography rely on the cyclic property of groups we have to ensure that all groups used are cyclic.
Therefore, we usually use groups of prime order $p$ since groups of prime order are cyclic.
Another useful feature of prime order groups is that all elements of \GG, except the identity, are generators of \GG.
Working in subgroups of $\ZZ_p$ it is further useful to know that $\ZZ^\ast_p=\{x\in{1,\dots,p-1}~|~\gcd(x,p)=1\}$ is a cyclic group.

\subsection{Computational Assumptions}
In this section we recall some computational assumptions in the group context that are believed to be hard.
The \ac{DLP} is the basis of all assumptions in groups.

\begin{definition}[\acl{DLP}]\label{def:dlp}
Let \GG denote a group of order $p$ with generator $g$.
The \ac{DLP} in \GG states that given a random element $h\rin\GG$ it is hard to compute $x$ such that $h=g^x$.
\eod
\end{definition}

\noindent
Several \ac{DLP}-based assumptions have been proposed.
The two most important ones are the \ac{DDH} and \ac{CDH} assumptions.

\begin{definition}[\acl{DDH}]\label{def:ddh}
Let \GG denote a group of order $p$ with generator $g$.
The \ac{DDH} assumption in \GG states that given $(g,g^a,g^b,g^c)\in\GG^4$ it is hard to determine whether $c=ab$ for random scalars $a,b,c\rin\ZZ_p$.
\eod
\end{definition}

\begin{definition}[\acl{CDH}]\label{def:cdh}
Let \GG denote a group of order $p$ with generator $g$.
The \ac{CDH} assumption in \GG states that given $(g,g^a,g^b)\in\GG^3$ it is hard to compute $g^{ab}$ for random scalars $a,b\rin\ZZ_p$.
\eod
\end{definition}

\section{Definitions}
Asymptotic notation allows us to describe the behaviour of a function when its arguments tend towards some limit.
As mentioned earlier, security models used in this work are from the computational world.
In particular, running time and success or advantage probabilities of algorithms, \ie adversaries, are modelled as functions on a security parameter.
Security is therefore only given for \emph{reasonable} security parameters.
To express this asymptotic notation is used.

\begin{definition}[Negligible Functions]\label{def:negligible}
A function $f$ is \emph{negligible} if for every polynomial $p(\cdot)$ there exists an $N\in\NN$ such that for all $n\in\NN$ with $n>N$ it holds that $f(n)<1/p(n)$.
\eod
\end{definition}

\begin{definition}[Asymptotic Notation]\label{def:asymptotic}
Let $f(n)$ and $g(n)$ denote functions from $\NN_0$ to $\RR_{\geq 0}$.
\begin{itemize}
	\item $f(n)=\cO(g(n)):$ There exist $c,N\in\NN_0$ such that for all $n>N$ it holds that $f(n)\leq c\cdot g(n)$.
	\item $f(n)=\Omega(g(n)):$ There exist $c,N\in\NN_0$ such that for all $n>N$ it holds that $f(n)\geq c\cdot g(n)$.
	\item $f(n)=\Theta(g(n)):$ Both $f(n)=\cO(g(n))$ and $f(n)=\Omega(g(n))$ hold.
	\item $f(n)=o(g(n)):$ $\lim_{m\rightarrow\infty}\frac{f(n)}{g(n)}=0$
	\item $f(n)=\omega(g(n)):$ $\lim_{m\rightarrow\infty}\frac{f(n)}{g(n)}=\infty$
\end{itemize}
\eod
\end{definition}

\paragraph{Probabilistic Polynomial-Time}
We often use the phrase \ac{PPT} to describe an efficient algorithm.
The actual definition of \ac{PPT}, first defined \cite{gill1977}, is given for \ac{PP}, how \ac{PPT} is usually called in complexity theory, as follows:

\begin{definition}[\acl{PP}]\label{def:ppt}
\ac{PP} denotes the class of decision problems solvable by a \ac{PTM} $A$ such that
\begin{itemize}
	\item $A$ runs in polynomial-time,
	\item at least $1/2$ of the computation paths accept when the answer is `yes', and
	\item less than $1/2$ of the computation paths accept when the answer is `no'.
\end{itemize}
\eod
\end{definition}

\noindent
The informal description for \aclp{PTM} is given in Definition \ref{def:ptm}.
We refer the reader to works concerned with complexity theory like \cite{santos1969,WaterlooComplexity} for more on \ac{PTM} and a formal definition.

\begin{definition}[\acl{PTM} \cite{gill1977}]\label{def:ptm}
A \ac{PTM} is a Turing machine with distinguished states called coin-tossing states.
For each coin-tossing state, the finite control unit specifies two possible next states.
The computation of a \ac{PTM} is deterministic except that in coin-tossing states the machine tosses an unbiased coin to decide between the two possible next states.
\end{definition}

\noindent
Note that the running time is always parametrised with the security parameter \secpar.
You can think of \ac{PPT} as a notion for ``feasible strategies'' or ``efficient algorithms'' running in time polynomial in \secpar.
In other words, this means that for some constants $a$ and $c$ the algorithm runs in time $a\cdot \secpar^c$ with security parameter \secpar \cite{katz2008introduction}.

\paragraph{Chosen-Ciphertext Attacks}
\ac{CCA} defines security for encryption schemes.
Note that we always refer to adaptive \ac{CCA}-security, i.e. \ac{CCA}2, when talking about \ac{CCA}-security.

\begin{definition}[IND-CCA2 Security]\label{def:indcca2}
An encryption scheme $\Pi=(\KGen,\Enc,\Dec)$ is \emph{IND-CCA2} secure if for all \ac{PPT} adversaries $\cA$ there exists a negligible function $\varepsilon(\cdot)$ such that :
\[\Adv_{\Pi,\cA}^{\ccatwo}(\secpar)=\left|\Pr[\Exp^{\ccatwo}_{\Pi,\cA}(\secpar)=1]-\frac12\right|\leq\varepsilon(\secpar)\]

\noindent
$\Exp^{\ccatwo}_{\Pi,\cA}(\secpar):$\\
\hspace*{2em}$(\pk,\sk)\ralgout\KGen(\secpar),~b\rin\bits$\\
\hspace*{2em}$(m_0,m_1)\gets\cA^{\Enc_\pk(\cdot),\Dec_\sk(\cdot)}(\secpar,\pk)$\\
\hspace*{2em}$c\gets\Enc_\pk(m_b)$\\
\hspace*{2em}$b'\gets\cA^{\Enc_\pk(\cdot),\Dec_\sk(\cdot)}(\secpar,\pk,m_0,m_1,c)$\\
\hspace*{2em}return $b=b'$
\eod
\end{definition}

\noindent
Note that $\cA$ must not query the decryption oracle with $c$.

\paragraph{The Random Oracle Model}
Many cryptographic proofs are only possible in the \emph{random oracle model} where a public randomly chosen black-box function $H$ is available.
The other popular model is the so-called \emph{standard model} where no such function exists.
While this function $H$ does not actually exist in the real world it is useful in many proves.
It is usually instantiated with a cryptographic hash function.
The function $H$ is queried on an input $x$ and returns the ``hash value'' of $x$.
It is consistent, such that $y=H(x)$ for all $y\gets H(x)$.
The output $y$ of $H$ is uniformly at random such that one can think of $H$ as drawing a random element $y$ each time it is queried on a new $x$.
If $H$ has seen $x$ before, it returns the previously chosen element $y$.

%\subsection{Notations}
%In addition to common notations (we do not recall here), we give here all notations used throughout this work.
%\fk[inline]{See / add what exactly I need later !}
%\begin{itemize}
%	\item If $A$ is an algorithm, then $x\algout A(y)$ denotes running $A$ with input $y$ and storing the result in $x$.
%	\item If $A$ is a randomized algorithm, then $x\ralgout A(y;r)$ denotes running $A$ with input $y$ and randomness $r$, and storing the result in $x$.
%	\item If $U$ is a set, then $x\rin U$ denotes that $x$ is chosen uniformly at random form $U$.
%	\item A variable $y$ is assigned to $x$ by $x:=y$.
%	\item The boolean operations \emph{and} and \emph{or} are denoted by $\wedge$ and $\vee$.
%	\item Exclusive or (xor) is denoted by $\oplus$.
%	\item Set of binary strings of length $n$ $\bits^n$.
%	\item The length of a binary string $x$ is denoted by $|x|$, the bit-length of an integer $y$ is denoted by $\|y\|$.
%	\item $\bigo,\Theta,\Omega,\omega$
%	\item Since $\log_2$ is the most used logarithm we denote it by $\log$.
%	\item The security parameter is denoted $\secpar$.
%	\item Oracle access to $O(\cdot)$ for algorithm $A$ is denoted by $A^{O(\cdot)}$.
%	\item Public/private key-pairs are denoted by $(\pk,\sk)$
%	\item Negligible functions are denoted by $\varepsilon(\cdot)$.
%	\item \aclp{PRF} are denoted by \PRF.
%	\item \ZZ, \NN, \RR, $\NN_0$, $\NN_{\geq 0}$
%\end{itemize}
