\section{Modeling Passwords and Policies}

\subsection{Passwords}
We adopt the reversible, structure-preserving encoding scheme from \cite{KieferM14c} that (uniquely) maps strings of printable ASCII characters to integers.
We use \pwd for the ASCII password string, $c_i=\pwd[i]$ for the $i$-th ASCII character in \pwd, and integer $\pi$ for the encoded password string.
The encoding proceeds as follows: $\pi\gets\pwdint(\pwd)=\sum_{i=0}^{n-1}b^{i} (\ASCII(c_i)-32)$ for the password string \pwd and $\pi_i\gets\chrint(c_i)$ $=\ASCII(c_i)-32$ for the $i$-th \emph{unshifted} ASCII character in \pwd.
Note that $n$ denotes the length of \pwd and $b\in\NN$ is used as shift base.
(We refer to \cite{KieferM14c} for a discussion on the shift base $b$. Note, however, that shift base related attacks on the password verifier from \cite{KieferM14c} are not possible in our two-server setting.) The \ASCII function returns the decimal ASCII code of a character.
In our protocol we will consider the case where password strings \pwd are chosen uniformly at random or according to some distribution with min-entropy $\beta$ from dictionary \cD.

\begin{remark}\label{rem:passwords}
While password distribution is important for the security of a password registration protocol in the verifier-based PAKE setting \cite{KieferM14c}, the password distribution plays a different role in the two-server setting.
Since the server stores only a password share instead of a password verifier, offline dictionary attacks from an attacker controlling only one of the two servers are impossible.
% (Note that this does not mean online dictionary attacks on a registered password are not possible. However, this is out of scope of the security model for a password registration protocol.)
The notion for 2BPR proposed in this work is therefore independent of the password, chosen by the client.
Note however that the password strength still continues to play a role in the usage of 2PAKE/2PASS protocols, where it influences the probability of successful online dictionary attacks.
% That said, one has to be careful not to assume that the client's password choice is not important for the overall security of the system.
% However, this only comes into play in subsequent use of the password but \emph{not} in the setup.
\end{remark}

\subsection{Password Sharing}
We focus on the additive password sharing of client passwords, i.e. $\pi=\share_0+\share_1 \mod q$ over a prime-order group $G_q$.
Such sharing has been used in various two-server PAKE protocols, including  \cite{Katz2005,Yang_Deng_Bao_2006,Jin_Wong_Xu_2007,Kiefer14}.
To be used in combination with two-server password authenticated secret sharing protocols such as \cite{CamenischLN2012} one can define the password as $g^\pi$ and thus create a multiplicative sharing $g^\pi=g^{\share_0}g^{\share_1}$.
Password shares are created as $\share_0\rin\ZZ_q$ and $\share_1=\pi-\share_0 \mod q$.
We remark that other sharing options such as xor \cite{Brainard2003,SzydloK05} have been used in literature but these are not supported by our protocol.


\subsection{Password Policies}
We represent password policies using an approach based on \cite{KieferM14c}, i.e. a password policy $f=(R,\pmin)$ consists of a simplified regular expression $R$ that defines ASCII subsets that must be present in the chosen password string and the minimum length \pmin of the password string.
Note that we do not require a maximum password length \pmax.
$R$ is defined over the \emph{four} ASCII subsets $\Sigma=\{d,u,l,s\}$ with digits $d$, upper case letters $u$, lower case letters $l$ and symbols $s$, and gives the minimum frequency of a character from the subset that is necessary to fulfil the policy; for instance, $R=ulld$ means that the password string must contain at least one upper case letter, two lower case letters and one digit.
In the two-server password setting each of the two servers may have its own policy, i.e. $f_0$ and $f_1$.
The registered passwords must comply with their \emph{mutual policy} defined as $f=f_0\cap f_1=(\max(R_0,R_1),\max(\pmin_0,\pmin_1))$, where $\max(R_0,R_1)$ is the regular expression with the maximum number of characters from each of the subsets $u,l,d,s$ from the two expressions $R_0$ and $R_1$.
%we define the intersection of two policies $f_0$ and $f_1$ that passwords registered through our 2BPR protocol will need to fulfil, namely
A mutual policy is fulfilled, i.e. $f(\pwd)=\true$, iff $f_0(\pwd)=\true$ and $f_1(\pwd)=\true$, and not fulfilled, i.e. $f(\pwd)=\false$, iff $f_0(\pwd)=\false$ or $f_1(\pwd)=\false$.
We mainly operate on the integer representation $\pi$ of a password \pwd throughout this paper and sometimes write $f(\pi)$, which means $f(\pwd)$ for $\pi\gets\pwdint(\pwd)$.
Further note that a character $c_i\in\pwd$ is called \emph{significant} if it is necessary to fulfil a policy expression $R$ and we say the corresponding set $R_j\in R$ is the according \emph{significant set}.
