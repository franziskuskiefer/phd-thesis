\section{Modelling Passwords and Policies}
Passwords and policies in the two-server setting are defined similar to the single-server setting from Chapter \ref{ch:vpake} but require additional treatment for here.

\subsection{Passwords}
We adopt the reversible, structure-preserving encoding scheme introduced in Section \ref{sec:passwords} of Chapter \ref{ch:vpake} that (uniquely) maps strings of printable \ac{ASCII} characters to integers.
We also use \pwd for the \ac{ASCII} password string, $c_i=\pwd[i]$ for the $i$-th \ac{ASCII} character in \pwd, and integer $\pi$ for the encoded password string.
% The encoding proceeds as follows: $\pi\gets\pwdint(\pwd)=\sum_{i=0}^{n-1}b^{i} (\ASCII(c_i)-32)$ for the password string \pwd and $\pi_i\gets\chrint(c_i)$ $=\ASCII(c_i)-32$ for the $i$-th \emph{unshifted} \ac{ASCII} character in \pwd.
% Note that $n$ denotes the length of \pwd and $b\in\NN$ is used as shift base.
% (We refer to \cite{KieferM14c} for a discussion on the shift base $b$. Note, however, that shift base related attacks on the password verifier from \cite{KieferM14c} are not possible in our two-server setting.) The \ASCII function returns the decimal \ac{ASCII} code of a character.
% In our protocol we will consider the case where password strings \pwd are chosen uniformly at random or according to some distribution with min-entropy $\beta$ from dictionary \cD.

\begin{remark}\label{rem:passwords}
While password distribution is important for the security of a password registration protocol in the verifier-based PAKE setting from Chapter \ref{ch:vpake} the password distribution plays a different role in the two-server setting.
Since the server stores only a password share instead of a password verifier, offline dictionary attacks from an attacker controlling only one of the two servers are impossible.
% (Note that this does not mean online dictionary attacks on a registered password are not possible. However, this is out of scope of the security model for a password registration protocol.)
The notion for \ac{2BPR} proposed in this chapter is therefore independent of the password, chosen by the client.
Note however that the password strength still continues to play a role in the usage of \ac{2PAKE}/\ac{PPSS} protocols, where it influences the probability of successful online dictionary attacks.
% That said, one has to be careful not to assume that the client's password choice is not important for the overall security of the system.
% However, this only comes into play in subsequent use of the password but \emph{not} in the setup.
\end{remark}

\subsection{Password Sharing}
We focus on the additive password sharing of client passwords, \ie $\pi=\share_0+\share_1 \mod p$ over a prime-order group $\GG_p$.
Such sharing has been used in various two-server \ac{PAKE} protocols, including \cite{Katz2012a,Yang_Deng_Bao_2006,Jin_Wong_Xu_2007}.
To be used in combination with two-server \ac{PPSS} protocols such as the one proposed by \citet{Camenisch2012} one can define the password as $g^\pi$ and thus create a multiplicative sharing $g^\pi=g^{\share_0}g^{\share_1}$.
Password shares are created as $\share_0\rin\ZZ_p$ and $\share_1=\pi-\share_0 \mod p$.
We remark that other sharing options such as xor \cite{BrainardJKS03,SzydloK05} have been used in literature but these are not supported by the proposed protocol.


\subsection{Password Policies}
We represent password policies using an approach based on the policy definition from Chapter \ref{ch:vpake} Section \ref{sec:passwords}, \ie a simplified password policy $f=(R,\pmin)$ consists of a simplified regular expression $R$ that defines \ac{ASCII} subsets that must be present in the chosen password string and the minimum length \pmin of the password string.
$R$ is defined over the four \ac{ASCII} subsets $\Sigma=\{d,u,l,s\}$ with digits $d$, upper case letters $u$, lower case letters $l$ and symbols $s$, and gives the minimum frequency of a character from the subset that is necessary to fulfil the policy.
In the two-server password setting each of the two servers may have its own policy, \ie $f_0$ and $f_1$.
The registered passwords must comply with their \emph{mutual policy} defined as $f=f_0\cap f_1=(\max(R_0,R_1),\max(\pmin_0,\pmin_1))$, where $\max(R_0,R_1)$ is the policy expression with the maximum number of characters from each of the subsets $u,l,d,s$ from the two expressions $R_0$ and $R_1$.
%we define the intersection of two policies $f_0$ and $f_1$ that passwords registered through our 2BPR protocol will need to fulfil, namely
A mutual policy is thus fulfilled, \ie $f(\pwd)=\true$, iff $f_0(\pwd)=\true$ and $f_1(\pwd)=\true$, and not fulfilled, \ie $f(\pwd)=\false$, iff $f_0(\pwd)=\false$ or $f_1(\pwd)=\false$.
We mainly operate on the integer representation $\pi$ of a password \pwd throughout this chapter and therefore sometimes write $f(\pi)$, which means $f(\pwd)$ for $\pi\gets\pwdint(\pwd)$.
Further recall that a character $c_i\in\pwd$ is called \emph{significant} if it is necessary to fulfil a policy expression $R$ and we say the corresponding set $R_j\in R$ is the according \emph{significant set}.
