\section{Modeling Passwords and Policies} \label{sec:passwords}
In the following we model passwords and their dictionaries. Note that password strings are typically mapped to integers before they are processed in cryptographic operations. For our purposes such integer mapping must be able to preserve password structures. In particular, the way a password string is composed from single characters must remain visible from the resulting integer value. As part of password modeling we describe an appropriate encoding scheme that maps password strings defined over the alphabet of printable ASCII characters to integers while preserving their structures. We further model and define password policies as regular expressions over different ASCII character sets.

\subsection{Password Strings and Dictionaries}
We consider \emph{password strings} $\pwd$ over the \emph{ASCII alphabet} $\Sigma$ containing all 94 \emph{printable} ASCII characters.\footnote{Although we do not consider password strings consisting of other characters, our approach is easily adaptable to UTF-8 and other character sets.} We split $\Sigma=d\cup u\cup l\cup s$ into four subsets:
\begin{itemize}
	\item set of \emph{\bf digits} $d=[0-9]$ (or ASCII codes $[48-57]$),
	\item set of \emph{\bf upper case letters} $u=[A-Z]$ (or ASCII codes $[65-90]$)
	\item set of \emph{\bf lower case letters} $l=[a-z]$ (or ASCII codes $[97-122]$)
	\item set of \emph{\bf symbols} $s=[$!\textquotedbl\#\$\%\&'()*+,-./~:;\textless=\textgreater?@~[\textbackslash]\^{}\_`~\{\textbar\}\textasciitilde$]$ (or ASCII codes $[33-47,58-64,91-96,123-126]$)
\end{itemize}

\noindent
By $\cD$ we denote a \emph{general dictionary} containing all strings that can be formed from printable ASCII characters, i.e. all power sets of $\Sigma$.
A \emph{password string} $\pwd=(c_{0},\dots,c_{n-1})\in\Sigma^n\subset\cD$ of length $n$ is an ordered set of characters $c_i\in\Sigma$.


\subsection{Structure-Preserving Mapping of Password Strings to Integers}\label{sec:pwdencoding}
In the following we show how a password string $\pwd$ can be mapped to an integer $\pi$ for further cryptographic processing in a way that preserves the character composition of $\pwd$ and makes it possible to efficiently reconstruct $\pwd$ from $\pi$. 

%We define the encoding in the following with a variable base $b$ and discuss possible choices for it in Remark \ref{rem:basischoice} at the end of Section \ref{sec:strucphash}.
%Note that the choice of base $b=95$ in the proceedings version of this work \cite{KieferM14b} renders the password hashing protocol in Section \ref{sec:strucphash} insecure.

\subsubsection{Mapping of Password Characters to Integers}
In order to preserve the character structure of a password string $\pwd$ upon its mapping to an integer we first define a \emph{character mapping} function $\chrint:\Sigma\mapsto\ZZ_{95}$ for any printable ASCII character $c\in\Sigma$ that internally uses its decimal ASCII code $\ASCII(c)$ to output an integer in $\ZZ_{95}$:
\[
\chrint(c) =
\left\{
	\begin{array}{ll}
		\bot  & \mbox{if } \ASCII(c) < 32 \\
		\ASCII(c)-32 & \mbox{if } 33 \leq \ASCII(c) \leq 126 \\
		\bot & \mbox{if } 126 < \ASCII(c)
	\end{array}
\right.
\]


\subsubsection{Position-Dependent Mapping of Password Characters to Integers}
A printable ASCII character $c\in\Sigma$ may appear at any position $i\in[0, n-1]$ in a password string $\pwd\in\Sigma^n$. For every position $i$ we require a different integer to which $c_i\in\pwd$ can be mapped to.
Assuming a reasonable upper bound $\pmax$ on the password length $n$, i.e. $n\leq \pmax$, and some \emph{shift base}\footnote{In the proceedings version~\cite{KieferM14b}, the shift base was not parameterized and was set to be 95. While this choice guarantees that each password string $\pwd$ formed of printable ASCII characters will be mapped to a unique integer $\pi$ it is not sufficient for secure processing of $\pi$ within the password hashing scheme, as discussed in Remark \ref{rem:basischoice}.} $b\in\NN$, we define four integer sets $\Omega_x$, $x\in\Sigma' = \{d, u, l, s\}$, where $d$, $u$, $l$, $s$ are the identifiers of the four ASCII character subsets that were used to define $\Sigma$ as follows:
\begin{itemize}
	\item $\Omega_d = \{b^i\chrint(c)\}$ for all digits $c\in d$ and $i=0,\ldots,\pmax-1$. Note that $|\Omega_d|= 10\pmax$.
	\item $\Omega_u = \{b^i\chrint(c)\}$ for all upper case letters $c\in u$ and $i=0,\ldots,\pmax-1$. Note that $|\Omega_u|= 26\pmax$.
	\item $\Omega_l = \{b^i\chrint(c)\}$ for all lower case letters $l\in u$ and $i=0,\ldots,\pmax-1$. Note that $|\Omega_l|= 26\pmax$.
	\item $\Omega_s = \{b^i\chrint(c)\}$ for all symbols $c\in s$ and $i=0,\ldots,\pmax-1$. Note that $|\Omega_s|= 32\pmax$.
\end{itemize}
Any password character $c_i\in\pwd$, $i\in[0,\pmax-1]$ can therefore be mapped to one of the four sets $\Omega_x$, $x\in\Sigma'$ with the \emph{position-dependent character mapping} function $\ichrint:\Sigma\mapsto\Omega_{x}$, defined as
\[
\ichrint(c, i)= b^i\chrint(c),
\]
where the shift base $b\in\NN$ is a public constant.
We write $\pi_i \gets \ichrint(c, i)$ for the integer value of the $i$th character $c_i\in\pwd$.
%Let $\Omega_\Sigma=\Omega_d\cup\Omega_u\cup\Omega_l\cup\Omega_s$.

\subsubsection{Mapping of Password Strings to Integers}


A \emph{password mapping} function $\pwdint:\Sigma^n\mapsto\ZZ_{b^{\pmax}}$ that maps any password string $\pwd=(c_{0},\dots,c_{n-1})\in\Sigma^n$ to an integer in a larger set $\ZZ_{b^{\pmax}}$ in a way that preserves the $i$th position of each character $c_i$ is defined as follows:
\[
\pwdint(\pwd) = \sum_{i=0}^{n-1}b^{i} \chrint(c_{i}) = \sum_{i=0}^{n-1}\ichrint(c_i, i) \textrm{ for } c_i\in\pwd
\]
We will use $\pwd$ to denote a password string and $\pi \gets \pwdint(\pwd)$ for its integer value. Note that $\pi = \sum_{i=0}^{n-1}\pi_i$. %The special constant $\emptyset$ is used to denote zero and $\emptyset$ is not a valid password character.
The mapping computed through $\pwdint$ is injective and reversible. For example, $\pi = 883318$ with $b=100$ is the integer value of password string $\pwd = (\textrm{2},\textrm{A},\textrm{x})$. The string can be recovered by concatenation of $883318\mod 100=18~ \widehat{=}$ 2 at position 0, $(797353\mod 100^2)-(797353\mod 100)=3300=33\cdot 100^1~ \widehat{=}$ A at position $1$ and $797353-(797353\mod95^2)=880000=88\cdot 100^2~ \widehat{=}$ x at position $2$.

\subsection{Alternative Password Mapping}
\mynote{add set-protocols mapping}


\subsection{Password Policies}\label{sec:policies}
A \emph{password policy} $f=(R,\pmin,\pmax)$ is modeled using a \emph{regular expression} $R$ over $\Sigma'=\{d, u, l, s\}$, a \emph{minimum length} $\pmin$ and a \emph{maximum length} $\pmax$ that a password string $\pwd$ must fulfill.\footnote{The way password policies are modeled in this work is suitable for policies that put restrictions on the password length and the nature of password characters. Other types of policies, e.g. search for natural words in a password (cf.~Dropbox password-meter, \url{https://tech.dropbox.com/2012/04/zxcvbn-realistic-password-strength-estimation}) are currently not supported by our framework and thus left for future work.} We write $f(\pwd)=\true$ to indicate that the policy is satisfied by the password string $\pwd$.
%The upper limit $n$ can be picked such that sufficiently long passwords are admitted and will have impact on the efficiency of the corresponding zero-knowledge proofs for policy compliance.
For example, %The following examples illustrate how regular expressions $R$ should be used to define policies:
\begin{itemize}
	\item $f=(\mathtt{ds}, 6, 10)$ means that $\pwd$ must have between 6 and 10 characters with at least one digit and one symbol.
	\item $f=(\mathtt{uss}, 8, 12)$ means that $\pwd$ must have between 8 and 12 characters with at least one upper-case letter and two symbols.
	\item $f=(\mathtt{duls}, 8, 16)$ means that $\pwd$ must have between 8 and 16 characters with at least one character of each type.
\end{itemize}
\begin{remark}Note that in practice password policies do not specify $\pmax$. We leave it for the server administrator to decide whether $\pmax$ should be mentioned explicitly in $f$ or fixed in the system to allow for all reasonable password lengths.\end{remark}

\section{Randomised Password Hashing} \label{sec:strucphash}
A \emph{password hashing} scheme $\Pi$ that is used to compute password verification information for later use in VPAKE protocols from \cite{BenhamoudaP13} is defined as follows:
\begin{itemize}
	\item $\PSetup(\secpar)$ generates password hashing parameters \paramP. These parameters contain implicit descriptions of random salt spaces $\mathbb{S}_P$ and $\mathbb{S}_H$.
	\item $\PPHSalt(\paramP)$ generates a random pre-hash salt $s_P\rin\mathbb{S}_P$.
	\item $\PPreHash(\paramP, \pwd, s_P)$ outputs the pre-hash value $P$.
	\item $\PHSalt(\paramP)$ generates a random hash salt $s_H\rin\mathbb{S}_H$.
	\item $\PHash(\paramP, P, s_P, s_H)$ outputs the hash value $H$.
\end{itemize}
In the above syntax the algorithm $\PPreHash$ is \emph{randomised} with a pre-hash salt $s_P$, which extends the syntax from \cite{BenhamoudaP13}, where $\PPreHash$ is deterministic (and realised in constructions as a random oracle output $\cH(\pwd)$). In contrast we are interested in algebraic constructions of both $\PPreHash$ and $\PHash$ to allow for efficient proofs of knowledge involving pre-hash values $P$. The randomisation of $\PPreHash$ further increases the complexity of an offline dictionary attack that recovers $\pwd$ from $P$ since it removes the ability of an attacker to pre-compute pairs $(P,\pwd)$ and use them directly to recover $\pwd$ (see also Section~\ref{sec:pwreg}). We write $H\gets \HashP(\pwd,r)$ to denote $H\gets \PHash(\paramP,P,s_P$, $s_H)$ with $P\gets\PPreHash(\paramP,\pwd,s_P)$, where $r=(s_P,s_H)$ combines the randomness used in \PHash and \PPreHash. A secure $\Pi$ must satisfy the following security properties. Note that password-hiding is a new property that is used in ZKPPC to ensure that password hashes $H$ do not leak any information about $\pwd$. The remaining four properties are from \cite{BenhamoudaP13}, updated where necessary to account for the randomised $\PPreHash$:
\begin{itemize}
\item Password hiding: For all PPT algorithms $A=(A_1,A_2)$ where $A_1$ on input $\paramP\gets\PSetup(\secpar)$ outputs two equal-length password strings $\pwd_0$ and $\pwd_1$  and where $A_2$ on input $H\gets\PHash(\paramP,P,s_P,s_H)$, where $s_H\gets\PHSalt(\paramP)$, $s_P\gets\PPHSalt(\paramP)$, and $P\gets\PPreHash(\paramP,\pwd_b, s_P)$ for a random bit $b\rin\bits$ outputs bit $b'$ there exists a negligible function $\varepsilon(\cdot)$ such that
$|\Pr[b'=b]-\frac12|\leq\varepsilon(\secpar)$.
%For all PPT algorithms $A$ there exists a negligible function $\varepsilon(\cdot)$ such that
%\[\Pr[(i,\pwd)\gets A^{\HashPO(\cdot)}(\paramP);~ \Finalise(i,\pwd) = 1]\leq \varepsilon(\secpar),\]
%where $\paramP\gets\PSetup(\secpar)$ and each $i$th invocation of $\HashPO(\cdot)$ returns $H\gets\PHash(\paramP,P,s_P,s_H)$, where $s_H\gets\PHSalt(\paramP)$, $s_P\gets\PPHSalt(\paramP)$, $P\gets\PPreHash(\paramP,\pwd, s_P)$, $\pwd\rin\cD$, and stores $T[i]\gets\pwd$. $\Finalise(i,\pwd) = 1$ if $T[i]=\pwd$.

\item Pre-image resistance (called tight one-wayness in \cite{BenhamoudaP13}): For all PPT algorithms $A$ running in time at most $t$, there exists a negligible function $\varepsilon(\cdot)$ such that
\[\Pr[(i,P)\gets A^{\HashPO(\cdot)}(\paramP);~ \Finalise(i,P) = 1]\leq \frac{\alpha t}{2^\beta t_\PPreHash}+\varepsilon(\secpar),\]
for small $\alpha$ and $t_\PPreHash$ being the running time of \PPreHash, where $\paramP\gets\PSetup(\secpar)$ and each $i$th invocation of $\HashPO(\cdot)$ returns $(H,s_H)$ with $H\gets\PHash(\paramP,P,s_P,s_H)$ and stores $T[i]\gets\PPreHash(\paramP,\pwd, s_P)$, where $s_H\gets\PHSalt(\paramP)$, $s_P\gets\PPHSalt(\paramP)$, and $\pwd\rin\cD$.
$\Finalise(i,P) = 1$ if $T[i]=P$. (Note that $\HashPO(\cdot)$ does not return $s_P$.)

\item Second pre-image resistance: For all PPT algorithms $A$ there exists a negligible function $\varepsilon(\cdot)$ such that for $P'\gets A(\paramP, P, s_H)$
\[\Pr[P'\not=P \wedge \PHash(\paramP,P,s_H)=\PHash(\paramP,P',s_H)] \leq \varepsilon(\secpar),\]
with $\paramP\gets\PSetup(\secpar), s_P\gets\PPHSalt(\paramP), s_H\gets\PHSalt(\paramP)$ and $P\gets\PPreHash$ $(\paramP, \pwd, s_P)$ for any $\pwd\in\cD$.

\item Pre-hash entropy preservation: For all polynomial time samplable dictionaries $\cD$ with min-entropy $\beta$, and any PPT algorithm $A$, there exists a negligible function $\varepsilon(\secpar)$ such that for $(P,s_P)\gets A(\paramP)$ with $\paramP\gets\PSetup(\secpar)$ and random password $\pwd\rin\cD$:
\[\Pr[s_P\in\SSS_P \wedge P=\PPreHash(\paramP,\pwd,s_P)] \leq 2^{-\beta} + \varepsilon(\secpar).\]

\item Entropy preservation: For all polynomial time samplable dictionaries $\cD$ with min-entropy $\beta$, and any PPT algorithm $A$, there exists a negligible function $\varepsilon(\secpar)$ such that for $(H,s_P,s_H)\gets A(\paramP)$
\[\Pr[s_P\in\SSS_P \wedge s_H\in\SSS_H \wedge H=\HashP(\paramP,\pwd,s_P,s_H)] \leq 2^{-\beta} + \varepsilon(\secpar),\]
where $\paramP\gets\PSetup(\secpar)$ and $\pwd\rin\cD$.
\end{itemize}

%\noindent
%The use of a pre-hash salt is motivated by the fact that algebraic hash functions usually allow to pre-compute certain values.
%Using these pre-computed values and the salt it is easy for a server to retrieve the password.
%Using separate salts for pre-hash and hash functions allows us to prevent the server from precalculating password hashes.

%While random oracle-like password hashing is not usually vulnerable to precalculations, it seems difficult to build precalculation-secure, pure algebraic password hashing schemes without additional randomness.

%The main reason for the authors of \cite{BenhamoudaP13} not to use randomness in \PPreHash seems the increased round complexity in VPAKE protocols using these password hashes.
%However, it is possible to build one-round VPAKE protocols from their framework using randomness in \PPreHash, as we will see later.


%Further note that we assume implicit opportunistic encoding, i.e. if the input \pwd to the hashing is a character string it applies $\Encode$ first.
%We consider (second) pre-image resistance and entropy preservation from \cite{BenhamoudaP13} as security properties for password hashing.
%Due to space limitations we refer to Appendix \ref{app:pwdhashing} for formal definition of those properties.
%Note that we change pre-image resistance by considering time to compute \PPreHash instead of \PHash.
%This allows us to prove our algebraic password hashing secure due to the additional pre-hash randomness.\footnote{A stronger notion where both \PPreHash and \PHash times are considered is possible, but difficult to reach in a purely algebraic setting.}
%While those definitions are password hashing related, we further consider common one-wayness security of \HashP to ensure security against an attacker that does not know the according salts.
%Note that one-wayness from \cite{BenhamoudaP13} is called pre-image resistance here to separate it from the common one-wayness.
%
%\myparagraph{One-wayness}
%We use the common definition for one-wayness with syntax appropriate for the used definition of password hashing.
%A password hashing protocol $\Pi$ is a one-way function if for all PPT algorithms $A$, there exists a negligible function $\varepsilon(\cdot)$ such that for $\pwd'\gets A^{\HashPO(\cdot)}(\paramP)$:
%$\Pr[ \Finalise(i,\pwd') = 1]\leq \varepsilon(\secpar)$,
%where $\HashPO$ returns $H\gets\HashP(\paramP,\pwd,r)$ for $s_H\gets\PHSalt(\paramP)$, $s_P\gets\PPHSalt(\paramP)$, global parameters $\paramP\gets\PSetup(\secpar)$ and random password $\pwd\rin\cD$ on the $i$-th invocation, and stores $T[i]\gets\pwd$.
%\Finalise returns $1$ if $T[i]=\pwd'$, otherwise $0$.

\subsection{Randomised Password Hashing from Pedersen Commitments}\label{sec:pwhashped}
We introduce a randomised password hashing scheme $\Pi=(\PSetup,\PPHSalt,\allowbreak\PPreHash,\allowbreak\PHSalt$, $\PHash)$ for ASCII-based passwords using Pedersen commitments. We assume that $\pi\gets\pwdint(\pwd)$ for an appropriate choice of the shift base $b$ (see Remark~\ref{rem:basischoice}) and construct $\Pi$ as follows:
\begin{itemize}
	\item $\PSetup(\secpar)$ generates $\paramP= (p,g,h,\secpar)$ where $g$, $h$ are independent generators of a cyclic group $G$ of prime order $p$ of length $\secpar$.
	\item $\PPHSalt(\paramP)$ generates a pre-hash salt $s_P\rin\Zrp$.
	\item $\PPreHash(\paramP, \pi, s_P)$ outputs the pre-hash value $P=g^{s_P\pi}$.
	\item $\PHSalt(\paramP)$ generates a hash salt $s_H\rin\Zrp$.
	\item $\PHash(\paramP, P, s_P, s_H)$ outputs hash value $H=(H_1, H_2)=(g^{s_P}, Ph^{s_H})$.
\end{itemize}
Observe that $H_2=H_1^\pi h^{s_H}$, i.e., $H_1$ can be seen as a fresh generator that is used to compute the Pedersen commitment $H_2$.
%Let $\paramP=(p,g,h,\secpar)\gets\PSetup(\secpar)$,
%$\ZZ_p\ni_R s_P\gets\PPHSalt(\paramP)$,
%$\ZZ_p\ni_R s_H\gets\PPHSalt(\paramP)$,
%$P=g^{s_P\cdot\pi}\gets\PPreHash(\paramP,\pwd,s_P)$ and
%$H=(H_1,H_2)=(g^{s_P}, Ph^{s_H})\gets\PHash(\paramP, P, s_P, s_H)$.
%Note that this allows us to compute $H_2$ from $(\pwd,H_1,s_H)$ as $H_2=H_1^\pi h^{s_H}$. %=\HashP(\paramP, \pwd, H_1, s_H)
The security properties of our password hashing scheme $\Pi$ follow from the properties of the underlying cyclic group $G$ and from the security of Pedersen commitments. We argue informally:
\begin{itemize}
\item
The \emph{password hiding} property of the scheme, assuming that $\pwd_0$ and $\pwd_1$ are mapped to corresponding integers $\pi_0$ and $\pi_1$ in $\ZZ_{b^n}$, is perfect and holds based on the perfect hiding property of the Pedersen commitment scheme. Note that the adversary receives the corresponding hash value $H=(H_1, H_2)=(g^{s_P}, Ph^{s_H})$, where $H_2= g^{s_P\pi}h^{s_H}$ is a Pedersen commitment on $\pi$ with respect to two independent bases $g^{s_P}$ and $h$. The ability of $A$ to distinguish between $\pi_0$ and $\pi_1$ can thus be turned into the attack on the hiding property of the commitment scheme.

\item
The \emph{pre-image resistance} holds since $s_P$ and $s_H$ are randomly chosen on every invocation of $\HashPO(\cdot)$ with a negligible probability for a collision and $H_2$ is a perfectly hiding commitment with bases $g^{s_P}$ and $h$. Therefore, for any given output $(H=(H_1, H_2), s_H)$ of $\HashPO(\cdot)$, $A$ must perform $2^\beta$ exponentiations $H_1^{\pi^\ast}$, one for each candidate $\pi^\ast$, in order to find $P = H_2h^{-s_H}$. This roughly corresponds to $2^\beta$ invocations of $\PPreHash$.

\item
The \emph{second pre-image resistance} holds since $H_1$ is uniform in $G$ and $H_2$ is a computationally binding commitment with bases $g^{s_P}$ and $h$. Note that for any $P'$ generated by $A$, $H_1^\pi h^{s_H}$ $=P'h^{s_H}$ is true only if $P'=H_1^\pi$.
	
	\item
The \emph{pre-hash entropy} and \emph{hash entropy} preservation hold since $H_1$ is a generator of $G$ such that for every $(P,s_P)$ chosen by the pre-hash entropy adversary, $\Pr[P=H_1^\pi]\leq2^{-\beta}+\varepsilon(\secpar)$, and for every $(H,s_H)$ chosen by the hash entropy adversary, $\Pr[H_2=H_1^\pi h^{s_H}]\leq2^{-\beta}+\varepsilon(\secpar)$ for a random $\pwd\rin\cD$.

%	\item Entropy preservation holds since $\Pr[H=s_P^\pwd h^{s_H}]\leq2^{-\beta}+\varepsilon(\secpar)$ for any randomly chosen password \pwd and the attacker's has to output $(H,s_H)$.
\end{itemize}

\subsection{Choosing the Shift Base $b$}\label{rem:basischoice}
As Benhamouda and Pointcheval point out in \cite{BenhamoudaP13}, the encoded password (integer) $\pi$ in $P=H_1^\pi$ can be computed from $(H, s_H)$ in time $O(\sqrt{n})$ for $\pi \in [0,n-1]$, e.g. using variants of the Pollard's kangaroo algorithm, and if $n$ corresponds to the dictionary size $|\cD|$ then the above password hashing scheme is no longer pre-image resistant. We observe, however, that
%be defined in the proceedings version of this work \cite{KieferM14b} using shift base $b=95$ is not pre-image resistant since discrete logarithms $x\in [1, n]$ can be computed in time $O(\sqrt{n})$, e.g. using algorithms like Pollard's kangaroo, baby-step-giant-step, etc. If $b=95$ then an attacker can recover$ 
the size of $n$ is \emph{larger} than the dictionary size $|\cD|$. To see this, first note that 
\[n=\sum_{i=0}^{|\pwd|-1}b^i\cdot 93,\] 
which denotes the largest possible value of $\pi \gets \pwdint(\pwd)$ obtained from a password string of length $|\pwd|$ (normalised to $0$ being the smallest possible $\pi$). 
In order to see the difference to the dictionary size $|\cD|$ we analyse how the password length $|\pwd|$ impacts the interval of the corresponding integer $\pi$: for example, if $|\pwd| = 1$ then $\pi$ is in $[1,94]$; if $|\pwd|=2$ then $\pi$ is in $[b+1,b+94]\cup \dots \cup[94b+1,94b+94]$; if $|\pwd|=3$ then $\pi$ is in  $[1+b+b^2,94+b+b^2]\cup\dots\cup[1+94b+94b^2,94+94b+94b^2]$, and so on. 
%In particular, a dictionary $\cD$ consists of encoded passwords from ranges $[1,94]$ for password length $1$, $[b+1,b+94],...,[94b+1,94b+94]$ for password length $2$, $[1+b+b^2,94+b+b^2],...,[1+94b+94b^2,94+94b+94b^2]$ for password length $3$ and so on.
That is in contrast to the size of the dictionary $\cD$ (assuming for now that policy $f$ does not restrict the character choice), which corresponds to the number of $|\pwd|$-tuples containing characters chosen out of all $94$ printable ASCII characters and is therefore given by 
\[|\cD| = 94^{|\pwd|}\]
and therefore independent of the shift base $b$.
In order to guarantee the pre-image resistance of the password hashing scheme we need to ensure that $|\cD| \approx \sqrt{n}$.
%, which makes $O(\sqrt{n})$ not more efficient than a brute-force attack on dictionary $\cD$. 
This can be achieved by choosing the shift base $b$ such that computing $\pi$ from $P=H_1^\pi$ becomes as difficult as a brute-force search over the dictionary $\cD$. 
In particular, the shift base $b$ must be chosen such that 
\begin{equation}\label{eq:hashlimit}
  |\mathcal{D}|^2\leq \sum_{i=0}^{|\pwd|-1}b^i\cdot 93.
\end{equation}
Since the required value of $b$ depends on $|\pwd|$ as well as policy $f$ it is possible for a specific password policy $f=(R,\pmin,\pmax)$ to compute the optimal value for $b$ using the password length restrictions $\pmin,\pmax$ and regular expression $R$, and by this to optimise the performance of the password hashing scheme (and of our ZKPPC protocol) with respect to the given policy. 
Using dictionary size 
\[|\cD_f|=94^{|\pwd|-|R|}\prod_{i=0}^{|R|-1}|R_i|,\]
for policy $f$ allows to compute an optimal $b$ such that Eq. \ref{eq:hashlimit} holds for all policy compliant passwords.
If such optimisation is not required then we recommend setting $b=10^5$, which should be a safe choice for all sensible policies and password lengths (in contrast to $b=95$ that was used in~\cite{KieferM14b}).
% A shift base of $b=10^5$ ensures in particular that $|\mathcal{D}|^2\leq n$ for all passwords of at least $5$ characters and allowing for some leeway defining regular expression $R$.
Figure \ref{fig:choiceb} depicts relations among password length and optimal shift base $b$ for the general case (without regular expression) and with regular expression $R=\mathtt{duls}$.
First it shows that $b=10^5$ is a safe choice.
Further, it shows that the regular expression has a significant influence on the choice of an optimal base $b$, i.e. can be used to tweak performance of a ZKPPC protocol.
In the general case we see that base $b$ grows exponentially the shorter the password gets, which is responsible for the relatively large suggested default base $b=10^5$.
With regular expression $\mathtt{duls}$ in contrast $b$ has to \emph{grow} with increasing password length.
This is due to the restriction on the character sets from the regular expression.


\begin{figure}[tbh]
\centering
\begin{tikzpicture}[x=1cm,y=1cm]

 \draw[latex-latex, thin, draw=gray, ->] (0,0)--(6,0) node [right] {$b$}; 
 \draw[latex-latex, thin, draw=gray, ->] (0,0)--(0,8) node [above] {$|\pwd|$}; 
 
   \foreach \y in  {1,2,3,4,5,6,7}
  \draw[shift={(0,\y)},color=black] (3pt,0pt) -- (-3pt,0pt);
  \foreach \y in {1,2,3,4,5,6,7} 
  \draw[shift={(0,\y)},color=black] (0pt,0pt) -- (-3pt,0pt) node[left] 
  {$\the\numexpr\y+3$};

  
 \draw[latex-latex, thin, draw=gray, ->] (7,0)--(7,8) node [above] {$|\pwd|$}; 
 \draw[latex-latex, thin, draw=gray, ->] (7,0)--(13,0) node [right] {$b$};
 
   \foreach \y in  {1,2,3,4,5,6,7}
  \draw[shift={(7,\y)},color=black] (3pt,0pt) -- (-3pt,0pt);
  \foreach \y in {1,2,3,4,5,6,7} 
  \draw[shift={(7,\y)},color=black] (0pt,0pt) -- (-3pt,0pt) node[left] 
  {$\the\numexpr\y+3$};

  
  \foreach \x in  {1,2,3,4,5}
  \draw[shift={(\x,0)},color=black] (0pt,3pt) -- (0pt,-3pt);
  \foreach \x in {1,3,5} 
  \draw[shift={(\x,0)},color=black] (0pt,0pt) -- (0pt,-3pt) node[below] 
  {\small $\the\numexpr\x*1000$}; 
  \foreach \x in {2,4} 
  \draw[shift={(\x,0)},color=black] (0pt,0pt) -- (0pt,-10pt) node[below] 
  {\small $\the\numexpr\x*1000$};
  
  \foreach \x in  {1,2,3,4,5}
  \draw[shift={(\the\numexpr\x+7,0)},color=black] (0pt,3pt) -- (0pt,-3pt);
  \foreach \x in {1,3,5} 
  \draw[shift={(\the\numexpr\x+7,0)},color=black] (0pt,0pt) -- (0pt,-3pt) node[below] 
  {\small $\the\numexpr\x*10000$}; 
  \foreach \x in {2,4} 
  \draw[shift={(\the\numexpr\x+7,0)},color=black] (0pt,0pt) -- (0pt,-10pt) node[below] 
  {\small $\the\numexpr\x*10000$};
  
  
% no R
% l = 3 -> b >= 86128
% l = 4 -> b >= 40320
% l = 5 -> b >= 27587
% l = 6 -> b >= 21969
% l = 7 -> b >= 18875
% l = 8 -> b >= 16936
% l = 9 -> b >= 15613
% l = 10 -> b >= 14656
\foreach \Point in {(\the\numexpr4.0320+7, 1), (\the\numexpr2.7587+7, 2), (\the\numexpr2.1969+7, 3), (\the\numexpr1.8875+7, 4), (\the\numexpr1.6936+7, 5), (\the\numexpr1.5613+7, 6), (\the\numexpr1.4656+7, 7)}{
    \node at \Point {\textbullet};
}


% R=duls
% l = 4 -> b >= 796
% l = 5 -> b >= 1452
% l = 6 -> b >= 2084
% l = 7 -> b >= 2651
% l = 8 -> b >= 3149
% l = 9 -> b >= 3582
% l = 10 -> b >= 3960
% l = 20 -> b >= 6042
% l = 40 -> b >= 7343
% l = 60 -> b >= 7818
% l = 80 -> b >= 8064
% l = 100 -> b >= 8215
\foreach \Point in {(0.796, 1), (1.452, 2), (2.084, 3), (2.651, 4), (3.149, 5), (3.582, 6), (3.960, 7)}{
    \node at \Point {$\circ$};
}

% to ensure that the points are being properly centered:
\draw [dotted, gray] (0,0) grid (6,8);
\draw [dotted, gray] (7,0) grid (13,8);
% \node [red] at (3,2.5) {\textbullet};
% \node [blue] at (3,-2.5) {$\circ$};

\end{tikzpicture}
\caption{Optimal Choice for $b$\\\textbullet without regular expression $R$\\$\circ$ with regular expression $R=\mathtt{duls}$}\label{fig:choiceb}
\end{figure}

% Considering password policies exemplified in Section \ref{sec:policies} we can define the following optimised basis.
% \begin{itemize}
% 	\item $\cD_{f,6}$ with $f=(\mathtt{ds}, 6, 10)$ contains $24,983,966,720$ passwords, which leads to a basis $b\geq 14,424$.
% % 	      $\cD_{f,10}$ with $f=(\mathtt{ds}, 6, 10)$ contains $1,950,620,603,331,461,120$ passwords, which leads to a basis $b\geq 11601$.
% 	      
% 	\item $\cD_{f,8}$ with $f=(\mathtt{uss}, 8, 12)$ contains $195,394,606,923,776$ passwords, which leads to a basis $b\geq 12,110$.
% % 	      $\cD_{f,12}$ with $f=(\mathtt{uss}, 8, 12)$ contains $15,255,413,614,534,691,127,296$ passwords, which leads to a basis $b\geq 10,799$.
% 	
% 	\item $\cD_{f,8}$ with $f=(\mathtt{duls}, 8, 16)$  contains $16,889,161,502,720$ passwords, which leads to a basis $b\geq 6,016$.
% \end{itemize}
%To the best of our knowledge no variant of the Kangaroo algorithm exists that allows to take the smaller subsets of correct solutions into account since the computation of possible solutions is performed by splitting the exponent (password) and therefore destroying the ability to use the fact that valid solutions come from a smaller subset of valid integer values.
%Further note that using Pollard's kangaroo on every subset separately is not efficient as each subset only holds $94$ possible solutions.
