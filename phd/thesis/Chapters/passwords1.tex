\section{Modelling Passwords and Policies} \label{sec:passwords}

In the following we describe how to model passwords, dictionaries they are chosen from, and password policies. 
Password strings have to be mapped to integers before they are used in cryptographic operations. 
For our purposes such an integer mapping must be able to preserve the password structure. 
In particular, the way a password string is composed from single characters must remain visible from the resulting integer value. 
As part of password modelling we describe an appropriate encoding scheme that maps password strings defined over the alphabet of printable \ac{ASCII} characters to integers while preserving their structure. 
We further model and define password policies as some kind of regular expressions over different \ac{ASCII} character sets.
In addition to the main model we describe an alternative way of modelling that is used in Section \ref{sec:spc-bpr}

\subsection{Password Strings and Dictionaries}
We consider \emph{password strings} $\pwd$ over the \emph{\ac{ASCII}  alphabet} $\Sigma$ containing all 94 \emph{printable} \ac{ASCII}  characters.
Although we do not consider password strings consisting of other characters, our approach is easily adaptable to \ac{UTF-8} and other character sets.
We split $\Sigma=d\cup u\cup l\cup s$ into four subsets:
\begin{itemize}
	\item set of \emph{\bf digits} $d=[0-9]$ (or \ac{ASCII}  codes $[48-57]$),
	\item set of \emph{\bf upper case letters} $u=[A-Z]$ (or \ac{ASCII}  codes $[65-90]$)
	\item set of \emph{\bf lower case letters} $l=[a-z]$ (or \ac{ASCII}  codes $[97-122]$)
	\item set of \emph{\bf symbols} $s=[$!\textquotedbl\#\$\%\&'()*+,-./~:;\textless=\textgreater?@~[\textbackslash]\^{}\_`~\{\textbar\}\textasciitilde$]$ (or \ac{ASCII}  codes $[33-47,58-64,91-96,123-126]$)
\end{itemize}

\noindent
By $\cD$ we denote a \emph{general dictionary} containing all strings that can be formed from printable \ac{ASCII}  characters, \ie all power sets of $\Sigma$.
A \emph{password string} $\pwd=(c_{0},\dots,c_{n-1})\in\Sigma^n\subset\cD$ of length $n$ is an ordered set of characters $c_i\in\Sigma$.

\subsection{Password Mapping --- From Strings to Integers}\label{sec:pwdencoding}
In the following we show how a password string $\pwd$ can be mapped to an integer $\pi$ for further cryptographic processing in a way that preserves the character composition of $\pwd$ and makes it possible to efficiently reconstruct $\pwd$ from $\pi$. 

%We define the encoding in the following with a variable base $b$ and discuss possible choices for it in Remark \ref{rem:basischoice} at the end of Section \ref{sec:strucphash}.
%Note that the choice of base $b=95$ in the proceedings version of this work \cite{KieferM14b} renders the password hashing protocol in Section \ref{sec:strucphash} insecure.

\subsubsection{Mapping of Password Characters to Integers}
In order to preserve the character structure of a password string $\pwd$ upon its mapping to an integer we first define a \emph{character mapping} function $\chrint:\Sigma\mapsto\ZZ_{95}$ for any printable \ac{ASCII}  character $c\in\Sigma$ that internally uses its decimal \ac{ASCII}  code $\ASCII(c)$ to output an integer in $\ZZ_{95}$:
\[
\chrint(c) =
\left\{
	\begin{array}{ll}
		\bot  & \mbox{if } \ASCII(c) < 32 \\
		\ASCII(c)-32 & \mbox{if } 33 \leq \ASCII(c) \leq 126 \\
		\bot & \mbox{if } 126 < \ASCII(c)
	\end{array}
\right.
\]


\subsubsection{Position-Dependent Mapping of Password Characters to Integers}
A printable \ac{ASCII}  character $c\in\Sigma$ may appear at any position $i\in[0, n-1]$ in a password string $\pwd\in\Sigma^n$. For every position $i$ we require a different integer to which $c_i\in\pwd$ can be mapped to.
Assuming a reasonable upper bound $\pmax$ on the password length $n$, \ie $n\leq \pmax$, and some \emph{shift base} $b\in\NN$, we define four integer sets $\Omega_x$, $x\in\Sigma' = \{d, u, l, s\}$, where $d$, $u$, $l$, $s$ are the identifiers of the four \ac{ASCII}  character subsets that were used to define $\Sigma$ as follows:
\begin{itemize}
	\item $\Omega_d = \{b^i\chrint(c)\}$ for all digits $c\in d$ and $i=0,\ldots,\pmax-1$. Note that $|\Omega_d|= 10\pmax$.
	\item $\Omega_u = \{b^i\chrint(c)\}$ for all upper case letters $c\in u$ and $i=0,\ldots,\pmax-1$. Note that $|\Omega_u|= 26\pmax$.
	\item $\Omega_l = \{b^i\chrint(c)\}$ for all lower case letters $c\in u$ and $i=0,\ldots,\pmax-1$. Note that $|\Omega_l|= 26\pmax$.
	\item $\Omega_s = \{b^i\chrint(c)\}$ for all symbols $c\in s$ and $i=0,\ldots,\pmax-1$. Note that $|\Omega_s|= 32\pmax$.
\end{itemize}
Any password character $c_i\in\pwd$, $i\in[0,\pmax-1]$ can therefore be mapped to one of the four sets $\Omega_x$, $x\in\Sigma'$ with the \emph{position-dependent character mapping} function $\ichrint:\Sigma\mapsto\Omega_{x}$, defined as
\[
\ichrint(c, i)= b^i\chrint(c),
\]
where the shift base $b\in\NN$ is a public constant.
We write $\pi_i \gets \ichrint(c, i)$ for the integer value of the $i$th character $c_i\in\pwd$.
%Let $\Omega_\Sigma=\Omega_d\cup\Omega_u\cup\Omega_l\cup\Omega_s$.

\subsubsection{Mapping of Password Strings to Integers}


A \emph{password mapping} function $\pwdint:\Sigma^n\mapsto\ZZ_{b^{\pmax}}$ that maps any password string $\pwd=(c_{0},\dots,c_{n-1})\in\Sigma^n$ to an integer in a larger set $\ZZ_{b^{\pmax}}$ in a way that preserves the $i$th position of each character $c_i$ is defined as follows:
\[
\pwdint(\pwd) = \sum_{i=0}^{n-1}b^{i} \chrint(c_{i}) = \sum_{i=0}^{n-1}\ichrint(c_i, i) \textrm{ for } c_i\in\pwd
\]
We will use $\pwd$ to denote a password string and $\pi \gets \pwdint(\pwd)$ for its integer value. Note that $\pi = \sum_{i=0}^{n-1}\pi_i$. %The special constant $\emptyset$ is used to denote zero and $\emptyset$ is not a valid password character.
The mapping computed through $\pwdint$ is injective and reversible. For example, $\pi = 883318$ with $b=100$ is the integer value of password string $\pwd = (\textrm{2},\textrm{A},\textrm{x})$. The string can be recovered by concatenation of $883318\mod 100=18~ \widehat{=}$ 2 at position 0, $(797353\mod 100^2)-(797353\mod 100)=3300=33\cdot 100^1~ \widehat{=}$ A at position $1$ and $797353-(797353\mod95^2)=880000=88\cdot 100^2~ \widehat{=}$ x at position $2$.



\subsection{Password Policies}\label{sec:policies}
A password policy is defined to impose requirements on password complexity in terms of the minimum number of characters, minimal number of character classes, and minimal number of characters in each class.
A \emph{password policy} $f=(R,\pmin,\pmax)$ is modelled using some sort of \emph{regular expression} $R$ over $\Sigma'=\{d, u, l, s\}$, a \emph{minimum length} $\pmin$ and a \emph{maximum length} $\pmax$ that a password string $\pwd$ must fulfil.
The way password policies are modelled in this work is suitable for policies that put restrictions on the password length and the nature of password characters. Other types of policies, \eg search for natural words in a password (cf.~Dropbox password-meter, \url{https://tech.dropbox.com/2012/04/zxcvbn-realistic-password-strength-estimation}) are currently not supported by our framework and thus left for future work.
We write $f(\pwd)=\true$ to indicate that the policy is satisfied by the password string \pwd.
%The upper limit $n$ can be picked such that sufficiently long passwords are admitted and will have impact on the efficiency of the corresponding zero-knowledge proofs for policy compliance.
For example, %The following examples illustrate how regular expressions $R$ should be used to define policies:
\begin{itemize}
	\item $f=(\mathtt{ds}, 6, 10)$ means that $\pwd$ must have between 6 and 10 characters with at least one digit and one symbol.
	\item $f=(\mathtt{uss}, 8, 12)$ means that $\pwd$ must have between 8 and 12 characters with at least one upper-case letter and two symbols.
	\item $f=(\mathtt{duls}, 8, 16)$ means that $\pwd$ must have between 8 and 16 characters with at least one character of each type.
\end{itemize}

\noindent
A character $c_i\in\pwd$ is called \emph{significant} if it is necessary to fulfil a policy expression $R$ and say the according set $R_j\in R$ is the according \emph{significant set}.
For every $R_j\in R$ the first occurrence of a character $c_i\in R_j$ is considered significant.
Note that $\Sigma$, and thus $d,u,l$ and $s$, in this work can refer to the set of encoded characters, or the set of \ac{ASCII}  characters, depending on the context.

\begin{remark}
Note that in practice password policies do not specify \pmax. 
We leave it for the server administrator to decide whether \pmax should be mentioned explicitly in $f$ or fixed in the system to allow for all reasonable password lengths.
A password policy is called simplified if \pmax is omitted, \ie a simplified policy is given by $f=(R,\pmin)$.
\end{remark}

% \subsubsection{Simplified Password Policies}
% A \emph{password policy} is defined as a tuple $f=(R,\pmin)$, where $R$ is a policy expression over $\Sigma=\{d, u, l, s\}$, where $d$ denotes digits, $u$ upper case letters, $l$ lower case letters, $s$ symbols, and \pmin defines the minimum length of a password.
% We do not specify an upper limit on the password length in this simplified password policy.
% A policy expression $R$ over $\Sigma$ is a simplified regular expression that only specifies the sets necessary for a string to fulfil the expression.
% In particular, it specifies the minimum number of occurrences of elements from $\Sigma$ in the password string, e.g., $R=dl$ requires \pwd to have at least one digit and one lower case letter, and $R=ssd$ requires \pwd to have at least two symbols and one digit.
% We write $f(\pwd)=\true$ to indicate that the policy is satisfied by the password string \pwd.

\subsection{An Alternative Definition}
Instead of the previously described definition of passwords and policies we propose an alternative description that giving a set-theoretical approach.
Note that only the mapping of passwords to cryptographic elements and the policy definition changes, \emph{not} the definition of dictionaries, character sets, or passwords.

\subsubsection{An Alternative Policy Definition} \label{sec:lsss}
An alternative approach of defining password policies is a set-theoretical representation, \ie \emph{monotone access structures} \cite{ito89}.
A policy $P$ defines a pair $(\cS,\Gamma_\cS)$ where $\cS$ is a set and $\Gamma_\cS$ is an access structure over $\cS$. 
The access structure is a subset of the powerset $2^\cS$. 
We say an access structure $\Gamma_\cS$ is monotonic if for each element in $\Gamma_\cS$, all its superset is also in $\Gamma_\cS$. 
We say a set $\cC$ satisfies a policy $P$, written as $P(\cC)=\true$, if $\cC\in \Gamma_\cS$. 
A set $\cC$ that satisfies $P$ is called an authorised set. 
Access structures capture many complex access control and authorisation policies. For example, $\cS$ can be a set of credentials and $\Gamma_\cS$ defines subsets of credentials that are required for authorisation.

It has long been known that an access structure can be mapped to a \ac{LSSS} \cite{ito89,bei96}. 
Given an access structure $\Gamma_\cS$ defined over $\cS$, one can choose a secret and split it into a set of shares accordingly. 
Each share is associated with an element in $\cS$ and the following holds: 
(1) any set of shares can reconstruct the secret if the elements associated with the shares form an authorised set, and 
(2) any set of shares does not reveal any information about the secret if the elements associated with the shares do not form an authorised set.
In this way, checking whether a set satisfies a policy is equivalent to checking whether a set of shares can reconstruct the secret. 
There are several generic approaches to map access structures to \ac{LSSS}, in this paper we will use the \ac{MSP} approach introduced by \citet{bei96}. 

\subsubsection{Alternative Password Mapping}
% In this paper, we consider a password to be in the basic format of a finite length string of printable characters (\ac{ASCII} , \ac{UTF-8}, etc.). 
% We do not consider other forms of passwords such as graphical passwords \cite{suo05}. 
% It is a common practice to partition the password alphabet into character classes, e.g., upper case, lower case, symbols and digits. 
% These character classes can be seen as disjoint subsets of the alphabet. 
% % Another common feature of password policies is a minimum number of characters a password must provide.
% For example, every valid password must have at least one character from each class and eight characters overall.
% %Let $P=(\min,\{t_i, A_i\})$ thus denote a password policy with minimum password length \min and a set of threshold, character set pairs.

An alternative approach of mapping passwords to cryptographically usable elements is given in the set theoretical setting.
To this end we first define a mapping from passwords (character strings) to a set, which can be used in cryptographic set operations.
First we observe that passwords are arbitrary strings and as such can have repeated characters. 
So the collection of characters in a password forms a multiset, not a set. 
The problem is that some direct conversion from a character multiset (password) to a set results in lost characters.
For example, if client chooses ``pa\$\$w0rd'', the resulting set would be $\{p,a,\$,w,0,r,d\}$.
This can be solved by pre-processing the characters in the dictionary and passwords.

The dictionary pre-processing step converts the dictionary into a set that can later be used in the password registration protocol. 
This step is done by the server as follows. 
Let $\Sigma=\sigma_1\cup\dots\cup \sigma_m$ be the alphabet where $\sigma_i$ is a character class (digits, lower case, etc.). 
The server transforms it into $\Sigma'$ based on password policy $f$. 
For each $\sigma_i$, there is a threshold $t_i$ in policy $f$ that says at least $t_i$ characters from $\sigma_i$ need to appear in the password. 
If $t_i = 0$, then the serve just skips all characters in this class $\sigma_i$ because they do not have to be checked. 
Otherwise, the server creates an empty set $\sigma'_i$, appends an index (from $1$ to $t_i$) to each character in $\sigma_i$, and puts the $t_i$ copies of indexed characters into $\sigma'_i$. 
For example, if $\sigma_i=l$ contains lower case characters and $t_i=2$, then $\sigma'_i=\{a1,a2,b1,b2,\ldots,z1,z2\}$. 
The union of $\sigma_i$ builds the set $\Sigma'$ to be used later 
This step has to be done only once as long as policy $f$ does not change.

The password pre-processing is performed as follows. 
We define a function $\psi$ for the client to convert a password (character string) into a set. 
For the $i$th character in the password, the function creates a substring of the password including the first $i$ characters in the password. 
Then the function appends to the $i$th character the number of occurrence of itself in the substring and puts the result into a set. 
For example, ``pa\$\$w0rd'' will be converted into $\{p1,a1,\$1,\$2,w1,01,r1,d1\}$. 
 


\subsection{Password Distributions and Min-Entropy} \label{sec:min-entropy}
Intuitively, a password hashing scheme should be considered secure (and a BPR protocol resistant to dictionary attacks) if an attacker can not retrieve the password from its hash more efficiently than by performing a brute-force attack over the dictionary. Therefore, security definitions for password hashing (cf. Section \ref{app:hash-security}) and dictionary attack resistance rely on the notion of min-entropy $\beta$.

The dictionary $\cD$, from which the passwords are chosen, has min-entropy $\beta$ such that efficient sampling of the dictionary allows retrieving the password from its hash with probability in $\beta$.
Although passwords in cryptographic research are often assumed to be uniformly at random distributed low-entropy secrets, we consider a somewhat more realistic password model.
In particular, we use passwords as character strings where the distribution of characters depends on the used character sets $\omega$, character positions and the password string itself.
We thus use a definition of password min-entropy commonly used in password security research \cite{ShayKKLMBCC10,KomanduriSKMBCCE11,MazurekKVBCCKSU13}, which captures the difficulty of brute-force attacks on passwords chosen from certain dictionaries.
As discussed in \cite{ShayKKLMBCC10,KomanduriSKMBCCE11}, this definition can capture many realistic password creation models.
Let $\DD_\omega$ denote the probability distribution in password \pwd of characters from a character set $\omega\in\{\Sigma,d,u,l,s\}$.
Min-entropy for $\pwd=(c_0,\dots,c_{n-1})$ is then defined according to \citet{shannon48} as
\[
  \beta_{\cD_{f,l}} = -\max_{\pwd\in \cD_{f,l}} \sum_{i=0}^{n-1} [\DD_\Sigma(c_i)\lg(\DD_\Sigma(c_i))].
\]
Note that definitions for min-entropy of $\cD$ and $\cD_f$ are equivalent to the definition for $\cD_{f,l}$.
While this may be surprising at first glance, one has to consider that while the policy restricts the character space, it does not restrict the positions where these characters appear, \ie an adversary cannot exclude any characters at any position.\footnote{Note that we exclude the special case where every character in a password is significant and the policy expression $R$ does not use all four available character sets. Changes for these can be easily incorporated.}

% We define the probability distribution $\DD_\omega$ of a character set $\omega\in\{\Sigma,d,u,l,s\}$ at position $i$ in password $\pwd=(c_0,\dots,c_{n-1})$ as $\DD_\omega=\Pr[c_i | i \wedge \pwd \wedge c_i\in\omega]$.
% Definition of the condition $i \wedge \pwd \wedge c_i\in\omega$ in the probability is left open for future work.
% This allows us to define min-entropy of a dictionary $\cD$, or in our case of $\cD_{f,l}$.
% Let $\DD_{\cD_{f,l}}=\Pr[\pwd]$ denote the password distribution in a dictionary $\cD_{f,l}$.
% Min-entropy $\beta_{\cD_{f,l}}$ of the dictionary is then given by $\beta_{\cD_{f,l}}=\min_{\pwd\in\cD_{f,l}} \log_2 (1/\DD_{\cD_{f,l}})$.
% In particular
% %, $\min_{\pwd\in\cD_f} \log_2 (1/\Pr_\DD[\pwd])$ and $\min_{\pwd\in\cD_{f,l}} \log_2 (1/\Pr_\DD[\pwd])$ respectively.
% % Min-entropy for $\cD_{f,l}$ is therfore given by
% \[
% \beta_{\cD_{f,l}} = \min_{\pwd\in \cD_{f,l}} \log_2 \left(\left( \prod_{i=0}^{n-1} \DD_{\Sigma}\right)^{-1}\right). %\Pr_{\DD_{\Sigma}}[c_i | i \wedge \pwd \wedge c_i\in\omega]
% \]
% % where $\DD_\Sigma$ is the distribution of \ac{ASCII}  characters and $\pwd=(c_0,\dots,c_{n-1})$.
% % The problem with this definition as $\beta_\cD$ is with less than $6.6$ very small.
% % This is due to the missing minimum length in $\cD$.
% % Therefore, to get a useful definition, we consider only $\cD_l$ where passwords have a minimum length of $l$ characters.
% In this case $\beta_{\cD_l}$ can has an upper bound of $\log_2(94^l)$ for uniformly at random distributed characters.
% % We further have dictionaries $\cD_f$ and $\cD_{f,l}$ with $f=(R,\pmin)$ and $l\geq\pmin$.
% % The regular expression $R$ defines which character sets from $\Sigma$ have to be present in every $\pwd\in\cD_f$, while $n$ gives a minimum size.
% % While these dictionaries have a different definition, their min-entropy is equivalent to $\cD$ (with according length restrictions).
% %, \ie the min-entropy is given by $\beta_{\cD_{f,l}} \leq \log_2(94^{l-m} \cdot \prod_{i=1}^{m} |R_i|)$.

\begin{remark}
Investigation of how character properties influence the distribution \DD of characters and therefore passwords and what are the exact properties, e.g., password length, neighbouring characters, character distribution in a language etc., is out of scope of this work.
Further, although min-entropy seems a reasonable measure for modelling dictionary attack resistance of BPR protocols,
we stress that min-entropy alone might not be sufficient to estimate the real password strength. 
As shown in \cite{MazurekKVBCCKSU13}, password strength is a complex matter (due to a variety of real-world attacks) and can hardly be captured through a single metric like min-entropy.
%We therefore stress that the definition of min-entropy is not necessarily equivalent to password strength.
%But this is not the focus of this work and may be worth exploring in future in order to refine security guarantees.
% However, note that our model can be easily extended to capture other, more complex metrics.
% While this work is concerned with the question how to efficiently perform blind password registrations with policy checks, we have to consider a certain type of password entropy.
\end{remark}


%****************** Randomised Password Hashing ****************************

\section{Randomised Password Hashing} \label{sec:strucphash}
A \emph{password hashing} scheme $\Pi$ that is used to compute password verification information for later use in VPAKE protocols from \cite{BenhamoudaP13} is defined as follows:
\begin{itemize}
	\item $\PSetup(\secpar)$ generates password hashing parameters \paramP. These parameters contain implicit descriptions of random salt spaces $\mathbb{S}_P$ and $\mathbb{S}_H$.
	\item $\PPHSalt(\paramP)$ generates a random pre-hash salt $s_P\rin\mathbb{S}_P$.
	\item $\PPreHash(\paramP, \pwd, s_P)$ outputs the pre-hash value $P$.
	\item $\PHSalt(\paramP)$ generates a random hash salt $s_H\rin\mathbb{S}_H$.
	\item $\PHash(\paramP, P, s_P, s_H)$ outputs the hash value $H$.
\end{itemize}
In the above syntax the algorithm $\PPreHash$ is \emph{randomised} with a pre-hash salt $s_P$, which extends the syntax from \cite{BenhamoudaP13}, where $\PPreHash$ is deterministic (and realised in constructions as a random oracle output $\cH(\pwd)$). In contrast we are interested in algebraic constructions of both $\PPreHash$ and $\PHash$ to allow for efficient proofs of knowledge involving pre-hash values $P$. The randomisation of $\PPreHash$ further increases the complexity of an offline dictionary attack that recovers $\pwd$ from $P$ since it removes the ability of an attacker to pre-compute pairs $(P,\pwd)$ and use them directly to recover $\pwd$ (see also Section~\ref{sec:pwreg}). We write $H\gets \HashP(\pwd,r)$ to denote $H\gets \PHash(\paramP,P,s_P$, $s_H)$ with $P\gets\PPreHash(\paramP,\pwd,s_P)$, where $r=(s_P,s_H)$ combines the randomness used in \PHash and \PPreHash. A secure $\Pi$ must satisfy the following security properties. Note that password-hiding is a new property that is used in ZKPPC to ensure that password hashes $H$ do not leak any information about $\pwd$. The remaining four properties are from \cite{BenhamoudaP13}, updated where necessary to account for the randomised $\PPreHash$:
\begin{itemize}
\item Password hiding: For all PPT algorithms $A=(A_1,A_2)$ where $A_1$ on input $\paramP\gets\PSetup(\secpar)$ outputs two equal-length password strings $\pwd_0$ and $\pwd_1$  and where $A_2$ on input $H\gets\PHash(\paramP,P,s_P,s_H)$, where $s_H\gets\PHSalt(\paramP)$, $s_P\gets\PPHSalt(\paramP)$, and $P\gets\PPreHash(\paramP,\pwd_b, s_P)$ for a random bit $b\rin\bits$ outputs bit $b'$ there exists a negligible function $\varepsilon(\cdot)$ such that
$|\Pr[b'=b]-\frac12|\leq\varepsilon(\secpar)$.
%For all PPT algorithms $A$ there exists a negligible function $\varepsilon(\cdot)$ such that
%\[\Pr[(i,\pwd)\gets A^{\HashPO(\cdot)}(\paramP);~ \Finalise(i,\pwd) = 1]\leq \varepsilon(\secpar),\]
%where $\paramP\gets\PSetup(\secpar)$ and each $i$th invocation of $\HashPO(\cdot)$ returns $H\gets\PHash(\paramP,P,s_P,s_H)$, where $s_H\gets\PHSalt(\paramP)$, $s_P\gets\PPHSalt(\paramP)$, $P\gets\PPreHash(\paramP,\pwd, s_P)$, $\pwd\rin\cD$, and stores $T[i]\gets\pwd$. $\Finalise(i,\pwd) = 1$ if $T[i]=\pwd$.

\item Pre-image resistance (called tight one-wayness in \cite{BenhamoudaP13}): For all PPT algorithms $A$ running in time at most $t$, there exists a negligible function $\varepsilon(\cdot)$ such that
\[\Pr[(i,P)\gets A^{\HashPO(\cdot)}(\paramP);~ \Finalise(i,P) = 1]\leq \frac{\alpha t}{2^\beta t_\PPreHash}+\varepsilon(\secpar),\]
for small $\alpha$ and $t_\PPreHash$ being the running time of \PPreHash, where $\paramP\gets\PSetup(\secpar)$ and each $i$th invocation of $\HashPO(\cdot)$ returns $(H,s_H)$ with $H\gets\PHash(\paramP,P,s_P,s_H)$ and stores $T[i]\gets\PPreHash(\paramP,\pwd, s_P)$, where $s_H\gets\PHSalt(\paramP)$, $s_P\gets\PPHSalt(\paramP)$, and $\pwd\rin\cD$.
$\Finalise(i,P) = 1$ if $T[i]=P$. (Note that $\HashPO(\cdot)$ does not return $s_P$.)

\item Second pre-image resistance: For all PPT algorithms $A$ there exists a negligible function $\varepsilon(\cdot)$ such that for $P'\gets A(\paramP, P, s_H)$
\[\Pr[P'\not=P \wedge \PHash(\paramP,P,s_H)=\PHash(\paramP,P',s_H)] \leq \varepsilon(\secpar),\]
with $\paramP\gets\PSetup(\secpar), s_P\gets\PPHSalt(\paramP), s_H\gets\PHSalt(\paramP)$ and $P\gets\PPreHash$ $(\paramP, \pwd, s_P)$ for any $\pwd\in\cD$.

\item Pre-hash entropy preservation: For all polynomial time samplable dictionaries $\cD$ with min-entropy $\beta$, and any PPT algorithm $A$, there exists a negligible function $\varepsilon(\secpar)$ such that for $(P,s_P)\gets A(\paramP)$ with $\paramP\gets\PSetup(\secpar)$ and random password $\pwd\rin\cD$:
\[\Pr[s_P\in\SSS_P \wedge P=\PPreHash(\paramP,\pwd,s_P)] \leq 2^{-\beta} + \varepsilon(\secpar).\]

\item Entropy preservation: For all polynomial time samplable dictionaries $\cD$ with min-entropy $\beta$, and any PPT algorithm $A$, there exists a negligible function $\varepsilon(\secpar)$ such that for $(H,s_P,s_H)\gets A(\paramP)$
\[\Pr[s_P\in\SSS_P \wedge s_H\in\SSS_H \wedge H=\HashP(\paramP,\pwd,s_P,s_H)] \leq 2^{-\beta} + \varepsilon(\secpar),\]
where $\paramP\gets\PSetup(\secpar)$ and $\pwd\rin\cD$.
\end{itemize}

%\noindent
%The use of a pre-hash salt is motivated by the fact that algebraic hash functions usually allow to pre-compute certain values.
%Using these pre-computed values and the salt it is easy for a server to retrieve the password.
%Using separate salts for pre-hash and hash functions allows us to prevent the server from precalculating password hashes.

%While random oracle-like password hashing is not usually vulnerable to precalculations, it seems difficult to build precalculation-secure, pure algebraic password hashing schemes without additional randomness.

%The main reason for the authors of \cite{BenhamoudaP13} not to use randomness in \PPreHash seems the increased round complexity in VPAKE protocols using these password hashes.
%However, it is possible to build one-round VPAKE protocols from their framework using randomness in \PPreHash, as we will see later.


%Further note that we assume implicit opportunistic encoding, \ie if the input \pwd to the hashing is a character string it applies $\Encode$ first.
%We consider (second) pre-image resistance and entropy preservation from \cite{BenhamoudaP13} as security properties for password hashing.
%Due to space limitations we refer to Appendix \ref{app:pwdhashing} for formal definition of those properties.
%Note that we change pre-image resistance by considering time to compute \PPreHash instead of \PHash.
%This allows us to prove our algebraic password hashing secure due to the additional pre-hash randomness.\footnote{A stronger notion where both \PPreHash and \PHash times are considered is possible, but difficult to reach in a purely algebraic setting.}
%While those definitions are password hashing related, we further consider common one-wayness security of \HashP to ensure security against an attacker that does not know the according salts.
%Note that one-wayness from \cite{BenhamoudaP13} is called pre-image resistance here to separate it from the common one-wayness.
%
%\myparagraph{One-wayness}
%We use the common definition for one-wayness with syntax appropriate for the used definition of password hashing.
%A password hashing protocol $\Pi$ is a one-way function if for all PPT algorithms $A$, there exists a negligible function $\varepsilon(\cdot)$ such that for $\pwd'\gets A^{\HashPO(\cdot)}(\paramP)$:
%$\Pr[ \Finalise(i,\pwd') = 1]\leq \varepsilon(\secpar)$,
%where $\HashPO$ returns $H\gets\HashP(\paramP,\pwd,r)$ for $s_H\gets\PHSalt(\paramP)$, $s_P\gets\PPHSalt(\paramP)$, global parameters $\paramP\gets\PSetup(\secpar)$ and random password $\pwd\rin\cD$ on the $i$-th invocation, and stores $T[i]\gets\pwd$.
%\Finalise returns $1$ if $T[i]=\pwd'$, otherwise $0$.

\subsection{Randomised Password Hashing from Pedersen Commitments}\label{sec:pwhashped}
We introduce a randomised password hashing scheme $\Pi=(\PSetup,\PPHSalt,\allowbreak\PPreHash,\allowbreak\PHSalt$, $\PHash)$ for \ac{ASCII}-based passwords using Pedersen commitments. We assume that $\pi\gets\pwdint(\pwd)$ for an appropriate choice of the shift base $b$ (see Remark~\ref{rem:basischoice}) and construct $\Pi$ as follows:
\begin{itemize}
	\item $\PSetup(\secpar)$ generates $\paramP= (p,g,h,\secpar)$ where $g$, $h$ are independent generators of a cyclic group $G$ of prime order $p$ of length $\secpar$.
	\item $\PPHSalt(\paramP)$ generates a pre-hash salt $s_P\rin\Zrp$.
	\item $\PPreHash(\paramP, \pi, s_P)$ outputs the pre-hash value $P=g^{s_P\pi}$.
	\item $\PHSalt(\paramP)$ generates a hash salt $s_H\rin\Zrp$.
	\item $\PHash(\paramP, P, s_P, s_H)$ outputs hash value $H=(H_1, H_2)=(g^{s_P}, Ph^{s_H})$.
\end{itemize}
Observe that $H_2=H_1^\pi h^{s_H}$, \ie $H_1$ can be seen as a fresh generator that is used to compute the Pedersen commitment $H_2$.
%Let $\paramP=(p,g,h,\secpar)\gets\PSetup(\secpar)$,
%$\ZZ_p\ni_R s_P\gets\PPHSalt(\paramP)$,
%$\ZZ_p\ni_R s_H\gets\PPHSalt(\paramP)$,
%$P=g^{s_P\cdot\pi}\gets\PPreHash(\paramP,\pwd,s_P)$ and
%$H=(H_1,H_2)=(g^{s_P}, Ph^{s_H})\gets\PHash(\paramP, P, s_P, s_H)$.
%Note that this allows us to compute $H_2$ from $(\pwd,H_1,s_H)$ as $H_2=H_1^\pi h^{s_H}$. %=\HashP(\paramP, \pwd, H_1, s_H)
The security properties of our password hashing scheme $\Pi$ follow from the properties of the underlying cyclic group $G$ and from the security of Pedersen commitments. We argue informally:
\begin{itemize}
\item
The \emph{password hiding} property of the scheme, assuming that $\pwd_0$ and $\pwd_1$ are mapped to corresponding integers $\pi_0$ and $\pi_1$ in $\ZZ_{b^n}$, is perfect and holds based on the perfect hiding property of the Pedersen commitment scheme. Note that the adversary receives the corresponding hash value $H=(H_1, H_2)=(g^{s_P}, Ph^{s_H})$, where $H_2= g^{s_P\pi}h^{s_H}$ is a Pedersen commitment on $\pi$ with respect to two independent bases $g^{s_P}$ and $h$. The ability of $A$ to distinguish between $\pi_0$ and $\pi_1$ can thus be turned into the attack on the hiding property of the commitment scheme.

\item
The \emph{pre-image resistance} holds since $s_P$ and $s_H$ are randomly chosen on every invocation of $\HashPO(\cdot)$ with a negligible probability for a collision and $H_2$ is a perfectly hiding commitment with bases $g^{s_P}$ and $h$. Therefore, for any given output $(H=(H_1, H_2), s_H)$ of $\HashPO(\cdot)$, $A$ must perform $2^\beta$ exponentiations $H_1^{\pi^\ast}$, one for each candidate $\pi^\ast$, in order to find $P = H_2h^{-s_H}$. This roughly corresponds to $2^\beta$ invocations of $\PPreHash$.

\item
The \emph{second pre-image resistance} holds since $H_1$ is uniform in $G$ and $H_2$ is a computationally binding commitment with bases $g^{s_P}$ and $h$. Note that for any $P'$ generated by $A$, $H_1^\pi h^{s_H}$ $=P'h^{s_H}$ is true only if $P'=H_1^\pi$.
	
	\item
The \emph{pre-hash entropy} and \emph{hash entropy} preservation hold since $H_1$ is a generator of $G$ such that for every $(P,s_P)$ chosen by the pre-hash entropy adversary, $\Pr[P=H_1^\pi]\leq2^{-\beta}+\varepsilon(\secpar)$, and for every $(H,s_H)$ chosen by the hash entropy adversary, $\Pr[H_2=H_1^\pi h^{s_H}]\leq2^{-\beta}+\varepsilon(\secpar)$ for a random $\pwd\rin\cD$.

%	\item Entropy preservation holds since $\Pr[H=s_P^\pwd h^{s_H}]\leq2^{-\beta}+\varepsilon(\secpar)$ for any randomly chosen password \pwd and the attacker's has to output $(H,s_H)$.
\end{itemize}

\subsection{Choosing the Shift Base}\label{rem:basischoice}
As Benhamouda and Pointcheval point out in \cite{BenhamoudaP13}, the encoded password (integer) $\pi$ in $P=H_1^\pi$ can be computed from $(H, s_H)$ in time $O(\sqrt{n})$ for $\pi \in [0,n-1]$, e.g. using variants of the Pollard's kangaroo algorithm, and if $n$ corresponds to the dictionary size $|\cD|$ then the above password hashing scheme is no longer pre-image resistant. We observe, however, that
%be defined in the proceedings version of this work \cite{KieferM14b} using shift base $b=95$ is not pre-image resistant since discrete logarithms $x\in [1, n]$ can be computed in time $O(\sqrt{n})$, e.g. using algorithms like Pollard's kangaroo, baby-step-giant-step, etc. If $b=95$ then an attacker can recover$ 
the size of $n$ is \emph{larger} than the dictionary size $|\cD|$. To see this, first note that 
\[n=\sum_{i=0}^{|\pwd|-1}b^i\cdot 93,\] 
which denotes the largest possible value of $\pi \gets \pwdint(\pwd)$ obtained from a password string of length $|\pwd|$ (normalised to $0$ being the smallest possible $\pi$). 
In order to see the difference to the dictionary size $|\cD|$ we analyse how the password length $|\pwd|$ impacts the interval of the corresponding integer $\pi$: for example, if $|\pwd| = 1$ then $\pi$ is in $[1,94]$; if $|\pwd|=2$ then $\pi$ is in $[b+1,b+94]\cup \dots \cup[94b+1,94b+94]$; if $|\pwd|=3$ then $\pi$ is in  $[1+b+b^2,94+b+b^2]\cup\dots\cup[1+94b+94b^2,94+94b+94b^2]$, and so on. 
%In particular, a dictionary $\cD$ consists of encoded passwords from ranges $[1,94]$ for password length $1$, $[b+1,b+94],...,[94b+1,94b+94]$ for password length $2$, $[1+b+b^2,94+b+b^2],...,[1+94b+94b^2,94+94b+94b^2]$ for password length $3$ and so on.
That is in contrast to the size of the dictionary $\cD$ (assuming for now that policy $f$ does not restrict the character choice), which corresponds to the number of $|\pwd|$-tuples containing characters chosen out of all $94$ printable \ac{ASCII} characters and is therefore given by 
\[|\cD| = 94^{|\pwd|}\]
and therefore independent of the shift base $b$.
In order to guarantee the pre-image resistance of the password hashing scheme we need to ensure that $|\cD| \approx \sqrt{n}$.
%, which makes $O(\sqrt{n})$ not more efficient than a brute-force attack on dictionary $\cD$. 
This can be achieved by choosing the shift base $b$ such that computing $\pi$ from $P=H_1^\pi$ becomes as difficult as a brute-force search over the dictionary $\cD$. 
In particular, the shift base $b$ must be chosen such that 
\begin{equation}\label{eq:hashlimit}
  |\mathcal{D}|^2\leq \sum_{i=0}^{|\pwd|-1}b^i\cdot 93.
\end{equation}
Since the required value of $b$ depends on $|\pwd|$ as well as policy $f$ it is possible for a specific password policy $f=(R,\pmin,\pmax)$ to compute the optimal value for $b$ using the password length restrictions $\pmin,\pmax$ and regular expression $R$, and by this to optimise the performance of the password hashing scheme (and of our ZKPPC protocol) with respect to the given policy. 
Using dictionary size 
\[|\cD_f|=94^{|\pwd|-|R|}\prod_{i=0}^{|R|-1}|R_i|,\]
for policy $f$ allows to compute an optimal $b$ such that Eq. \ref{eq:hashlimit} holds for all policy compliant passwords.
If such optimisation is not required then we recommend setting $b=10^5$, which should be a safe choice for all sensible policies and password lengths (in contrast to $b=95$ that was used in~\cite{KieferM14b}).
% A shift base of $b=10^5$ ensures in particular that $|\mathcal{D}|^2\leq n$ for all passwords of at least $5$ characters and allowing for some leeway defining regular expression $R$.
Figure \ref{fig:choiceb} depicts relations among password length and optimal shift base $b$ for the general case (without regular expression) and with regular expression $R=\mathtt{duls}$.
First it shows that $b=10^5$ is a safe choice.
Further, it shows that the regular expression has a significant influence on the choice of an optimal base $b$, \ie can be used to tweak performance of a ZKPPC protocol.
In the general case we see that base $b$ grows exponentially the shorter the password gets, which is responsible for the relatively large suggested default base $b=10^5$.
With regular expression $\mathtt{duls}$ in contrast $b$ has to \emph{grow} with increasing password length.
This is due to the restriction on the character sets from the regular expression.


\begin{figure}[tbh]
\centering
\begin{tikzpicture}[x=1cm,y=1cm]

 \draw[latex-latex, thin, draw=gray, ->] (0,0)--(6,0) node [right] {$b$}; 
 \draw[latex-latex, thin, draw=gray, ->] (0,0)--(0,8) node [above] {$|\pwd|$}; 
 
   \foreach \y in  {1,2,3,4,5,6,7}
  \draw[shift={(0,\y)},color=black] (3pt,0pt) -- (-3pt,0pt);
  \foreach \y in {1,2,3,4,5,6,7} 
  \draw[shift={(0,\y)},color=black] (0pt,0pt) -- (-3pt,0pt) node[left] 
  {$\the\numexpr\y+3$};

  
 \draw[latex-latex, thin, draw=gray, ->] (7,0)--(7,8) node [above] {$|\pwd|$}; 
 \draw[latex-latex, thin, draw=gray, ->] (7,0)--(13,0) node [right] {$b$};
 
   \foreach \y in  {1,2,3,4,5,6,7}
  \draw[shift={(7,\y)},color=black] (3pt,0pt) -- (-3pt,0pt);
  \foreach \y in {1,2,3,4,5,6,7} 
  \draw[shift={(7,\y)},color=black] (0pt,0pt) -- (-3pt,0pt) node[left] 
  {$\the\numexpr\y+3$};

  
  \foreach \x in  {1,2,3,4,5}
  \draw[shift={(\x,0)},color=black] (0pt,3pt) -- (0pt,-3pt);
  \foreach \x in {1,3,5} 
  \draw[shift={(\x,0)},color=black] (0pt,0pt) -- (0pt,-3pt) node[below] 
  {\small $\the\numexpr\x*1000$}; 
  \foreach \x in {2,4} 
  \draw[shift={(\x,0)},color=black] (0pt,0pt) -- (0pt,-10pt) node[below] 
  {\small $\the\numexpr\x*1000$};
  
  \foreach \x in  {1,2,3,4,5}
  \draw[shift={(\the\numexpr\x+7,0)},color=black] (0pt,3pt) -- (0pt,-3pt);
  \foreach \x in {1,3,5} 
  \draw[shift={(\the\numexpr\x+7,0)},color=black] (0pt,0pt) -- (0pt,-3pt) node[below] 
  {\small $\the\numexpr\x*10000$}; 
  \foreach \x in {2,4} 
  \draw[shift={(\the\numexpr\x+7,0)},color=black] (0pt,0pt) -- (0pt,-10pt) node[below] 
  {\small $\the\numexpr\x*10000$};
  
  
% no R
% l = 3 -> b >= 86128
% l = 4 -> b >= 40320
% l = 5 -> b >= 27587
% l = 6 -> b >= 21969
% l = 7 -> b >= 18875
% l = 8 -> b >= 16936
% l = 9 -> b >= 15613
% l = 10 -> b >= 14656
\foreach \Point in {(\the\numexpr4.0320+7, 1), (\the\numexpr2.7587+7, 2), (\the\numexpr2.1969+7, 3), (\the\numexpr1.8875+7, 4), (\the\numexpr1.6936+7, 5), (\the\numexpr1.5613+7, 6), (\the\numexpr1.4656+7, 7)}{
    \node at \Point {\textbullet};
}


% R=duls
% l = 4 -> b >= 796
% l = 5 -> b >= 1452
% l = 6 -> b >= 2084
% l = 7 -> b >= 2651
% l = 8 -> b >= 3149
% l = 9 -> b >= 3582
% l = 10 -> b >= 3960
% l = 20 -> b >= 6042
% l = 40 -> b >= 7343
% l = 60 -> b >= 7818
% l = 80 -> b >= 8064
% l = 100 -> b >= 8215
\foreach \Point in {(0.796, 1), (1.452, 2), (2.084, 3), (2.651, 4), (3.149, 5), (3.582, 6), (3.960, 7)}{
    \node at \Point {$\circ$};
}

% to ensure that the points are being properly centered:
\draw [dotted, gray] (0,0) grid (6,8);
\draw [dotted, gray] (7,0) grid (13,8);
% \node [red] at (3,2.5) {\textbullet};
% \node [blue] at (3,-2.5) {$\circ$};

\end{tikzpicture}
\caption[Choosing a shift base]{Optimal Choice for $b$\\$\circ$ with regular expression $R=\mathtt{duls}$ \hspace*{5em} \textbullet without regular expression $R$ }\label{fig:choiceb}
\end{figure}

% Considering password policies exemplified in Section \ref{sec:policies} we can define the following optimised basis.
% \begin{itemize}
% 	\item $\cD_{f,6}$ with $f=(\mathtt{ds}, 6, 10)$ contains $24,983,966,720$ passwords, which leads to a basis $b\geq 14,424$.
% % 	      $\cD_{f,10}$ with $f=(\mathtt{ds}, 6, 10)$ contains $1,950,620,603,331,461,120$ passwords, which leads to a basis $b\geq 11601$.
% 	      
% 	\item $\cD_{f,8}$ with $f=(\mathtt{uss}, 8, 12)$ contains $195,394,606,923,776$ passwords, which leads to a basis $b\geq 12,110$.
% % 	      $\cD_{f,12}$ with $f=(\mathtt{uss}, 8, 12)$ contains $15,255,413,614,534,691,127,296$ passwords, which leads to a basis $b\geq 10,799$.
% 	
% 	\item $\cD_{f,8}$ with $f=(\mathtt{duls}, 8, 16)$  contains $16,889,161,502,720$ passwords, which leads to a basis $b\geq 6,016$.
% \end{itemize}
%To the best of our knowledge no variant of the Kangaroo algorithm exists that allows to take the smaller subsets of correct solutions into account since the computation of possible solutions is performed by splitting the exponent (password) and therefore destroying the ability to use the fact that valid solutions come from a smaller subset of valid integer values.
%Further note that using Pollard's kangaroo on every subset separately is not efficient as each subset only holds $94$ possible solutions.
