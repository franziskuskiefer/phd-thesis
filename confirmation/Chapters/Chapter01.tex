%************************************************
\chapter{Introduction}\label{ch:introduction}
%************************************************
\fk[inline]{Fix nomenclature (\eg algorithm, protocol, primitive ...)}

This work at hand is concerned with cryptography from low-entropy secrets, better known as \emph{passwords}.
In contrast to conventional cryptographic algorithms and protocols, secrets have low-entropy here, such that an adversary is able to iterate through all possible secrets in reasonable time.
This leads directly to the inherent threat of everything we are dealing with in this work: so-called dictionary attacks (cf. Section \ref{sec:introdictionaryattacks}).
So why do we want to perform cryptography with passwords at all, when they comprise intrinsic attack possibilities?

\begin{quote}
``Humans are incapable of securely storing high-quality cryptographic keys, and they have unacceptable speed and accuracy when performing cryptographic operations.''~\cite{Kaufmann02}
\end{quote}

\noindent
We do not care about humans lack of speed and accuracy in performing cryptographic operations in this work.
But their inability to remember high-quality cryptographic keys is a major problem.
Even though there exist several standardized authentication tokens for humans, we heavily rely on passwords.
In general we distinguish between three big categories of authentication data \cite{Burr11}:
\begin{itemize}
	\item Something you know (\eg password, \ac{PIN})
	\item Something you have (\eg mobile phone, cryptographic key)
	\item Something you are (\eg fingerprint)
\end{itemize}
But since everything someone has or is may be stolen or duplicated, knowledge (of a password) is a very important factor in authenticating a human (as long as he does not write down his passwords).
To perform cryptography where humans input a secret, those have to be human-memorable.
This leads to the necessity of password-based cryptography as everything humans are able to remember correctly is rather short and has not much entropy.

\section{Password Based Cryptography}
Password based cryptography can be classified (with some exceptions) as symmetric within the field of cryptographic research.
In contrast to other cryptographic mechanisms where \emph{keys} are chosen from a \emph{key space}, \emph{passwords} are chosen from a \emph{dictionary}.
Note that dictionaries in our case denote sets of characters rather than \emph{real} lexicon like dictionaries.
When talking about those we use the term \emph{lexicon}.

Even though password is the the umbrella term, we distinguish between several types of passwords, we discuss in the following.

\paragraph{Password}
The term \emph{password} usually refers to character strings chosen from a dictionary consisting of alpha-numeric characters \texttt{a-z}, \texttt{A-Z}, \texttt{0-9} and special characters like \texttt{\$,\%,@} and so on.
While many (online) services have password policies in place, humans tend to choose easy memorable and therefore mainly easy to guess passwords that can be found in a lexicon \cite{Florencio2007,Gaw2006}.
To encourage users to use stronger passwords, password-strength meters are used by many services.
A recent study \cite{Ur2012} on password-strength meters shows that this could have a significant impact on user's password-strength.

\paragraph{\accl{PIN}}
\acp{PIN} are rather short passwords, chosen from the numeric dictionary containing numbers \texttt{0-9}.
They are mainly used to secure access to the actual authentication token like credit card or other smart cards.
Since the card is a second authentication factor (\emph{something you have}) here, a short \ac{PIN} is sufficient to reach reasonable security.
The smaller dictionary containing only numbers is most presumable due to practical restrictions on the input device, used to enter the \ac{PIN}.

\paragraph{\accl{OTP}}
While most passwords are meant to be memorized by humans, \acp{OTP} are used only once and therefore don not have to be memorised.
They are mainly chosen from the numeric dictionary \texttt{0-9}.
\acp{OTP} have become more popular recently as a second factor, in addition to the ``regular'' password, in two-factor authentication, \eg Google \cite{Google2Factor}, Facebook \cite{FB2Factor}, Twitter \cite{Twitter2Factor} and GitHub \cite{Github2Factor}.
The most popular standards for \ac{OTP} that are implemented in the Google Authenticator App \cite{GAuthenticator} for example, are the \acs{HMAC} based algorithm specified in RFC 4226 \cite{rfc4226} and the time based algorithm specified in RFC 6238 \cite{rfc6238}.

\paragraph{\accl{TAN}}
\acp{TAN} are special \acp{OTP} mainly used to authorise single financial transactions in online banking applications.
These, mainly short passwords, are usually drawn from the numeric dictionary containing numbers \texttt{0-9}.
They can be seen as transaction bound \acp{OTP}.

\subsection{The inherent Threat of Dictionary Attacks}\label{sec:introdictionaryattacks}
As mentioned already, most cryptographic mechanisms use high-entropy secrets such that it is impossible for anyone (any algorithm) to traverse all possible secrets from the particular secret space in tolerable time.
Password based schemes in contrast assume low-entropy secrets that are drawn from a polynomial sized dictionary.
Therefore, it is feasible for a program to walk through the entire dictionary and just try every possible secret (password).
This kind of attack is called \emph{dictionary attack} and is inherent to every password based algorithm.
There exist several approaches to deal with this flaw in each specific field of password based research.
We will point out the particular mechanism when dealing with each field.

In addition to a brute force dictionary attack that iterates an entire dictionary, it is often sufficient to perform a lexicon attack that traverses only a relatively small dictionary containing for example words of the English language.

\paragraph{Poorly Chosen Passwords}
While dictionary attacks are unavoidable, poorly chosen passwords worsen the situation.
By choosing passwords that are available in lexicons or easy to derive from other parameters like the public e-mail address, the search space for brute-force attacks gets significantly smaller.
Password cracking programs like \emph{John the Ripper} \cite{JohnTheRipper} gather lexicons of different languages in password lists \cite{JohnTheRipperWordlist}.
Not only language lexicons can be used for brute-force attacks on passwords, but password lists of often used passwords that got somehow leaked.
Using for example a list with the most used passwords (\eg \cite{XatoPwds}), brute-force attacks on online accounts may be performed very efficiently.

\section{Outline}

\paragraph{Outline of this Work}
This work consists of ???\fk{How many chapters do we have?} chapters.
We gave a motivational overview on low-entropy cryptography and passwords already and deepen this discussion in the remainder of this introduction.
While the introduction stays informal in most points, the subsequent Chapter \ref{ch:background} on background of low-entropy cryptography complete the information on research in this area, necessary for the remainder of this work.
\fk[inline]{Finish Outline}

\paragraph{Outline of this Chapter}
We continue this introduction by giving a brief informal overview on the most important approaches in the research area of provable security.
In the subsequent sections we discuss different areas in the general field of password based cryptography.
We start with password based authentication and key exchange protocols, which comprise the biggest part of password based cryptography research.
The second part is concerned with general cryptographic primitives from low entropy secrets and other password based protocols.
We finally conclude this chapter with an overview of mechanisms to simplify password related tasks or avoid problems related with password handling.

% *****************************************************************************
% Section: On Provable Security
% *****************************************************************************

\section{On Provable Security}
In this section we briefly discuss general security models used in the field of provable security.
We discuss more specialised methods used in password based cryptography later.
To prove a cryptographic primitive secure, reductionist arguments are used.
Thereby, the attempt is made to reduce the security of the primitive under consideration to a problem that is believed to be hard\footnote{Note that this work is not concerned with the question whether specific problems are actual hard. We consider only well investigated problems, which the cryptography community beliefs are hard.}.
Thus, given a (tight) reduction from the primitive to an underlying problem, one could be confident that the primitive is secure as long as the underlying problem is actually hard to solve.
For an insightful and skeptical discussion on this kind of security verification we refer to the ``Another Look'' series by \citeauthor{AnotherLook} \cite{AnotherLook}.

Cryptographic primitives can only be proven secure in the light of the abilities of an adversary.
Therefore, before being able to proof anything about a cryptographic primitive it has to be modelled considering an appropriate (hostile) environment.
It always has to be kept in mind that the resulting proof is only valid within the used model.
A well-known model is the so-called \emph{Dolev-Yao} model.

\paragraph{The World of Dolev-Yao}
A formal view on adversarial behaviour and how to model security protocols goes back to works from \citeauthor{Needham78} \citeyear{Needham78} \cite{Needham78} and \citeauthor{Dolev83} \citeyear{Dolev83} \cite{Dolev83}, the Dolev-Yao model. It models cryptographic protocols by assuming black-box cryptographic primitives and an adversary that is able to control the entire communication channel between the participating parties.
Since all primitives are seen in a black-box way, this model is not sufficient to analyse the actually used cryptographic primitives.

\paragraph{The Computational World}
In this work at hand however we are concerned with a more realistic approach of modelling, the computational modelling.
The foundation of today's computational models are laid in the 1980's\fk{more citations} \cite{Goldwasser82}.
The computational world consists of bit-string messages and cryptographic primitives performing computations on them.
In this world the attacker is mostly modelled as a \ac{PPT} Turing machine \cite{Turing37}.
We give necessary formal definitions in Chapter \ref{ch:background}.

There are basically two different ways of modelling and proving in this computational world, we describe in the following: game-based and simulation-based.

\subsection{Game-Based Security}
Game-based security models allow the (\ac{PPT}) adversaries to play a game against a challenger.\fk{Needs citation!!!}
During this game (or experiment) he is allowed to query a set of \emph{oracles} to simulate and interact with the primitive or protocol under consideration.
Eventually, the adversary outputs his answer to the challenger who decides whether the adversary wins the game or not.
Oracles are black-box functions, the adversary can query.
On (possibly empty) input by the adversary they return the result of their computation.
\fk[inline]{Rewrite above!}

The actual security definition and strength of the adversary depends on the oracles available to the adversary.
There are different commonly used ways of modelling security.

\subsection{Simulation-Based Security}
The idea of simulation-based security proofs is due to \citeauthor{Goldreich87} \cite{Goldreich87} \citeyear{Goldreich87}.
In contrast to the previously described game-based approach simulation-based security does not define a challenger and oracles the adversary interacts with, but an ideal functionality the actual protocol should mimic in the real world.
The adversaries goal is to distinguish between the execution in the ideal world with a perfectly secure algorithm or protocol, and the real world with the actual algorithm or protocol.
Security reasoning is then done as follows: since the ideal primitive does not leak any information other than publicly observable and its behaviour is indistinguishable from the real world, the real world primitive is also secure.

\paragraph{\accl{UC}}
The \ac{UC} framework proposed by \citeauthor{Canetti2001a} \cite{Canetti2001a} in \citeyear{Canetti2001a} is a popular general purpose simulation-based security model.
It overcomes an inherent shortcoming of most security models as it allows for secure arbitrary composition of with other primitives once a primitive is proven secure.
Most security models only allow to derive statements about the algorithm in a shielded environment.
As soon as the algorithm is used with other primitives, and thus sharing \ac{IO} channels and data, the security can not be guaranteed anymore.
The \ac{UC} framework introduces another adversary \UCZ, the environment, that generates all inputs for all parties, and reads their outputs.
A protocol is secure in \ac{UC} when it realizes a given ideal functionality \UCF, such that for any real-world adversary \A, interacting with the protocol, there exists an ideal-world adversary \UCS, such that no environment \UCZ can decide whether it interacts with \A and the real-world protocol, or with \UCS and the ideal functionality \UCF.

% *****************************************************************************
% Section: Password Based Authentication and Key Exchange
% *****************************************************************************

\section{Password Based Authentication and Key Exchange}
In this section we give a broad overview on password based authentication and key exchange protocols found in literature.
We start with the most popular protocol \emph{\ac{PAKE}} before investigating more specialized protocol classes like threshold \ac{PAKE} or group \ac{PAKE}.

\subsection{Password Based Authenticated Key Exchange}
The notion of \emph{\ac{PAKE}} was introduced by \citeauthor{bellovin92}~\cite{bellovin92} and corresponding security models were initially developed by \citeauthor{Bellare2000}~\cite{Bellare2000}, \citeauthor{Boyko2000}~\cite{Boyko2000}, and \citeauthor{Goldreich01}~\cite{Goldreich01}.
The first and maybe best known \ac{PAKE} protocols include SPEKE \cite{Jablon96} and EKE \cite{bellovin92,Bellare2000}.
Until now, numerous subsequent work explored the notion of \acl{PAKE} in depth.
\ac{PAKE} allows two parties, holding low-entropy keys, to negotiate a common session key.
Despite the key exchange functionality it authenticates the two parties explicitly or implicitly.
They aim to protect against offline dictionary attacks but require restriction on the number of failed password trials as all password based protocols.
One of the most promising applications of \ac{PAKE} protocols is the online authentication of users.
It is considered as a more secure alternative to the nowadays mainly deployed approach of transmitting the password over a secure channel (\ac{HTTPS}) and let the server perform a check against his stored credential.
The standard model of \ac{PAKE} does not require any \ac{PKI}, which is necessary for the secure channel, and assumes that only a low-entropy secret, \ie a human memorable password, is shared between both parties.
\footnote{Please note that this applies only for the authentication process \emph{not} for the registration process.
There is no mechanism to store the password on the server without sending it there in a secure way.
This is the key distribution problem, implicit to all symmetric key protocols.}
Thereby, \ac{PAKE} protocols solve the problem of potential password leakage, inherent to the approach based on secure channels.

In general, all PAKE models (see~\cite{Pointcheval2012} for a recent overview) take into account unavoidable online dictionary attacks and aim to guarantee security against offline dictionary attacks.
While many PAKE constructions require a constant number of communication rounds~\cite{Gennaro2003,Abdalla2005,Gennaro2008,Katz2009a,Katz2011}; recent frameworks by \citeauthor{Katz2011} \cite{Katz2011} and \citeauthor{Benhamouda2013} \cite{Benhamouda2013} offer optimal one-round PAKE.

In addition to the aforementioned approaches that are tailored to the password-based setting there exist several more general authentication and key exchange frameworks such as~\cite{Camenisch2010,Blazy2012} that also lend themselves to the constructions of (somewhat less practical) PAKE protocols.

A problem intrinsic to all \ac{PAKE} protocols is the issue of \emph{server compromise}.
Servers store passwords in databases to retrieve them when necessary.
Even though salted hashing of passwords is more or less standard, leakage of password databases is a big security problem.
Different approaches have been proposed to mitigate the impact of server compromises.
\citeauthor{Gentry2006} propose the first general technique \cite{Gentry2006}, proven in the \ac{UC} framwork, to make arbitrary \ac{PAKE} protocols secure resilient to server compromise.\fk{more cites}
\fk[inline]{Asymmetric and symmetric PAKE}
In the field of \acl{PAKE}, security models aim at \ac{AKE}-security \cite{Bellare1993,Bellare1995} that may be seen in one of the following general settings.

\paragraph{Game-Based \acc{PAKE}-Security}
The original game-based \ac{PAKE} models in~\cite{Bellare2000,Boyko2000} incorporate the \ac{FtG} approach where the semantic security of the session key is considered with respect to one particular session, referred to as a \emph{test session}, determined by the adversary through one call to a \Test oracle.
The adversary has furthermore access to oracles that allow him to eavesdrop on protocol executions, take actively part in executions and corrupt protocol participants.
\citeauthor{Abdalla2005} \cite{Abdalla2005} proposed a \ac{RoR} approach to model semantic security of PAKE protocols by allowing polynomially-many queries to a \Test oracle.
They showed not only that their \ac{RoR} approach leads to stronger security but were also able to simplify the model.
The models in~\cite{Bellare2000,Abdalla2005} remain the most popular game-based PAKE models, adopted in the analysis of many protocols, including the random oracle-based protocols~\cite{Abdalla2006,Abdalla2005b} and protocols requiring a \ac{CRS}~\cite{Gennaro2003,Gennaro2008,Katz2009a}.

%TODO move
% by removing the \Reveal oracle that provides the adversary with established session keys.
%The adversary in~\cite{Abdalla2005} is thus left with access to the \Test oracle, the \Execute oracle that models passive attacks, and the \Send oracle accounting for active attacks.

\paragraph{\accl{FtG}}
The term \acl{FtG} goes back to \citeauthor{Bellare97} \cite{Bellare97} whose definition of \ac{FtG}-security for symmetric encryption is based on \cite{Goldwasser84} \cite{Micali86}.
In our research focus of password-based cryptography one of the first formal models for \ac{PAKE}, proposed by \citeauthor{Bellare2000} in \cite{Bellare2000}, employs the \ac{FtG} approach.
The security requirement there is that an adversary must not be able to decide whether a given bit-string is the real key computed by honest parties performing the protocol, or a random element from the key space.
But the adversary has only one change to retrieve such a test key.

\paragraph{\accl{RoR}}
The term \acl{RoR} has been introduced by \citeauthor{Bellare97} in \cite{Bellare97} in a different context and a different meaning.
The notion of \ac{RoR} in the context of \acl{AKE} protocols has been introduced by \citeauthor{Abdalla2005} to strengthen and simplify the \ac{FtG} approach used in the original model from \cite{Bellare2000} towards the \ac{RoR} approach.
In the \ac{AKE} context \ac{RoR} allows the adversary to query \emph{multiple} keys before deciding whether all of them have been computed by honest parties performing the protocol, or all of tem  have been randomly chosen from the key space.


\paragraph{Simulation-Based \acc{PAKE}-Security}
Simultaneously with the first game-based models in \cite{Bellare2000,Boyko2000}, the first simulation-based \ac{PAKE} model has been proposed by \citeauthor{Goldreich01}~\cite{Goldreich01}.
Their work also comprises the first (and until now the only, but fairly inefficient) protocol that is built from general secure multi-party computation techniques but does not require any setup assumptions nor random oracles.\footnote{The work in \cite{Goldreich01} is concerned with the general possibility of such a protocol rather than building a practical one.}
The protocol has been subsequently simplified at the cost of weakened security in~\cite{NguyenV04}.
While the model from \cite{Goldreich01} is hardly used in the analysis of \ac{PAKE} protocols, a stronger simulation-based model in the framework of \acl{UC}~\cite{Canetti2001a} has been proposed by \citeauthor{Canetti2005}~\cite{Canetti2005} later.
In contrast to game-based \ac{PAKE} protocols, \ac{UC}-secure protocols require setup assumptions, with CRS being the most popular one~\cite{Katz2011}, albeit ideal ciphers / random oracles~\cite{Abdalla2008} and stronger hardware-based assumptions~\cite{cryptoeprint:2012:537} have also been used.

\fk[inline]{Check if I missed any PAKE paper}
\fk[inline]{Add newer protocols (2013)}

\subsection{Other \accl{PAKE} Protocols}
Besides the straight-forward case where a single client-sever pair uses a common password to authenticate each other and establish a secure channel, there exist several other scenarios we discuss in the following.

\paragraph{Three-Party \accl{PAKE}}
The three-party setting considers two humans who want to securely communicate with each other.
Since sharing passwords with everyone else is not practical, a trusted server (the third party) come into play.
Thus, the users have to share only one password with the trusted sever, which assists then in the three-party protocol between the two users.
The initial three-party setting in the password case is due to \citeauthor{Abdalla2005} \cite{Abdalla2005}.
\fk[inline]{What to do with \cite{Abdalla2007}?}
\fk[inline]{More papers?}

\paragraph{Group \accl{PAKE}}
General group protocols can be modified for the password case, \eg \cite{Bresson02,BrChPo05}.
\fk[inline]{read those and write sth....\\
\cite{Kim2004,Abdalla2006,Dutta2006}\\
Efficient, standard model \cite{Bohli2006}\\
group protocols~\cite{AbdallaP06,AbdallaBVS07}
}

\paragraph{Threshold Password Based Authenticated Key Exchange}
threshold-based PAKE protocols, e.g.~\cite{Abdalla2005b}, where the client's password is shared amongst two (or possibly more) servers that jointly authenticate the client

\paragraph{\accl{PPSS}}
\ac{PPSS} allows to share a secret, \eg a symmetric key, among several servers, protected by a password. \cite{Bagherzandi2011}\fk{there is one other paper}

\paragraph{Security against Malicious Servers}

\subsection{Multi-Factor Authenticated Key Exchange}

% *****************************************************************************
% Section: How to Ease the Pain
% *****************************************************************************

\section{How to Ease the Pain}
Mechanisms to avoid password based authentication problems
\begin{itemize}
	\item password manager
	\item SSO
	\item ...
\end{itemize}

% *****************************************************************************
% Section: Strong Cryptography from Low-Entropy Secrets
% *****************************************************************************

\section{Strong Cryptography from Low-Entropy Secrets}
PKE, independent from PAKE

\subsection{Password Protected Secret Sharing}

\subsection{Password Based Encryption}

%This bundle for \LaTeX\ has two goals:
%\begin{enumerate}
%    \item Provide students with an easy-to-use template for their
%    Master's
%    or PhD thesis. (Though it might also be used by other types of
%    authors
%    for reports, books, etc.)
%    \item Provide a classic, high-quality typographic style that is
%    inspired by \citeauthor{bringhurst:2002}'s ``\emph{The Elements of
%    Typographic Style}'' \citep{bringhurst:2002}.
%    \marginpar{\myTitle \myVersion}
%\end{enumerate}
%The bundle is configured to run with a \emph{full} 
%MiK\TeX\ or \TeX Live\footnote{See the file \texttt{LISTOFFILES} for
%needed packages. Furthermore, \texttt{classicthesis} 
%works with most other distributions and, thus, with most systems 
%\LaTeX\ is available for.} 
%installation right away and, therefore, it uses only freely available 
%fonts. (Minion fans can easily adjust the style to their needs.)
%
%People interested only in the nice style and not the whole bundle can
%now use the style stand-alone via the file \texttt{classicthesis.sty}.
%This works now also with ``plain'' \LaTeX.
%
%As of version 3.0, \texttt{classicthesis} can also be easily used with 
%\mLyX\footnote{\url{http://www.lyx.org}} thanks to Nicholas Mariette 
%and Ivo Pletikosi\'c. The \mLyX\ version of this manual will contain
%more information on the details.
%
%This should enable anyone with a basic knowledge of \LaTeXe\ or \mLyX\ to
%produce beautiful documents without too much effort. In the end, this
%is my overall goal: more beautiful documents, especially theses, as I
%am tired of seeing so many ugly ones.
%
%The whole template and the used style is released under the
%\textsmaller{GNU} General Public License. 
%
%If you like the style then I would appreciate a postcard:
%\begin{center}
% André Miede \\
% Detmolder Straße 32 \\
% 31737 Rinteln \\
% Germany
%\end{center}
%The postcards I received so far are available at:
%\begin{center}
% \url{http://postcards.miede.de}
%\end{center}
%\marginpar{A well-balanced line width improves the legibility of
%the text. That's what typography is all about, right?}
%So far, many theses, some books, and several other publications have 
%been typeset successfully with it. If you are interested in some
%typographic details behind it, enjoy Robert Bringhurst's wonderful book.
%% \citep{bringhurst:2002}.
%
%\paragraph{Important Note:} Some things of this style might look
%unusual at first glance, many people feel so in the beginning.
%However, all things are intentionally designed to be as they are,
%especially these:
%\begin{itemize}
%    \item No bold fonts are used. Italics or spaced small caps do the
%    job quite well.
%    \item The size of the text body is intentionally shaped like it
%    is. It supports both legibility and allows a reasonable amount of
%    information to be on a page. And, no: the lines are not too short.
%    \item The tables intentionally do not use vertical or double
%    rules. See the documentation for the \texttt{booktabs} package for
%    a nice discussion of this topic.\footnote{To be found online at \\
%    \url{http://www.ctan.org/tex-archive/macros/latex/contrib/booktabs/}.}
%    \item And last but not least, to provide the reader with a way
%    easier access to page numbers in the table of contents, the page
%    numbers are right behind the titles. Yes, they are \emph{not}
%    neatly aligned at the right side and they are \emph{not} connected
%    with dots that help the eye to bridge a distance that is not
%    necessary. If you are still not convinced: is your reader
%    interested in the page number or does she want to sum the numbers
%    up?
%\end{itemize}
%Therefore, please do not break the beauty of the style by changing
%these things unless you really know what you are doing! Please.
%
%
%\section{Organization}
%A very important factor for successful thesis writing is the
%organization of the material. This template suggests a structure as
%the following:
%\begin{itemize}
%    \marginpar{You can use these margins for summaries of the text
%    body\dots}
%    \item\texttt{Chapters/} is where all the ``real'' content goes in
%    separate files such as \texttt{Chapter01.tex} etc.
% %  \item\texttt{Examples/} is where you store all listings and other
% %  examples you want to use for your text.
%    \item\texttt{FrontBackMatter/} is where all the stuff goes that
%    surrounds the ``real'' content, such as the acknowledgments,
%    dedication, etc.
%    \item\texttt{gfx/} is where you put all the graphics you use in
%    the thesis. Maybe they should be organized into subfolders
%    depending on the chapter they are used in, if you have a lot of
%    graphics.
%    \item\texttt{Bibliography.bib}: the Bib\TeX\ database to organize
%    all the references you might want to cite.
%    \item\texttt{classicthesis.sty}: the style definition to get this
%    awesome look and feel. Does not only work with this thesis template
%    but also on its own (see folder \texttt{Examples}). Bonus: works
%    with both \LaTeX\ and \textsc{pdf}\LaTeX\dots and \mLyX.
%    \item\texttt{ClassicThesis.tcp} a \TeX nicCenter project file.
%    Great tool and it's free!
%    \item\texttt{ClassicThesis.tex}: the main file of your thesis
%    where all gets bundled together.
%    \item\texttt{classicthesis-config.tex}: a central place to load all 
%    nifty packages that are used. In there, you can also activate 
%    backrefs in order to have information in the bibliography about 
%    where a source was cited in the text (\ie, the page number).
%    
%    \emph{Make your changes and adjustments here.} This means that you  
%    specify here the options you want to load \texttt{classicthesis.sty} 
%    with. You also adjust the title of your thesis, your name, and all 
%    similar information here. Refer to \autoref{sec:custom} for more 
%    information.
%    
%		This had to change as of version 3.0 in order to enable an easy 
%		transition from the ``basic'' style to \mLyX.
%    
%\end{itemize}
%In total, this should get you started in no time.
%
%
%\section{Style Options}\label{sec:options}
%There are a couple of options for \texttt{classicthesis.sty} that
%allow for a bit of freedom concerning the layout:
%\marginpar{\dots or your supervisor might use the margins for some
%    comments of her own while reading.}
%\begin{itemize}
%	\item General:
%		\begin{itemize}
%			\item\texttt{drafting}: prints the date and time at the bottom of
%    each page, so you always know which version you are dealing with.
%    Might come in handy not to give your Prof. that old draft.
%		\end{itemize}
%	
%	\item Parts and Chapters:
%		\begin{itemize}
%			\item\texttt{parts}: if you use Part divisions for your document,
%    you should choose this option. (Cannot be used together with 
%    \texttt{nochapters}.)
%    
%			\item\texttt{nochapters}: allows to use the look-and-feel with 
%    classes that do not use chapters, \eg, for articles. Automatically
%    turns off a couple of other options: \texttt{eulerchapternumbers}, 
%    \texttt{linedheaders}, \texttt{listsseparated}, and \texttt{parts}. 
%    
%	    \item\texttt{linedheaders}: changes the look of the chapter
%	    headings a bit by adding a horizontal line above the chapter
%	    title. The chapter number will also be moved to the top of the
%	    page, above the chapter title.
%    
%		\end{itemize}
%
%  \item Typography:
%		\begin{itemize}
%				\item\texttt{eulerchapternumbers}: use figures from Hermann Zapf's
%    Euler math font for the chapter numbers. By default, old style
%    figures from the Palatino font are used.
%    
%        \item\texttt{beramono}: loads Bera Mono as typewriter font. 
%    (Default setting is using the standard CM typewriter font.)
%    \item\texttt{eulermath}: loads the awesome Euler fonts for math. 
%    (Palatino is used as default font.)
%    
%		    \item\texttt{pdfspacing}: makes use of pdftex' letter spacing
%		    capabilities via the \texttt{microtype} package.\footnote{Use 
%		    \texttt{microtype}'s \texttt{DVIoutput} option to generate
%		    DVI with pdftex.} This fixes some serious issues regarding 
%		    math formul\ae\ etc. (\eg, ``\ss'') in headers. 
%		    
%		    \item\texttt{minionprospacing}: uses the internal \texttt{textssc}
%		    command of the \texttt{MinionPro} package for letter spacing. This 
%		    automatically enables the \texttt{minionpro} option and overrides
%		    the \texttt{pdfspacing} option.
%    
%		\end{itemize}  
%
%	\item Table of Contents:
%		\begin{itemize}
%			 \item\texttt{tocaligned}: aligns the whole table of contents on
%		    the left side. Some people like that, some don't.
%		    
%		    \item\texttt{dottedtoc}: sets pagenumbers flushed right in the 
%		    table of contents.
%
%			\item\texttt{manychapters}: if you need more than nine chapters for 
%	    your document, you might not be happy with the spacing between the 
%	    chapter number and the chapter title in the Table of Contents. 
%	    This option allows for additional space in this context. 
%	    However, it does not look as ``perfect'' if you use
%	    \verb|\parts| for structuring your document.
%		    
%		\end{itemize}
%    
%	\item Floats:
%		\begin{itemize}
%    \item\texttt{listings}: loads the \texttt{listings} package (if not 
%    already done) and configures the List of Listings accordingly.
%    
%    \item\texttt{floatperchapter}: activates numbering per chapter for
%    all floats such as figures, tables, and listings (if used).	
%    
%	    \item\texttt{subfig}(\texttt{ure}): is passed to the \texttt{tocloft} 
%	    package to enable compatibility with the \texttt{subfig}(\texttt{ure}) 
%	    package. Use this option if you want use classicthesis with the
%	    \texttt{subfig} package.
%    	
%%    \item\texttt{listsseparated}: will add extra space between table
%%    and figure entries of different chapters in the list of tables or
%%    figures, respectively. % Deprecated as of version 2.9.
%		\end{itemize}    
% 
%% 	\item\texttt{a5paper}: adjusts the page layout according to the
%%    global \texttt{a5paper} option (\emph{experimental} feature).
%%    \item\texttt{minionpro}: sets Robert Slimbach's Minion as the 
%%    main font of the document. The textblock size is adjusted 
%%    accordingly.    
%
%   \end{itemize}
%The best way to figure these options out is to try the different
%possibilities and see, what you and your supervisor like best.
%
%In order to make things easier in general, 
%\texttt{classicthesis-config.tex} 
%contains some useful commands that might help you.
%
%
%\section{Customization}\label{sec:custom}
%%(As of v3.0, the Classic Thesis Style for \LaTeX{} and \mLyX{} share
%%the same two \texttt{.sty} files.)
%This section will give you some hints about how to adapt 
%\texttt{classicthesis} to your needs.
%
%The file \texttt{classicthesis.sty}
%contains the core functionality of the style and in most cases will
%be left intact, whereas the file \texttt{classic\-thesis-config.tex}
%is used for some common user customizations. 
%
%The first customization you are about to make is to alter the document
%title, author name, and other thesis details. In order to do this, replace
%the data in the following lines of \texttt{classicthesis-config.tex:}%
%\marginpar{Modifications in \texttt{classic\-thesis-config.tex}%
%}
%
%\begin{lstlisting}[frame=lt]
%% **************************************************
%% 2. Personal data and user ad-hoc commands
%% **************************************************
%\newcommand{\myTitle}{A Classic Thesis Style\xspace} 
%\newcommand{\mySubtitle}{An Homage to...\xspace} 
%\end{lstlisting}
%
%Further customization can be made in \texttt{classicthesis-config.tex}
%by choosing the options to \texttt{classicthesis.sty} 
%(see~\autoref{sec:options}) in a line that looks like this:
%
%\begin{lstlisting}[frame=lt]
%\PassOptionsToPackage{eulerchapternumbers,drafting,listings,subfig,eulermath,parts}{classicthesis}
%\end{lstlisting}
%
%If you want to use backreferences from your citations to the pages
%they were cited on, change the following line from:
%\begin{lstlisting}[breaklines=false,frame=lt]
%\setboolean{enable-backrefs}{false} % true false
%\end{lstlisting}
%to
%\begin{lstlisting}[breaklines=false,frame=lt]
%\setboolean{enable-backrefs}{true} % true false
%\end{lstlisting}
%
%Many other customizations in \texttt{classicthesis-config.tex} are
%possible, but you should be careful making changes there, since some
%changes could cause errors.
%
%Finally, changes can be made in the file \texttt{classicthesis.sty},%
%\marginpar{Modifications in \texttt{classicthesis.sty}%
%} although this is mostly not designed for user customization. The
%main change that might be made here is the text-block size, for example,
%to get longer lines of text.
%
%
%\section{Issues}\label{sec:issues}
%This section will list some information about problems using
%\texttt{classic\-thesis} in general or using it with other packages.
%
%Beta versions of \texttt{classicthesis} can be found at the following 
%Google code repository:
%\begin{center}
%	\url{http://code.google.com/p/classicthesis/}
%\end{center}
%There, you can also post serious bugs and problems you encounter.
%
%\subsection*{Compatibility with the \texttt{glossaries} Package}
%If you want to use the \texttt{glossaries} package, take care of loading it 
%with the following options:
%\begin{verbatim}
%	\usepackage[style=long,nolist]{glossaries}
%\end{verbatim}
%Thanks to Sven Staehs for this information. 
%
%
%\subsection*{Compatibility with the (Spanish) \texttt{babel} Package}
%Spanish languages need an extra option in order to work with this template:
%\begin{verbatim}
%	\usepackage[spanish,es-lcroman]{babel}
%\end{verbatim}
%Thanks to an unknown person for this information (via Google Code issue reporting). 
%
%
%\paragraph{Further information for using \texttt{classicthesis} with Spanish (in addition to the above)}
%In the file \texttt{ClassicThesis.tex} activate the language: 
%\begin{verbatim}
%	\selectlanguage{spanish}
%\end{verbatim}
%	
%In order to get the bibliography style right, you can use the following:
%\begin{verbatim}
%	\bibliographystyle{babplain}
%\end{verbatim}
%
%For this, it is necessary to load the package:
%\begin{verbatim}
%	\usepackage[spanish,fixlanguage]{babelbib}
%	\selectbiblanguage{spanish}
%\end{verbatim}
%
%If there are issues changing \verb|\tablename|, \eg, using this:
%\begin{verbatim}
%	\renewcommand{\bibname}{Referencias}
%	\renewcommand{\tablename}{Tabla}
%\end{verbatim}
%
%This can be solved by passing \texttt{es-tabla} parameter to \texttt{babel}:
%\begin{verbatim}
%	\PassOptionsToPackage{es-tabla,spanish,es-lcroman,english}{babel}
%	\usepackage{babel}
%\end{verbatim}
%
%But it is also necessary set \texttt{spanish} in the \verb|\documentclass|.
%
%Thanks to Alvaro Jaramillo Duque for this information. 
%
%
%\subsection*{Compatibility with the \texttt{pdfsync} Package}
%Using the \texttt{pdfsync} package leads to linebreaking problems with the \texttt{graffito} command. 
%Thanks to Henrik Schumacher for this information. 
%
%
%
%\section{Future Work}
%So far, this is a quite stable version that served a couple of people
%well during their thesis time. However, some things are still not as
%they should be. Proper documentation in the standard format is still
%missing. In the long run, the style should probably be published
%separately, with the template bundle being only an application of the
%style. Alas, there is no time for that at the moment\dots it could be
%a nice task for a small group of \LaTeX nicians.
%
%Please do not send me email with questions concerning \LaTeX\ or the
%template, as I do not have time for an answer. But if you have
%comments, suggestions, or improvements for the style or the template
%in general, do not hesitate to write them on that postcard of yours.
%
%
%\section{Beyond a Thesis}
%It is easy to use the layout of \texttt{classicthesis.sty} without the
%framework of this bundle. To make it even easier, this section offers 
%some plug-and-play-examples.
%
%The \LaTeX -sources of these examples can be found in the folder 
%with the name \texttt{Examples}. They have been tested with  
%\texttt{latex} and \texttt{pdflatex} and are easy to compile. To 
%assure you even a bit more, PDFs built from the sources can also 
%be found in the folder. 
%%(It might be necessary to adjust the path to 
%%\texttt{classicthesis.sty} and \texttt{Bibliography.bib} within the 
%%examples.)
%
%\lstinputlisting[caption=An Article]%
%    {Examples/classicthesis-article.tex}
%    
%\lstinputlisting[caption=A Book]%
%    {Examples/classicthesis-book.tex}
%
%\lstinputlisting[caption=A Curriculum Vit\ae]%
%    {Examples/classicthesis-cv.tex}
%
%
%\section{License}
%\paragraph{GNU General Public License:} This program is free software;
%you can redistribute it and/or modify
% it under the terms of the \textsmaller{GNU} General Public License as
% published by
% the Free Software Foundation; either version 2 of the License, or
% (at your option) any later version.
%
% This program is distributed in the hope that it will be useful,
% but \emph{without any warranty}; without even the implied warranty of
% \emph{merchantability} or \emph{fitness for a particular purpose}.
% See the
% \textsmaller{GNU} General Public License for more details.
%
% You should have received a copy of the \textsmaller{GNU} General
% Public License
% along with this program; see the file \texttt{COPYING}.  If not,
% write to
% the Free Software Foundation, Inc., 59 Temple Place - Suite 330,
% Boston, \textsmaller{MA} 02111-1307, \textsmaller{USA}.

%*****************************************
%*****************************************
%*****************************************
%*****************************************
%*****************************************




