%************************************************
\chapter{PAKE and Malicious Servers}\label{ch:malservers}
%************************************************
In this chapter we summarise contributions from Appendix \ref{paper:2pake} on distributed \aclp{SPHF} and two-server \ac{PAKE} as a mechanism to handle malicious servers and server compromise in \acl{PAKE}.
We further give an outlook on future work in this area and on blind password policy checking.

%************************************************
\section{Distributed Smooth Projective Hashing \& Two-Server PAKE} \label{sec:twopake}
%************************************************
This section is based on unpublished work presented in Appendix \ref{paper:2pake}.
We recall high-level intuition and most important findings of Appendix \ref{paper:2pake} and discuss future work on this topic.

While smooth projective hashing got a lot attention in the last ten years it has been used only in two-party protocols.
Smooth projective hashing allows to compute a hash value of a word from a language $L$ in two different ways.
Either by using a secret hashing key on the element, or utilising the public projection key and some secret information proving that the particular element is part of a specific subset under consideration.
Despite this neat feature, smooth projective hash values guarantee to be uniformly distributed in their domain as long as the input element is not from the language $L$.
These features make them a quite popular building block in many protocols such as \ac{CCA}-secure public key encryption, blind signatures, password authenticated key exchange, oblivious transfer, zero-knowledge proofs, commitments and verifiable encryption.
\ac{SPHF} are due to Cramer and Shoup \cite{Cramer2001} who used them to construct \ac{CCA}-secure public key encryption schemes and analyse mechanisms proposed in \cite{Cramer1998}.

We introduce the notion of \emph{distributed smooth projective hashing} that allows distributed computation of the hash value and demonstrate its use on the popular use case of \acl{PAKE}.
In particular, we use the distributed computation to build two-server password authenticated key exchange.
Similar to previous work on smooth projective hashing, this allows us to ``explain'' an earlier protocol on two-server password authenticated key exchange due to \citeauthor{Katz2012a} \cite{Katz2012a}.
The distributed \ac{SPHF} is actually a generalisation of the concepts used in \cite{Katz2012a}.
We use the \ac{SPHF} framework from \cite{Benhamouda2013} on cyclic groups $\GG$ of prime order and focus on languages of ciphertexts to prove general statements on distributed \acp{SPHF}.
A language $L$ is indexed by a parameter \texttt{aux}, consisting of global public information and secret variable information $\mathtt{aux}'$.
In our setting of languages of ciphertext the public part of \texttt{aux} is essentially the \ac{CRS} containing the public key of the used encryption scheme.
We define that a distributed \ac{SPHF} consists of two \texttt{Hash} functions.
One uses a set of words and hashing keys instead of a single word, hash key pair while the other operates on a set of hashing keys and a single word.
We introduce an extended notion of smooth projective hashing that allows us later to distribute the computation of their functions.

For distributed \SPHFF we require the existence of a function $h:\PP\mapsto\PP^x$ such that all elements of a word in $L_\aux$ are words of their respective language $L_{h(\aux)[i]}$ and the combination of the elements in ciphertexts $C$ are correct in respect to the combined $\aux'$.
More precisely, for every ciphertext $C$ of the message in $\aux'$, there exists a modified decryption algorithm $\Dec'$ and a combining function $g$ such that $\Dec'_{\pi}(C_0)=\Dec'_{\pi}(g(C_1,\dots,C_x))$ for $C_0\gets\Enc_{\pk}^\cL(\aux)$ and $C_i\gets\Enc_{\pk}^\cL(h(\aux)[i])$ for $i=1,\dots,x$, where $\pi$ denotes the secret key according to the public key $\pk$ in the \ac{CRS}.
Let $\odot$ be defined as follows: for $a\in \GG,~ r\in\ZZ_p,~ b\in \GG^{x\times y},~ s\in\ZZ_p^{1\times y},~ c\in \GG^{y\times 1}:$~ $a\odot r = r\odot a = a^r\in \GG$, using common matrix and vector operations on it. 

\begin{definition}[\SPHFF]\label{def:symgensphf}
Let $L_\aux$ denote a language such that $C=(C_0,C_1,\dots,C_x)\in L_\aux$ if there exists a witness $w=(w_0,w_1,\dots,w_x)$ proving so and there exist functions $h(\aux)=(\aux_1,\dots,\aux_x)$ and $g:\GG^l\mapsto\GG^{l'}$ as described above.
An extended smooth projective hash function for language $L_\aux$ with $\Gamma\in\GG^{k\times n}$ consists of the following 6 algorithms:

\begin{itemize}
	\item $\hk_i\ralgout\HKGen(L_\aux)$ generates a hashing key $\hk_i\in\ZZ_p^{1\times n}$ for language $L_\aux$.
	
	\item $\hp_i\algout\PKGen(\hk,L_\aux)$ derives the projection key $\hp_i=\Gamma \odot \hk\in\GG^{1\times k}$.
	
	\item $h^x\algout\Hash^x(\hk_0,L_\aux,C_1,\dots,C_x)$ outputs the hash value
	$$h^x=\Theta^x_{\aux}(C_1,\dots,C_x)\odot\hk_0=\prod^{n}_{j=1} \theta_\aux^j(C_1,\dots,C_x)^{\hk_0[j]}.$$
	
	\item $h^x\algout\PHash^x(\hp_0,L_\aux,C_1,\dots,C_x,w_1,\dots,w_x)$ returns the hash value where $\lambda^i=\Omega(w_i,C_i)$
	\[h^x=\prod^{x}_{i=1}(\lambda^i\odot \hp_0).\] 
	
	\item $h^0\algout\Hash^0(\hk_1,\dots,\hk_x,L_\aux,C_0)$ outputs the hash value 
	\[h^0=\prod^{x}_{i=1}(\Theta_{\aux}^0(C_0)\odot\hk_i)=\Theta_{\aux}^0(C_0)\odot \sum^x_{i=1}\hk_i.\]
	
	\item $h^0\algout\PHash^0(\hp_1,\dots,\hp_x,L_\aux,C_0,w_0)$ returns the hash value where $\lambda^0=\Omega(w_0,C_0)$
	\[h^0=\prod^{x}_{i=1}(\lambda^0\odot \hp_i).\]
\end{itemize}
\eod
\end{definition}

\noindent
Using \SPHFF we can define distributed computation protocols for $\PHash^x$ and $\Hash^0$ as follows which leads directly to the two-server \ac{PAKE} framework given in Figure \ref{fig:twopake}.
\begin{itemize}
\item $\PHash^x_D$ is executed between $x$ parties $P_1,\dots,P_x$.
	Each $P_i$ performs $\PHash^x_D$ on input $(\hp_0,\aux_i,\allowbreak C_1,\dots,C_x,w_i)$ such that $P_1$ eventually holds $h^x$ while all $P_i$ for $i>1$ do not learn anything about $h^x$.
	
	\item $\Hash^0_D$ is executed between $x$ parties $P_1,\dots,P_x$.
	Each $P_i$ performs $\Hash^0_D$ on input $(\aux_i,\sk_i,\hk_i,\allowbreak \pk_1,\dots,\pk_x,C_0,\dots,C_x)$ such that $P_1$ eventually holds $h^0$ and all $P_i$ for $i>1$ do not learn anything about $h^0$.
\end{itemize}

\noindent
Despite the new two-server \ac{PAKE} framework, \SPHFF allows us to ``explain'' the two-server protocol due to \citeauthor{Katz2012a} \cite{Katz2012a}.
Their protocol is based on the first two-party \ac{PAKE} using (implicitly) smooth projective hashing \cite{KatzOY01}.

\begin{sidewaysfigure}[htbp]
\centering
\begin{tikzpicture}[scale=0.6, every node/.style={scale=0.6}]
\matrix (m)[matrix of nodes, column  sep=.6cm,row  sep=1mm,
		nodes={draw=none, anchor=center,text depth=1pt},
		column 2/.style={nodes={minimum width=8em}},
		column 3/.style={nodes={minimum width=20em}},
		column 4/.style={nodes={minimum width=8em}}]{
	{$\bm{C}$} & & {$\bm{S_1}$} \\ [-1mm]
	
	$\crs,\pwd$ & & $\crs,\pwd_1$ & \\ [2mm]
	
	$\hk_0\gets \HKGen(L_\aux)$ & &
	$\hk_1\gets \HKGen(L_\aux)$ & \\
	
	$\hp_0\gets \PKGen(\hk_0)$ & &
	$\hp_1\gets \PKGen(\hk_1)$ & \\
	
	$C_{0}\gets\Enc_{\pk}^\cL(\ell_0,\pwd;r_0)$ & 
	$\hp_0,C_{0},\hp_1,C_1$ &
	$C_1\gets\Enc_\pk^\cL(\ell_1,\pwd_1,r_1)$ &
	$\hp_2,C_2$ \\
	
	check $C_1,C_2$ & & check $C_0,C_2$ & \\
	
	$h^0\gets\PHash^0(\hp_1,\hp_2,L_\aux,C_{0},r_0)$ & &
	$h^0\gets\Hash^0_{D}$ & \hspace*{8em} \\

	& & \hspace*{10em} & \hspace*{8em} \\
	
	$h^x\gets\Hash^x(\hk_{0},L_\aux,C_1,C_2)$ & &
	$h^x\gets\PHash^x_{D}$ & \hspace*{8em} \\ 
	
	$\sk_1=h^0 h^x$ & 
	& $\sk_1=h^0 h^x$ & \hspace*{8em} \\ 
};

\draw[<->,dashed] (m-5-2.south west)--(m-5-2.south east);
\draw[->,dashed] (m-5-4.south east)--(m-5-4.south west);

\draw[<->] (m-7-4.south west)--(m-7-4.south east);
\draw[<->] ([yshift=2pt] m-9-4.south west)--([yshift=2pt] m-9-4.south east);
\draw[dashed] ([xshift=0pt,yshift=0pt] m-7-3.north west) -- ([xshift=0pt,yshift=4pt] m-7-4.north east);
\draw[dashed] ([xshift=0pt,yshift=0pt] m-7-3.north west) -- ([xshift=0pt,yshift=0pt] m-9-3.south west);
\draw[dashed] ([xshift=0pt,yshift=5pt] m-8-3.south west) -- ([xshift=0pt,yshift=5pt] m-8-4.south east);
\draw[dashed] ([xshift=0pt,yshift=0pt] m-9-3.south west) -- ([xshift=0pt,yshift=-4pt] m-9-4.south east);

\end{tikzpicture}
\caption[Two-Server PAKE framework using \SPHFF]{Two-Server PAKE framework using \SPHFF
\\{\tiny Dashed lines denote broadcast messages.}}
\label{fig:twopake}
\end{sidewaysfigure}

%************************************************
\section{Future Work}
%************************************************
Two-server password-based protocols are less popular than common two-party protocols and thus lack the diversity of protocols one can find in the two-party setting.
It therefore offers many possibilities for future research such as \ac{UC}-secure two-server \ac{PAKE} or password-only two-server secret sharing.

%************************************************
\paragraph{Blind Password Policy Checker} \label{sec:policies}
%************************************************
A so far unexplored field in password-based cryptography and closely related to two-server \ac{PAKE} protocols is the are of secure password policy checking.
While \ac{PAKE} protocols solve the problem of transmitting passwords when logging in, it still has to be stored on the server and thus is vulnerable to server compromise.
Two-server \ac{PAKE} protocols allow to perform a \ac{PAKE} protocol without the need to disclose, i.e. store, the entire password on one server.
In this case however it is not possible for a server to check that the client uses a password that is policy conform.
We therefore research the new field of \emph{blind password policy checking} that allows a server that knows only a share of a password to verify that the actual password is conform with his policy.


