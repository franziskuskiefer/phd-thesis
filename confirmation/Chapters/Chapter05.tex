%************************************************
\chapter{PAKE and Malicious Servers}\label{ch:malservers}
%************************************************
In this chapter we summarise contributions from Appendix \ref{paper:2pake} on distributed \aclp{SPHF} and two-server \ac{PAKE} as a mechanism to handle malicious servers and server compromise in \acl{PAKE}.
We further give an outlook on planned work on blind password policy checking.

%************************************************
\section{Distributed Smooth Projective Hashing \& Two-Server PAKE} \label{sec:twopake}
%************************************************
This section is based on unpublished work presented in Appendix \ref{paper:2pake}.
We recall high-level intuition of Appendix \ref{paper:2pake} and discuss future work on this topic.

While smooth projective hashing got a lot attention in the last ten years it has been used only in two-party protocols.
Smooth projective hashing allows to compute a hash value of a word from a language $L$ in two different ways.
Either by using a secret hashing key on the element, or utilising the public projection key and some secret information proving that the particular element is part of a specific subset under consideration.
Despite this neat feature, smooth projective hash values guarantee to be uniformly distributed in their domain as long as the input element is not from the language $L$.
These features make them a quite popular building block in many protocols such as \ac{CCA}-secure public key encryption, blind signatures, password authenticated key exchange, oblivious transfer, zero-knowledge proofs, commitments and verifiable encryption.
\ac{SPHF} are due to Cramer and Shoup \cite{Cramer2001} who used them to construct \ac{CCA}-secure public key encryption schemes and analyse mechanisms proposed in \cite{Cramer1998}.

We introduce the notion of \emph{distributed smooth projective hashing} that allows distributed computation of the hash value and demonstrate its use on the popular use case of \acl{PAKE}.
In particular, we use the distributed computation to build two-server password authenticated key exchange.
Similar to previous work on smooth projective hashing, this allows us to ``explain'' an earlier protocol on two-server password authenticated key exchange due to \citeauthor{Katz2012a} \cite{Katz2012a}.
The distributed \ac{SPHF} is actually a generalisation of the concepts used in \cite{Katz2012a}.
We use the \ac{SPHF} framework from \cite{Benhamouda2013} on cyclic groups $\GG$ of prime order and focus on languages of ciphertexts to prove general statements on distributed \acp{SPHF}.
A language $L$ is indexed by a parameter \texttt{aux}, consisting of global public information and secret variable information $\mathtt{aux}'$.
In our setting of languages of ciphertext the public part of \texttt{aux} is essentially the \ac{CRS} containing the public key of the used encryption scheme.
We define that a distributed \ac{SPHF} consists of two \texttt{Hash} functions.
One uses a set of words and hashing keys instead of a single word, hash key pair while the other operates on a set of hashing keys and a single word.
We introduce an extended notion of smooth projective hashing that allows us later to distribute the computation of their functions.
\fk[inline]{finish....}

%************************************************
\section{Blind Password Policy Checker} \label{sec:policies}
%************************************************
\fk[inline]{ToDo: Policies}
